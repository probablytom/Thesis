\labelledsec{Aspect Orientation in Simulation \& Modelling}{ao_and_modelling}

% Aspect orientation as it applies to modelling

Having discussed aspect orientation as it is used in a simulation context, it is
natural to investigate its use in modelling research, too.

Simulation and modelling are similar topics and are often combined into a single
study. However, their goals differ. Simulation typically involves the study of
processes or behaviour: there is an expectation that simulations are
\emph{executed} or \emph{run}. This often produces data. The intent of modelling
is more structural in nature: models are typically observed or analysed to gain
insights. Quoting \citeauthor{smintro}'s introduction~\cite{smintro}:

\begin{displayquote}
    Modeling \emph{[sic]} is the process of producing a model; a model
    is a representation of the construction and working of
    some system of interest. A model is similar to but
    simpler than the system it represents.
    \newline{}
    [ \ldots{} ]
    \newline{}
    A simulation of a system is the operation of a model of the system. The
    model can be reconfigured and experimented with; usually, this is
    impossible, too expensive or impractical to do in the system it represents.
    The operation of the model can be studied, and hence, properties concerning
    the behavior of the actual system or its subsystem can be inferred.
\end{displayquote}

\citeauthor{smintro}'s definition implies that to simulate is to operate a
model. Whether this model is constructed for the purpose of simulation or for
study in its own right, a simplified representation of the system being studied
is implicitly required for any simulation. However, modelling does not imply
simulation. Models can be studied for their own merits, and many modelling
frameworks exist which are made explicitly for their own study, without regard
to their use in simulation\footnote{Consider UML, a well-studied modelling
framework which is generally not used for any kind of simulation --- depending
on the use case, it often cannot be --- and for which many alternatives now
exist specifically to address this
limitation~\cite{opm_original,ExecutableBPMNMitsyuk,obashimethodology}.}. Aspect
orientation has seen some study in modelling, particularly for \sociotechnical
modelling, and while aspect-oriented \sociotechnical modelling is not generally
researched with subsequent simulation in mind, an important body of work is
still present, and therefore important to discuss.

\subsection{Aspect Orientation in Business Process Modelling}
Aspect orientation for \sociotechnical systems is particularly well studied in
the business process modelling
community\cite{Machado_2011,Cappelli_AOBPM}~\inline{find more citations for AOBPM}


\subsection{MAML \& SWARM}
\inline{Is MAML/SWARM really modelling, or simulation? Simulation, right?}


\subsection{BPMN \& aspect orientation}
\inline{There's tons here. Go through the remarkable read folder. Particularly
anything Claudia Capelli's worked on, see
\cite{da2020implementation,Cappelli_AOBPM}. Also the widely cited work on aspect
orientation in BPEL\cite{charfi2007ao4bpel}, and the work on precedence of
aspect application in \cite{jalali2012aspect}, }


% \section{Aspect Orientation \& Simulation}\label{sec:ao_and_simulation}
\labelledsec{Aspect Orientation \& Simulation}{ao_and_simulation}

\inline{The simulation section \emph{badly} needs revisiting.}

Surprisingly, little literature exists pertaining specifically to the use of
aspect-orientation in a simulation context. Aspect orientation is often applied
to modelling as discussed in \cref{sec:ao_and_modelling}, used to compose a
perspective of the world from individual parts, but in a way which isn't
necessarily executable or able to produce data.

Early in the history of aspect orientation as an emerging paradigm, there was
some interest in its use for scientific simulation. \cite{gulyas1999use} discuss
that computer simulations require code for both observation of a simulation and
the simulation itself, and that misuse of this could cause what is in effect a
kind of Hawthorne Effect\inline{Does hawthorne effect need a citation?}, where
the inclusion of observation code intertwined with simulation code might
influence the outcome of an experiment. They suggest that improving simulation
technologies could combat this approach. Aspect Orientation, being developed
specifically with obliviousness in mind, is an ideal candidate which
\citeauthor{gulyas1999use} identify.

Much of the literature concerning aspect-oriented programming and simulation
focuses on tooling support for aspect-oriented simulation, rather than
investigations into its efficacy. For example, attempts have been made to
integrate aspect orientation into new
tools~\cite{DEVSaspectorientation2008aksu,ribault2008OSA,ribault2010osif,} or
into existing
ones~\cite{chibani2019using,DEVSaspectorientation2008aksu,wallis2018caise}.
Typically, these papers identify a need for aspect orientation in simulation
frameworks --- the argument often revolves around a need for increased
modularity, and occasionally around better structuring of the simulations
themselves\footnote{See \citeauthor{chibani2014practical}'s series of papers on
the topic~\cite{chibani2013toward,chibani2014practical,chibani2019using}, which
culminate in an implementation of a suite of aspects for simulation purposes.
Unfortunately, the work still does not produce significant case studies showing
the benefits of the technique in practice. Some empirical measurements are made.
A lack of significant real-world evidence that the technique works is a major
criticism of aspect orientation as a paradigm~\cite{steimann06paradoxical},
however, and the use of these empirical measurements designed for different
paradigms as a sign of success hints that satisfactory results in the
measurement are being treated as more important than empirical effectiveness, an
instance of Goodhart's Law: ``when a measure becomes a target, it ceases to be a
good measure''\cite{strathern1997improving}.\inline{refactor this footnote into
its own paragraph underneath its currently enclosing one.}} --- but past a small example to
demonstrate how the tooling can be used, little additional development is
performed. No significant case studies or refactoring of existing codebases of
notable scale are provided. This is important because aspect orientation's main
strength is pragmatic in nature. If no real-world testing is conducted, it is
hard to conclude that the community's suite of modelling tools contribute
anything useful when developing simulations.

Some experiments specifically using aspect orientation in the implementation of
process-based simulations also exist\cite{Ionescu_2009}~\inline{include more!}.
For example, \citeauthor{Ionescu_2009} apply aspect orientation in a nuclear
disaster prevention simulation. Their motivation is that code can become complex
to maintain over time and changes to the scientific zeitgeist or to regulatory
requirements leads to costly technical debt. Aspect orientation therefore allows
developers to separate functionality into distinct modules more easily, without
disturbing the underlying codebase.

\subsection{Aspect-oriented L-Systems}
Aspect-orientation is also applied in other simulation paradigms.
\citeauthor{Cieslak_2011} investigated the use of aspect orientation in L-system
based simulations~\cite{Cieslak_2011}. An L-system\cite{lindenmayer1968lsystem}
is defined by a set of symbols, an initial string composed of these symbols, and
a set of rules for rewriting substrings. While being a powerful tool for
representing fractal structures, they were originally conceived of for plant
modelling (and still see the most use in this field).

\citeauthor{Cieslak_2011} note that some details of plant modelling are actually
cross-cutting concerns against many plants or families of plants. To represent
these, they introduce a new language to describe plant models which makes use of
aspect orientation to represent these cross-cutting concerns. They test the
approach by representing carbon dynamics, apical dominance and biomechanics as
cross-cutting concerns that are integrated into a previously published model of
kiwifruit shoot development. \citeauthor{Cieslak_2011} hope that these
cross-cutting concerns might work in other models too, but this is untested. The
use of an aspect in a new model, when developed for another, seems untested in
the community's literature writ large and is a noted omission in the conclusion
of this particular work.



\subsection{AOP and simulation tooling}

\inline{Here, I should be discussing: japrosim work by chibani
\cite{chibani2019using,chibani2014practical,chibani2013toward}; OSIF \& OSA in
\cite{ribault2008OSA,ribault2010osif}; 
}
