\chapter{Relevant Literature}\label{chap:lit_review}

\revnote{Include a review of the survey in
\citet{masuhara2003modeling}.}

This thesis presents an \aspectoriented{} approach to simulation \& modelling
and to experiment design, with tooling to support these endeavours and an
empirical assessment of the paradigm's application in the domain of simulation
\& modelling. In particular, the presented research makes use of aspects to
augment models. The research presented in later chapters is motivated in this
chapter through a review of the related literature.

Research related to this work comes from a variety of fields which do not
overlap significantly, meaning that this review must cover segments of several
partly related fields. The presented material is eclectic in nature as a result.

\inline{FINISH THIS AFTER EDITING / REARRANGING THE LIT REVIEW}
The review begins with an overview of \aop{}'s design philosophy



% philosophy toward software engineering and modularity, and some criticisms of
% its high-level design. Other sections, in particular
% \cref{sec:dynamic_aop_review} and \cref{sec:ao_and_modelling}, expand on this
% introduction by investigating implementations, technologies and use cases of
% particular interest.\revnote{Reword if I move criticisms to another section.}
% \revnote{I believe this para was mangled by emacs somehow, and is the end of a
%   drafted final paragraph of this introduction. Keeping it because it wasn't
%   committed so can't restore the original para, and ideally I'll recreate it
%   when redrafting this chapter.}

\section{\AOP{}}
\label{aspect_oriented_programming_litreview}

\subsection{Motivations \& Philosophy Underlying \AspectOrientation{}}
\label{review_aop_motivations_and_philosophy}

Modularity is considered a key trait of maintainable, flexible, and legible
programs~\cite{Parnas_1972,dijkstra1968letters}. Modern modular design
techniques are often concerned with segmenting a program's logic or data into
separate units. The standard approach to designing with modularity in mind in
many industrially-relevant languages today is object-oriented programming, where
data structures and selected program logic are modularised as
classes.

Some concepts in software engineering are not easily modularised by techniques
such as object orientation. Guarding against unsafe concurrency usage, manual
memory management, and logging are all examples of program components which are
not readily modularised using an object-oriented or similar approach, because
they occur alongside the logic fulfilling a program's requirements. These parts
of a program are common across many modules; however, this common logic does not
form traditional modules. Such concerns of a program are termed ``cross-cutting
concerns'' in \aop{} parlance. Programmers looking to separate cross-cutting
concerns into separate modules seek to address two
problems they give rise to~\cite{kiczales1997aspect}:

\begin{enumerate}
    \item \emph{``Tangling''}, where program logic essential for
    a program's intended purpose is intermixed with ancillary code addressing
    cross-cutting concerns, thereby making essential logic more difficult to
    maintain;
    \item \emph{``Scattering''}, where program logic for
    cross-cutting concerns is strewn throughout a codebase, making maintenance
    of this code more difficult.
\end{enumerate}

The existence of cross-cutting concerns is expected to make maintenance of both
ancillary program logic and a program's core logic more difficult because of
these traits. To address them, \citet{kiczales1997aspect} introduced the notion
of aspect-oriented programming. The paradigm is simple to define through its
unique software concepts:

\begin{itemize}
    \item A \emph{``join point''} defines some point in a program's execution
    (usually the moment of invocation or return of some function or method)
    where additional logic is required.
    \item \emph{``Advice''} implements some behaviour such as logging, which
    which can conceptually happen anywhere in program execution (i.e. what's
    defined would typically represent behaviour which cuts across many parts of
    a codebase).
    \item An \emph{``aspect''} is constructed by composing this advice with
    \emph{``point cuts''}: sets of join points that define all moments in
    program execution where associated advice is intended to be invoked.
    \item An \emph{``aspect weaver''} then adds the functionality defined by
    each aspect by adding the functionality defined by its advice at each join
    point defined by its point cut.
\end{itemize}

The implementation of join points, and advice, or weaving is a matter left for
\aspectorientation{} frameworks and \aspectoriented{} languages to define. In
employing the technique, \aop{} aims to separate cross-cutting concerns into
aspects, removing the aforementioned repetitive code from the logic implementing
a program's functional behaviour so that additional pieces of functionality ---
logging, authentication, and so on --- can be maintained in only one place in a
codebase (thereby simplifying their maintenance and comprehension), and
remaining program logic can be understood and maintained without the overhead
imposed by the previously tangled cross-cutting concerns.

\citet{kiczales1997aspect} see these engineering concepts as universal
throughout business logic, motivating the \aspectoriented{} approach for the
first time. The authors present an implementation of \aop{} in Lisp, and compare
implementations by way of SLOC count in an emitted C program to a comparable,
non-AOP implementation, with two examples (its use in image processing and
document processing). They find the idea --- which they note is "young" and note
many areas where research might help it to grow --- can successfully separate
systemic implementation concerns such as memory management in a way that reduces
program bloat and simplifies implementation. It is noted that measuring the
benefits of their approach quantitatively is challenging.

Tooling followed the theoretical work presented by \citet{kiczales1997aspect}
with a demonstration and subsequent technical description of AspectJ, a Java
extension for aspect oriented
programming~\cite{AspectJLanguageAndTools,aspectj_intro}. AspectJ was introduced
to satisfy the research community's need for a tool with which to demonstrate
the aspect-oriented paradigm in case studies. The tool is intended to serve as
``the basis of an empirical assessment of aspect-oriented
programming''~\cite{aspectj_intro}. The library makes use of standard
aspect-oriented concepts: Pointcuts, Join Points, and Advice, bundled together
in Aspects. They define ``dynamic'' and ``static'' cross-cutting, by which they
refer to join points at specific points in the execution of a program, and join
points describing specific types whose functionality is to be altered in some
way. Their paper describes only ``dynamic'' cross-cutting, but presents tooling
support, architectural detail of its implementation, and the representation \&
definition of pointcuts in AspectJ. AspectJ is compared to other tools for
\aspectorientation{} and related decomposition-focused paradigms, and the authors
are explicit about their approach being distinct from metaprogramming in, for
example, Smalltalk or Clojure.

\citet{filman2000aspect} isolate properties of \aspectorientation{} which they
assert are definitive of the paradigm. Specifically, they claim that \aop{}
should be considered in terms of ``quantification'' and ``obliviousness'':

\begin{displayquote}
AOP can be understood as the desire to make quantified statements about the
behaviour of programs, and to have these quantifications hold over programs
written by oblivious programmers.~[\ldots{}]~We want to be able to say, ``This
code realises his concern. Execute it whenever these circumstances hold.''
\end{displayquote}

These concepts became core concepts in \aspectorientation{} literature alongside
``tangling'' and ``scattering''. \citeauthor{filman2000aspect} give no concrete
definition of the terms in their original paper and cite no sources for the
terms. This thesis therefore adopts the definitions:

\begin{description}
  \item[Quantification:] the specification of points in a program in
  which that program should change;
  \item[Obliviousness:] the property of a codebase that it contains no lexical
  or conceptual reference to advice which might be applied to it, and of the
  programmer of a target program that their code may be amended by an aspect
  programmer.
\end{description}

\citeauthor{filman2000aspect} write about \aspectorientation{} as a paradigm
independent of a particular application or implementation. They are therefore
able to arrive at conclusions about the paradigm in the abstract and can
identify concerns for future researchers in the field and design goals for
developers of \aspectoriented{} tooling. They note:

\begin{displayquote}
  Better AOP systems are more oblivious. They minimise the degree to which
  programmers (particularly the programmers of the primary functionality) have
  to change their behaviour to realise the benefits of AOP. It's a really nice
  bumper sticker to be able to say, ``Just program like you always do, and we'll
  be able to add the aspects later.'' (And change your mind downstream about
  your policies, and we'll painlessly transform your code for that,
  too.)~\cite{filman2000aspect}
\end{displayquote}

Here, \citeauthor{filman2000aspect} theorise about the advantages they expect
\aspectorientation{} to bring software engineers. However, the notion that
obliviousness is a straightforward advantage of \aop{} has been contested in
more recent literature. This is discussed further in
\cref{subsec:aop-criticisms}.

Aside from the benefits of \aop{}'s obliviousness, claims made by
\citeauthor{filman2000aspect} such as the ``painless transformation'' of code
when a developer changes their mind during the development process --- another
theorised benefit of obliviousness --- is incompatible with earlier writing on
modularity. \citet{yourdon1979structured} assert:

\begin{displayquote}
  The more that we must know of module B in order to understand module A, the
  more closely connected A is to B. The fact that we must know something about
  another module is a priori evidence of some degree of interconnection even if
  the form of the interconnection is not known.
\end{displayquote}

\Aspectorientation{}'s critics describe similar incompatibilities with existing
best-practices~\cite{przybylek2010wrong,Constantinides04aopconsidered}, as well
as the lack of empirical evidence for the benefits of
obliviousness~\cite{steimann06paradoxical}. Claims about ``better''
aspect-oriented systems being more oblivious should therefore be regarded as
\emph{suggestions} from the literature. While obliviousness and quantification
are useful concepts in discussing research in the field, they also give context
for the research community's perspective that obliviousness and quantification
are design goals for those using
\aop{}~\cite{AspectCplusplusDesignImpl,kell2008survey,Charfi2006AspectOrientedWL}
(though \citet{leavens2007multiple} suggest they may be best applied in
moderation).

\Aop{} is thus a contentious paradigm which addresses limitations
of existing methods using a novel approach, but its design philosophy also
attracts sceptics. More specific criticisms of \aop{} are reviewed in
\cref{subsec:aop-criticisms}; specific implementations of \aop{} language
extensions and frameworks are reviewed first.


\subsection{Approaches to Weaving Aspects}
\label{sec:dynamic_aop_review}

The design of the weavers in different \aspectorientation{} frameworks and
language extensions is pertinent to the research in this thesis, because the
research introduced 

To review and compare different \aop{} frameworks, it is important to
differentiate their weavers. Many designs for weavers exist. Some weavers
introduce aspects when code is compiled, calcifying the program's behaviour and
disallowing extension at a later time. Others weave advice as a program
executes. To flexibly support the needs of a developer of a scientific model,
this thesis focuses on the latter type of weaver. These weavers are referred to
as ``dynamic''.

\citet{dynamicAOchitchyan} present a review of \aspectorientation{} frameworks
with dynamic weavers. They compare AspectWerkz, JBoss, Prose, and Nanning
Aspects through the lens of the authors' prior work on dynamic reconfiguration
of software systems. By comparing different implementations of dynamic weaving,
\citeauthor{dynamicAOchitchyan} contribute a categorisation of the tools'
approaches: \revnote{Need citations for these technologies}

\begin{enumerate}
\item ``Total hook'' weaving, where aspect hooks are woven at all possible
points;
\item ``Actual hook'' weaving, where aspect hooks are woven where required;
\item ``Collective'' weaving, where aspects are woven directly into the executed
code, ``collecting the aspects and base in one unit''.
\end{enumerate}

As \citeauthor{dynamicAOchitchyan} focus on software reconfiguration rather than
the mechanics and design of dynamic aspect weaving, their analysis of the
reviewed tools is of less relevance to the work presented in this thesis than
their generalisation of dynamic weaving. However, their review discusses the
trade-offs of the three approaches identified. \citeauthor{dynamicAOchitchyan}
propose that total hook weaving allows flexibility in the evolution of a
software product, at the expense of the performance of that product; this
contrasts collected weaving, which shifts overhead out of the codebase and into
the maintenance effort. Actual hook weaving is positioned as a compromise
between the two, offering the best approach for none of their criteria but never
compromising so much as to offer the worst, either. This suggests merit in a
tool designed to flexibly offer any weaving approach appropriate for the task at
hand.

\citeauthor{dynamicAOchitchyan} note that one could use many of
the systems they describe in practice. Though the paper is an early publication
in the field, no tool the authors review offers all three dynamic weaving
approaches, and none offers collective weaving alongside either kind of hook
weaving.

\subsection{AspectJ}

Following their seminal work on \aspectorientation{}, \citet{aspectj_intro}
published an \aop{} extension to Java called AspectJ, as discussed earlier in
\cref{review_aop_motivations_and_philosophy}. AspectJ is still actively
developed, and is currently maintained by the Eclipse organisation. It is also
the most well-adopted tool for \aop{} today, and its language for specifying
aspects is adopted by other tools such as Spring
AOP~\cite{introducing_spring_aop_chapter_integration_with_aspectj}. AspectJ can
weave aspects at compile time or when join points are loaded, resolving the
trade-offs between collective weaving and total hook weaving by allowing
developers to select the weaving method appropriate for their use-case.

The motivations behind developing AspectJ were those of \aop{}, as the paradigm
lacked an implementation at the time. Later \aspectorientation{} research
efforts have since been folded into the project, such as AspectWerks' dynamic
weaving mechanisms, or have been reproduced within it, such as support within
the JVM~\cite{golbeck2008lightweight} and late binding of
aspects~\cite{golbeck2010late}. The popularity of the framework has led it to
see some industry adoption~\cite{aspectj_in_industrial_projects}, although its
use in real-world settings has been found to introduce a trade-off between the
legibility of a codebase and the ease with which that codebase can be
updated~\cite{przybylek2018empirical}.
\revnote{
I feel like AspectJ should probably have a larger section here, but honestly,
it's hard to know what to say about it\ldots{} hopefully this is sufficient.
Only revisit if there's time.
}

\revnote{
  Ideally there'd be a section for Spring AOP here too as another
  industry-focused \aop{} offering, but I don't think it's too important as I'm
  not aware of it bringing any advancements of interest for the research. Maybe
  make reference to that here, given it's kind of related to AspectJ (in that
  they have the same audience?)\ldots{} or revisit if there's time I suppose.
}


\subsection{PROSE}

Another implementation of \aop{} with dynamic weaving is
PROSE~\cite{popovici2002PROSE,popovici2003JITaspects}, a library which achieves
dynamic weaving by use of a Just-In-Time compiler for Java.
\citeauthor{popovici2002PROSE} motivate the project be identifying aspect
orientation as a possible solution to software's increasing need for adaptivity.
Mobile devices, for example, could enable a required feature by applying an
aspect as a kind of hotfix, thereby adapting over time to a user's needs. Other
uses of dynamic \aspectorientation{} they identify are in the process of
software development: as aspects are applied to a compiled, live product, the
join points being used can be inspected by a developer to see whether the
correct pointcut is used. If not, a developer could use dynamic weaving to
remove a misapplied aspect, rewrite the pointcut, and weave again without
recompiling and relaunching their project.

\citet{popovici2003JITaspects} indicate that some performance issues may prevent
dynamic \aspectorientation{} from being useful in production software, but that
it presented opportunities in a prototyping or debugging context. The PROSE
project explores dynamic weaving as it could apply in a development context, but
the authors do not appear to have investigated dynamic weaving as it could apply
to simulation contexts, or others where aspect-oriented software does not
constitute a product.

Their observation that aspect-oriented programming could be used for the purpose
of adaptation or software prototyping instead of modularisation is an example of
an alternative use case for \aspectorientation{} suggested in
\citeauthor{steimann06paradoxical}'s critique~\cite{steimann06paradoxical}
discussed in \cref{subsec:aop-criticisms}.



\subsection{Nu}

\citeauthor{rajan2006nu_towardsao_invocation} propose a new aspect-oriented
invocation mechanism, which they call
``Bind''~\cite{rajan2006nu_towardsao_invocation}. Bind's design is motivated by
opportunities to improve modularity from a design perspective:
\citeauthor{rajan2006nu_towardsao_invocation} assert that scattering and
tangling can be introduced into a codebase \emph{after} weaving with some weaver
implementations. This complicates the use of compiled aspect-oriented code, and
the development and execution of unit tests on such a codebase. Bind alleviates
this issue by allowing a developer to choose when aspects are applied. This new
mechanism is presented as an alternative to the weaving of aspect hooks for
load-time registration into target code in the style of
PROSE~\cite{popovici2002PROSE,popovici2003JITaspects} and the direct weaving of
aspect invocations in the style of AspectJ~\cite{aspectj_intro}.

In order to demonstrate Bind's approach to simplifying post-weave codebases,
\citet{rajan2006nu_towardsao_invocation} present ``Nu'', an \aspectorientation{}
framework written in .NET supporting Bind. Nu's design is explained and its
implementation presented, which aim to promote granularity in join point
specification. What results is a flexible model for \aspectorientation{} which
is demonstrated to satisfactorily emulate many other paradigms and tools, such
as \aspectorientation{} in AspectJ and subject-orientation in
HyperJ.\revnote{Might not be an issue but I'm planning to drop SOP due to time
constraints, so this may stick out. Might not need defining though, presented in
this way I don't think the specifics of SOP matter all that much.}

\citet{dyerNUmasters} give a more thorough technical explanation of the design
and implementation of the Bind mechanism and the implementation of the Nu
framework. A more technical discussion is presented, in particular on
implementation details including optimisation and benchmarking, largely against
AspectJ. The implementation discussed is a Java implementation, rather than the
.Net implementation presented by \citet{rajan2006nu_towardsao_invocation}. Many
\aspectorientation{} frameworks are language-specific; the existence of Nu's
implementation on multiple platforms highlights the work's main contribution in
the design of the Bind primitive, rather than the framework itself.

It is noted that it is ``very common in aspect-oriented programming research
literature to provide language extensions to support new properties of
aspect-like constructs''.\revnote{Who notes this?}
\citeauthor{rajan2006nu_towardsao_invocation} look to provide similar extensions
to virtual machines, noting that their mechanism is a suitable candidate to
introduce \aspectorientation{} directly in a language's virtual machine. They
position the project as a general model of aspect-oriented programming which can
flexibly represent a variety of existing approaches. Bind fulfils this aim by
providing a single mechanism which can be used to achieve many weaving
techniques; however, case studies demonstrating that Bind can be used to make
\aspectorientation{} more practically effective for software developers does not
appear to have been published.


\subsection{Dynamic Weaving in Embedded Systems}

\citet{gilani2004family} observe that while there are different approaches
to dynamically weaving aspects, no approach is suitable for an embedded
environment. This is due to these systems' low power and
limited memory. \citeauthor{gilani2004family} propose a framework for these
situations through which weavers can be assessed for their suitability in a
given environment, or generated from a set of desired features.

\citeauthor{gilani2004family} define families of weavers by grouping the the
environments they can be suitably applied to, separating them in particular by
their trade-off between dynamism and resource use. There is an implication that
dynamism and resource use are broadly proportional, presumably because even a
carefully crafted ``actual hook weaver'' or JIT-compiled ``collective weaver''
in the parlance of \citet{dynamicAOchitchyan} carries runtime overhead by virtue
of the weaving mechanism used. A ``collective weaver'' embeds aspect invocations
directly into their join points, which eliminates the need for an intermediary
hook invocation or additional compilation. The embedded systems of interest to
\citeauthor{gilani2004family} have memory in the range of \lstinline{~30kb},
where these overheads could represent significant resource use.

Aspect oriented programming's criticism is sometimes that it doesn't know what
it ``aims to be good for''~\cite{steimann06paradoxical}, and the dynamic weaving
of aspects may not be good for resource-constrained environments. As
\citeauthor{gilani2004family} identify a trade-off between dynamism and resource
economy, a highly constrained environment would not be able to take advantage of
dynamic weaving by definition. Additionally, because the anticipated benefits of
aspect-oriented programs are not observed in practice and others have noted the
inefficiencies introduced by dynamic
weaving~\cite{dyerNUmasters,dynamicAOchitchyan}, the paradigm is difficult to
apply to any use-case where conservative resource usage is a concern. We might
observe that researchers should seek other contexts in which to apply aspect
oriented programming.

Support for this observation can be found in criticisms of \aop{} from
\citet{steimann06paradoxical}, whose position that alternative use-cases of
\aspectorientation{} should be explored is discussed in
\cref{subsec:aop-criticisms}. Examples of alternative uses include
\citeauthor{popovici2002PROSE}'s proposal of using \aspectorientation{} for
software prototyping, the proposal presented by \citet{gulyas1999use} that
simulation \& modelling is a more appropriate field, or \citet{Cieslak_2007}'s
use of \aspectorientation{} in modelling plant growth using L-systems. These
proposed uses of \aspectorientation{} involve a different resource economy than
that imposed by embedded systems. A discussion of \aspectorientation{}'s use in
simulation \& modelling can be found in \cref{sec:ao_and_modelling}.


\subsection{Criticisms of \AspectOrientation{}}
\label{subsec:aop-criticisms}

As referenced in \cref{review_aop_motivations_and_philosophy}, the design
philosophy underpinning \aop{} has its detractors. The opposition to
\aspectorientation{} is of particular relevance to the work presented in this
thesis, which should be understood within the context of some perceived
weaknesses of the paradigm. The following chapters introduce contributions which
address some criticisms of \aop{}, so broad criticisms of the paradigm --- and
the work published in awareness of those weaknesses --- motivates some research
presented in later chapters.


\citeauthor{Constantinides04aopconsidered} published an early critique of
aspect-oriented programming~\cite{Constantinides04aopconsidered} which notes
similarities between the paradigm's core concepts and \lstinline{GO-TO}
statements. \citeauthor{Constantinides04aopconsidered} uses the comparison to
demonstrate fundamental issues with the paradigm's philosophy: use of
\lstinline{GO-TO} statements is widely accepted to be bad
practice~\cite{dijkstra1968letters}, and their infamy has become an in-joke for
academics and language enthusiasts~\cite{clark73comefrom}. They note that
\lstinline{GO-TO} statements disorientate a programmer by way of ``destroying
their coordinate system'' --- referring to a developer's technique for
navigating and understanding code --- which results in uncertainty about both a
program's flow of execution and the state of a program at different points of
its flow. \citeauthor{Constantinides04aopconsidered} uses the comparison to
\lstinline{GO-TO} statements to question whether aspect-oriented programs can
have a consistent coordinate system for developers: \lstinline{GO-TO} statements
lack obliviousness as they are visible in disrupted code, wheras aspects are not
represented structurally within a program due to their oblivious design. This
complicates a developer's understanding of where and how flow is interrupted.
They draw comparison between aspects and \lstinline{COME-FROM} statements, an
April Fools' joke where a claimed improvement over \lstinline{GO-TO} is
developed by removing the latter's structural component~\cite{clark73comefrom},
and conclude that existing engineering techniques provide similar benefits
without a trade-off in program legibility. In particular, Dynamic Dispatch is
identified as a preferred alternative.


\citeauthor{steimann06paradoxical} makes a similar but more thorough critique of
aspect-oriented programming~\cite{steimann06paradoxical}. They express concern
that the popularity of aspect-oriented programming --- which was nearly 10 years
old at time of publication --- was founded on the perception that it would solve
real-world engineering problems, yet no proof existed that it was effective in
practice. \citeauthor{steimann06paradoxical} notes that most papers are
theoretical in their discussion on tooling, that examples were typically
repetitive, and that the community's discussion was concerned more with what
\aspectorientation{} could be used for than whether it worked in practice. They
present a comparison between \aspectorientation{} and object-orientation, where
\aspectorientation{}'s claimed properties and principles are examined in detail,
and the impact on software engineering is reasoned about from a skeptical
perspective. They compare the paradigms' specific claims that they both support
improved modularity against classic papers on the subjects.\footnote{In
  particular, they compare against literature by \citet{parnas_1972}.}.
\citeauthor{steimann06paradoxical} presents a philosophical examination of
\aspectorientation{} and assesses the paradigm against its purported merits,
discussing whether we should accept the claims made by the aspect-oriented
programming community. \Aspectorientation{}'s promise of unprecedented
modularisation is presented as unfulfilled, and
\citeauthor{steimann06paradoxical} reflects critically on the state of
\aspectorientation{} research at the time. However, they conclude by proposing
that the aspect-oriented approach could have a legitimate alternative use-case
--- other than as a \emph{general-purpose} technique for modularising
cross-cutting concerns --- where it could be shown to be effective empirically.

Similar sentiments are shared by \citet{przybylek2010wrong}, who examines \aop{}
in the context of language designers' quest to achieve maintainable modularity
in system design. They frame the design goals of \aspectorientation{} as being
to represent issues that ``cannot be represented as first-class entities in the
adopted language''. \citeauthor{przybylek2010wrong} questions whether the
modularity offered by \aspectorientation{} can really be said to make code more
modular. In particular, they distinguish between lexical separation of concerns
and the separation of concerns originally discussed by
\citet{djikstra_scientific_thought}. They assess principles of modularity ---
modular reasoning, interface design, and a decrease in coupling --- and find
that the aspect-oriented paradigm can detrimentally impact the expected benefits
of proper modularisation in a program by ``violating basic software engineering
principles'' such as information hiding and structured programming. They suggest
that \aspectorientation{}'s claimed benefits are a myth repeated often enough to
be believed.
% \citeauthor{przybylek2010wrong} presents a critical review of aspect
% orientation literature, but marries their critique with mention of others'
% solutions to the problems identified: they point to many papers which suggest
% improvements to the standard AOP approach which might reduce it's negative
% impact or make it more practically viable\revnote{Which papers?!}.

\revnote{This review is arguably unfinished --- there are more papers commented below
  this note which are probably worth
  reading / ensuring I'm citing. Tim has noted that the review on early /
  foundational literature of AOP seems incomplete, and I think the commented
  papers to review below could be good candidates to address this.}

%%  ## References to pick up & review:
%%  
%%  On criticisms of AOP
%%  - Dantas & Walker 2006, from What Is Wrong With AOP
%%  - Leavens & Clifton 2007, from What Is Wrong With AOP
%%  - Tourwe, Brichau & Gybels 2003, from What Is Wrong With AOP
%%  - Wampler 2007, from What Is Wrong With AOP
%%      - "Most AO languages in use today are based on structural information about
%%      join points, such as naming conventions and package structure, rather than
%%      the logical patterns of the software
%%  
%%  On AOP and decreasing coupling
%%  - Yourdon & Constantine 1979, from What is Wrong With AOP
%%      - "The fact that we must know something about another module is a priori
%%      evidence of some degree of interconnection even if the form of the
%%      interconnection is not known" 



\subsection{Alternatives to \AspectOrientation{}}

Aspect-oriented programming's goal of modularising cross-cutting concerns is
shared by other paradigms. As discussed in
\cref{review_aop_motivations_and_philosophy}, the work first introducing aspect
orientation by \citet{kiczales1997aspect} makes note of similarities to
reflection, metaprogramming, and program transformation. They also observe that
other disciplines have introduced ``aspectual decomposition'' independently.
Some modelling \& decomposition techniques which are related to
\aspectorientation{} are discussed here.


\subsubsection{System Diagrams}\label{subsec:system_diagrams_as_aspects}

The example of pre-existing aspectual decomposition by way of diagramming given
by \citet{kiczales1997aspect} is in physical engineering. To give a concrete
example from their description, differing types of diagrams when engineering a
system such as thermal and electric diagrams of a heater are described as
``aspectual'' because of the modular nature of the diagrams; though there might
be many diagrams of different kinds, they compose together to give an overview
of the system being designed.

Similar diagramming techniques have independently arisen in other domains since.
The Obashi dataflow modelling methodology\cite{obashimethodology} by
\citeauthor{obashimethodology} models all possible paths of dataflow through
``B\&IT'' (business and IT) diagrams, where business-specific concerns (people,
locations, groups, and business processes such as payroll, stock-check or
budgeting) are modelled alongside IT concerns such as applications supporting
business processes and the software and hardware infrastructure supporting them.
Modelling dataflows in this way allows for a comprehensive understanding of
assets and business processes. However, in order to understand how data flows
between specific assets within a B\&IT, sub-graphs (``DAVs'', or Dataflow
Analysis Views) denote specific pathways through which data flows between source
and sink assets. Alternatively, a B\&IT can be viewed as a composition of all
possible DAVs within an organisation. Dataflows are therefore broken into
different diagramming techniques and specific business concerns can be described
independently of others, even if these concerns interact in their dataflow
pathways (and, therefore, cutting across each other). Obashi therefore allows
for the aspectual decomposition of business processes, through the description
of an organisation by individual dataflow analysis views, which compose into an
overall model of a system in a B\&IT diagram. Obashi models are an instance of
\aspectorientation{} which were designed for simplicity and
comprehension\cite{obashimethodology,seow2011obashi}, but trade this for
domain-specificity.

Other diagrams of systems, called ``bigraphs'', were produced by
\citet{bigraphs_and_their_algebra}. Bigraphs are a category-theoretic process
calculus able to simulate petri nets, \pi{}-calculus, and
\lambda{}-calculus~\cite{space_and_motion_of_communicating_agents} but are also
used in the simulation \& modelling of real-world systems such as
networks~\cite{bigraph_network_model} and ubiquitous
computing~\cite{bigraph_lions_impalas}. A key property of bigraphs are their
``ports'', which denote possible points of composition a given model can have
with others. Bigraphs' compositional property is reminiscent of the use of
aspect-orientation for composition~\cite{gulyas1999use,Cieslak_2007}. While the
existence of a port means bigraphs cannot be composed obliviously, they are a
powerful tool for formal modelling and are used in a similar manner.



\subsubsection{Metaprogramming}\label{metaprogramming_as_an_aop_alternative}

\revnote{Wouldn't the paragraph below fit better in the section on program mutation?}
\reviewresponse{I'm not sure exactly what section this is, but have reworded the
  below to fit better within this section (BCA is a relative of AOP) and added a
cref to where I \emph{think} it might be added. Unsure though. To discuss and
fix the cref. I haven't added notes on BCA anywhere else. Or, was it intended to
refer to the metaprogramming notes in the next para? If so, I've since collected
these into a subsubsec.}

Metaprogramming is identified as a precursor to \aspectorientation{} by the
original paper on the paradigm~\cite{kiczales1997aspect}. Research into the use
of metaprogramming to simplify the composition of software modules was
undertaken at a similar time by \citet{keller1998binary}, who introduce a
technique they name ``Binary Component Adaptation'' (BCA). Their research into
BCA the difficulties involved in the integration of software components,
particularly considering their evolution over time where components are re-used
with differing requirements. By modifying binaries directly, incompatibilities
between a program and one of that program's dependencies can be resolved by way
of mutating either after compilation. Their implementation defines a
representation for the modification of pre-existing Java class binaries, the
output of which can be verified as also being valid Java class binaries.
\Citeauthor{keller1998binary} claim that BCA allows for dynamic modification of
programs with little overhead. They believe BCA is unique in its combination of
features, which include engineering concerns such as typechecking code which is
subject to adaptation and its obliviousness to source implementation, as well as
guarantees that modifications are valid even for later iterations of the program
subject to adaptation.\footnote{BCA shares concepts with \aspectorientation{}, but
  is also a promising technology for the introduction of process variance; see
  \cref{subsec:variations_in_sm} for a discussion.}

Metaobject protocols describe the properties of an object's class (including,
for example, its position within a class hierarchy) in an adaptable
manner~\cite{kiczales1991art}. \citet{espakaspect} note that this technique can
be used to implement \aspectorientation{}, therefore providing at a minimum the
same functionality, though they achieve this through reflective programming
techniques and are designed with metaprogramming as a primary goal as opposed to
modularisation of cross-cutting
concerns~\cite{kiczales1991art,sullivan2001aspect}.

\revnote{
  Subject-Oriented Programming would go here, but I haven't written it up and
  I'm running out of time. I don't think it's the most essential thing to
  include --- it's relevant to AOP's history but not really to the work I'm
  presenting in later chapters. Write up if there's time\ldots{}
}

% \subsubsection{Subject-oriented Programming}


% Subject-oriented programming is a paradigm which \revnote{Incomplete! Add subject-oriented
%   programming here.} \inline{Note: there's an argument that SOP has seen more
%   success in practice than AOP: Golang's interfaces are subjective views on a
%   struct, and I believe Rust's Traits are too --- confirm this! --- by contrast,
%   there's some limited use of AOP in the wild, but related design patterns have
%   arguably supplanted them for modularising cross-cutting concerns (I'm thinking
%   of Decorators). I'm not aware of any literature which backs this up.}

% relevant papers:

% \begin{itemize}
% \item Harold Ossher, Peri Tarr, William Harrison, Stanley Sutton, N Degrees of
%   Separation: Multi-Dimensional Separation of Concerns, Proceedings of 1999
%   International Conference on Software Engineering
% \item Subject-oriented programming: a critique of pure objects
% \item Whatever originally defined SOP
%   \item Founding papers for Hyper/J and/or Hyper/C
%   \end{itemize}



\subsubsection{Engineering Techniques with Related Aims}
\label{engineering_techniques_as_aop_alternative}

Multiple-dispatch, where methods on objects are chosen to be run based on the
properties of the parameters passed at point of invocation, allows for oblivious
decomposition without the need for a weaver~\cite{dozsa2008lisp}, although this
does not support the goals of \aspectorientation{} in totality. For example, a
programmer might want their program to exhibit differing behaviour when methods
are called with differently-typed arguments, which is supported by multiple
dispatch. However, they might instead want their program to exhibit some
additional behaviour whenever a method is invoked, such as logging, but might
not want to implement logging alongside the rest of their method implementation
for clarity or length reasons. Multiple dispatch therefore offers comparable but
different functionality to a software engineer.

Engineering patterns such as decorators provide similar functionality to
aspects~\cite{friesel2017annotations}, in that cross-cutting concerns can be
separated into their own module, but they differ in their approach to
obliviousness: decorators annotate areas of a codebase they are applied to, and
therefore do not offer obliviousness as aspects do. Decorators allow for
additional functionality to be written as a separate module and applied as a
wrapper around a function definition.\revnote{A diagram or Python example might
  help here, as per Tim's suggestion below. Add one if there's time.} A function with a wrapped
definition is replaced by a function returned by the decorator, which takes the
original definition as an argument~\citep{gof_design_patterns}. Additional logic
can therefore be simply applied before or after a function. While this achieves
a similar effect to \aspectorientation{} --- as logic is added before and/or after
the original logic of a function --- its design principles are different as
annotating the function's definition directly marks it as altered, and so the
original definition cannot be oblivious to the change.

\revnote{Would some illustrative examples of these different approaches help?}
\reviewresponse{Yes, I'm sure they'd help! Diagrams to add here if there's time.}



\section{\AspectOrientation{} in Simulation \& Modelling}
\label{sec:ao_and_modelling}

This thesis is concerned with the use of \aspectorientation{} in simulation \&
modelling codebases; it is therefore necessary to review related research in
simulation and modelling. This section reviews literature contributing
aspect-oriented tooling and the philosophy of aspect-oriented experiment design
in the simulation \& modelling community.

\subsection{Suitability of \AspectOrientation{} in Experimental Codebases}
\label{review_gulyas_use_of_aop_in_research_codebases}

\citet{gulyas1999use} observed that, in the study of complex systems through
software models, the codebase produced serves two purposes: the
experimental subject, and the observational apparatus used to conduct the
experiment itself. They use this framing to identify that the observation of a
program's state constitutes a cross-cutting concern. In order to reduce
scattering and tangling in experimental codebases, \citeauthor{gulyas1999use}
theorise that aspect-oriented programming may separate the logic of observation
from the core logic of a simulation or model.

The approach is demonstrated via their Multi-Agent Modelling Language (MAML). MAML
was designed to enable aspect-oriented simulation of agent-based models and was
implemented using the SWARM simulation system~\citep{hiebeler1994swarm}, a
domain-agnostic framework for agent-based simulation. While SWARM takes the form
of a collection of C libraries, MAML is implemented as a domain-specific
modelling language, allowing it to support aspect-oriented programming as a
language feature.

MAML's \aspectorientation{} effectively makes use of Observer patterns to measure
a simulation's state. This enables a researcher to observe an experiment without
the necessary logic being scattered or tangled in their domain model.
\citeauthor{gulyas1999use} find that aspect-oriented programming provides an
intuitive and straightforward method by which simulated experimental systems can
be composed. They note that MAML's simplicity and its philosophy on modelling
are more ``satisfactory'' than Swarm's standard approach. The team report that
MAML's implementation was more complex than initially conceived: the
\lstinline{patch} unix tool was intended for use as their weaver, though the
team eventually developed a transpiler from MAML to Swarm instead (which they
name \lstinline{xmc}.). The deciding factors for the development of a custom
transpiler are not discussed.

In addition to presenting tooling for aspect oriented simulation,
\citet{gulyas1999use} theorise about the potential benefit of applying aspect
orientation to simulation \& modelling. They observe that there may be benefits
beyond improvements to modularity and a reduction of tangling \& scattering. In
particular, their work discusses specific scenarios in which the \emph{type} of
separation of concerns offered by \aspectorientation{} is desirable, and the
engineering approach to achieving the aim reasonable.\revnote{T: If there's time
when redrafting, note which scenarios they distinguish between specifically.}
This distinguishes the work in comparison to most other research on aspect
orientation. Many papers describe the expected benefits by simply drawing from
existing literature and the claims made in \citet{kiczales1997aspect}'s first
paper on the subject, rather than drawing on empirical
evaluation~\cite{steimann06paradoxical,przybylek2010wrong,przybylek2018empirical}
as discussed in \cref{subsec:aop-criticisms}.

This is a rare example of philosophy on aspect oriented programming's
suitability in a particular use-case. \citeauthor{gulyas1999use}'s work is of
particular importance in this review, because the domain they identify as
particularly suitable is simulation \& modelling, which is the subject of this
thesis. That \aspectorientation{} might be well suited to separating observer and
experiment partially motivates research in later chapters, which investigates
the use of aspects to \emph{modify} simulated behaviour rather than simply
observing it. Further discussion follows in \cref{sec:lit_discussion}.



\subsection{\AspectOrientation{} in Discrete Event Simulation}

\subsubsection{Aspect-Orientated Implementations of Simulation Frameworks}

\citet{chibani2013toward} observed that simulation frameworks and the codebases
built upon them can exhibit cross-cutting concerns such as event handling,
resource sharing, and the restoring the state of a simulation run. They
investigated the introduction of \aspectorientation{} to Discrete Event Simulation
(DES) frameworks to address tangling and scattering that may arise from the
cross-cutting concerns they identified. They contribute a discussion of
aspect-oriented programming's potential application to DES codebases, and detail
the avenues available for research in the field. \citet{chibani2013toward}
identify Japrosim~\cite{abdelhabib2008japrosim,belattar2014yet} --- a DES
framework previously developed by the research team, which was designed to model
domains through process interaction in Java --- as an example of an
existing framework which exhibits the cross-cutting concerns they describe.

Later, \citet{chibani2019using} identified opportunities for the use of aspect
orientation in simulation tooling, aiming to increase ``modularity,
understandability, maintainability, reusability, and testability'' by applying
the paradigm~\cite{chibani2019using}. They present a case study of an
application of \aspectorientation{} to simulation tooling by identifying
cross-cutting concerns in Japrosim, a discrete event simulation framework, and
propose an aspect-oriented redesign of the tool using AspectJ.
\citeauthor{chibani2019using} describe Japrosim's existing object-oriented
design, followed by aspect oriented variations of some design elements,
including concurrent process management and in Japrosim's graphical animation
features. A similar survey of areas in which Japrosim's source might benefit
from the application of \aspectorientation{} is presented by
\citeauthor{chibani2014practical} in an earlier
work~\cite{chibani2014practical}. In both cases, the main contribution noted is
the design itself. Evaluating the main improvements between the presented
aspect-oriented design and the existing object-oriented one is left to future
work.

\citet{chibani2019using}'s later work presented an implementation of their
proposal and provided a quantitative evaluation of that implementation. The
quantitative evaluation provides measurements based on \citet{martin1994oo}'s
object-oriented design metrics and demonstrates a greater independence of
packages in their aspect oriented version of Japrosim than in the original.
However, the intended aim of aspect orientation is not to decouple existing
packages, but to isolate those packages' cross-cutting concerns into new ones.
It is therefore unclear that their quantitative evaluation achieves its
improvements as a result of aspect orientation. No further discussion of their
results is provided, and it is possible that the improvement is due to the
rewriting necessary in their maintenance of the Japrosim source, rather than due
to their use of aspect orientation specifically.

Similarly to
\citeauthor{chibani2019using}~\cite{chibani2019using,chibani2013toward,chibani2014practical},
\citeauthor{DEVSaspectorientation2008aksu} observe that there are advantages in
adopting \aspectorientation{} when developing a simulation
framework~\cite{DEVSaspectorientation2008aksu}. Examining the DES framework
Simkit~\cite{introducing_simkit_des}, they motivate two different applications
of \aspectorientation{}: a refactoring of the framework itself to better manage
cross-cutting concerns within its codebase; and aspect-oriented tooling for use
by modellers who represent cross-cutting concerns within their models.
Opportunities for improvements in production and development are discussed, and
some implementation notes are detailed, although no concrete implementation or
evaluation is provided; the work instead proposes design alterations, and the
authors ``leave it as a future work \emph{[sic]} to explore the usability and
efficiency'' of \aspectorientation{} used idiomatically alongside Java's
existing reflection offerings. The existence of multiple attempts to refactor
differing simulation packages with \aspectorientation{} indicates potential for
modellers in the use of aspect-oriented patterns, but the real-world utility of
the techniques are omitted. As is common in \aspectorientation{} literature,
\citeauthor{chibani2019using} and \citeauthor{DEVSaspectorientation2008aksu}
both defer to the general claims that \aspectorientation{} improves modularity
of cross-cutting concerns and can eliminate code smells such as tangling and
scattering. \citeauthor{chibani2014practical} do present some
quantitative evaluation, but this is flawed as previously described.



\subsubsection{Aspect-Orientated Implementations of Simulations \& Models}

Research projects applying \aspectorientation{} to the implementation of
simulation frameworks~\cite{chibani} fails to provide a case study which evaluates
their technique with real-world examples. However, case studies do exist which
evaluate \aspectorientation{}'s suitability in maintaining model source code.

\citeauthor{ionescu2009aspect} identify an increased demand for computational power in
simulation execution on supercomputers~\cite{ionescu2009aspect}. Existing
known-good models might be unsuitable for the extreme requirements of code
efficiency modellers contend with, but running the code in different
environments requires modifications to make the code suitable in the
environment. These modifications are subject to regulations and introduce risk
of a reduction in quality during maintenance. The authors propose an
aspect-oriented solution to the problem, where aspects modify the simulation
codebase with minimal overhead. An implementation of a real-world model for
disaster prevention is presented, and assessed both by comparison against an
equivalent non-aspect-oriented codebase and by assessment of the aspect-oriented
variant's scalability and reliability in both cluster and multi-cluster
environments. They find that a comparative analysis of generated code and of
their simulations in various configurations both indicate that their
simulation's aspect-oriented implementation is suitable for use in disaster
prevention, implying that \aspectorientation{} could be suitable in scenarios with
comparable requirements.

This work provides evidence that aspect-oriented programming is suitable for
supporting modifications to simulations and models. This contrasts the lack of
evidence for the paradigm's success in improvements to modularity and
maintainability~\cite{przybylek2010wrong,Constantinides04aopconsidered}, and
supports \citet{steimann06paradoxical}'s suggestion that \aspectorientation{}
might be suitable for other uses --- as well as the suggestion made by
\citet{gulyas1999use} that the paradigm is well suited to the requirements of
simulation \& modelling codebases.

As it is a rare example of aspect-oriented case studies, the methodology
employed by \citeauthor{ionescu2009aspect} is important to highlight. Their
evaluation measures code quality improvements, which are claimed using similar
measurements in early \aspectorientation{} research~\cite{kiczales1997aspect}.
Their code analysis makes use of significant lines of code as a core metric,
which doesn't reliably reflect code quality. As \citet{rosenberg1997some}
explains:

\begin{displayquote}
    \textelp{}the best use of SLOC is not as a predictor of quality itself (for
such a prediction would simply reduce to a claim about size, not quality), but
rather as a covariate adjusting for size in using another metric.
\end{displayquote}

Although \citet{ionescu2009aspect} evaluate code quality, the methodology
employed to measure improvements is unreliable.

Improvements in code quality are those which have come under scrutiny by the
critical papers reviewed in \cref{subsec:aop-criticisms}. The results presented
by \citeauthor{ionescu2009aspect} do not satisfy critics' requests for empirical
evidence of improved code quality. This does not impact their aspect-oriented
models' viability: their study demonstrates that their models can be augmented
to support new supercomputing environments without lack of performance. The
models described in this work satisfy that aim: their models are also evaluated
in their performance. Model performance is a priority in supercomputing
contexts, where execution time is financially expensive and energy-intensive.
Quantitative evaluation of their models' execution time shows less than 5\%
slowdown compared to a non-aspect-oriented implementation.
\citeauthor{ionescu2009aspect} deem this a reasonable trade-off for the
engineering improvements they observe.

\citeauthor{ionescu2009aspect}'s application of \aspectorientation{} to
supercomputing \& disaster prevention simulations meet their performance
requirements and demonstrate a modelling technique which adapts existing models
for use in new environments without directly modifying pre-existing source code.
This result is notable with regards the contributions presented in this thesis, which
similarly aim to augment a pre-existing model without directly modifying its
source code --- though this thesis' targets model reuse and design
simplification rather than for avoiding the regulatory overhead and financial
cost of maintaining models which run on supercomputers.


\subsection{\AspectOriented{} L-Systems}
\Aspectorientation{} is also applied in other simulation paradigms.
\citet{Cieslak_2011} investigated the use of \aspectorientation{} in L-system
based simulations. An L-system\cite{lindenmayer1968lsystem} is defined by a set
of symbols, an initial string composed of these symbols, and a set of rules for
rewriting substrings. While being a powerful tool for representing fractal
structures, they were originally conceived of for modelling in botanical
research.

\citeauthor{Cieslak_2011} note that some details of plant modelling are actually
cross-cutting concerns against many plants or families of plants, such as carbon
dynamics, apical dominance and biomechanics. To represent these, they introduce
a new language to describe plant models which makes use of \aspectorientation{} to
represent these cross-cutting concerns. They successfully test the approach by
representing carbon dynamics, apical dominance and biomechanics as cross-cutting
concerns that are integrated into a previously published model of kiwifruit
shoot development. \citeauthor{Cieslak_2011} hope that these cross-cutting
concerns might work in other models too, but this is untested. The use of an
aspect developed for use in one model and reused in another seems untested in
the community's literature and is a
noted omission in the conclusion of this particular work.



%%%%%
%%%%% MARK MARK MARK here when redrafting 09-10-2023
%%%%%




\subsection{\AspectOrientation{} \& Business Process Modelling}\label{subsec:ao_and_bpm_review}


% As business processes models represent a kind of \sociotechnical system, and
% this thesis offers tooling for and results in the modelling of \sociotechnical
% systems, . Additionally,
% related work undertaken before this PhD develops on software engineering
% processes that lend themselves well to the same modelling paradigms as business
% processes (see \cref{chap:prior_work}), and there also exists interest in
% modelling behavioural variance within the business process modelling community
% (see \cref{sec:dynamism_in_sm}). This overlap necessitates a review of related
% literature within the business process modelling field. \inline{Add Charfi \&
% Cappelli's work before Jalali's in this subsec}

Several projects within the business process modelling research community make
use of \aspectorientation{} to design modelling languages which produce less
monolithic business process models~\cite{Cappelli_AOBPM,da2020implementation}
and simplify the composition of models~\cite{charfi2007ao4bpel}. Business
process modelling research is relevant to this thesis' contributions, as
business processes are inherently \sociotechnical and later chapters present
tooling for and results in the modelling and simulation of \sociotechnical
systems using aspect-oriented techniques. In addition, some research conducted
prior to this thesis developed software engineering processes that are
conceptually similar to business process modelling (see \cref{chap:prior_work}).
There also exists interest in modelling behavioural variance within the business
process modelling community (see \cref{sec:dynamism_in_sm}), which is relevant
to this thesis' concern with the aspect-oriented representation of changes to
processes and modelled behaviours.

\citeauthor{charfi2007ao4bpel} observed opportunities integrating BPEL, an
executable business-process modelling language, with aspect-oriented
concepts~\cite{charfi2007ao4bpel}. This is because when BPEL systems are
composed together the logic being composed can lack the flexibility required by
BPEL's use cases. The specific use-case examined is web service definition,
where changes affecting composition of multiple component parts can affect many
areas of a final result, making modification error-prone. The authors
specifically seek to support dynamic workflow definitions --- ``adaptive
workflows'' --- which BPEL's existing extension mechanisms do not sufficiently
support but which \aspectorientation{} literature does discuss (see the \aop{}
implementations discussed in \cref{aspect_oriented_programming_litreview}). Therefore, they look to
construct an aspect-oriented BPEL extension. Using the case study of modelling a
travel agency's web services, they create an aspect-oriented extension by first
defining how such an extension would be represented graphically in BPEL's
workflow diagrams. Further detail is added to arrive at a technical definition
with XML representations, weaving mechanics, and eventually the construction of
a BPEL dialect, AO4BPEL. The authors find that their pointcut system (which
describes join points on both processes and BPEL messages), support for adaptive
workflows, and aspect-oriented approach to workflow process modelling make
AO4BPEL unique at the time of publication, though related AOP implementations
exist in each individual area of their contributions. The work is weakened by
brittle semantics around pointcuts, join points, and the temporal nature of
workflow modelling. For example, they note that defining contingent behaviour
--- only applying an aspect conditionally, based on a trace through a simulation
of a modelled system --- would allow weaving advice only when model
state deems this appropriate. They also call for more generally theoretical AOP
research, which mirrors the requests some critics of aspect orientation research
make (as noted in \cref{subsec:aop-criticisms}). The contingent application of
model adaptation is a motivating case for some work presented in this thesis;
see \cref{chap:prior_work} for a discussion.

\citet{Charfi2006AspectOrientedWL} described AO4BPEL in detail and presented a
generalisation of the notation developed for it which applies to any graphical
workflow modelling language. Accompanying this are some examples of its use
building a framework for enforcing certain requirements of BPEL models, and use
of that framework to develop aspect-oriented frameworks for enforcing security
and reliability within AO4BPEL models.

In later work, \citeauthor{charfi2010AO4BPMN} defined a similar aspect-oriented
dialect of BPMN they name AO4BPMN~\cite{charfi2010AO4BPMN}, after asserting that
the concerns addressed by
AO4BPEL~\cite{Charfi2006AspectOrientedWL,charfi2007ao4bpel} in the field of
executable process languages also apply to business-process modelling languages,
and can be solved similarly. The generalised notation of aspectual
workflow models presented in \citeauthor{Charfi2006AspectOrientedWL}'s
thesis~\cite{Charfi2006AspectOrientedWL} are applied to BPMN to produce an
aspect-oriented language specifically for process modelling, as opposed to
executable business process modelling. \revnote{can you say something explicit
about the difference and benefits of this?}\reviewresponse{I don't think I can
off the top of my head right now, but I'm leaving this here to consider more if
there's time later on.}

\Citeauthor{Cappelli_AOBPM} also note that cross-cutting concerns exist in
business process models, and are specifically motivated by monolithic design
approaches common in business process modelling languages. Like
\citeauthor{kiczales1997aspect}, they claim that a lack of modularity in
business process models leads to cross-cutting concerns scattered throughout a
model~\cite{Cappelli_AOBPM}. To alleviate the issue, they propose a
meta-language, AOPML, which incorporates \aspectorientation{} in a metamodel of
business process modelling languages, and instantiate it within their own
dialect of BPMN. Using a model of a steering committee as a case study, and
separating cross-cutting concerns such as logging, the paper proposes reducing
complexity and repetition graphically, thereby in a manner more in keeping with
the language design philosophies of popular business process modelling
languages, the design and use of which are typically
graphical~\cite{OMG-BPMN-SPEC,opm_original,OMG-UML-SPEC}. They note that this is
in contrast to other applications of \aspectorientation{} in business process
modelling --- specifically AO4BPMN --- where aspect definitions written in
XML concern not only the advice to be applied but also their relevant join
points, as in general programming \aspectorientation{} implementations such as
AspectJ. In this way, AOPML exhibits the spirit of business process\revnote{Can
you say something more precise than spirit here, e.g. objectives, intentions at
least?}\reviewresponse{Yes, but I kind of like this phrasing! I want to get at
the community / idioms as well as the technical intent. Leaving this here to
reconsider if there's time later on.}
modeling more stringently than does \citeauthor{Charfi2006AspectOrientedWL}'s
notation for aspect-oriented workflow modelling.

The difference between \citeauthor{charfi2010AO4BPMN}'s approach in designing
AO4BPMN~\cite{charfi2010AO4BPMN} and \citeauthor{Cappelli_AOBPM}'s approach in
designing AOPML~\cite{Cappelli_AOBPM} highlights design decisions taken when
introducing \aspectorientation{} in a new domain. There is an opportunity for a
domain-specific \aspectorientation{} framework to align its design with the
traditions and idioms already present in models within that domain, but doing so
may break the traditions and idioms which already exist in aspect-oriented
approaches in other domains.  Comparing the approaches of
\citeauthor{charfi2010AO4BPMN} and \citeauthor{Cappelli_AOBPM} demonstrates that
there may be no clear ``best'' design approach when blending pre-existing
modelling paradigms, such as the graphical modelling languages used in
business-process modelling and the abstract concepts of \aspectorientation{}.
The discussion around whether it is more desirable to adapt existing design
elements of aspect-oriented frameworks to a given domain or adapt that domain's
existing modelling traditions, idioms, conventions or syntax to incorporate
aspect orientation as it is used elsewhere is outside the scope of this thesis.

New concepts within the design of \aspectorientation{} frameworks are addressed in
the business process modelling community. \Citeauthor{jalali2012aspect} note
that aspect oriented modelling frameworks often do not explicitly model the
precedence of aspect application~\cite{jalali2012aspect}. They address this
limitation by defining a mechanism to be used in capturing multiple concerns as
aspects, where the invocation of advice must follow a certain precedence. The
aim of the work is not to propose tooling around the precedence of aspect
application so much as to contribute to aspect oriented design theory, providing
a notation for precedence which is broadly applicable. The precedence model is,
put simply, that a mapping exists for each application of advice to join point
such that the mapping defines an ordering on advice for that join point. The
definition defines ``AOBPMN'', a formalised dialect of BPMN supporting aspect
orientation with precedence. A case study is provided where AOBPMN is
instantiated within a coloured Petri net. Their study expands on existing work
by research teams led by Capelli~\cite{Cappelli_AOBPM,da2020implementation} and
by Charfi~\cite{charfi2007ao4bpel}, in that it develops a mature formalism for
and model of \aspectorientation{} as applied to business process modelling.
However, \citeauthor{jalali2012aspect} note that their case study is limited in
scale. No tooling or evaluation of the practical benefit of their approach is
provided.\revnote{add something early on giving an example of how it might be a problem?}


\section{Process Variance in Simulation \& Data Generation}
\label{sec:dynamism_in_sm}
\label{subsec:variations_in_sm}


Research presented in later chapters concerns the application of \aop{} to the
codebase of a model to represent changes within it. In particular, we are
interested in representing changes to simulated behaviour. This section reviews
techniques where variations on already-modelled behaviour are represented in
some way.

\subsection{Discussion of Variation \& Motivations for Variations in Process Models}
\label{subsec:variation_sm_motivations}

Difficulties arise when obtaining real-world datasets for many reasons. For
example, large empirical datasets are typically produced by organisations which
would prefer some level of secrecy around their operations, making publishing
those operations for the investigation of research teams unlikely. Researchers
collecting these datasets describe a \textquote{lengthy
process}~\cite{bpi_ten_years_of_datasets} and explain that traces of real-world
processes are hard to obtain because \textquote{higher management [can be]
worried about the risks} of publishing such
datasets~\cite{bpi_ten_years_of_datasets}. Another factor contributing to the
difficulty of collecting empirical datasets is that it is often infeasible to do
so, either because there is a need to study the process before implementing it
(making synthetic datasets the only option available to researchers), because
the process is not yet fully understood (making simulation of many variants of
that process useful in aiding understanding), or because the dataset itself is
of interest to researchers rather than the real-world system that produced it
(such as in the case of evaluating process mining
techniques~\cite{van2004process,agrawal1998mining}).\revnote{Citations for all
difficulties in collecting empirical datasets would be nice\ldots{}} The
generation of datasets with specific properties is its own field of research;
the literature of interest specifically concerns techniques for the modification
of a model or simulation to generate those datasets.

The reviewed techniques allow a programmer to include some (separately defined)
modification of a process, or represent the modifications of a varied process
within simulated output~\cite{stocker2013secsy,stocker2014secsy}. The benefit of
this is that possible changes to a process can be described once and applied to
that process where appropriate, resulting in datasets which are affected by the
modification. Changes to processes might describe attempts to circumvent
security protocols, laziness or confusion in a human actor within the model, or
random ``noise'' to produce synthetic log traces containing aberrations which
mimic those found in noisy empirical datasets. In all cases, behavioural
variations can be described as some alteration to a process and applied to
either a model or the product of that model (datasets or log traces) to
represent the same alteration introduced at an arbitrary point in the
simulation.

This decouples the expected behaviour encoded in the original model from
simulated behaviour, which is obtained by composing the model and behavioural
variation using a given technique's method for doing so. This approach to
modelling behavioural variation allows the same altered behaviour, which would
otherwise be described in many disparate points in a model, to instead be
written once and introduced wherever required. The observation that the same
variation might appear in many areas of a model, and that the variation can be
separated from the model and introduced where necessary, frames the modelling of
these variations in the same way as \aspectorientation{} frames cross-cutting
concerns. The work presented in this thesis explicitly applies changes to
processes and simulated behaviour as aspects in the same manner. Therefore,
although this aspectual connection is not made explicit in much of the
literature to date, it is important to review literature on simulation and
modelling which modularises these variations; this section reviews that
literature. The work reviewed is highly relevant to the contributions in this
thesis, in particular because the core motivations of this field are shared by
the research presented in later chapters. For this reason, the section leads
by discussing those motivations and their relationship to
\aspectorientation{} in detail.


\subsection{SecSY}

Research undertaken by \citet{stocker2013secsy} aims to synthesise process logs
which are representative of attackers' efforts to compromise the security of a
modelled system. Their research project, named ``SecSY'', is an attempt to
address issues arising from the difficulty of retrieving representative log
traces for security-critical systems in which attacker activity is present. Logs
are developed by process simulation through ``well-structured'' models, a
mathematical property on which transformations were previously defined by
\citeauthor{vanhatalo2009refined}~\cite{vanhatalo2009refined}. The authors
develop a tool for the simulation of a process using well-structured process
models, and apply transformations to both the model before execution and the log
it produces through the trace of a simulation.

They conclude that their tool is performant and verify it can produce logs
representing security violations by way of analysis through PROM, a popular
framework for process mining, and pre-defined security constraints on their
models. They also note some limitations: log traces cannot be interleaved (due
to a lack of parallel simulation of processes), may be incomplete (missing
violations), and that mutated models and traces are not guaranteed to be sound
by construction. However, they see their proposal as a necessary step in
realistic data generation for business processes. A further weakness of the work
is that model and trace modification techniques are limited: processes can be
added or removed, but complex graph transformations are only permissible when
representable through the composition of the mutation primitives they provide,
on which there are only three for processes: swapping \lstinline{AND} and
\lstinline{XOR} definitions of process gateways, and swapping process order.
Mutations cannot be applied contingent on the state of a simulation run, for
example, representing a decision taken by an attacker based on what had already
happened.

In later work, \citeauthor{stocker2014secsy} detail the
technical aspects of SecSY, their tool for implementing the generation of
synthetic logs which use their technique~\cite{stocker2013secsy} to represent
security violations of security-critical business processes. A Java implementation
of SecSY is described, which simulates well-structured models and applies
mathematically-defined transformations on the model being simulated (before
simulation occurs) and the logs obtained through simulation traces. An
improvement on earlier work is that custom transformers can be written. However,
a limitation of the original work remains, which is that users cannot easily
dictate the degree to which variations are applied.

\subsection{Cross-Organisational Process Mining}

\citeauthor{pourmasoumi2015business} also address the need for access to
variations on business processes, though for the development of a research
field, ``cross-organisational process mining''~\cite{pourmasoumi2015business}.
Process mining can require many process logs, as does the benchmarking and
evaluation of process mining techniques. Traces from business processes which
are similar but not identical can produce log traces which reflect that
similarity, but also reflect the variations in different instances of those
processes. These log traces exhibiting variation can be used in the training and
analysis of process mining tooling and techniques, which must contend with
natural variation present in the execution of real-world traces. To support the
field, log trace generation from a variety of process models is therefore
required. Such logs are not in adequate supply, as explained in
\cref{sec:dynamism_in_sm}. The authors' approach to the problem is to present an
algorithm for the mutation of business processes, such that simulation against
variations of the business process can produce process logs reflecting those
variations. Their algorithm makes use of structure tree representations of
processes, which models processes as trees and permits conversion to BPMN models
and Petri nets~\cite{buijs2014flexible}. \citeauthor{pourmasoumi2015business}
make use of this constraint to demonstrate that their models are
block-structured, a mathematical constraint on model structure which 95\% of
models have been shown to comply with~\cite{chenthesis}. Their contribution is a
set of transformations on structure trees and block-structured models, and an
algorithm applying these transformations to process models, and a tool which
implements it built on PLG, a process log generation tool. They conclude that
tools such theirs can be used to generate log traces representing process
variation, in such a manner as to satisfy the requirements of the process mining
research community.

\citeauthor{pourmasoumi2015business} describe a list of model transformations
they explain is \textquote{not intended to be
comprehensive}~\cite{pourmasoumi2015business}, which makes its full potential
unclear. Additionally, processes their tool applies to must be block-structured.
The importance of this requirement is that it limits their model transformation
technique to business process models. It is not demonstrated that models of
processes in other domains satisfy the condition, such as the flow of
data~\cite{obashimethodology} or behaviour of human or technical actors in
\sociotechnical systems~\cite{wallis2018caise}. Finally, the tool is limited by
its lack of capacity to represent variations which are applied contingent on
system state. A use case for modelling behavioural variance is to model changes
which are impossible to anticipate from the vantage of a modeller, such as a
\sociotechnical system's human actors' mistakes, security exceptions and
violations, or corrective actions to mitigate undesirable system state.
\citeauthor{pourmasoumi2015business}'s tool produces variations on a process
model, but modelling behaviour which is expected to vary within iterations of
the \emph{same} model is outwith the scope of their project.



\subsection{Executable BPMN Models}

\citeauthor{ExecutableBPMNMitsyuk} are motivated by the field of process
mining's requirement for datasets of process logs made from well-understood
process models, defined in a high-level manner~\cite{ExecutableBPMNMitsyuk}.
They demonstrate a technique for generating event logs from BPMN models by
introducing algorithms for the direct simulation of BPMN models and the
collection of traces from those simulations. While their approach does not
support the simulation of all BPMN concepts, notably message passing, they
provide a tool which produces log traces for a BPMN model through PROM, a
standard tool within the process mining community~\cite{van2005prom}. This
results in their technique providing high-level model simulation through
already-standard tooling, meaning adopters of the technique need not rely on
dedicated tooling which may not be compatible with other researchers' process
mining techniques.

The algorithms presented by \citeauthor{ExecutableBPMNMitsyuk} simulate
processes described by BPMN models, but don't include any provisions for
representing variance. However, the technique could be combined with
\aspectorientation{} techniques for BPMN as discussed in
\cref{ao_and_bpm_review}~\cite{charfi2010AO4BPMN,Cappelli_AOBPM} to represent
alternate behaviour applied contingently. Demonstrating the viability of this
approach is an avenue of research beyond the scope of this thesis. However, the
motivation of the work mirrors that of other research projects reviewed in this
section: a need for synthetic datasets of traces through a process, for use in
scenarios where empirical datasets are difficult to obtain.


\subsection{BPMN Extensions for Process Variation}

\citeauthor{Machado_2011} note that there are operational costs to the
inefficient modelling of business processes. Specifically, they note that costs
can be replicated across automated processes, and failure to identify such
scenarios give rise to these operational costs. This work's core motivation is
that the representation of variation in process models would allow for the
capture of a replicated process once, with instances of similar processes
described as deviations from that captured blueprint. On this basis,
\citeauthor{Machado_2011} extend BPMN to support the notion of individual
processes as transformations of an underlying process, i.e. that a given process
can sometimes be expressed as a deviation from some pattern, and is therefore
define-able as the composition of that pattern and a variation upon it. Their
approach is illustrated through two small, broadly similar business processes
initially modelled in BPMN and then represented in Haskell, allowing the authors
to demonstrate their representation of variability as process deviation with
realistic examples.

While the work presented makes no empirical evaluation of their technique,
\citeauthor{Machado_2011} note that their industrial partner responded
positively to the research presented in this publication and that further
technical improvements are to be made (support for around advice, and for
quantification). They also express an intent to conduct a real-world evaluation
in the HR domain. While we are unaware of any real-world evaluation undertaken
by this research time to date, some formal proofs that the transformations their
tool supports are always well-formed have been
contributed~\cite{machado2012formal}.

\revnote{Review \citet{Mitsyuk_2016} and other work with Shugurov if there's time\ldots{}}


\section{Discussion}\label{sec:lit_discussion}


This thesis draws on the idea that, in response to
  \citeauthor{steimann06paradoxical} asking what \aspectorientation{} is good for,
\citeauthor{gulyas1999use} would seem to answer, ``simulation \&
  modelling''. This discussion concludes the literature review by summarising
  the observations made and motivating the work presented in future chapters.

\revnote{Worth observing that \citet{ionescu2009aspect} actually
  produce a case studdy of using AOP and their code quality evaluation uses SLOC
(unreliable), but they do find that they can augment models successfully (in
their context, successfully is without significant loss of performance). Backs
up claim in \citet{gulyas1999use} that AOP is useful for simulation \& modelling
and claim in \citet{steimann06paradoxical} that AOP ``doesn't know what it aims
to be good for'' (paraphrased, it's quoted in the ciriticisms section and used
elsewhere too). There's loads of AOP lit but we need to explore using it in new
ways, and actually buildinng models \& running simulations using it can be
motivated by existing literature. I don't believe I make reference to
\citet{ionescu2009aspect} in this section at present.}

% Points to make here:
% \begin{itemize}
% \item Aspect orientation allows for a clean separation of concerns. There has
% been a lot of promising research in the field. 
\Aop{} is designed to permit highly modular software
engineering in scenarios where cross-cutting concerns are identified, by
isolating them as separate modules~\cite{kiczales1997aspect}. The aim in
employing aspect oriented programming is to reduce tangling, where a
cross-cutting concern is intertwined within a program's main concern, and
scattering, where the same cross-cutting concern is re-written at many points in
a program's source. Aspects which modularise that concern can be written once,
separately from a program's main concern, and re-introduced to each point in a
program to which the aspect relates by way of weaving. An \aspectorientation{}
framework must therefore be able to quantify the points at which the aspect
applies~\cite{filman2000aspect}. Aspects specify both advice (the implementation
of its associated cross-cutting concern) and the join point defining where in a
target program that advice should be woven. \Aspectorientation{} therefore implies
that the source aspects are applied to are oblivious to their
application~\cite{filman2000aspect}.

% \item However, we lack empirical results which show the paradigm delivers what
% its original proponents claim.
The design of the paradigm is intended to increase modularity in the software
applying it~\cite{kiczales1997aspect,filman2000aspect,przybylek2018empirical},
and its proponents often claim this modularity as a benefit of the
aspect-oriented approaches of their
research~\cite{gilani2004family,charfi2007ao4bpel,Cappelli_AOBPM,jalali2012aspect,chibani2013toward}.
However, its critics question the reasoning around these benefits, and note that
there is little empirical study into whether aspect oriented programs truly
benefit as a result of this
modularity~\cite{Constantinides04aopconsidered,steimann06paradoxical,przybylek2010wrong,przybylek2018empirical}.

% Lead into variance and why there's potential for aop to work, although we'd
% need some kind of empirical proof that it did, partly becaue of aop's track
% record and partly because we'd be automatically rewriting models so they might
% not reflect what their authors intended anymore if we're not careful.
One appropriate application area may be in the representation of behavioural
variation in simulation \& modelling. The application of aspects to models is
already well-studied~\cite{DEVSaspectorientation2008aksu,chibani2013toward},
particularly within aspect-oriented business process
modelling~\cite{charfi2007ao4bpel,Cappelli_AOBPM,jalali2012aspect}, where
modelling behavioural variation has also seen some prior
research~\cite{Machado_2011,stocker2013secsy,pourmasoumi2015business,ExecutableBPMNMitsyuk}. 
Outside of business process modelling, \aspectorientation{} would reasonably be
expected to support the development and observation of models
themselves~\cite{gulyas1999use}. 

Research opportunities at the intersection of \aspectorientation{} and
behavioural variation in modelling occur because behavioural variation is an
example of a cross-cutting concern. Changes to expected behaviour such as
laziness, boredom, confusion or misunderstanding can impact many parts of a
process in a \sociotechnical system, but modelling the variation caused by any
one of these requires similar alterations to behaviour in many disparate parts
of the model they occur within. Behavioural variations are therefore both
scattered and tangled, and constitute a cross-cutting concern which might be
well suited to modularisation in aspects. They may also cut across models,
requiring a re-implementation for each codebase if not modularised.

% NEXT: note the weaknesses in existing work and what needs to be done to show
% that AOP and behavioural variation in S&M do work well together.
Existing \aspectorientation{} techniques and behavioural variation modelling
techniques are ill-equipped to take advantage of this alignment. That behaviour
changes when it varies is tautological; however, changes supported by existing
\aspectorientation{} techniques weave advice before, after, or around their join
points, and therefore do not alter the definition itself. In some systems, some
variations may be representable as additions inserted before and/or after some
other behaviours, but such techniques are unsuitable for representing
modifications of the target behaviour itself, or behaviours which should be
omitted instead of added. Additionally, join points available to an aspectual
programmer may not be granular enough to permit representing the changes they
require in such as system, and \aspectorientation{}'s principle of obliviousness
opposes the modification of target code to make new join points available.
Techniques which would directly rewrite target source are typically extremely
low-level, and therefore ill-suited to most modelling
applications~\cite{keller1998binary}. Other techniques which permit defining
changes to processes at a high level may allow a modeller to \emph{describe}
intended changes (such as the high-level annotations supported in
AOPML~\cite{Cappelli_AOBPM}), but such techniques are intended for human
interpretation, not machine execution for simulation purposes. These techniques
permit describing behavioural variation within another process, but only by
virtue of the flexibility of natural-language annotation, and are therefore
unsuitable for simulation and modelling purposes.

Such techniques also lack executable notion of ``state''. Real-world behavioural
variance can often be contingent on the environment an actor exists within.
While variations such as security breaches might be predictable (by identifying
weaknesses in existing processes, for example), variance in \sociotechnical
systems often occurs in the behaviour of human actors. This might be in response
to a degraded mode~\cite{johnson2007degradedmodes}, where behaviour naturally
drifts to a functional --- but undesirable --- state, or due to an individual
making a mistake, forgetting procedure, or being in a state which alters their
behaviour, such as tiredness or drunkenness~\cite{aranTheatreThesis}. A
framework for modelling behavioural variation using aspects should therefore
apply that aspect to a simulated system contingent on the state of that system
at a given point in time. High-level modelling technologies such as BPMN and OPM
are executable~\cite{ExecutableBPMNMitsyuk,opm_original}, but it is not trivially
evident that executable versions of these technologies are compatible with
aspect-oriented modifications of their modelling
language~\cite{charfi2010AO4BPMN,Cappelli_AOBPM}. As noted, low-level program
transformation technologies are also unsuitable. Techniques for applying
variations to models exist, but are unsuitable for simulation (and therefore
cannot represent application based on temporal state)~\cite{stocker2013secsy},
produce individual models representing each possible instance of a
variation~\cite{pourmasoumi2015business}, do not support the dynamic weaving of
aspects for contingent behaviour~\cite{Machado_2011}, or attempt to represent
the changes one would expect in the output of a simulation by executing an
unmodified simulation and amending its output
directly~\cite{shugurov2014generation}. None of these techniques are suitable
for representing a behavioural change which is applied contingent on state.
Incidentally, these techniques for modelling behavioural variation also lack
support for the alteration of a process definition, or changes ``inside'' a
process definition, as discussed earlier, which also makes them unsuitable for
simulated behavioural variation.

\subsection{Research Questions}\label{subsec:rqs}

A tool which dynamically weaves definitions of behavioural variations for
contingent application and which is capable of expressing changes within a join
point rather than before or after it prompts a demonstration of the benefits
that tool achieves. Contingent application of behavioural variation, and the
ability to define changes to processes specifically, would fulfil the
opportunity in marrying \aspectorientation{} and modelling behavioural
variation, but the benefit offered by such a tool is unclear. The introduction
of oblivious modification to a model may break its representation of its
real-world analogue, making such a model difficult to reason about. Added to
this is the complexity of the tool's capacity to rewrite any join point's
definition. Though \aop{} literature often lacks case studies demonstrating the
benefits of the approach, it is particularly important to investigate whether
such a tool could produce realistic models, and whether the expected benefits of
aspect orientation as applied to the model hold in practice. Further, the
research should demonstrate that the resulting aspectual modules are re-usable
when modelling other systems: if so, aspects would represent modifications to
models which cut across different model codebases.

This is investigated in the research presented in the following chapters. This
literature review motivates the following research questions, which those
chapters address:

\begin{researchquestion}
\begin{description}
  \item[RQ1] \rqone{}
  \item[RQ2] \rqtwo{}
  \item[RQ3] \rqthree{}
  \item[RQ4] \rqfour{}
\end{description}
\end{researchquestion}

To address these, new tooling is produced in \cref{chap:pdsf_rewrite} which
offers new types of aspects which weave their advice within their targets,
rather than before or after them; a system is presented in \cref{chap:rpglite}
around which a model can be constructed to investigate \aspectoriented{}
simulation \& modelling. That model --- and the aspects applied to it --- is
described in \cref{chap:experiment_setup}; this chapter addresses the first
research question while laying the foundation for experiments which address the
latter three. Those experiments are discussed in
\cref{chap:experimental_results}. Before discussing these responses to the
literature, related work which predates the contributions of later chapters is
explained. \Cref{chap:prior_work} Distinguishes prior research from that
contributed in this thesis.

% \item Criticism of aspect orientation notes that there is no clear application
% area for the paradigm (and some areas where it _is_ applied seem unsuitable)
% \item One promising area of application is in modelling, as demonstrated by its
% use in business-process modelling literature.
% \item in addition, in business-process modelling literature we find research
% motivation for --- and research towards --- the modelling of \emph{variations}
% on behaviour. These behavioural variations appear well-suited to modelling as
% aspects, as they typically separate the variation from the domain model in which
% variation is observed, and a given variation might occur in many areas of a
% model. 
% \item However, projects modelling these variations do not support contingently
% applied behavioural variation.
% \item Also lacking in the literature is a method for applying transformations
% \emph{within} a process using aspects. Aspect orientation techniques apply
% alterations before, after, or around their targets.
% \item We therefore see a need for:
%     \begin{enumerate}
%         \item Behavioural variation modelled as aspects
%         \item Tooling supporting a more ``natural'' representation of variation,
%         i.e. a variation of a process applied to that process (rather than
%         additional behaviour appended to its start or finish)
%         \item Contingent behavioural variation, i.e. variations applied during
%         simulation execution
%         \item Empirical evidence that aspect oriented models:
%         \begin{enumerate}
%             \item Allow separation of behavioural variance into aspects
%             \item Allow new modelling techniques, such as
%             compositionally-defined hypothesised behaviour
%             \item Provide the typically defined benefits of aspect orientation,
%             i.e. reusable components in different modelling scenarios (i.e.
%             aspects are ``oblivious'' to the codebase they are applied to)
%         \end{enumerate}
%         \item In addition, we must be certain that a model with variations
%         applied still represents the originally specified model. That is to say,
%         we want to fulfil the opportunities observed in the literature, while
%         demonstrating that applying variation to a model in order to more
%         realistically represent its domain (a model of human behaviour including
%         varied behaviour the mistakes we'd expect to see exhibited by human
%         actors is arguably more ``realistic'' than one without those mistakes)
%         can be done in a ``safe'' way, i.e. modifying the model programmatically
%         does not change the model in a way which makes it \emph{less} reflective
%         of reality by mistake.
%     \end{enumerate}
% \end{itemize}

% The underlying hypothesis is:

% # Aspect orientation is well suited to modelling behavioural variations

% In order to investigate this hypothesis, we must investigate:

% ## Can we modify a model while keeping it at least as plausibly realistic as the
% original (if not more so)?
% ## Can aspect orientation provide the benefits claimed in the paradigm's
% literature in the simulation and modelling domain?

% To support those investigations, we need:

% ## A suitable system to model in which we can assert the ``realism'' of a model
% before and after the application of variance
% ## Aspect oriented tooling which supports application of behavioural variation
% ## Ideally also, aspect oriented tooling which addresses the shortcomings of
% other aspect orientation frameworks (no ``inside'' join point, engineering
% difficulties such as obliviousness' impact on legibility)



% Old research questions were:

% 1. Can we fit model details per-player to get realistic player behaviour?
% 2. Can we cluster based on accuracy of different models for each player?
% 3. Can we generate predictive data for unseen models?

% The contributions we want to demonstrate are:

% 1. Behavioural variation is well-represented by aspect orientation
% 2. Tooling which makes modelling behavioural variation feasible using aspect
%    orientation
% 3. Tooling which improves on aspect orientation's shortcomings (to make AOP
%     more viable to apply to s&m in practice)
% 4. Showing that data produced by aspects is model-specific 


% First draft new research questions:

% 1. Can we demonstrate that a model of human actors can be made realistic
%    through aspect oriented behavioural variation (rather than breaking the
%    model's representation of its domain)?
% 2. Can we demonstrate that our aspects represent real-world variation in
%    actors' behaviour, rather than the emergent change to an overall system?
%    (by clustering, we demonstrate that many people exhibit the _same_ changes,
%    but not everybody does, i.e. the aspect accurately represents a subset of
%    our modelled actors)
% 3. Can aspects applied successfully to one model be expected to see the same
%    success when applied to other models?

% Research questions found in an early .md file detailling the thesis structure:
% - RQ: Can we fit model details per-player to get realistic player behaviour?
% - RQ: Can we cluster based on accuracy of different models for each player?
% - RQ: Can we generate predictive data for unseen models?

%%% Local Variables:
%%% mode: latex
%%% TeX-master: "../thesis"
%%% End:
