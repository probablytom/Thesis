\chapter{Relevant Literature}\label{chap:lit_review}

%\labelledsec{Aspect Orientation}{dynamic_aop_review}
%
%In \cref{kiczales1997aspect}, \citeauthor{kiczales1997aspect} see these
%engineering concepts as universal throughout business logic, motivating the
%aspect-oriented approach for the first time. The authors present an
%implementation of AOP in Lisp, and compare implementations by way of e.g. SLOC
%count in an emitted C program to a comparable, non-AOP implementation, with two
%examples (its use in image processing and document processing). They find the
%idea --- which they note is "young" and note many areas where research might
%help it to grow --- can successfully separate systemic implementation concerns
%such as memory management in a way that reduces program bloat and simplifies
%implemenation. It is noted that measuring the benefits of their approach
%quantitatively is challenging.
%
% Aspect oriented programming (AOP) is a technique for isolating `cross-cutting
% concerns'' in a codebase in separate parts of that
% codebase\cite{kiczales1997aspect}. For the benefit of the reader, a short
% explanation of aspect orientation precedes the discussion of its literature and
% its relevance to the work presented in this thesis.


% \subsection*{Literature Review Structure}\label{sec:lit_review_structure}

This thesis presents an aspect-oriented approach to simulation and experimental
design, tooling to support these endeavours, some empirical assessment of both
the paradigm and its application in the domain of simulation and modelling. In
particular, the presented research makes use of aspects to model hypotheses and
the complexities of variable and unpredictable behaviour in the simulation of
\sociotechnical systems.

Relevant literature in this topic typically comes from a variety of fields which
do not overlap significantly, meaning that this review must cover segments of
several partly related fields. The presented material is eclectic in nature as a
result. We therefore present context for unfamiliar readers as well as a
motivating case for the work to follow in three distinct areas, which are, in
order~\revnote{don't use a description list here, this should be a regular paragraph}:

\begin{description}

    \item[\Cref{lit_review_AOP_explainer},] which introduces aspect orientation and
    gives a background on the field;
    \item[\Cref{sec:dynamic_aop_review},] which details some existing approaches
    to the dynamic weaving of aspects;
    \item[\Cref{sec:ao_and_modelling},] which discusses the use of aspect
    orientation in simulation and modelling; and
    \item[\Cref{sec:dynamism_in_sm},] which outlines literature on the modelling
    of process variance, particularly in simulation and modelling.

\end{description}


\section{Aspect Orientation}\label{lit_review_AOP_explainer}

Often, software engineers have to repetitively handle an issue in some codebase,
where the issue is pervasive across many parts of the codebase and is
necessarily interwoven through its core functionality. Common examples of this
are guarding against unsafe concurrency usage, memory management, and logging.
Modularity is considered a key trait of maintainable, flexible and legible
programs~\cite{Parnas_1972}. Modern design techniques often centre around the
structure of a program to increase its modularity, with object-oriented
programming being the standard approach to designing with modularity in mind in
many industrially-relevant languages today. \inline{Citation needed?}

Approaches to modularity typically section codebases into units of
functionality. Concerns such as logging and memory management happen in
effectively all areas of a codebase, as a result of engineering needs and the
properties of a project's language and environment. As a result, common devices
employed with the aim of increasing modularity are unable to strip these
``cross-cutting concerns'' into some separate, modular unit. Programmers
separating these concerns into additional modules are expected to see two key
benefits:

\begin{enumerate}
    \item A reduction of \emph{``tangling''}, where program logic essential for
    a program's intended purpose is intermixed with ancillary code addressing
    cross-cutting concerns, thereby making essential logic more difficult to maintain;
    \item A reduction of \emph{``scattering''}, where program logic for
    cross-cutting concerns is strewn throughout a codebase, making maintenance
    of this code more difficult.
\end{enumerate}

The existence of cross-cutting concerns therefore expected to make maintenance
of both ancillary program logic and a program's core logic more difficult.
Addressing this, \citeauthor{kiczales1997aspect} introduced the notion of
aspect-oriented programming~\cite{kiczales1997aspect}. Aspects are best
described through their component parts:

\begin{itemize}
    \item A \emph{``join point''} defines some point in a program's execution
(usually the moment of invocation or return of some function or method).
    \item \emph{``Advice''} defines some behaviour, such as emitting a logline, which
    can conceptually happen anywhere in program execution (i.e. what's defined
    would typically represent behaviour which cuts across many parts of a
    codebase).
    \item An \emph{``aspect''} is constructed by composing this advice with
    \emph{``point cuts''}: sets of join points that define all moments in
    program execution where associated advice is intended to be invoked.
    \item An \emph{``aspect weaver''} then adds the functionality defined by
    each aspect by adding the functionality defined by its advice at each join
    point defined by its point cut.
\end{itemize}

The definition of join points and advice, or how weaving occurs, is a matter
left for aspect orientation frameworks and languages to define. In employing the
technique, aspect oriented programming aims to separate cross-cutting concerns
into aspects, removing the aforementioned repetitive code from the logic
implementing a program's functional behaviour so that additional pieces of
functionality --- logging, authentication, and so on --- can be maintained in
only one place in a codebase (thereby simplifying their maintenance and
comprehension), and remaining program logic can be understood and maintained
without the overhead imposed by the previously tangled cross-cutting concerns.

In \cite{kiczales1997aspect}, \citeauthor{kiczales1997aspect} see these
engineering concepts as universal throughout business logic, motivating the
aspect-oriented approach for the first time. The authors present an
implementation of AOP in Lisp, and compare implementations by way of e.g. SLOC
count in an emitted C program to a comparable, non-AOP implementation, with two
examples (its use in image processing and document processing). They find the
idea --- which they note is "young" and note many areas where research might
help it to grow --- can successfully separate systemic implementation concerns
such as memory management in a way that reduces program bloat and simplifies
implementation. It is noted that measuring the benefits of their approach
quantitatively is challenging.

Tooling followed the theoretical work presented by
\citeauthor{kiczales1997aspect}~\cite{kiczales1997aspect} with a demonstration
and subsequent technical description of AspectJ, a Java extension for aspect
oriented programming~\cite{AspectJLanguageAndTools,aspectj_intro}. AspectJ was
introduced to satisfy the research community's need for a tool with which to
demonstrate the aspect-oriented paradigm in case studies. The tool is intended
to serve as ``the basis of an empirical assessment of aspect-oriented
programming''~\cite{aspectj_intro}. The library makes use of standard
aspect-oriented concepts: Pointcuts, Join Points, and Advice, bundled together
in Aspects. They define ``dynamic'' and ``static'' cross-cutting, by which they
refer to join points at specific points in the execution of a program, and join
points describing specific types whose functionality is to be altered in some
way. Their paper describes only ``dynamic'' cross-cutting, but presents tooling
support, architectural detail of its implementation, and the representation \&
definition of pointcuts in AspectJ. AspectJ is compared to other tools for
aspect orientation and related decompositional paradigms, and the authors are
explicit about their approach being distinct from metaprogramming in, for
example, Smalltalk or Clojure.


\citeauthor{filman2000aspect} isolate properties of aspect orientation which
they assert are definitive of the paradigm~\cite{filman2000aspect}. Specifically, they claim that aspect
oriented programming should be considered ``quantification'' and
``obliviousness'':

\begin{blockquote}
AOP can be understood as the desire to make quantified statements about the
behaviour of programs, and to have these quantifications hold over programs
written by oblivious programmers.~[\ldots{}]~We want to be able to say, ``This
code realises his concern. Execute it whenever these circumstances hold.
\end{blockquote}

These concepts, alongside ``tangling'' and ``scattering'', became core
concepts in aspect orientation literature. This is in spite of
\citeauthor{filman2000aspect} giving no concrete definition of the terms in
their original paper (nor citing a source for definition). For the purposes of
this thesis, we therefore provide the following definitions of the terms:

\begin{description}
  \item[Quantification] is the property of specifying specific points in a
  program in which that program should change;
  \item[Obliviousness] is the property of a codebase that it contains no
  lexical or conceptual reference to advice which might be applied to it, and of
  the programmer of a target program that their code may be amended by an aspect programmer.
\end{description}

\citeauthor{filman2000aspect} write about aspect orientation \emph{``qua
programming language''}, and so theorise around aspect orientation as a paradigm
independent of a particular instantiation. They are therefore able to arrive at
conclusions about the paradigm in the abstract, and can identify concerns for
future investigation for researchers in the field and design goals for
developers of aspect orientation tooling. They note:

\begin{blockquote}
Better AOP systems are more oblivious. They minimize the degree to which
programmers (particularly the programmers of the primary functionality) have to
change their behaviour to realise the benefits of AOP. It's a really nice bumper
sticker to be able to say, ``Just program like you always do, and we'll be able
to add the aspects later.'' (And change your mind downstream about your
policies, and we'll painlessly transform your code for that, too.)
\end{blockquote}

Whether obliviously designed aspect-oriented systems achieve their intended
goals empirically is outside the scope of their work, and the lack of empirical
evidence for this is discussed in \cref{subsec:aop-criticisms}. Claims made such
as changing one's mind while developing or maintaining a program and having that
``painlessly transformed'' --- an effect of the aforementioned programmer's
obliviousness --- is incompatible with earlier writing on modularity.
\citeauthor{yourdon1979structured} assert~\cite{yourdon1979structured}: 

\begin{blockquote}
The more that we must know of module B in order to understand module A, the more closely connected A is to B. The fact that we must know something about another module is a priori evidence of some degree of interconnection even if the form of the interconnection is not known. 
\end{blockquote}

Aspect orientation's critics describe similar incompatibilities with existing
best-practices~\cite{przybylek2010wrong,Constantinides04aopconsidered}, as well
as the lack of empirical evidence for the benefits of
obliviousness~\cite{steimann06paradoxical}. Claims about ``better''
aspect-oriented systems being more oblivious should therefore be regarded as
\emph{suggestions} from the literature, and while obliviousness and
quantification are useful concepts in discussing research in the field. They
also give context for the research community's perspective that obliviousness
and quantification are design goals for aspect oriented
programmers~\cite{AspectCplusplusDesignImp,kell2008survey,aspect oriented
workflow} (though other researchers suggest they may be best applied in
moderation~\cite{leavens2007multiple}).

\subsection{Alternative methods in Aspect Orientation}

Aspect-oriented programming's goal of modularising cross-cutting concerns is
shared by other paradigms. Seminal work in aspect orientation makes note of
similarities to use of reflection, metaprogramming \& program transformation,
and subject-oriented programming~\cite{kiczales1997aspect,aspectj_intro}. They
also observe that other disciplines have introduced ``aspectual decomposition''
independently.

The example of pre-existing aspectual decomposition by way of diagramming given
by \citeauthor{kiczales1997aspect}~\cite{kiczales1997aspect} is in physical
engineering. To give a concrete example from their description, differing types
of diagrams when engineering a system such as thermal and electric diagrams of a
heater are described as ``aspectual'' because of the modular nature of the
diagrams; though there might be many diagrams of different kinds, they compose
together to give an overview of the system being designed.

Similar diagramming techniques have independently arisen in other domains since.
The Obashi dataflow modelling methodology\cite{obashimethodology} by
\citeauthor{obashimethodology} models all possible paths of dataflow through
``B\&IT'' (business and IT) diagrams, where business-specific concerns (people,
locations, groups, and business processes such as payroll, stock-check or
budgeting) are modelled alongside IT concerns such as applications supporting
business processes and the software and hardware infrastructure supporting them.
Modelling dataflows in this way allows for a comprehensive understanding of
assets and business processes. However, in order to understand how data flows
between specific assets within a B\&IT, sub-graphs (``DAVs'', or Dataflow
Analysis Views) denote specific pathways through which data flows between source
and sink assets. Alternatively, a B\&IT can be viewed as a composition of all
possible DAVs within an organisation. Dataflows are therefore broken into
different diagramming techniques and specific business concerns can be described
independently of others, even if these concerns interact in their dataflow
pathways (and, therefore, cutting across each other). Obashi therefore allows
for the aspectual decomposition of business processes, through the description
of an organisation by individual dataflow analysis views, which compose into an
overall model of a system in a B\&IT diagram. Obashi models are an instance of
aspect orientation which were designed for simplicity and
comprehension\cite{obashimethodology,seow2011obashi}, but trade this for
domain-specificity.

Research conducted by \Citeauthor{keller1998binary}~\cite{keller1998binary}
investigates solutions to the difficulties involved in the integration of
software components and their evolution over time, where those components are
re-used with differing requirements. By modifying binaries directly,
incompatibilities in a program and one of that program's dependencies can be
resolved by way of mutating either after compilation. Their implementation
defines a representation for the modification of pre-generated Java class
binaries, the output of which can be verified as also being valid Java class
binaries. \Citeauthor{keller1998binary} claim that BCA allows for dynamic
modification of programs with little overhead. They believe BCA is unique in its
combination of features, which include engineering concerns such as typechecking
code which is subject to adaptation and its obliviousness to source
implementation, as well as guarantees that modifications are valid even for
later iterations of the program subject to adaptation.

Other engineering techniques can be used to modularise cross-cutting concerns.
For example, metaobject protocols describe the properties of an object's class
(including, for example, its position within a class hierarchy) in an adaptable
manner~\cite{kiczales1991art}. These can be used to implement aspect
orientation~\cite{espakaspect}, therefore providing at a minimum the same
functionality, though they achieve this through their reflective qualities and
are designed with metaprogramming as a primary goal as opposed to modularisation
of cross-cutting concerns~\cite{kiczales1991art,sullivan2001aspect}.
Multiple-dispatch, where methods on objects are chosen to be run based on the
properties of the parameters passed at point of invocation, allows for oblivious
decomposition without the need for a weaver~\cite{dozsa2008lisp}, although this
does not support the goals of aspect orientation in totality. For example, a
programmer might want their program to exhibit differing behaviour when methods
are called with differently-typed arguments, which is supported by multiple
dispatch. However, they might instead want their program to exhibit some
additional behaviour whenever a method is invoked, such as logging, but might
not want to implement logging alongside the rest of their method implementation
for clarity or length reasons. Multiple dispatch therefore offers comparable but
different functionality to a software engineer. Engineering patterns such as
decorators provide similar functionality to
aspects~\cite{friesel2017annotations}, in that cross-cutting concerns can be
separated into their own module, but they differ in their approach to
obliviousness: decorators annotate areas of a codebase they are applied to, and
therefore do not offer obliviousness as aspects do. \inline{Add a paragraph
after this on subject-oriented programming.}

More notes~\inline{Add notes on subject-oriented programming, bigraphs, maybe holons?}
%% subject oriented programming

\labelledsubsec{Criticisms of aspect-oriented programming}{aop-criticisms}

The growing aspect-oriented programming research community collected both
proponents and detractors of the paradigm. The developments in
aspect orientation pertinent to the research presented in this thesis are
discussed in later sections of this literature review. However, common
criticisms of aspect-oriented programming are important to present in two
regards:

\begin{enumerate}
    \item Discussions of advancements in the field pertinent to the work
    presented in this thesis should be understood within the context of some
    perceived weaknesses in the field, which helps to frame an understanding of
    literature reviewed in this chapter,
    \item The presented work addresses some criticisms of aspect-oriented
    programming, meaning that the criticisms of the paradigm writ large and
    properties of work published in awareness of those weaknesses will motivate
    some research presented in later chapters.
\end{enumerate}

An early piece of scepticism in the aspect orientation community is
\cite{Constantinides04aopconsidered}, in which
\citeauthor{Constantinides04aopconsidered} see AOP's core concepts as having
significant similarities to \lstinline{GO-TO} statements, which have
historically been the subject of some derision in the literature.
\cite{Constantinides04aopconsidered} is, in spirit, a child of
\citeauthor{dijkstra1968letters}'s \citetitle{dijkstra1968letters}. The authors
note that the notion of unstructured control flow makes reasoning about a
program complicated --- disorientating a programmer by way of ``destroying their
coordinate system'', leaving them unsure about both a program's flow of
execution and the states at different points of that flow --- and discuss
whether aspect orientated programs can have a consistent "coordinate system" for
developers. They note that, while Go To statements are at least visible in
disrupted code, the AOP concept of obliviousness makes such reasoning even more
difficult than Go To statements, as even the understanding of where and how flow
is interrupted is not represented structurally within an aspect-oriented
program. They compare aspects to a Come From statement, noting that the concept
is a literal April Fools' joke for programming language enthusiasts who claim
they've found an improvement over Go To statements. The authors conclude that
existing techniques, specifically Dynamic Dispatch in OOP, provide similar
benefits without the trade-off in legibility of a program's intended execution.


A similar and more thorough critique of the aspect oriented paradigm can be
found in \citeauthor{steimann06paradoxical}'s
\citetitle{steimann06paradoxical}~\cite{steimann06paradoxical}. The concern in
this paper is that the popularity of aspect oriented programming --- which was
nearly 10 years old at time of publication --- was founded on a perception that
it assisted in engineering more than it was proof that such assistance viable in
practice. The author notes that most papers are theoretical in their discussion
on tooling, that examples were typically repetitive, and that the community's
discussion concerned more what aspect orientation is good for than what it
actually is in practice. AOP is compared against OOP, AOP's claimed properties
and principles are examined in detail, and the impact on software engineering is
reasoned about from a sceptical perspective, comparing claims such as improved
modularity against classic papers on the subjects (such as Parnas' work on the
same). The paper presents a philosophical examination of aspect orientation,
assessing the paradigm against its purported merits and discussing whether we
should expect, rationally, that the claims made by the AOP research community
would hold true. The paper ends noting some benefits of AOP that do hold true
under rational scrutiny, and notes that the true utility of AOP may be very
different to those purported by the community. Overall, the paper is a
philosophical and critical reflection on the state of AOP research and the
community's zeitgeist at the time, claims around which are not well-evidenced in
literature. In particular, the author sees AOP's promise of unprecedented
modularisation as unfulfillable. 

Similar sentiments are shared by
\citeauthor{przybylek2010wrong}~\cite{przybylek2010wrong}, who looks to examine aspect oriented
programming within the context of language designers' quest to achieve
maintainable modularity in system design. They frame the design goals of aspect
orientation as being to represent issues that ``cannot be represented as
first-class entities in the adopted language''. The paper discusses whether the
modularity offered by aspect orientation actually makes code more modular. In
particular, they distinguish between lexical separation of concerns and the
separation of concerns originally discussed by
\citeauthor{djikstra_scientific_thought} in
\citetitle{djikstra_scientific_thought}. They assess the principles of
modularity --- modular reasoning, interface design, and a decrease in coupling,
for example --- and find that from a theoretical perspective, there are many
reasons to believe that the aspect-oriented paradigm can detrimentally impact
the expected benefits of proper modularisation in a program. They conclude that
the benefits touted by AOP are a myth repeated often enough to be believed, but
point to many papers which suggest improvements to the standard AOP approach
which might reduce it's negative impact or make it more practically viable.
\citeauthor{przybylek2010wrong} presents a critical review of aspect orientation
literature, but often hints at others' solutions to the problems identified too.


%%  ## References to pick up & review:
%%  
%%  On criticisms of AOP
%%  - Dantas & Walker 2006, from What Is Wrong With AOP
%%  - Leavens & Clifton 2007, from What Is Wrong With AOP
%%  - Filman & Friedman 2001, from What Is Wrong With AOP
%%  - Constantinides, Scotinides & Störzer 2004, from What Is Wrong With AOP
%%  - Tourwe, Brichau & Gybels 2003, from What Is Wrong With AOP
%%  - Wampler 2007, from What Is Wrong With AOP
%%      - "Most AO languages in use today are based on structural information about
%%      join points, such as naming conventions and package structure, rather than
%%      the logical patterns of the software
%%  
%%  On AOP and decreasing coupling
%%  - Yourdon & Constantine 1979, from What is Wrong With AOP
%%      - "The fact that we must know something about another module is a priori
%%      evidence of some degree of interconnection even if the form of the
%%      interconnection is not known" 


%% NOTE: MOVE THIS TO A REGULAR SECTION FILE
\section{Dynamic Aspect Weaving}\label{sec:dynamic_aop_review}

One implementation of dynamic weaving is
PROSE~\cite{popovici2002PROSE,popovici2003JITaspects}, a library which achieves
dynamic weaving by use of a Just-In-Time compiler for Java. The authors saw
aspect orientation as a solution to software's increasing need for adaptivity:
mobile devices, for example, could enable a required feature by applying an
aspect as a kind of ``hotfix'', thereby adapting over time to a user's needs.
Other uses of dynamic aspect orientation they identify are in the process of
software development: as aspects are applied to a compiled, live product, the
join points being used can be inspected by a developer to see whether the
pointcut used is correct. If not, a developer could use dynamic weaving to
remove a misapplied aspect, rewrite the pointcut, and weave again without
recompiling and relaunching their project.

Indeed, the conclusion \citeauthor{popovici2003JITaspects} provide in
\cite{popovici2003JITaspects} indicates that some performance issues 
% generalised by \citeauthor{dynamicAOchitchyan} in \cite{dynamicAOchitchyan
may prevent
dynamic aspect orientation from being useful in production software, but that
it presented opportunities in a prototyping or debugging context.

PROSE explores dynamic weaving as it could apply in a development context, but
the authors do not appear to have investigated dynamic weaving as it could apply
to simulation contexts, or others where software making use of aspects does not
constitute a product.

% MARK: dynamicAOchitchyan

The performance issues noted by \citeauthor{popovici2003JITaspects} are explored
in more detail by \citeauthor{dynamicAOchitchyan} in \cite{dynamicAOchitchyan}.
\citeauthor{dynamicAOchitchyan} present a review of early dynamic aspect orientation
techniques. The paper reviews AspectWerkz, JBoss, Prose, and Nanning Aspects
through the lens of the authors' prior work on dynamic reconfiguration of
software systems and their generalisation of dynamic aspect orientation
approaches: 

\begin{enumerate}
\item ``Total hook'' weaving, where aspect hooks are woven at all possible
points; 
\item ``Actual hook'' weaving, where aspect hooks are woven where required;
\item ``Collective'' weaving, where aspects are woven directly into the executed
code, ``collecting the aspects and base in one unit''. 
\end{enumerate}

Because of the paper's focus on software reconfiguration (rather than the
mechanics and design of dynamic aspect weaving specifically), the analysis of
the tools presented in the paper is of less relevance to the work presented in
this thesis than their generalisation of dynamic weaving. The trade-offs of the
three generalised philosophies are discussed. \citeauthor{dynamicAOchitchyan}
propose that total hook weaving allows flexibility in the evolution of a
software product, at the expense of the performance of that product; this
contrasts collected weaving, which shifts overhead out of the codebase and into
the maintenance effort. Actual hook weaving is positioned as a compromise
between the two, offering the best approach for none of their criteria but never
compromising so much as to offer the worst, either. This suggests merit in a
tool designed to flexibly offer any weaving approach appropriate for the task at
hand. It's explicitly noted that, in practice, one could use many of the systems
they describe. Though the paper is an early publication in the field, no tool
the authors review offers all three, and none offers collective weaving
alongside either kind of hook weaving.


In contrast, \citeauthor{gilani2004family} note that, while there are different
approaches to dynamically weaving aspects, no approach is suitable for an
embedded environment. This is due to these systems' low power and available
memory. \citeauthor{gilani2004family} therefore propose a framework through
which weavers can be assessed for suitability in a given environment, or
generated from a set of possible features (where, presumably, features would be
enabled and disabled as per an environment's needs). Their families of weavers
are defined by the similarities of the requirements in domains they are applied
to, and specifically defined by their trade-off between dynamism and resource use
(asserted to be broadly proportional). It is unclear that a carefully crafted
``actual hook weaver'', or JIT-compiled ``collective weaving'', in the parlance
of \citeauthor{dynamicAOchitchyan} (see \citep{dynamicAOchitchyan}), would be
meaningfully less efficient than static weaving in all but the extreme
application areas outlined in the paper (embedded systems with resources in the
range of 30kb memory).

Aspect oriented programming's criticism can often be that it doesn't know what
it ``aims to be good for'', and its application in such extreme environments is
arguably mistaken from the off. The families outlined in
\citeauthor{gilani2004family}'s publication are unnecessary if dynamic aspects
are not required in their target environments.
\Citeauthor{steimann06paradoxical}'s critique of aspect-oriented programming,
contrasted against these families, presents an interesting question. If the
goals of dynamism and resource efficiency are at odds, and
\citeauthor{steimann06paradoxical}'s stance that aspect-oriented programs do not
earn its proponents' plaudits in practice~\inline{don't be fancy here}, what can
dynamic aspect weaving be appropriately applied to? In what environment does the
trade-off presented by dynamic weaving not necessitate a theory like
\citeauthor{gilani2004family}'s in the first place? Arguably, that environment
is not found in low-resource systems, and a take-away of \cite{gilani2004family}
could be that researchers should seek other contexts in which to apply aspect
oriented programming. \footnote{As discussed alongside \cite{gulyas1999use} in
\cref{sec:ao_and_modelling}, simulation \& modelling might present a more
appropriate field.}\inline{Note from discussion with Tim to work into this:
``the anticipated benefits of aspect-oriented programs are not observed in
practice''}~\inline{Both paragraphs on Gilani2004 need to be heavily edited.}




%% MARK MARK MARK

% This is a good position to add notes on other dynamic weavers

%% MARK MARK MARK



\citeauthor{rajan2006nu_towardsao_invocation} propose a new aspect-oriented
invocation mechanism, which they call
``Bind''~\cite{rajan2006nu_towardsao_invocation}. Bind's design is motivated by
perceived opportunities to improve modularity from a design perspective. The
impact of ``scattering'' and ``tangling'' in a codebase after weaving in some
aspect orientation implementations leads to a more complicated post-weave
codebase, which in turn leads to increased difficulty including the compilation
of aspect-oriented code and the development and execution of unit tests on said
code. In order to demonstrate Bind's approach to simplifying post-weave
codebases, the design of ``Nu'', an aspect orientation framework in .NET
supporting Bind, is explained and an implementation presented. They present Bind
as an alternative to the weaving of aspect hooks (for load-time registration)
into target code, in the style of PROSE (see
\cite{popovici2002PROSE,popovici2003JITaspects}), or the weaving of calls
directly, in the style of AspectJ (see \cite{aspectj_intro}). Bind's model for
aspect orientation is to apply or remove implementations of cross-cutting
concerns to arbitrary sets of join points at a time of a developer's choosing.
Nu's model for this is designed with the aim of granularity of join point
specification. What results is a flexible model for aspect orientation which is
demonstrated to satisfactorily emulate many other models for aspect oriented
programming, such as the models of AspectJ, HyperJ, and Adaptive Programming. It
is noted that it is ``very common in aspect-oriented programming research
literature to provide language extensions to support new properties of
aspect-like constructs'', and note that their work is similar to (yet distinct
from) weaving approaches in run-time \& load-time weaving, support for aspect
orientation directly in a language's virtual machine, and work towards general
models of aspect oriented programming (models which can represent a variety of
existing approaches). Their approach is flexible and considered enough to
warrant impact in the introduction of aspect orientation within virtual machines
(because they require no direct support), and in their ability to represent
different weaving approaches, arguably \emph{because} their approach is general
enough in design to approach the general model worked towards, which qualifies
their satisfaction of their motivation to provide a model distinct to the
approaches initially discussed. In line with the complaints of AOP's critics,
this does not seem to qualify the satisfaction with which they achieve their
practical engineering goals.

Relatedly, \citeauthor{dyerNUmasters} explain in more depth than in the design
and implementation of the Bind mechanism and the implementation of the Nu
framework~\cite{dyerNUmasters}. A more technical discussion is presented, in
particular on implementation details including optimisation and benchmarking,
largely against AspectJ. Notably, the implementation discussed is a Java
implementation, rather than the .Net implementation presented in
\cite{rajan2006nu}. Many aspect orientation frameworks are language-specific;
the existence of Nu's implementation on multiple platforms highlights the work's
most interesting facet being the design of the Bind primitive, rather than the
framework itself. In a research area where tooling papers are common but the
lack of design philosophy \& analysis of case studies is frequent fodder for
critics, the novelty of the Bind mechanism distinguishes this series of papers.


\labelledsec{Aspect Orientation in Simulation \&
Modelling}{ao_and_modelling}

Aspect orientation as applied to simulating systems, and building models of
systems, has been researched from several approaches.

A very early example of aspect orientation in simulation \& modelling is
presented by \citeauthor{gulyas1999use}~\cite{gulyas1999use}.
\citeauthor{gulyas1999use} observes that, in the study of complex systems
through software models, the software developed typically serves two purposes:
the experimental subject, and the observational apparatus used to conduct the
experiment itself. Arguing that the separation of these roles ought to make both
the implementation of an experimental system and its later analysis simpler,
\citeauthor{gulyas1999use} proposes the use of aspect orientation as a means of
separating what they perceive as cross-cutting concerns of systems modelling.
They present their Multi-Agent Modelling Language, a language implemented in
Objective-C via the Swarm simulation package and designed for aspect-oriented
simulation of agent-based models. Their aspect orientation effectively makes use
of Observer patterns to measure a pre-constructed system under simulation,
without the observations being an intrinsic component of the simulated system.
They find that AOP provides an intuitive and straightforward method by which
simulated experimental systems can be composed, and that MAML's simplicity and
its philosophy on modelling are more ``satisfactory'' than Swarm's standard
approach, though the paper betrays that its implementation was more complex than
initially conceived: the \lstinline{patch} unix tool was intended for use as
their weaver, though the team eventually developed a transpiler from MAML to
Swarm instead (which they name \lstinline{xmc}.). The deciding factors for the
development of a custom transpiler are not discussed.

\citeauthor{gulyas1999use}'s work presents not only tooling for
aspect oriented simulation, but some reasoning \& philosophy on the potential
benefit of using aspect orientation in these endeavours that extends further
than the conclusions of modularity through separation of concerns and a
reduction of tangling \& scattering. In particular, their work discusses
specific scenarios in which the \emph{type} of separation of concerns offered by
aspect orientation is desirable, and the engineering approach to achieving the
aim reasonable. This distinguishes the work in comparison to many aspect
orientation papers reviewed in this chapter. Many papers describe the expected
benefits by simply drawing from existing literature such as
\cite{kiczales1997aspect}. The fact that a rare example of detailed reasoning
about the appropriateness of aspect oriented programming in a particular domain
is highlighted because the domain in question is simulation \& modelling; the
subject of this thesis. That aspect orientation might be well suited to
separating observer and experiment motivates, in part, this thesis' work showing
the plausible realism of a simulation in which behaviour is modified by aspects.
Put another way: this thesis draws on the idea that, in response to
\citeauthor{steimann06paradoxical} asking what aspect orientation is good for,
\citeauthor{gulyas1999use} would seem to answer, ``simulation \& modelling'', a
premise this thesis shares.

\citeauthor{chibani2013toward} discuss two issues in object-oriented
programming: ``tangling'', where separate design elements of a program are woven
within each other in program source, and ``scattering'', where a single design
element is strewn throughout the source, rather than being contained within a
single area of the codebase.~\inline{Remove the description of tangling and
scattering from here and include in a review of more foundational AOP literature
early in the lit review. They don't belong buried so far down in this chapter.}
They propose that aspect orientation solves these problems, and identify that
there are potential benefits in discrete event simulation code in both regards,
making DES frameworks with aspect oriented primitives a potentially fruitful
contribution to the research community. \citeauthor{chibani2013toward} identify
cross-cutting concerns in DES codebases, including event handling, resource
sharing, and the restoration of a simulation run. The contribution of the paper
is the discussion of AOP's potential application to DES codebases, and detail of
the avenues available for research in the field. Japrosim is presented as a
motivating example of an existing DES framework which they see as ripe for the
aspect oriented enhancements they identify.

In a later publication~\cite{chibani2019using}, \citeauthor{chibani2019using}
identify opportunities for the use of aspect orientation in simulation tooling,
aiming to increase ``modularity, understandability, maintainability,
reusability, and testability'' by applying the paradigm~\cite{chibani2019using}.
They present a case study of an application of aspect orientation to simulation
tooling by identifying cross-cutting concerns in Japrosim, a discrete event
simulation framework, and propose an aspect-oriented redesign of the tool using
AspectJ. \citeauthor{chibani2019using} describe Japrosim's existing
object-oriented design, followed by aspect oriented variations of some design
elements, including concurrent process management and in Japrosim's graphical
animation features. A similar survey of areas in which Japrosim's source might
benefit from the application of aspect orientation is presented by
\citeauthor{chibani2014practical} in an earlier
work~\cite{chibani2014practical}. In both cases, the main contribution noted is
the design itself. Counting the main improvements between the presented
aspect-oriented design and the existing object-oriented one is left to future
work in the authors' later publication~\cite{chibani2019using}, although a
concrete implementation is linked to and some quantitative evaluation of that
implementation presented in their earlier
publication~\cite{chibani2014practical}. The quantitative evaluation provides
measurements based on \citeauthor{martin1994oo}'s object-oriented design metrics
and demonstrates a greater independence of packages in their aspect oriented
version of Japrosim than in the original. However, the intended aim of aspect
orientation is not to decouple existing packages, but to isolate those packages'
cross-cutting concerns into new ones. It is therefore unclear that their
quantitative evaluation achieves its improvements as a result of aspect
orientation. No further discussion of their results is provided, and it is
possible that the improvement is due to the rewriting necessary in their
maintenance of the Japrosim source, rather than due to their use of aspect
orientation specifically.

In a manner similar to
\citeauthor{chibani2019using}'s~\cite{chibani2019using,chibani2013toward,chibani2014practical},
\citeauthor{DEVSaspectorientation2008aksu} observe that there are opportunities
to be found in a simulation framework able to take advantage of aspect
orientation~\cite{DEVSaspectorientation2008aksu}. Examining the DEVS framework
Simkit, their proposal for aspect-oriented programming adoption is two-fold:
refactoring of the framework itself and aspect-oriented tooling for use by
modellers, who represent cross-cutting concerns within their models.
Opportunities for improvements in production and development are discussed, and
some implementation notes are detailed, although no concrete implementation or
evaluation is provided; the work instead proposes design alterations, and the
authors ``leave it as a future work \emph{[sic]} to explore the usability and
efficiency'' of aspect orientation used idiomatically alongside Java's existing
reflection offerings. The existence of multiple attempts to refactor differing
simulation packages with aspect orientation indicates potential for modellers in
the use of aspect-oriented patterns, but the real-world utility of the
techniques are omitted. \citeauthor{chibani2019using} and
\citeauthor{DEVSaspectorientation2008aksu} both seem to defer to the zeitgeist
wisdom of the aspect orientation community in their unproven claim that it
improves modularity and maintainability of a codebase.
\footnote{\citeauthor{chibani2014practical} do present some quantitative
evaluation, but this is flawed as previously described.}

Neither \citeauthor{gulyas1999use} nor
\citeauthor{DEVSaspectorientation2008aksu} detail a case study of their
techniques with real-world examples. However, \citeauthor{ionescu2009aspect} do
in their work identifying an increased demand for computational power in
simulation execution on supercomputers~\cite{ionescu2009aspect}. Existing
known-good models might be unsuitable for the extreme requirements of code
efficiency modellers contend with, but running the code on different
environments requires modifications for suitability in different environments,
around which there are regulations and risks of a reduction in quality during
maintenance. The authors propose an aspect-oriented solution to the problem,
where aspects modify the simulation codebase with minimal overhead. An
implementation of a real-world model for disaster prevention is presented, and
assessed both by comparison against an equivalent non-aspect-oriented codebase
and by assessment of the aspect-oriented variant's scalability and reliability
in both cluster and multi-cluster environments. They find that a comparative
analysis of generated code and of their simulations in various configurations
both indicate that their simulation's aspect-oriented implementation is suitable
for use in disaster prevention, implying that aspect orientation could be
suitable in scenarios with comparable requirements.

That the authors can conclude that aspect orientation is suitable in a
real-world use case constrained by the requirements of supercomputer use seems
promising for the aspect-oriented paradigm, which is criticised for its lack of
practical
evaluation~\cite{Constantinides04aopconsidered,steimann06paradoxical,przybylek2010wrong}.
As it is therefore a rare example of aspect-oriented case studies, their
evaluation methodology is important to highlight. Their code analysis makes use
of significant lines of code as a core metric, which doesn't reliably reflect
code quality; quoting \citeauthor{rosenberg1997some}'s \citetitle{rosenberg1997some}~\cite{rosenberg1997some}:
\begin{displayquote}
    (\ldots{})~the best use of SLOC is not as a predictor of quality
itself (for such a prediction would simply reduce to a claim about size, not
quality), but rather as a covariate adjusting for size in using another
metric.
\end{displayquote}
This is important because a common claim in the aspect-oriented
literature for which there is little empirical evidence is that aspect
orientation improves the ``quality'' of a codebase, but the related claims made
by \citeauthor{ionescu2009aspect}~\cite{ionescu2009aspect} are unreliable.

It is important to note that improvements in code quality specifically are those
which have come under scrutiny by the critical papers reviewed in
\cref{subsec:aop-criticisms}. The results presented by
\citeauthor{ionescu2009aspect} do not satisfactorily address the requests for
empirical evidence of improved code quality in these reviewed criticisms. This
does not impact their aspect-oriented models' suitability given the motivations
of \citeauthor{ionescu2009aspect}, which are that models should be amendable for
new supercomputing settings without lack of performance. The models described in
this work satisfy that aim. These models are also evaluated by way of
performance, an important factor in supercomputer use where execution time is
financially expensive and power-intensive. Quantitative evaluation of their
simulation's execution time shows less than 5\% slowdown compared to a
non-aspect-oriented implementation. \citeauthor{ionescu2009aspect} deem this a
reasonable trade-off for the engineering improvements they observe. Their
application to the supercomputing \& disaster prevention simulation domains
therefore seem satisfactory by way of performance, and meet
\citeauthor{ionescu2009aspect}'s aim of demonstrating a modelling technique
which permits adapting existing models for use in new environments without
directly maintaining the original model's source.

This result is notable with regards the findings presented in this thesis, which
similarly aim to alter a pre-existing model without directly altering it,
although for purposes such as model reuse and simplification of experimental
design rather than for avoiding the regulatory overhead and financial cost of
maintaining the models described by
\citeauthor{ionescu2009aspect}~\cite{ionescu2009aspect}.



\labelledsubsec{Aspect Orientation \& Business Process Modelling}{ao_and_bpm_review}


% As business processes models represent a kind of \sociotechnical system, and
% this thesis offers tooling for and results in the modelling of \sociotechnical
% systems, . Additionally,
% related work undertaken before this PhD develops on software engineering
% processes that lend themselves well to the same modelling paradigms as business
% processes (see \cref{chap:prior_work}), and there also exists interest in
% modelling behavioural variance within the business process modelling community
% (see \cref{sec:dynamism_in_sm}). This overlap necessitates a review of related
% literature within the business process modelling field. \inline{Add Charfi \&
% Cappelli's work before Jalali's in this subsec}


Several projects within the business process modelling research community make
use of aspect orientation to design modelling languages which produce less
monolithic business process models~\cite{Cappelli_AOBPM,da2020implementation}
and simplify the composition of models~\cite{charfi2007ao4bpel}. As business
processes are inherently sociotechnical and this thesis presents tooling for and
results in the modelling and simulation of \sociotechnical systems using
aspect-oriented techniques, it is important to review this community's
literature.~\inline{Editing tip from Tim: don't call things "important", this
implies other things aren't. Rework areas where I do this.} This field is particularly relevant as work on this project prior to
this thesis' research models software engineering processes that are
conceptually similar to business process modelling (see \cref{chap:prior_work}),
and there also exists interest in modelling behavioural variance within the
business process modelling community (see \cref{sec:dynamism_in_sm}), which is
highly relevant to this thesis' concern with the representation of alteration to
process and modelled behaviour as aspects.

\citeauthor{charfi2007ao4bpel} see opportunities in integrating BPEL, an
executable business-process modelling language, with aspect-oriented
concepts~\cite{charfi2007ao4bpel}. This is because when BPEL systems are
composed together the static nature of the logic being composed is not always
appropriate for BPEL's use cases. The specific use-case examined is web service
definitions, where changes affecting composition of multiple component parts can
affect many areas of a final result, making modification error-prone. They
specifically seek to support dynamic workflow definitions --- ``adaptive
workflows'' --- which BPEL's existing extension mechanisms do not sufficiently
support, but the aspect-oriented literature discusses at length (an overview of
which is presented in \cref{sec:dynamic_aop_review}). Therefore, they look to
construct an aspect-oriented BPEL extension. Using the case study of modelling a
travel agency's web services, they create an aspect-oriented extension by first
defining how such an extension would be represented graphically in BPEL's
workflow diagrams. Further detail is added to arrive at a technical definition
with XML representations, weaving mechanics, and eventually the construction of
a BPEL dialect, AO4BPEL. The authors find that their pointcut system (which
describes join points on both processes and BPEL messages), support for adaptive
workflows, and aspect-oriented approach to workflow process modelling make
AO4BPEL unique at the time of publication, though related AOP implementations
exist in each individual area of their contributions. The work is weakened by
brittle semantics around pointcuts, join points, and the temporal nature of
workflow modelling. For example, they note that defining contingent behaviour
--- only applying an aspect conditionally, based on a trace through a simulation
of a modelled system --- would allow the application of advice only when model
state deems this appropriate.\footnote{The contingent application of model
adaptation is a motivating case for some work presented in this thesis; see
\cref{chap:prior_work} for a discussion.} They also call for more generally
theoretical AOP research, which mirrors the requests some critics of aspect
orientation research make (as noted in \cref{subsec:aop-criticisms}).

In a PhD thesis describing AO4BPEL in detail~\cite{Charfi2006AspectOrientedWL}
\citeauthor{Charfi2006AspectOrientedWL} presented a generalisation of the
notation developed for AO4BPEL, which applies to any graphical workflow
modelling language. Accompanying this are some examples of its use building a
framework for enforcing certain requirements of BPEL models, and use of that
framework to develop aspect-oriented frameworks for enforcing security and
reliability within AO4BPEL models.

In later work, \citeauthor{charfi2010AO4BPMN} define a similar aspect-oriented
dialect of BPMN they name AO4BPMN~\cite{charfi2010AO4BPMN}, after asserting that
the concerns addressed by
AO4BPEL~\cite{Charfi2006AspectOrientedWL,charfi2007ao4bpel} in the field of
executable process languages also apply to business-process modelling languages,
and can be solved similarly. The generalised notation of aspectual
workflow models presented in \citeauthor{Charfi2006AspectOrientedWL}'s
thesis~\cite{Charfi2006AspectOrientedWL} are applied to BPMN to produce an
aspect-oriented language specifically for process modelling, as opposed to
executable business process modelling.

\Citeauthor{Cappelli_AOBPM} also note that cross-cutting concerns exist in
business process models, and are specifically motivated by monolithic design
approaches common in business process modelling languages. Like
\citeauthor{kiczales1997aspect}, they claim that a lack of modularity in
business process models leads to cross-cutting concerns scattered throughout a
model~\cite{Cappelli_AOBPM}. To alleviate the issue, they propose a
meta-language, AOPML, which incorporates aspect orientation in a metamodel of
business process modelling languages, and instantiate it within their own
dialect of BPMN. Using a model of a steering committee as a case study, and
separating cross-cutting concerns such as logging, the paper proposes reducing
complexity and repetition graphically, thereby in a manner more in keeping with
the language design philosophies of popular business process modelling
languages, the design and use of which are typically
graphical~\cite{OMG-BPMN-SPEC,opm_original,OMG-UML-SPEC}. They note that this is
in contrast to other applications of aspect orientation in business process
modelling --- specifically AO4BPMN --- where aspect definitions are written in
XML concern not only the advice to be applied but also their relevant join
points, as in general programming aspect orientation implementations such as
AspectJ. In this way, the AOPML exhibits the spirit of business process
modelling more stringently than does \citeauthor{Charfi2006AspectOrientedWL}'s
notation for aspect-oriented workflow modelling.

The difference between \citeauthor{charfi2010AO4BPMN}'s approach in designing
AO4BPMN~\cite{charfi2010AO4BPMN} and \citeauthor{Cappelli_AOBPM}'s approach in
designing AOPML~\cite{Cappelli_AOBPM} highlights design decisions taken when
introducing aspect orientation in a new domain. There is an opportunity for a
domain-specific aspect orientation framework to align its design with the
traditions and idioms already present in models within that domain, but doing so
may break the traditions and idioms which already exist in aspect-oriented
approaches in other domains.  Comparing the approaches of
\citeauthor{charfi2010AO4BPMN} and \citeauthor{Cappelli_AOBPM} does surface that
there may be no clear ``best'' design approach when blending pre-existing
modelling paradigms, such as the graphical modelling languages used in
business-process modelling and the abstract concepts of aspect orientation. The
discussion around whether it is more desirable to adapt existing
design elements of aspect-oriented frameworks to a given domain or adapt that
domain's existing modelling traditions and idioms to incorporate aspect
orientation as it is used elsewhere is outside the scope of this thesis.

New concepts within the design of aspect orientation frameworks are addressed in
the business process modelling community. \Citeauthor{jalali2012aspect} note
that aspect oriented modelling frameworks often do not explicitly model the
precedence of aspect application~\cite{jalali2012aspect}. They address this
limitation by defining a mechanism to be used in capturing multiple concerns as
aspects, where the invocation of advice must follow a certain precedence. The
aim of the work is not to propose tooling around the precedence of aspect
application so much as to contribute to aspect oriented design theory, providing
a notation for precedence which is broadly applicable. The precedence model is,
put simply, that a mapping exists for each application of advice to join point
such that the mapping defines an ordering on advice for that join point. The
definition defines ``AOBPMN'', a formalised dialect of BPMN supporting aspect
orientation with precedence. A case study is provided where AOBPMN is
instantiated within a coloured Petri net. Their study expands on existing work
by research teams led by Capelli~\cite{Cappelli_AOBPM,da2020implementation} and
by Charfi~\cite{charfi2007ao4bpel}, in that it develops a mature formalism for
and model of aspect orientation as applied to business process modelling.
However, \citeauthor{jalali2012aspect} note that their case study is limited in
scale. No tooling or evaluation of the practical benefit of their approach is
provided.


\labelledsec{Process Variance in Simulation \& Data Generation}{dynamism_in_sm}

Brief intro of the section here~\inline{Brief intro of the process variance
section here}
%% Note motivating modelling with variance:
% - representing contingent behaviour
% - generating data with variation, for i.e. process mining.

% First review here should probably be the Aalst-co-written one on executable
% BPMN for log generation with variance

Typically, a simulation concerns a single process. This means that all expected
behaviour must be included within that process; complex or contingent behaviour
must be represented within it. The techniques reviewed here offer separately
including some modification of a process (or represent the modifications of
varied process within the simulated
output~\cite{stocker2013secsy,stocker2014secsy}). The benefit of this approach
is that possible changes to a process can be described once and applied to that
process where appropriate. Process changes might describe attempts to circumvent
security protocols, laziness or confusion in a human actor within the model, or
random ``noise'' so as to produce synthetic log traces containing aberrations
which mimic those found in noisy empirical datasets. In all cases, behavioural
variations can be described as some alteration to a process and applied to
either a model or the product of that model (datasets or log traces) to
represent the same alteration introduced at an arbitrary point of the
simulation.

This decouples the expected behaviour in the original model from simulated
behaviour, which is obtained by composing the model and behavioural variation
using a given technique's method for doing so. This approach to modelling
behavioural variation allows the same altered behaviour, which would otherwise
be described in many disparate points in a model, to instead be written once and
introduced wherever required. The observation that the same variation might
appear in many areas of a model, and that the variation can be separated from
the model and introduced where necessary, frames the modelling of these
variations in the same way as aspect orientation frames cross-cutting concerns.
The work presented in this thesis explicitly applies changes to processes and
simulated behaviour as aspects in the same manner. Therefore, although this
aspectual connection is not made explicit in much of the literature to date, it
is important to review literature on simulation and modelling which modularises
these variations; this section reviews that literature. The work reviewed is
highly relevant to the contributions in this thesis, in particular because the
core motivations of this field are shared by this thesis; the section therefore
leads with a subsection discussing those motivations, and their relationship to
aspect orientation, in detail.


\labelledsubsec{Discussion of Variation \& Motivations for Variations in Process Models}{variation_sm_motivations}


\citeauthor{ExecutableBPMNMitsyuk} are motivated by the field of process
mining's requirement for datasets of process logs made from well-understood
process models, defined in a high-level manner~\cite{ExecutableBPMNMitsyuk}.
They demonstrate a technique for generating event logs from BPMN models by
introducing algorithms for the direct simulation of BPMN models and the
collection of traces from those simulations. While their approach does not
support the simulation of all BPMN concepts, notably message passing, they
provide a tool which produces log traces for a BPMN model through PROM, a
standard tool within the process mining community~\cite{van2005prom}. This
results in their technique providing high-level model simulation through
already-standard tooling, meaning adopters of the technique need not rely on
dedicated tooling which may not be compatible with other researchers' process
mining techniques.

The algorithms presented by \citeauthor{ExecutableBPMNMitsyuk} simulate
processes described by BPMN models, but don't include any provisions for
representing variance. However, the technique could plausibly be combined with
aspect orientation techniques for BPMN as discussed in
\cref{ao_and_bpm_review}~\cite{charfi2010AO4BPMN,Cappelli_AOBPM} to represent
alternate behaviour applied contingently. Demonstrating the viability of this
approach is an avenue of research beyond the scope of this thesis. However, the
motivation of the work mirrors that of other research projects reviewed in this
section: a need for synthetic datasets of traces through a process, for use in
scenarios where empirical datasets are difficult to obtain.

Difficulties arise when obtaining real-world datasets for many reasons. For
example, large empirical datasets are typically produced by organisations which
would prefer some level of secrecy around their operations, making publishing
those operations for the investigation of research teams unlikely. Researchers
collecting these datasets describe a \textquote{lengthy
process}~\cite{bpi_ten_years_of_datasets} and explain that traces of real-world
processes are hard to obtain because \textquote{higher management [can be]
worried about the risks} of publishing such
datasets~\cite{bpi_ten_years_of_datasets}. Another factor contributing to the
difficulty of collecting empirical datasets is that they often cannot be
collected, either because there is a need to study the process before
implementing it (making synthetic datasets the only option available to
researchers \inline{Find citation for empirical datasets being the only ones
available to researchers, maybe for disaster prevention or similar?}), because
the process is not yet fully understood (making simulation of many variants of
that process useful in aiding understanding~\inline{find a citation for
simulating different systems for finding an optimal one. Arguably Genetic
programming \& hill climbing do similar things?}), or because the dataset itself
is of use to researchers, not the real-world system that produced it (such as in
the case of evaluating process mining
techniques~\cite{van2004process,agrawal1998mining}). \inline{Find some nice way
to round this subsection out, or refactor out subsections}~\inline{Important
to acknowledge somewhere in this subsection that synthetic data generation is a
well-researched field, but that generating logs from simulations with variance
is the specific area relevant to this thesis}

\labelledsubsec{Representing Variations in Process Models and their Outputs}{variations_in_sm}

Research undertaken by \citeauthor{stocker2013secsy}~\cite{stocker2013secsy}
aims to synthesise process logs which are representative of attackers' efforts
to compromise the security of a modelled system. Their research project, named
``SecSY'', is an attempt to address issues arising from the difficulty of
retrieving representative log traces for security-critical systems in which
attacker activity is present. Logs are developed by process simulation through
``well-structured'' models, a mathematical property on which transformations
were previously defined by
\citeauthor{vanhatalo2009refined}~\cite{vanhatalo2009refined}. The authors
develop a tool for the simulation of a process using well-structured process
models, and apply transformations to both the model before execution and the log
it produces through the trace of a simulation. They conclude that their tool is
performant, and verify it can produce logs representing security violations by
way of analysis through PROM, a popular framework for process mining, and
pre-defined security constraints on their models. They note that log traces
cannot be interleaved (due to a lack of parallel simulation of processes), may
be incomplete (missing violations), and that mutated models and traces are not
guaranteed to be sound by construction. However, they see their proposal as a
necessary step in realistic data generation for business processes. A weakness
of the work is that model and trace modifications are relatively rudimentary:
processes can be added or removed, but complex graph transformations are
presumably only permissible when representable through the composition of the
mutation primitives they provide, on which there are only three for processes:
swapping \lstinline{AND} and \lstinline{XOR} definitions of process gateways, and swapping process
order. Mutations cannot be applied contingent on the state of a simulation run,
for example, representing a decision taken by an attacker based on what had
already happened. In later work, \citeauthor{stocker2014secsy} detail the
technical aspects of SecSY, their tool for implementing the generation of
synthetic logs which use their technique~\cite{stocker2013secsy} to represent
security violations security-critical business processes. A Java implementation
of SecSY is described, which simulates well-structured models and applies
mathematically-defined transformations on the model being simulated (before
simulation occurs) and the logs obtained through simulation traces. An
improvement on earlier work is that custom transformers can be written. However,
a limitation of the original work remains, which is that users cannot easily
dictate the degree to which variations are applied.


\citeauthor{pourmasoumi2015business} also address the need for access to
variations on business processes, though for the development of a research
field, ``cross-organisational process mining''~\cite{pourmasoumi2015business}.
Process mining can require many process logs, as does the benchmarking and
evaluation of process mining techniques. Traces from business processes which
are similar but not identical can produce log traces which reflect that
similarity, but also reflect the variations in different instances of those
processes. These log traces exhibiting variation can be used in the training and
analysis of process mining tooling and techniques, which must contend with
natural variation present in the execution of real-world traces. To support the
field, log trace generation from a variety of process models is therefore
required. Such logs are not in adequate supply, as explained in
\cref{subsec:variation_sm_motivations}. The authors' approach to the problem is
to present an algorithm for the mutation of business processes, such that
simulation against variations of the business process can produce process logs
reflecting those variations. Their algorithm makes use of structure tree
representations of processes, which models processes as trees and permits
conversion to BPMN models and Petri nets~\cite{buijs2014flexible}.
\citeauthor{pourmasoumi2015business} make use of this constraint to demonstrate
that their models are block-structured, a mathematical constraint on model
structure which 95\% of models are shown to comply with~\cite{chenthesis}. Their
contribution is a set of transformations on structure trees and block-structured
models, and an algorithm applying these transformations to process models, and a
tool which implements it built on PLG, a process log generation tool. They
conclude that tools such theirs can be used to generate log traces representing
process variation, in such a manner as to satisfy the requirements of the
process mining research community.

\citeauthor{pourmasoumi2015business} describe a list of transformations they
explain is \textquote{not intended to be
comprehensive}~\cite{pourmasoumi2015business}, which makes its full potential
unclear. Additionally, processes their tool applies to must be block-structured.
The importance of this requirement is that it limits their technique to business
process models. It is not demonstrated that models of processes in other domains
satisfy the condition, such as the flow of data~\cite{obashimethodology} or
behaviour of human or technical actors in \sociotechnical
systems~\cite{wallis2018caise}. Finally, the tool is limited by its lack of
capacity to represent variations which are applied contingent on system state. A
use case for modelling behavioural variance is to model changes which are
impossible to anticipate from the vantage of a modeller, such as a
\sociotechnical system's human actors' mistakes, security exceptions and
violations, or corrective actions to mitigate undesirable system state.
\citeauthor{pourmasoumi2015business}'s tool produces variations on a process
model, but modelling behaviour which is expected to vary within iterations of
the \emph{same} model is outwith the scope of their project.
\inline{Shugurov \& Mitsyuk's work should follow this. REVIEW IT.}


%% Shugurov & Mitsyuk here
% Mitsyuk_2016
% shugurov2014generation


\citeauthor{Machado_2011} note that there are operational costs to the
inefficient modelling of business processes. Specifically, they note that
processes can be replicated across automated processes, and failure to identify
such scenarios give rise to these operational costs. This work's core motivation
is that the representation of variation in process models would allow for the
capture of a replicated process once, with instances of similar processes
described as deviations from that captured blueprint. On this basis,
\citeauthor{Machado_2011} extend BPMN to support the notion of individual
processes as transformations of an underlying process, i.e. that a given process
can sometimes be expressed as a deviation from some pattern, and is therefore
define-able as the composition of that pattern and a variation upon it. Their
approach is illustrated through two small, broadly similar business processes
initially modelled in BPMN and then represented in Haskell, allowing the authors
to demonstrate their representation of variability as process deviation with
realistic examples. While the work presented makes no empirical evaluation of
their technique, \citeauthor{Machado_2011} note that their industrial partner
responded positively to the research presented in this publication and that
further technical improvements are to be made (support for around advice, and
for quantification). They also express an intent to conduct a real-world
evaluation in the HR domain. While we are unaware of any real-world evaluation
undertaken by this research time to date, some formal proofs that the
transformations their tool supports are always well-formed~\cite{machado2012formal}.


\labelledsec{Discussion}{lit_discussion}

% Points to make here:

% \begin{itemize}
% \item Aspect orientation allows for a clean separation of concerns. There has
% been a lot of promising research in the field. 
Aspect orientated programming is designed to permit highly modular software
engineering in scenarios where cross-cutting concerns are identified, by
isolating them as separate modules~\cite{kiczales1997aspect}. The aim in
employing aspect oriented programming is to reduce tangling, where a
cross-cutting concern is intertwined within a program's main concern, and
scattering, where the same cross-cutting concern is re-written at many points in
a program's source. Aspects which modularise that concern can be written once,
separately from a program's main concern, and re-introduced to each point in a
program to which the aspect relates by way of weaving. An aspect-orientation
framework must therefore be able to quantify the points at which the aspect
applies~\cite{filman2000aspect}. Aspects specify both advice (the implementation
of its associated cross-cutting concern) and the join point defining where in a
target program that advice should be woven. Aspect orientation therefore implies
that the source aspects are applied to are oblivious to their
application~\cite{filman2000aspect}.

% \item However, we lack empirical results which show the paradigm delivers what
% its original proponents claim.
In theory, the design of the paradigm is such that it should be expected to
increase modularity in the software applying
it~\cite{kiczales1997aspect,filman2000aspect}\inline{more citations}, and its
proponents often claim this modularity as a benefit of the aspect-oriented
approaches of their
research~\cite{gilani2004family,charfi2007ao4bpel,Cappelli_AOBPM,jalali2012aspect,chibani2013toward}.
However, its critics question the reasoning around these benefits, and note that
there is little empirical study into whether aspect oriented programs truly
benefit as a result of this
modularity~\cite{Constantinides04aopconsidered,steimann06paradoxical,przybylek2010wrong}.

% Lead into variance and why there's potential for aop to work, although we'd
% need some kind of empirical proof that it did, partly becaue of aop's track
% record and partly because we'd be automatically rewriting models so they might
% not reflect what their authors intended anymore if we're not careful.
One appropriate application area may be in the representation of behavioural
variation in simulation and modelling. The application of aspects to models is
already well-studied~\cite{DEVSaspectorientation2008aksu,chibani2013toward},
particularly within aspect-oriented business process
modelling~\cite{charfi2007ao4bpel,Cappelli_AOBPM,jalali2012aspect}, where
modelling behavioural variation has also seen some prior
research~\cite{Machado_2011,stocker2013secsy,pourmasoumi2015business,ExecutableBPMNMitsyuk}. 
Outside of business process modelling, aspect orientation would reasonably be
expected to support the development and observation of models
themselves~\cite{gulyas1999use}. 

Research opportunities at the intersection of aspect orientation and the
modelling behavioural variation occur because behavioural variation is an
example of a cross-cutting concern. Changes to expected behaviour such as
laziness, boredom, confusion or misunderstanding can impact many parts of a
process in a \sociotechnical system, but modelling the variation caused by any
one of these requires similar alterations to behaviour in many
disparate parts of the model they occur within. Behavioural variations are
therefore both scattered and tangled, and constitute a cross-cutting concern
which might be well suited to modularisation in aspects.

% NEXT: note the weaknesses in existing work and what needs to be done to show
% that AOP and behavioural variation in S&M do work well together.
Existing aspect orientation techniques and behavioural variation modelling
techniques are ill-equipped to take advantage of this alignment. That behaviour
changes when it varies is tautological; however, changes supported by existing
aspect orientation techniques weave advice before, after, or around their join
points, and therefore do not alter the definition itself. In some systems, some
variations may be representable as additions inserted before and/or after some
other behaviours, but such techniques are unsuitable for representing
modifications of the target behaviour itself, or behaviours which should be
omitted instead of added. Additionally, join points available to an aspectual
programmer may not be granular enough to permit representing the changes they
require in such as system, and aspect orientation's principle of obliviousness
opposes the modification of target code to make new join points available.
Techniques which would directly rewrite target source are typically extremely
low-level, and therefore ill-suited to most modelling
applications~\cite{keller1998binary}. Other techniques which permit defining
changes to processes at a high level may allow a modeller to \emph{describe}
intended changes (such as the high-level annotations supported in
AOPML~\cite{Cappelli_AOBPM}), but such techniques are intended for human
interpretation, not machine execution for simulation purposes. These techniques
permit describing behavioural variation within another process, but only by
virtue of the flexibility of natural-language annotation, and are therefore
unsuitable for simulation and modelling purposes.

Such techniques also lack executable notion of ``state''. Real-world behavioural
variance can often be contingent on the environment an actor exists within.
While variations such as security breaches might be predictable (by identifying
weaknesses in existing processes, for example), variance in \sociotechnical
systems often occurs in the behaviour of human actors. This might be in response
to a degraded mode~\cite{johnson2007degradedmodes}, where behaviour naturally
drifts to a functional --- but undesirable --- state, or due to an individual
making a mistake, forgetting procedure, or being in a state which alters their
behaviour, such as tiredness or drunkenness~\cite{aranTheatreThesis}. A
framework for modelling behavioural variation using aspects should therefore
apply that aspect to a simulated system contingent on the state of that system
at a given point in time. High-level modelling technologies such as BPMN and OPM
are executable~\cite{ExecutableBPMNMitsyuk,opm_original}, but it is not trivially
evident that executable versions of these technologies are compatible with
aspect-oriented modifications of their modelling
language~\cite{charfi2010AO4BPMN,Cappelli_AOBPM}. As noted, low-level program
transformation technologies are also unsuitable. Techniques for applying
variations to models exist, but are unsuitable for simulation (and therefore
cannot represent application based on temporal state)~\cite{stocker2013secsy},
produce individual models representing each possible instance of a
variation~\cite{pourmasoumi2015business}, do not support the dynamic weaving of
aspects for contingent behaviour~\cite{Machado_2011}, or attempt to represent
the changes one would expect in the output of a simulation by executing an
unmodified simulation and amending its output
directly~\cite{shugurov2014generation}. None of these techniques are suitable
for representing a behavioural change which is applied contingent on state.
Incidentally, these techniques for modelling behavioural variation also lack
support for the alteration of a process definition, or changes ``inside'' a
process definition, as discussed earlier, which also makes them unsuitable for
simulated behavioural variation.

Should a tool exist which dynamically weaves definitions of behavioural
variations for contingent application and which is capable of expressing changes
within a join point rather than before or after it, a challenge would arise as
to demonstrating the benefits the tool achieves. Contingent application of
behavioural variation, and the ability to define changes to processes
specifically, would fulfil the opportunity in marrying aspect-orientation and
modelling behavioural variation, but the benefit offered by such a tool is
unclear. The introduction of oblivious modification to a model may break its
representation of its real-world analogue, making such a model difficult to
reason about. Added to this is the complexity of the tool's capacity to rewrite
any join point's definition. Though aspect oriented literature often lacks case
studies demonstrating the benefits of the approach, it is particularly important
to investigate whether such a tool could produce realistic models, and whether
the expected benefits of aspect orientation as applied to the model hold in
practice. In particular, would engineers benefit from the modularity afforded
by isolating behavioural variation into an aspect, and would the resulting
aspectual modules be re-usable when modelling other systems?

\labelledsubsec{Research Questions}{rqs}

Not sure exactly how to write this. Eeep!


% \item Criticism of aspect orientation notes that there is no clear application
% area for the paradigm (and some areas where it _is_ applied seem unsuitable)
% \item One promising area of application is in modelling, as demonstrated by its
% use in business-process modelling literature.
% \item in addition, in business-process modelling literature we find research
% motivation for --- and research towards --- the modelling of \emph{variations}
% on behaviour. These behavioural variations appear well-suited to modelling as
% aspects, as they typically separate the variation from the domain model in which
% variation is observed, and a given variation might occur in many areas of a
% model. 
% \item However, projects modelling these variations do not support contingently
% applied behavioural variation.
% \item Also lacking in the literature is a method for applying transformations
% \emph{within} a process using aspects. Aspect orientation techniques apply
% alterations before, after, or around their targets.
% \item We therefore see a need for:
%     \begin{enumerate}
%         \item Behavioural variation modelled as aspects
%         \item Tooling supporting a more ``natural'' representation of variation,
%         i.e. a variation of a process applied to that process (rather than
%         additional behaviour appended to its start or finish)
%         \item Contingent behavioural variation, i.e. variations applied during
%         simulation execution
%         \item Empirical evidence that aspect oriented models:
%         \begin{enumerate}
%             \item Allow separation of behavioural variance into aspects
%             \item Allow new modelling techniques, such as
%             compositionally-defined hypothesised behaviour
%             \item Provide the typically defined benefits of aspect orientation,
%             i.e. reusable components in different modelling scenarios (i.e.
%             aspects are ``oblivious'' to the codebase they are applied to)
%         \end{enumerate}
%         \item In addition, we must be certain that a model with variations
%         applied still represents the originally specified model. That is to say,
%         we want to fulfil the opportunities observed in the literature, while
%         demonstrating that applying variation to a model in order to more
%         realistically represent its domain (a model of human behaviour including
%         varied behaviour the mistakes we'd expect to see exhibited by human
%         actors is arguably more ``realistic'' than one without those mistakes)
%         can be done in a ``safe'' way, i.e. modifying the model programmatically
%         does not change the model in a way which makes it \emph{less} reflective
%         of reality by mistake.
%     \end{enumerate}
% \end{itemize}

% The underlying hypothesis is:

% # Aspect orientation is well suited to modelling behavioural variations

% In order to investigate this hypothesis, we must investigate:

% ## Can we modify a model while keeping it at least as plausibly realistic as the
% original (if not more so)?
% ## Can aspect orientation provide the benefits claimed in the paradigm's
% literature in the simulation and modelling domain?

% To support those investigations, we need:

% ## A suitable system to model in which we can assert the ``realism'' of a model
% before and after the application of variance
% ## Aspect oriented tooling which supports application of behavioural variation
% ## Ideally also, aspect oriented tooling which addresses the shortcomings of
% other aspect orientation frameworks (no ``inside'' join point, engineering
% difficulties such as obliviousness' impact on legibility)



% Old research questions were:

% 1. Can we fit model details per-player to get realistic player behaviour?
% 2. Can we cluster based on accuracy of different models for each player?
% 3. Can we generate predictive data for unseen models?

% The contributions we want to demonstrate are:

% 1. Behavioural variation is well-represented by aspect orientation
% 2. Tooling which makes modelling behavioural variation feasible using aspect
%    orientation
% 3. Tooling which improves on aspect orientation's shortcomings (to make AOP
%     more viable to apply to s&m in practice)
% 4. Showing that data produced by aspects is model-specific 


% First draft new research questions:

% 1. Can we demonstrate that a model of human actors can be made realistic
%    through aspect oriented behavioural variation (rather than breaking the
%    model's representation of its domain)?
% 2. Can we demonstrate that our aspects represent real-world variation in
%    actors' behaviour, rather than the emergent change to an overall system?
%    (by clustering, we demonstrate that many people exhibit the _same_ changes,
%    but not everybody does, i.e. the aspect accurately represents a subset of
%    our modelled actors)
% 3. Can aspects applied successfully to one model be expected to see the same
%    success when applied to other models?

% Research questions found in an early .md file detailling the thesis structure:
% - RQ: Can we fit model details per-player to get realistic player behaviour?
% - RQ: Can we cluster based on accuracy of different models for each player?
% - RQ: Can we generate predictive data for unseen models?


%%%% OLD LIT REVIEW INTRO

% The work presented in this thesis revolves around the combination of simulation
% \& modelling and aspect orientation. As discussed, some research on the topic
% was also done prior to this body of work~\inline{I assume we discuss PDSF
% existing in the intro}. There are therefore several things to discuss: work that
% was done on this project prior to this thesis must be discussed so as to make
% clear what is \emph{new} in this thesis, related work by others should make
% clear the context in which this research was done, and opportunities found in
% this related work for which PyDySoFu is well-suited must be identified to set
% the scene for the rest of this thesis.
% 
% With this in mind, \cref{sec:pdsf_early_work} discusses the earlier work contextualising this
% thesis, summarised in \cite{wallis2018caise}. As PyDySofu has some unusual
% features as an aspect orientation library, \cref{sec:dynamic_aop_review} follows
% this with a review of similar approaches in the literature.
% \cref{sec:ao_and_modelling} and \cref{sec:ao_and_simulation} then review aspect
% orientation as applied in simulation and modelling respectively, and
% \cref{sec:dynamism_in_sm} reviews literature on variable behaviour in simulation
% and modelling, a key strength of PyDySoFu. Finally, research opportunities
% identified in these various bodies of work are discussed in
% \cref{sec:research_opportunities}, which pave the way for the research composing
% the body of this thesis.
% 
% 
% \labelledsec{Early work on PyDySoFu}{pdsf_early_work}
% Some notes on where PDSF was before the PHD work.

PyDySoFu\footnote{Or ``PDSF'' for short.} is a Python library~\cite{pdsf_repo}
built for making changes to the source code of a Python function as it is
called, and before it is executed, while the original function definition
remains oblivious to the changes being made. It was originally developed as an
honours-level dissertation, which was built upon and detailled in a subsequent
paper~\cite{wallis2018caise}. This thesis furthers that original work. To be
clear about the work this thesis contains, the state of the project
\emph{before} this work began is briefly discussed here.

\subsection{PyDySoFu's implementation and features}

The original version of PyDySoFu\footnote{Further improvements have been made
through this research which improve on the design, but this section is to
discuss the state of the project before this work began, and the general
principles around it, which remain unchanged.} patched Python classes with
additional handlers. Attributes of Python objects are usually retrieved using
dot notation (i.e. \lstinline{object_id.attr_id}), which evaluates internally to
a call to \lstinline{object_class.__getattribute__(``attr_id'')}. PyDySoFu replaces a
class' built-in \lstinline{__getattribute__()} method with a new one, which
calls the original to acquire the required attribute. 

In the case where the required attribute is not callable, the value is returned
as normal. Callable attributes are modified, however. In this case, the
replacement \lstinline{__getattribute__()} also checks for a set of
manipulations to make to the original code. These can be applied before or after
the original code is run, as well as around it. A new function is returned
containing a reference to the originally sought attribute, but which will search
for these additional pieces of work before executing it, and can execute this
work before or after the call (or both). These pieces of work are referred to as
``advice'', adopting aspect orientation terminology.

As discussed further in \cref{subsec:pdsf_aop}, this approach is effectively an
implementation of a traditional aspect orientation framework. However, unlike
existing frameworks, PyDySoFu also supports a special kind of ``around'' advice:
before a function is called, it can be rewritten. This is done by applying
``before'' advice which retrieves the abstract syntax tree of the target
callable attribute using Python's \lstinline{inspect} module (its built-in
reflection), applying arbitrary transformations to the tree, and recompiling it
into a Python \lstinline{code} object (its representation of its
internal bytecode). At this point, many things are possible: the transformation
can be cached for later use, can replace the original callable's
\lstinline{code} object to make the transformation persistent, or can be
discarded after use. This transformed code is run in lieu of the original,
effectively enabling aspect orientation which can make adaptations \emph{inside}
a procedure as well as before and after its execution.

This approach also had some limitations:

\begin{itemize}
    \item Traditional pointcuts cannot target points inside a procedure, meaning
    that an aspect applied ``inside'' its target must manage the points where
    its transformation is applied manually.
    \item Importantly, a callable object's internal bytecode cannot be replaced
    in Python3, leading to a rewrite discussed in \inline{Make a crossreference
    to the discussion of PDSF's rewrite.}
    \item This method is significantly slower than other aspect orientation
    approaches, as rewriting a class' \lstinline{__getattribute__} method means
    that \emph{every} resolution of an object's attributes --- whether they are
    methods or values, and including a class' built-in ``magic'' methods ---
    incurs an overhead from the replaced \lstinline{__getattribute__}
    implementation. However slight this overhead can be made, affecting Python's
    built-in methods on classes means that rewriting the
    \lstinline{__getattribute__} method is unavoidably expensive due to the
    scale of these methods' use.
\end{itemize}

However, the goal of the original research was to develop a flexible
``proof-of-concept'' of aspect orientation adapting procedure definition at
runtime, which was successfully
achieved\cite{wallis2018caise,wallis2018genetic}. 

\subsection{Aspect Orientation \& PyDySoFu}\label{subsec:pdsf_aop}

The goals of ``changing a function's behaviour'' and maintaining
``obliviousness'' in the original definition of that function speak to the goals
of the aspect oriented programming paradigm\cite{kiczales1997aspect}. Quoting
their original definitions:

\begin{displayquote}
    Components are properties of a system, for which the implementation can be
    cleanly encapsulated in a generalized procedure. Aspects are properties
    for which the implementation cannot be cleanly encapsulated in a
    generalized procedure. Aspects and cross-cut components cross-cut each other
    in a system’s implementation.
    [ \ldots{} ]
    The key difference between
    AOP and other approaches is that AOP provides component and aspect languages
    with different abstraction and composition mechanisms.
\end{displayquote}

Generally, aspect orientation is percieved to be a technique for separation of
concerns. Any cross-cutting concerns can be separated from their components into
aspects applied where that concern arises. The strength of aspect orientation
lies in its compositional nature: developers can write short, maintainable
implementations of a procedure's core purpose (for example, business logic) and
ancillary concerns such as logging or security can be woven into this
implementation as preprocessing, compilation, or at runtime. This compositional
nature is what gives rise to aspect orientation's ``obliviousness'', as the
procedure targetted by a piece of advice is written without regard to that fact.

The original PyDySoFu implementation was an aspect orientation library focusing
on separating a function's definition from \emph{potential changes to it}. This
was used to model ``contingent behaviour'' --- behaviour sensitive to some
condition --- as an original, ``idealised'' definition of that behaviour, plus
some possible alterations. These changes might apply to many different
behaviours in the same manner, and therefore represent concerns which separate
cleanly into an aspect. An example would be the behaviour of a worker whose job
requires focus on allocated tasks. A lack of focus could be represented as steps
of the worker's tasks being executed in duplicate, out-of-order, or skipped.
Assuming aspects as described by \citeauthor{kiczales1997aspect} are able to
edit the definition or execution of a procedure\footnote{As opposed to simply
wrapping it with additional behaviour before and/or after execution}, such
contingent behaviours are well modelled as aspects.

To achieve this, a model was presented in \cite{wallis2018caise} wherein aspects
were developed which could change function \emph{definitions} on each invocation
of that function, contingent on program state. This allowed behavioural
adaptation to be simulated in an aspect-oriented fashion. In addition, a library
of behavioural adaptations called FUZZI-MOSS\inline{CITECITECITE} was developed
which implemented many cross-cutting, contingent behaviours in procedural
simulations of \sociotechnical systems.

One important contribution of this work is that PDSF aspects are effectively
able to operate \emph{inside} a target. In typical aspect orientation frameworks
such as AspectJ~\cite{aspectj_intro}, aspects operate by effectively prepending
or appending work to a target, referred to as ``before'' or ``after'' pointcuts
respectively. To do both is referred to as ``around''. By manipulating
procedures within Python directly, PDSF is able to manipulate its target from a
new perspective, adding (or removing) work during the target's
execution\footnote{Similarly to \cref{subsec:bca}, but in an aspect oriented
manner.}. Moreover, because weaving is performed dynamically, every execution of
a function may perform different operations.

\subsection{Opportunities presented by PyDySoFu}

PyDySoFu presented several oppportunities for future research. Some salient
properties of the original work include:

\begin{itemize}
    \item It provided an aspect orientation library which could weave and
    unweave aspects during program execution, without relying on anything other
    than Python's built-in language features. As discussed in
    \cref{sec:dynamic_aop_review}, this is supported by some early aspect
    orientation frameworks also, but AspectJ dominates in the world of aspect
    orientation frameworks and does not support weaving during program execution.
    \item It provided the capacity to weave aspects \emph{inside} targets, as
    opposed to around them, or at either end of their execution. So far as we
    are aware, no aspect orientation framework in research or industry has
    offered this feature, and its applications and potential are yet to be
    explored.
    \item Relatedly, PyDySoFu was used in the context of simulating behaviour which may change
    over time. Contingent behaviour being a cross-cutting concern is an
    innovation of the early research which suggests aspect orientation may have
    strong applications in \sociotechnical simulation \& modelling.
\end{itemize}

\inline{Do we need a brief explainer of what aspect orientation is before
jumping into outside lit? Or will this go in the introduction? already a little
in the earlier litrev subsections.}

The amount of potential investigation which can be done into the dynamic weaving
of target-altering / ``inside'' aspects in simulation \& modelling applications
is vast. While literature on the complete topic is absent, each individual
component of this research angle is well-studied on its own. These opportunities
might be related to existing literature through the following questions:

\begin{itemize}
    \item How does PyDySoFu compare to existing aspect orientation frameworks,
    particularly those with a focus on dynamic weaving? Related frameworks are
    summarised and compared in \cref{sec:dynamic_aop_review}.

    \item What is the use of aspect orientation in simulation \& modelling? How
    does the approach taken in \pdsf's prior work relate to existing
    approaches? This will be discussed for simulation in
    \cref{sec:ao_and_simulation}, and for modelling in
    \cref{sec:ao_and_modelling}.
    
    \item Variability is important to capture in any \sociotechnical model or
    simulation. How is variability treated in existing literature, and how does
    this relate to \pdsf's approach? This will be explored in
    \cref{sec:dynamism_in_sm}.
\end{itemize}


% 
% \labelledsec{Dynamism in AOP}{dynamic_aop_review}
% Dynamic methods in aspect orientation

Aspect orientation frameworks have supported ``dynamic behaviour'' in different
ways for a long time. This is largely through a technique referred to as dynamic–
or runtime–weaving. 

\inline{Describe for each framework in this section how it compares to PDSF}

\subsection{Dynamic and static weaving}

Dynamic weaving integrates advice into a target program during its execution, as
opposed to during compilation or a pre-processing step. The advantage of this is
flexibility: dynamic aspect-oriented approaches have been proposed for deploying
hotfixes in safety-critical scenarios where software systems cannot be taken
offline to apply patches~\cite{Ionescu_2009}, and in adaptive mobile scenarios
where software may need to alter its properties in response to its
environment~\cite{hveding2005aspect}, or when debugging code to apply potential
patches without reloading an entire software system~\cite{popovici2002PROSE}.

To meet these needs, software systems need to check for available aspects to
weave at any join point, as it is always possible that the set of applied advice
has changed since the program last encountered this point. The technique
therefore presents a tradeoff compared to traditional (static) aspect weaving,
as illustrated in \cite{dynamicAOchitchyan}. \citeauthor{dynamicAOchitchyan}
generalise this tradeoff by describing different mechanisms used to implement
aspect orientation into three main categories\footnote{Drawing from
\cite{popovici2002PROSE,popovici2003JITaspects} where ``PROSE'', a particularly
influential dynamic aspect orientation library, is detailed.}, each with their own
strengths:

\begin{description}
    \item[``Total hook weaving''] alters all join points where advice may be
    applied before runtime, so that during execution each join point ``watches''
    for applied advice. The benefit of this approach is that aspects can be
    applied at any point at runtime, but this flexibility is bought at the cost
    of maximum overhead: at all points where weaving \emph{may} be possible,
    checks for applied advice must be made.
    \item[``Actual hook weaving''] weaves hooks only to join points that are
    expected to be in use. This limits overhead from watching for applied
    advice, at the cost of flexibility: during program execution, advice may be
    applied or retracted \emph{only at specific points within the system}.
    \item[``Collected weaving''] weaves aspects directly into code at
    compilation / preprocessing\inline{surely this isn't dynamic, Ian...?!}, so as
    to collect advice and target codebase into a single unit. This provides
    exactly the necessary amount of overhead, and in many cases may result in
    requiring no ``watching'' for applied advice at all, but this limits a
    developer's ability to amend advice supplied at runtime.
\end{description}

There is an almost direct tradeoff between the number of potential join points
actively checking for applied advice at runtime, and the overhead of dynamism in
any aspect oriented framework, with ``total hook weaving'' providing complete
adaptability at the expense of checking at all possible points whether advice is
applied.

Another tradeoff could be seen to be the clarity of dynamically woven aspect
oriented code. Aspect orientation is already criticised for the lack of clarity
as to what woven code will \emph{do} when run, and where weaving can change
during program execution, static tools are less useful in making these
predictions. Some tools have been produced which do provide tooling for
achieving understanding as to what dynamically woven code will do when executed
(also called an ``Aspect Monitor'', as discussed in \cite{popovici2002PROSE}),
but they are often limited or missing from a dynamically weaving framework's
implementation (such as \cite{Baker_2002}). \inline{Find more citations for both
dynamic weavers with aspect monitors and without. Nanning aspects? Nu?}


% Dynamic AO libraries
\labelledsubsec{PROSE}{subsec:PROSE} One implementation of dynamic weaving is
PROSE\cite{popovici2002PROSE,popovici2003JITaspects}, a library which achieves
dynamic weaving by use of a Just-In-Time compiler for Java. The authors saw
aspect orientation as a solution to software's increasing need for adaptivity:
mobile devices, for example, could enable a required feature by applying an
aspect as a kind of ``hotfix'', thereby adapting over time to a user's needs.
Other uses of dynamic aspect orientation they identify are in the process of
software development: as aspects are applied to a compiled, live product, the
join points being used can be inspected by a developer to see whether the
pointcut used is correct. If not, a developer could use dynamic weaving to
remove a mis-applied aspect, rewrite the pointcut, and weave again without
recompiling and relaunching their project.

Indeed, the conclusion \citeauthor{popovici2003JITaspects} provide in
\cite{popovici2003JITaspects} indicates that the performance issues generalised
by \citeauthor{dynamicAOchitchyan} in \cite{dynamicAOchitchyan} may prevent
dynamic aspect orientation from being useful in production software, but that
it presented opportunities in a prototyping or debugging context.

PROSE explores dynamic weaving as it could apply in a development context, but
the authors do not appear to have investigated dynamic weaving as it could apply
to simulation contexts, or others where software making use of aspects does not
constitute a \emph{product}.


\subsection{Handi-Wrap}
Handi-Wrap\cite{Baker_2002} is a Java library allowing for dynamic weaving via a
third-party language designed for metaprogramming, called Maya\inline{Do I want a
citation for this? Probably not, but worth revisiting.}. At the time of
development Handi-Wrap's dynamic aspect weaving feature was novel: the aspect
orientation library of note, AspectJ, wove only statically\footnote{AspectJ now
supports what it calls ``load-time weaving'' --- that is, weaving aspects as
classes are loaded into the JVM --- but not weaving to things that are
\emph{already} loaded, meaning AspectJ still allows for only a particular
flavour of dynamic behaviour.}, and Handi-Wrap's purpose was to show that DSLs
for metaprogramming could pave a way to dynamic weaving.

\citeauthor{Baker_2002} implemented an aspect orientation framework which is
reasonably performant, weaves dynamically, and allows for aspect orientation
features to be implemented natively for greater control as compared to
Handi-Wrap's then competitor, AspectJ. As a tool, Handi-Wrap demonstrated a
promising approach to dynamic weaving, but the project appears to have enjoyed
less attention than similar work (such as PROSE, described in
\cref{subsec:PROSE}).

The technique used to implement Handi-Wrap (implementation via a
metaprogramming-specific DSL, Maya) is familiar, in that it shares a perspective
on dynamic weaving with early PyDySoFu work. The fuzzers used in
\cite{wallis2018caise} applied transformations to abstract syntax trees, not
unlike a LISP-style macro. To quote \cite{baker2002maya} by way of contrast:
``{\ttfamily{}Maya generalizes macro systems by treating grammar productions as
generic functions\ldots{}}''.\inline{Revisit this inline quote format} The two
approaches have clear differences. Most notably, PyDySoFu's entire
implementation \emph{and use} is performed in Python directly, and Maya's
intended purpose is metaprogramming in a more general sense. It is possible
that, while Maya provided a useful foundation to explore the dynamic weaving of
aspects, its lack of adoption as a language limited handi-wrap's reach;
nevertheless, it is encouraging to see another use of metaprogramming for
weaving aspects at runtime.

\subsection{Nu}
\inline{Some extra things here about Nu, such as \cite{dyerNUmasters}.}

Nu is an aspect orientation framework written in Java which achieves dynamic
weaving by way of the Nu virtual machine~\cite{dyerNUmasters}. This introduced
new primitives in Java for the application and removal of aspects:
\lstinline{BIND} and \lstinline{REMOVE}.\inline{Write more NU writeup — requires
more citations etc.}

% Alternative techniques
\subsection{Binary Component Adaptation}\label{subsec:BCA} Binary Component
Adaptation\cite{keller1998binary} (BCA) is a technique for performing
adaptations on software components after compilation. Though it works on
already-compiled code it does provide dynamic behaviour: the technique can adapt
software components via rewriting before or during the loading of its target.
Like some aspect orientation techniques\inline{which?!}, BCA adapts a Java class
loader to make its adaptations, but unlike aspect oriented approaches it does
not require access to the original source of the software. For scientific
simulation purposes, it could therefore be appealing in situations where
adaptations are made to another researcher's simulations --- assuming the
original source code is not published --- or in security settings investigating
trust in compilers and runtimes\cite{trustingtrust}. In the present context of
developing \sociotechnical simulations however, this does not appear to be an
advantage, particularly at a time when the source code of software components of
research projects are increasingly published.

An important distinction to be made is that BCA provides an example of runtime
adaptation, but does not enable an aspect oriented approach and is not developed
with separation of concerns in mind. It is presented here as a useful contrast
to PyDySoFu: it demonstrates an alternative technique for achieving dynamic
runtime source manipulation, even if the lack of separation of concerns means it
would not be well applied for this thesis.




% check Kell survey for more, I believe there's juicy stuff there, maybe in
% section 2



% 
% \section{Aspect Orientation in Simulation \& Modelling}\label{ao_and_modelling}

% Aspect orientation as it applies to modelling

Having discussed aspect orientation as it is used in a simulation context, it is
natural to investigate its use in modelling research, too.

Simulation and modelling are similar topics and are often combined into a single
study. However, their goals differ. Simulation typically involves the study of
processes or behaviour: there is an expectation that simulations are
\emph{executed} or \emph{run}. This often produces data. The intent of modelling
is more structural in nature: models are typically observed or analysed to gain
insights. Quoting \citeauthor{smintro}'s introduction~\cite{smintro}:

\begin{displayquote}
    Modeling \emph{[sic]} is the process of producing a model; a model
    is a representation of the construction and working of
    some system of interest. A model is similar to but
    simpler than the system it represents.
    \newline{}
    [ \ldots{} ]
    \newline{}
    A simulation of a system is the operation of a model of the system. The
    model can be reconfigured and experimented with; usually, this is
    impossible, too expensive or impractical to do in the system it represents.
    The operation of the model can be studied, and hence, properties concerning
    the behavior of the actual system or its subsystem can be inferred.
\end{displayquote}

\citeauthor{smintro}'s definition implies that to simulate is to operate a
model. Whether this model is constructed for the purpose of simulation or for
study in its own right, a simplified representation of the system being studied
is implicitly required for any simulation. However, modelling does not imply
simulation. Models can be studied for their own merits, and many modelling
frameworks exist which are made explicitly for their own study, without regard
to their use in simulation\footnote{Consider UML, a well-studied modelling
framework which is generally not used for any kind of simulation --- depending
on the use case, it often cannot be --- and for which many alternatives now
exist specifically to address this
limitation~\cite{opm_original,ExecutableBPMNMitsyuk,obashimethodology}.}. Aspect
orientation has seen some study in modelling, particularly for \sociotechnical
modelling, and while aspect-oriented \sociotechnical modelling is not generally
researched with subsequent simulation in mind, an important body of work is
still present, and therefore important to discuss.

\subsection{Aspect Orientation in Business Process Modelling}
Aspect orientation for \sociotechnical systems is particularly well studied in
the business process modelling
community\cite{Machado_2011,Cappelli_AOBPM}~\inline{find more citations for AOBPM}


\subsection{MAML \& SWARM}
\inline{Is MAML/SWARM really modelling, or simulation? Simulation, right?}


\subsection{BPMN \& aspect orientation}
\inline{There's tons here. Go through the remarkable read folder. Particularly
anything Claudia Capelli's worked on, see
\cite{da2020implementation,Cappelli_AOBPM}. Also the widely cited work on aspect
orientation in BPEL\cite{charfi2007ao4bpel}, and the work on precedence of
aspect application in \cite{jalali2012aspect}, }


% \section{Aspect Orientation \& Simulation}\label{sec:ao_and_simulation}
\labelledsec{Aspect Orientation \& Simulation}{ao_and_simulation}

\inline{The simulation section \emph{badly} needs revisiting.}

Surprisingly, little literature exists pertaining specifically to the use of
aspect-orientation in a simulation context. Aspect orientation is often applied
to modelling as discussed in \cref{ao_and_modelling}, used to compose a
perspective of the world from individual parts, but in a way which isn't
necessarily executable or able to produce data.

Early in the history of aspect orientation as an emerging paradigm, there was
some interest in its use for scientific simulation. \cite{gulyas1999use} discuss
that computer simulations require code for both observation of a simulation and
the simulation itself, and that misuse of this could cause what is in effect a
kind of Hawthorne Effect\inline{Does hawthorne effect need a citation?}, where
the inclusion of observation code intertwined with simulation code might
influence the outcome of an experiment. They suggest that improving simulation
technologies could combat this approach. Aspect Orientation, being developed
specifically with obliviousness in mind, is an ideal candidate which
\citeauthor{gulyas1999use} identify.

Much of the literature concerning aspect-oriented programming and simulation
focuses on tooling support for aspect-oriented simulation, rather than
investigations into its efficacy. For example, attempts have been made to
integrate aspect orientation into new
tools~\cite{DEVSaspectorientation2008aksu,ribault2008OSA,ribault2010osif,} or
into existing
ones~\cite{chibani2019using,DEVSaspectorientation2008aksu,wallis2018caise}.
Typically, these papers identify a need for aspect orientation in simulation
frameworks --- the argument often revolves around a need for increased
modularity, and occasionally around better structuring of the simulations
themselves\footnote{See \citeauthor{chibani2014practical}'s series of papers on
the topic~\cite{chibani2013toward,chibani2014practical,chibani2019using}, which
culminate in an implementation of a suite of aspects for simulation purposes.
Unfortunately, the work still does not produce significant case studies showing
the benefits of the technique in practice. Some empirical measurements are made.
A lack of significant real-world evidence that the technique works is a major
criticism of aspect orientation as a paradigm~\cite{steimann06paradoxical},
however, and the use of these empirical measurements designed for different
paradigms as a sign of success hints that satisfactory results in the
measurement are being treated as more important than empirical effectiveness, an
instance of Goodhart's Law: ``when a measure becomes a target, it ceases to be a
good measure''\cite{strathern1997improving}.\inline{refactor this footnote into
its own paragraph underneath its currently enclosing one.}} --- but past a small example to
demonstrate how the tooling can be used, little additional development is
performed. No significant case studies or refactoring of existing codebases of
notable scale are provided. This is important because aspect orientation's main
strength is pragmatic in nature. If no real-world testing is conducted, it is
hard to conclude that the community's suite of modelling tools contribute
anything useful when developing simulations.

Some experiments specifically using aspect orientation in the implementation of
process-based simulations also exist\cite{Ionescu_2009}~\inline{include more!}.
For example, \citeauthor{Ionescu_2009} apply aspect orientation in a nuclear
disaster prevention simulation. Their motivation is that code can become complex
to maintain over time and changes to the scientific zeitgeist or to regulatory
requirements leads to costly technical debt. Aspect orientation therefore allows
developers to separate functionality into distinct modules more easily, without
disturbing the underlying codebase.

\subsection{Aspect-oriented L-Systems}
Aspect-orientation is also applied in other simulation paradigms.
\citeauthor{Cieslak_2011} investigated the use of aspect orientation in L-system
based simulations~\cite{Cieslak_2011}. An L-system\cite{lindenmayer1968lsystem}
is defined by a set of symbols, an initial string composed of these symbols, and
a set of rules for rewriting substrings. While being a powerful tool for
representing fractal structures, they were originally conceived of for plant
modelling (and still see the most use in this field).

\citeauthor{Cieslak_2011} note that some details of plant modelling are actually
cross-cutting concerns against many plants or families of plants. To represent
these, they introduce a new language to describe plant models which makes use of
aspect orientation to represent these cross-cutting concerns. They test the
approach by representing carbon dynamics, apical dominance and biomechanics as
cross-cutting concerns that are integrated into a previously published model of
kiwifruit shoot development. \citeauthor{Cieslak_2011} hope that these
cross-cutting concerns might work in other models too, but this is untested. The
use of an aspect in a new model, when developed for another, seems untested in
the community's literature writ large and is a noted omission in the conclusion
of this particular work.



\subsection{AOP and simulation tooling}

\inline{Here, I should be discussing: japrosim work by chibani
\cite{chibani2019using,chibani2014practical,chibani2013toward}; OSIF \& OSA in
\cite{ribault2008OSA,ribault2010osif}; 
}

% 
% \section{Variable Behaviour in Simulation \& Modelling}\label{sec:dynamism_in_sm}
% Dynamic / contingent behaviours in simulation and models
% IMO this is where a lot of stuff like variance in process mining, sanitisation
% of logs, log noise removal / injection goes.

In \cite{wallis2018caise}, PyDySoFu was used to model behaviour that changed as
the simulation progressed. Behaviour undergoing variance appears in literature
from many fields, but some themes stand out. Researchers are often interested
in:

\begin{itemize}
    \item Removing small variations from datasets in order to mine the original
    process (that real-world actors might be deviating from), referred to as
    sanitisation,
    \item Inserting variations so as to produce datasets with 
\end{itemize}

\subsection{Business Process Modelling \& variation in behaviour}
% process mining for processes which exhibit natural variations
\inline{Is this going from data to a model? Models to data? Potentially multiple models either way depending on their use? Consider this and possibly restructure.}

In real-world business processes, natural variation is difficult to avoid. This
is because business processes are inherently \sociotechnical, and so can be
expected to exhibit at least slight variations due to the mistakes of human
actors executing those processes. Variations can effectively take two forms:

\begin{enumerate}
    \item Some variations are expected, where predictable shifts in behaviour
    emerge over time. Examples would be habits forming which deviate from
    prescribed processes, skipped steps, or paths of a fork in a process
    becoming effectively ignored as others become the default (essentially
    producing redundancy in the model).
    \item Unexpected variations can occur if an actor behaves erratically,
    information is improperly recorded in a log (and so \emph{seems} to exhibit
    variance), or if some accident occurs. This appears as random noise in
    collected data, and is difficult to statically embed in any model, as change
    might take myriad forms and occur at an arbitrary number of points in the
    process.
\end{enumerate}

As these two forms of variance must be modelled differently, they are treated
differently in a business process model exhibiting variance. Typically the
second is treated as noise: undesired and a distraction from a model built to
reflect a prescribed process. They are therefore removed via sanitisation, and
are discussed in \cref{subsec:review_synthetic_datasets}. The first, variation
which might be reflective of a model as it can be expected to be \emph{executed}
--- even when this was not an intended or prescribed version --- might be
interesting to modellers. These situations might arise, for example, where
sociotechnical variance within the context of the broader system is the specific
subject of investigation. Degraded modes in these systems are a good example of
this~\cite{johnson2007degradedmodes}.


\labelledsubsec{Variations in Process Models}{bpm_variation}
Discussing 


\subsection{Process mining \& variation in data}
% working with noisy data: sanitisation / coping strategies
\inline{Write a short subsection on the trend of santising data / coping with noisy datasets. Sometimes behavioural variance isn't desirable.}
\inline{Consider making noisy data a subsubsection of \cref{subsec:review_synthetic_datasets}.}


\inline{So, there's some stuff to cite on sanitisation and mining in the
    presence of noise --- see On process model synthesis based on event logs
    with noise, \cite{Mitsyuk_2016}.}

    \inline{Add more stuff to be cited here, at the very least...}

Process Mining is a field which necessarily deals with erroneous data. As
processes are identified within event logs sourced from real systems,
inconsistencies in data collected or execution of a prescribed process results
in data fed to a mining algorithm which is at best not indicative of the desired
result, and at worst indicative of a different one altogether. As a result,
variation in process logs is a subject of active research in the community.

There are two main research efforts involving mining on log data with variance:

\begin{enumerate}
    \item Some researchers look to minimise the impact of log variance on the
    outcome of process mining. This can be done through the development of
    mining algorithms which are able to cope somewhat with variance. Many
    algorithms attempting to solve this problem have been developed, but their
    effectiveness depends on the kind of variance present and the degree to
    which those different variances are expressed in the
    data\cite{Mitsyuk_2016}.
    \item Other researchers look to identify noise in event logs before they are
    mined, processing them to eliminate any variance before mining begins. This
    requires classification of noise and the removal of suspect
    traces\cite{Cheng2015logsanitization}.
\end{enumerate}

Another perspective on the problem is that noise cannot be successfully
eliminated, but that training on empirical noise limits a researcher's control
over an experiment. The argument here is usually along the lines that empirical
noise is effectively impossible to predict, exert control over, or classify
entirely, so any testing of tools using that data is flawed. Without knowing an
algorithm's response to specific kinds of variance, a researcher can't compare
one approach properly or reproducibly against another. It is therefore important
to \emph{produce logs with controlled kinds of variance}, so as to create a kind
of synthetic workspace where algorithms are tested against synthetic data with
known kinds and degrees of variance. Once they are reproducibly tested against
known good datasets, they can undergo empirical verification by using data
captured ``in the wild''.

Naturally, similar approaches exist outside of process mining, as the
requirement for synthetic data is a common one. In potentially sensitive data
collected on the public --- census or health data for example --- there may be a
need to publish data which is at least partially
synthetic~\cite{little1993statistical,Drechsler_2011,rubin1993discussion}.
\citeauthor{Drechsler_2011} presents an array of simple statistical methods for
producing this~\cite{Drechsler_2011}. \citeauthor{koenecke2020syntheticgenecon}
note that, depending on the nature of the data \emph{needed} for a given
application, different methods are appropriate, meaning a variety of techniques
are required~\cite{koenecke2020syntheticgenecon}, and give an overview of
methods suitable in economics. MetaSim\cite{kar2019metasim} produces data using
probabilistic grammars for training neural nets in a manner naturally resilient
to variance by including an appropriate amount via the trained grammar, thereby
injecting a guaranteed correct degree of noise. Admittedly neural nets are a
common source of synthetic data in modern literature and a research subject with
a growing need for training data, perhaps best exemplified in the community's
production of another kind of neural net specifically for this purpose:
Generative Adversarial Nets, or GANs~\cite{goodfellow2014generative}.

Approaches specific to the generation of synthetic event logs are also
abundant\cite{stocker2013secsy, pourmasoumi2015business, Loreti_2019,
Yousfi_2015, ExecutableBPMNMitsyuk}, and as PyDySoFu's original use was in
\sociotechnical modelling, this is our primary interest. \inline{Shugurov paper
from 2014 should be checked for citing here too. There's a useful summary in On
Business Process Variants Generation (Pourmasoumi 2015).}



% \inline{OK! So now we need to go through the lit, in particular the five
% articles cited above. Some notes below too.}

In \cite{stocker2013secsy, stocker2014secsy} \citeauthor{stocker2013secsy}
describe a method for injecting variance into synthetic event logs (``traces'').
In \cite{stocker2013secsy}, a method is described whereby security-specific
alterations to traces can be made which represent the behaviour of an attacker
in some \sociotechnical system. Variance can be injected by statically
manipulating a process before simulating it to generate traces or making
modifications to traces after simulation. A supporting tool, ``Secsy'', is
provided in \cite{stocker2014secsy}.

A similar approach approach is provided in \cite{pourmasoumi2015business}, where
alterations are made to a process before simulation occurs. In terms of
alterations made to the model directly (and not produced traces), Secsy only
supports a limited number of operations on a model: transformation of
\texttt{AND} and \texttt{OR} gateways to the alternative kind, and swapping the
ordering of modelled activities. The method proposed in
\cite{pourmasoumi2015business} is able to make use of a much broader gamut of
alterations, by limiting themselves to mutating only block-structured processes
which they represent as ``structure trees''. Working with a tree-like structure
allows for edits to be made which preserve the model's validity, and a table of
ten potential --- reportedly non-exhaustive --- modifications to a model are
suggested, far more than suggested in the various works on Secsy.


The authors claim that the limitation of requiring block-structured models does
not impact the broad applicability of their approach, as \citeauthor{chenthesis}
claims that around 95\% of BPMN models can be represented this
way~\cite{chenthesis}. However, that claim should be held with some scepticism.
The citations for this claim are \cite{Thom2009ActivityPI} and a paper by
\citeauthor{polyvyanny2010structuring} which is most likely
\cite{polyvyanny2010structuring}\footnote{The citation indicates a paper
presented a year earlier than \cite{polyvyanny2010structuring}, and the author
has given talks with the same title and published other works with similar
titles --- although it is possible a paper with the same name was published a
year earlier, and this could change \citeauthor{chenthesis}'s claim, some
confidence can be had that this is a simple referencing error or typo.}. The
first work checks 214 process models against a set of patterns which are
specifically not formalised, and the second presents some formal work on the
translation of process models following the block structure relied on in
\cite{pourmasoumi2015business} and \cite{chenthesis}. However, the two works
never cite each other, the application of the formal translations to the
patterns presented is non-trivial, and no further explanation as to the
application required appears to be presented in the thesis. One could suppose
that the translation of the patterns to block-structuring could be automated by
an implementation of the theory presented in \cite{polyvyanny2010structuring}.
In any case, \citeauthor{chenthesis} notes in \cite{chenthesis} that the
requirement of block-structuring on a process model is a limiting factor in the
application of their own work in their conclusions, and so the broad
applicability claimed in \cite{pourmasoumi2015business} should be taken with
healthsome caution. After some exhaustive citation reading we can conclude that
neither approach supports effective production of synthetic event logs
exhibiting a wide gamut of variances by statically manipulating a BPMN model
prior to simulation, although the existence of both methods suggests it would be
a valuable research outcome.


% ===== better applied to synthetic data sans variance =====
% In \cite{ExecutableBPMNMitsyuk}, \citeauthor{ExecutableBPMNMitsyuk} provide a
% more recent example of an argument for synthetic traces pertaining to 

\inline{In \cite{Loreti_2019}, we can find good things I should write about.}


\inline{In \cite{Yousfi_2015}, we can find good things I should write about.
However after going through it it's clearly more variability in \emph{models},
not variability in \emph{data}. Belongs in another subsection.}


\begin{enumerate}
    \item \cite{Yousfi_2015} --- unread, v interesting
    \item \cite{stocker2013secsy} for secsy, and \cite{stocker2014secsy}, the associated
    tooling paper
    \item \cite{pourmasoumi2015business} generates synthetic logs with variance,
    just like secsy, but instead of making edits to the process before
    simulation using a ``structure tree'' representation and identifying points
    suitable for mutation.
    \item \cite{Loreti_2019} --- unread, v interesting
    \item \cite{ExecutableBPMNMitsyuk} --- Aalst generating logs from models. No
    variance but they make the case that synthetic data is needed by the
    community \emph{and} it's a big name taking a swing, too. Could combine well
    with \cite{pourmasoumi2015business} to get variance without actually
    producing new techniques, assuming a limitation of the sim approach to
    block-structured models (which I think they already impose anyway…)
    % \item \cite{kar2019metasim} where probabilistic grammars are used to generate training
    % data for neural nets. Not directly related but an example of how broad
    % synthetic generation is (and a good case for there being a broad requirement
    % in research and industry for synthetic datasets too). Their approach
    % naturally introduces plenty of noise.
\end{enumerate}

% 
% % \section{Aspect Orientation \& Modelling}
% 
% 
% 
% 
% \labelledsec{Research Opportunities in the Literature}{research_opportunities}
% % putting these all together. There's lit in each section, but specifically AO
% % for representing cross-cutting behaviour, behaviour with variance, or
% % simulations produced by composition of speculative parts aren't well studied
% % on their own and warrant further study. It just so happens that PDSF is
% % well-positioned to fill this gap, as it has the necessary properties of a tool
% % to perform this research, where something like AspectJ [likely...?] falls
% % short.
% 
% One notable omission from the set of research themes outlined in
% \cref{sec:dynamism_in_sm} is that variations on processes aren't well
% studied in their own right. That is to say, behavioural variation is typically
% treated as a nuisance to the researcher or practicioner interested in a model or
% dataset, and the variations and their impact on simulations are not studied on
% their own. Some work exists\inline{cite work that focuses on variations
% specifically}, but the majority of this is done in the context of tooling, i.e.
% the representation of variation for their use in another research context, where
% they are not the subject.
% 
% As an aspect orientation framework capable of runtime adaptation of a target
% system which can manipulate a join point from inside \inline{what are our
% research opportunities given the above...?}