\chapter{Relevant Literature}

% Here's where the lit review goes.

% \todo{Decide on section ordering}
% \todo{Longlist papers to be included from devonthink}


% \section{Aspect Orientation}

% \subsection{AOP Techniques}
% Aspect orientation is an engineering technique designed to separate concerns
% in code, primarily in scenarios where the same concern appears in more than one
% place. It 


% \subsection{Aspect Orientation as applied to Modelling \& Simulation}


% AOP and BPMN is the big one here!

% SWARM \& MAML for the multi-agent approach

% Check the lit again in the JAPROSIM paper

% \section{Modelling and Simulation}

% \subsection{General Simulation Techniques}

% \subsection{Sociotechnical Modelling of Behaviour}




% \section{Aspect Orientation as applied to Simulation}

% Surprisingly little literature exists pertaining to aspect-oriented simulation.
% Aspect orientation is often applied in the context of \emph{modelling},
% composing together a perspective of the world which isn't necessarily executable
% or able to produce data.

% Early in the history of aspect orientation as an emerging paradigm, there was
% some interest in its use for scientific simulation. \cite{gulyas1999use} discuss
% that computer simulations require code for both observation of a simulation and
% the simulation itself, and that misuse of this could cause what is in effect a
% kind of Hawthorne Effect\todo{does this need a citation?}, where the inclusion
% of observation code intertwined with simuilation code might influence the
% outcome of an experiment. They suggest that improving simulation technologies
% could combat this approach. Aspect Orientation, being developed specifically
% with obliviousness in mind, is an ideal candidate which \citeauthor{gulyas1999use}
% identify.

% Much of the literature concerning aspect-oriented programming and simulation
% focus on tooling support for aspect-oriented simulation, rather than
% investigations into its efficacy. For example, attempts have been made to
% integrate aspect orientation into new
% tools~\cite{DEVSaspectorientation2008aksu}~\todo{throw more in here from excel},
% or into existing ones~\cite{chibani2019using}\todo{throw more in here from
% excel}. \todo{Fill out more}

% Some experiments specifically using aspect orientation in the implementation of
% process-based simulations also exist\cite{Ionescu_2009}~\todo{include more!}. For
% example, \citeauthor{Ionescu_2009} apply aspect orientation in a nuclear
% disaster prevention simulation. Their motivation is that code can become complex
% to maintain over time and changes to the scientific zeitgeist or to regulatory
% requirements become costly as technical debt mounts. Aspect orientation
% therefore allows developers to separate functionality into distinct modules more
% easily, without disturbing the underlying codebase.




% \section{Realism in Models} % Consider removing
% % Probably mostly stats?




% \section{Research Opportunities}



% \todo{Determine other sections that can be included here...!}



% === PROPOSED NEW STRUCTURE
% === It relies on the idea that the introduction tells the story leading up to
%     the PHD research i.e. the development and ideas behind the PDSF originally


Some notes here on what this chapter contains, and a brief description of its
layout. Two snappy paras.
\todo{Some notes here on what this chapter contains, and a brief description of its
layout. Two snappy paras.}



\section{Early work on PyDySoFu}
% Some notes on where PDSF was before the PHD work.

PyDySoFu\footnote{Or ``PDSF'' for short.} is a Python library~\cref{pdsf_repo}
built for altering the source of a Python function as it is called, and before
it is executed, while the original function definition remains oblivious to the
changes being made. It was originally developed as an honours-level
dissertation, which was built upon and detailled in a subsequent
paper~\cref{wallis2018caise}. This thesis furthers that original work. To be
clear about the work this thesis contains, the state of the project
\emph{before} this work began is briefly discussed here.

The goals of ``changing a function's behaviour'' and maintaining
``obliviousness'' in the original definition of that function speak to the goals
of the aspect oriented programming paradigm\cref{kiczales1997aspect}. Quoting
their original definitions:

\begin{displayquote}
    Components are properties of a system, for which the implementation can be
    cleanly encapsulated in a generalized procedure. Aspects are properties
    for which the implementation cannot be cleanly encapsulated in a
    generalized procedure. Aspects and cross-cut components cross-cut each other
    in a system’s implementation.
    [ \ldots{} ]
    The key difference between
    AOP and other approaches is that AOP provides component and aspect languages
    with different abstraction and composition mechanisms.
\end{displayquote}

Generally, aspect orientation is percieved to be a technique for separation of
concerns. Any cross-cutting concerns can be separated from their components into
aspects applied where that concern arises. The strength of aspect orientation
lies in its compositional nature: developers can write short, maintainable
implementations of a procedure's core purpose (for example, business logic) and
ancillary concerns such as logging or security can be woven into this
implementation as preprocessing, compilation, or at runtime. This compositional
nature is what gives rise to aspect orientation's ``obliviousness'', as the
procedure targetted by a piece of advice is written without regard to that fact.

The original PyDySoFu implementation was an aspect orientation library focusing
on separating a function's definition from \emph{potential changes to it}. This
was used to model ``contingent behaviour'' --- behaviour sensitive to some
condition --- as an original, ``idealised'' definition of that behaviour, plus
some possible alterations. These changes might apply to many different
behaviours in the same manner, and therefore represent concerns which separate
cleanly into an aspect. An example would be the behaviour of a worker whose job
requires focus on allocated tasks. A lack of focus could be represented as steps
of the worker's tasks being executed in duplicate, out-of-order, or skipped.
Assuming aspects as described by \citeauthor{kiczales1997aspect} are able to
edit the definition or execution of a procedure\footnote{As opposed to simply
wrapping it with additional behaviour before and/or after execution}, such
contingent behaviours are well modelled as aspects.

To achieve this, a model was presented in \cref{wallis2018caise} wherein aspects
were developed which could change function \emph{definitions} on each invocation
of that function, contingent on program state. This allowed behavioural
adaptation to be simulated in an aspect-oriented fashion. In addition, a library
of behavioural adaptations called FUZZI-MOSS\todo{CITECITECITE} was developed
which implemented many cross-cutting, contingent behaviours in procedural
simulations of sociotechnical systems.

One important contribution of this work is that PDSF aspects are effectively
able to operate \emph{inside} a target. In typical aspect orientation frameworks
such as AspectJ\cref{aspectj_intro}, aspects operate by effectively prepending
or appending work to a target, referred to as ``before'' or ``after'' pointcuts
respectively. To do both is referred to as ``around''. By manipulating
procedures within Python directly, PDSF is able to manipulate its target from a
new perspective, adding (or removing) work during the target's
execution\footnote{Similarly to \cref{subsec:bca}, but in an aspect oriented
manner.}. Moreover, because weaving is performed dynamically, every execution of
a function may perform different operations.

\subsection{Opportunities presented by PDSF}

PDSF presented several oppportunities for future research. Some salient
properties of the original work include:

\begin{itemize}
    \item 
\end{itemize}

\section{Dynamism in AOP}
% Dynamic methods in aspect orientation

Aspect orientation frameworks have supported ``dynamic behaviour'' in different
ways for a long time. This is largely through a technique referred to as dynamic–
or runtime–weaving.

\subsection{Dynamic and static weaving}

Dynamic weaving integrates advice into a target program during its execution, as
opposed to during compilation or a pre-processing step. The advantage of this is
flexibility: dynamic aspect-oriented approaches have been proposed for deploying
hotfixes in safety-critical scenarios where software systems cannot be taken
offline to apply patches\todo{CITECITECITE}, and in adaptive mobile scenarios where software
may need to alter its properties in response to its
environment\cref{hveding2005aspect}, or when debugging code to apply potential
patches without reloading an entire software system\cref{popovici2002PROSE}.

To meet these needs, software systems need to check for available aspects to
weave at any join point, as it is always possible that the set of applied advice
has changed since the program last encountered this point. The technique
therefore presents a tradeoff compared to traditional (static) aspect weaving,
as illustrated in \cref{dynamicAOchitchyan}. \citeauthor{dynamicAOchitchyan}
generalise this tradeoff by describing different mechanisms used to implement
aspect orientation into three main categories\footnote{Drawing from
\cref{popovici2002PROSE,popovici2003JITaspects} where ``PROSE'', a particularly
influential dynamic aspect orientation library, is detailled.}, each with their own
strengths:

\begin{description}
    \item[``Total hook weaving''] alters all join points where advice may be
    applied before runtime, so that during execution each join point ``watches''
    for applied advice. The benefit of this approach is that aspects can be
    applied at any point at runtime, but this flexibility is bought at the cost
    of maximum overhead: at all points where weaving \emph{may} be possible,
    checks for applied advice must be made.
    \item[``Actual hook weaving''] weaves hooks only to join points that are
    expected to be in use. This limits overhead from watching for applied
    advice, at the cost of flexibility: during program execution, advice may be
    applied or retracted \emph{only at specific points within the system}.
    \item[``Collected weaving''] weaves aspects directly into code at
    compilation / preprocessing\todo{surely this isn't dynamic, Ian...?!}, so as
    to collect advice and target codebase into a single unit. This provides
    exactly the necessary amount of overhead, and in many cases may result in
    requiring no ``watching'' for applied advice at all, but this limits a
    developer's ability to amend advice supplied at runtime.
\end{description}

There is an almost direct tradeoff between the number of potential join points
actively checking for applied advice at runtime, and the overhead of dynamism in
any aspect oriented framework, with ``total hook weaving'' providing complete
adaptability at the expense of checking at all possible points whether advice is
applied.

Another tradeoff could be seen to be the clarity of dynamically woven aspect
oriented code. 


% Dynamic AO libraries
\subsection{PROSE}
One implementation of dynamic weaving is PROSE, a library which achieves dynamic weaving by
use of a Just-In-Time compiler for Java. PROSE was developed with sever


\subsection{Handi-Wrap}
Handi-Wrap\cref{Baker_2002} is a Java library allowing for dynamic weaving via
a third-party language, Maya\todo{Do I want a citation for this? Probably not,
but worth revisiting.}. 



% Alternative techniques
\subsection{BCA}\label{subsec:BCA}
%check Kell survey for more, I believe there's juicy stuff there, maybe in
%section 2

\section{Aspect Orientation \& Modelling}

% Aspect orientation as it applies to modelling



\section{Aspect Orientation \& Simulation}

Surprisingly little literature exists pertaining to aspect-oriented simulation.
Aspect orientation is often applied in the context of \emph{modelling},
composing together a perspective of the world which isn't necessarily executable
or able to produce data.

Early in the history of aspect orientation as an emerging paradigm, there was
some interest in its use for scientific simulation. \cref{gulyas1999use} discuss
that computer simulations require code for both observation of a simulation and
the simulation itself, and that misuse of this could cause what is in effect a
kind of Hawthorne Effect\todo{does this need a citation?}, where the inclusion
of observation code intertwined with simuilation code might influence the
outcome of an experiment. They suggest that improving simulation technologies
could combat this approach. Aspect Orientation, being developed specifically
with obliviousness in mind, is an ideal candidate which \citeauthor{gulyas1999use}
identify.

Much of the literature concerning aspect-oriented programming and simulation
focus on tooling support for aspect-oriented simulation, rather than
investigations into its efficacy. For example, attempts have been made to
integrate aspect orientation into new
tools~\cref{DEVSaspectorientation2008aksu}~\todo{throw more in here from excel},
or into existing ones~\cref{chibani2019using}\todo{throw more in here from
excel}. \todo{Rethink this. The angle is, ``much of the lit strays from any kind
of real-world testing of sim tech --- why?''}

Some experiments specifically using aspect orientation in the implementation of
process-based simulations also exist\cref{Ionescu_2009}~\todo{include more!}. For
example, \citeauthor{Ionescu_2009} apply aspect orientation in a nuclear
disaster prevention simulation. Their motivation is that code can become complex
to maintain over time and changes to the scientific zeitgeist or to regulatory
requirements become costly as technical debt mounts. Aspect orientation
therefore allows developers to separate functionality into distinct modules more
easily, without disturbing the underlying codebase.





\section{Dynamism in Simulation and Modelling}

% Dynamic / contingent behaviours in simulation and models
% IMO this is where a lot of stuff like variance in process mining, sanitisation
% of logs, log noise removal / injection goes.




\section{Opportunities}

% putting these all together. There's lit in each section,
% but specifically AO in simulation and dynamic simulation aren't well studied
% on their own and warrant further study. It just so happens that PDSF is
% well-positioned to fill this gap, as it has the necessary properties of a
% tool to perform this research, where something like AspectJ [likely...?] falls
% short.