%\labelledsec{Aspect Orientation}{dynamic_aok_review\}
%
%In \cref{kiczales1997aspect}, \citeauthor{kiczales1997aspect} see these
%engineering concepts as universal throughout business logic, motivating the
%aspect-oriented approach for the first time. The authors present an
%implementation of AOP in Lisp, and compare implementations by way of e.g. SLOC
%count in an emitted C program to a comparable, non-AOP implementation, with two
%examples (its use in image processing and document processing). They find the
%idea --- which they note is "young" and note many areas where research might
%help it to grow --- can successfully separate systemic implementation concerns
%such as memory management in a way that reduces program bloat and simplifies
%implemenation. It is noted that measuring the benefits of their approach
%quantitatively is challenging.
%
% Aspect oriented programming (AOP) is a technique for isolating `cross-cutting
% concerns'' in a codebase in separate parts of that
% codebase\cite{kiczales1997aspect}. For the benefit of the reader, a short
% explanation of aspect orientation precedes the discussion of its literature and
% its relevance to the work presented in this thesis.


% \subsection*{Literature Review Structure}\label{sec:lit_review_structure}

This thesis presents an aspect-oriented approach to simulation and experimental
design, tooling to support these endeavours, some empirical assessment of both
the paradigm and its application in the domain of simulation and modelling. In
particular, the presented research makes use of aspects to model hypotheses and
the complexities of variable and unpredictable behaviour in the simulation of
\sociotechnical Their evaluation measures code quality improvements, which are
commonly claimed in aspect orientation research\revnote{add citations where code
quality improvements are claimed in AOP research}. systems.

Relevant literature in this topic typically comes from a variety of fields which
do not overlap significantly, meaning that this review must cover segments of
several partly related fields. The presented material is eclectic in nature as a
result. For this reason, the literature reviewed in this chapter includes
present context for unfamiliar readers as well as a motivating case for the work
which follows in later chapters.\revnote{There's a paragraph missing here
explaining which chapters follow, why, and in what order. I'm omitting it at the
moment because I suspect I'm about to reorder them.}

% ===== Original description list describing which chapter follows and in what order…:
% \begin{description}
%     \item[\Cref{lit_review_AOP_explainer},] which introduces aspect orientation and
%     gives a background on the field;
%     \item[\Cref{sec:dynamic_aop_review},] which details some existing approaches
%     to the dynamic weaving of aspects;
%     \item[\Cref{sec:ao_and_modelling},] which discusses the use of aspect
%     orientation in simulation and modelling; and
%     \item[\Cref{sec:dynamism_in_sm},] which outlines literature on the modelling
%     of process variance, particularly in simulation and modelling.
% \end{description}



philosophy toward software engineering and modularity, and some criticisms of
its high-level design. Other sections, in particular
\cref{sec:dynamic_aop_review} and \cref{sec:ao_and_modelling}, expand on this
introduction by investigating implementations, technologies and use cases of
particular interest.\revnote{Reword if I move criticisms to another section.}
\revnote{I believe this para was mangled by emacs somehow, and is the end of a
  drafted final paragraph of this introduction. Keeping it because it wasn't
  committed so can't restore the original para, and ideally I'll recreate it
  when redrafting this chapter.}

\subsection{Motivations \& Philosophy Underlying Aspect Orientation}
\label{review_aop_motivations_and_philosophy}

Modularity is considered a key trait of maintainable, flexible, and legible
programs~\cite{parnas_1972}. Modern modular design techniques are often
concerned with segmenting a program's logic or data into separate units. The
standard approach to designing with modularity in mind in many
industrially-relevant languages today is object-oriented programming, where
data structures and selected program logic are modularised as
classes.\inline{Cite some OO languages, papers
noting their industrial use / the ``success'' of object orientation.}

Some concepts in software engineering are not easily modularised by techniques
such as aspect orientation: guarding against unsafe concurrency usage, manual
memory management, and logging are all examples of program components which are
not readily modularised using an object-oriented or similar approach: the logic
of these ``cross-cutting concerns'' is common to all modules. Programmers
looking to separate cross-cutting concerns into separate modules look to
address two problems~\cite{kiczales1997aspect}:

\begin{enumerate}
    \item \emph{``Tangling''}, where program logic essential for
    a program's intended purpose is intermixed with ancillary code addressing
    cross-cutting concerns, thereby making essential logic more difficult to
    maintain;
    \item \emph{``Scattering''}, where program logic for
    cross-cutting concerns is strewn throughout a codebase, making maintenance
    of this code more difficult.
\end{enumerate}

The existence of cross-cutting concerns is therefore expected to make maintenance
of both ancillary program logic and a program's core logic more difficult.
Addressing this, \citeauthor{kiczales1997aspect} introduced the notion of
aspect-oriented programming~\cite{kiczales1997aspect}. Aspects are simple to
define through their component parts:

\begin{itemize}
    \item A \emph{``join point''} defines some point in a program's execution
(usually the moment of invocation or return of some function or method).
    \item \emph{``Advice''} defines some behaviour such as logging, which
    which can conceptually happen anywhere in program execution (i.e. what's defined
    would typically represent behaviour which cuts across many parts of a
    codebase).
    \item An \emph{``aspect''} is constructed by composing this advice with
    \emph{``point cuts''}: sets of join points that define all moments in
    program execution where associated advice is intended to be invoked.
    \item An \emph{``aspect weaver''} then adds the functionality defined by
    each aspect by adding the functionality defined by its advice at each join
    point defined by its point cut.
\end{itemize}

The definition of join points and advice, or how weaving occurs, is a matter
left for aspect orientation frameworks and languages to define. In employing the
technique, aspect oriented programming aims to separate cross-cutting concerns
into aspects, removing the aforementioned repetitive code from the logic
implementing a program's functional behaviour so that additional pieces of
functionality --- logging, authentication, and so on --- can be maintained in
only one place in a codebase (thereby simplifying their maintenance and
comprehension), and remaining program logic can be understood and maintained
without the overhead imposed by the previously tangled cross-cutting concerns.

In \cite{kiczales1997aspect}, \citeauthor{kiczales1997aspect} see these
engineering concepts as universal throughout business logic, motivating the
aspect-oriented approach for the first time. The authors present an
implementation of AOP in Lisp, and compare implementations by way of e.g. SLOC
count in an emitted C program to a comparable, non-AOP implementation, with two
examples (its use in image processing and document processing). They find the
idea --- which they note is "young" and note many areas where research might
help it to grow --- can successfully separate systemic implementation concerns
such as memory management in a way that reduces program bloat and simplifies
implementation. It is noted that measuring the benefits of their approach
quantitatively is challenging.

Tooling followed the theoretical work presented by
\citeauthor{kiczales1997aspect}~\cite{kiczales1997aspect} with a demonstration
and subsequent technical description of AspectJ, a Java extension for aspect
oriented programming~\cite{AspectJLanguageAndTools,aspectj_intro}. AspectJ was
introduced to satisfy the research community's need for a tool with which to
demonstrate the aspect-oriented paradigm in case studies. The tool is intended
to serve as ``the basis of an empirical assessment of aspect-oriented
programming''~\cite{aspectj_intro}. The library makes use of standard
aspect-oriented concepts: Pointcuts, Join Points, and Advice, bundled together
in Aspects. They define ``dynamic'' and ``static'' cross-cutting, by which they
refer to join points at specific points in the execution of a program, and join
points describing specific types whose functionality is to be altered in some
way. Their paper describes only ``dynamic'' cross-cutting, but presents tooling
support, architectural detail of its implementation, and the representation \&
definition of pointcuts in AspectJ. AspectJ is compared to other tools for
aspect orientation and related decomposition-focused paradigms, and the authors
are explicit about their approach being distinct from metaprogramming in, for
example, Smalltalk or Clojure.

\citeauthor{filman2000aspect} isolate properties of aspect orientation which
they assert are definitive of the paradigm~\cite{filman2000aspect}.
Specifically, they claim that aspect oriented programming should be considered
in terms of ``quantification'' and ``obliviousness'':

\begin{displayquote}
AOP can be understood as the desire to make quantified statements about the
behaviour of programs, and to have these quantifications hold over programs
written by oblivious programmers.~[\ldots{}]~We want to be able to say, ``This
code realises his concern. Execute it whenever these circumstances hold.
\end{displayquote}

These concepts became core concepts in aspect orientation literature alongside
``tangling'' and ``scattering''. \citeauthor{filman2000aspect} give no concrete
definition of the terms in their original paper and cite no sources for the
terms; for the purposes of this thesis, we therefore provide the following
definitions of the terms:

\begin{description}
  \item[Quantification] is the property of specifying specific points in a
  program in which that program should change;
\item[Obliviousness] is the property of a codebase that it contains no lexical
  or conceptual reference to advice which might be applied to it, and of the
  programmer of a target program that their code may be amended by an aspect
  programmer.
\end{description}

\citeauthor{filman2000aspect} write about aspect orientation \emph{``qua
  programming language''}, and so theorise around aspect orientation as a
paradigm independent of a particular application. They are therefore able to
arrive at conclusions about the paradigm in the abstract, and can identify
concerns for future investigation for researchers in the field and design goals
for developers of aspect orientation tooling. They note:

\begin{displayquote}
  Better AOP systems are more oblivious. They minimise the degree to which
  programmers (particularly the programmers of the primary functionality) have
  to change their behaviour to realise the benefits of AOP. It's a really nice
  bumper sticker to be able to say, ``Just program like you always do, and we'll
  be able to add the aspects later.'' (And change your mind downstream about
  your policies, and we'll painlessly transform your code for that,
  too.)~\cite{filman2000aspect}
\end{displayquote}

Here, \citeauthor{filman2000aspect} theorise about the advantages they expect
aspect orientation to bring software engineers. However, their writing follows
the original from \citet{kiczales1997aspect} by only three years, and in the
following decades no significant literature or case study seems to evidence
their expectations~\citep{steimann06paradoxical}. Criticisms of
aspect orientation's effectiveness are discussed further in
\cref{subsec:aop-criticisms}.

Aside from evidence of practical benefits of aspect orientation, claims made by
\citeauthor{filman2000aspect} such as the ``painless transformation'' of code
when a developer changes their mind during the development process --- a
theorised benefit of obliviousness --- is incompatible with earlier writing on
modularity. \citet{yourdon1979structured} assert:

\begin{displayquote}
  The more that we must know of module B in order to understand module A, the
  more closely connected A is to B. The fact that we must know something about
  another module is a priori evidence of some degree of interconnection even if
  the form of the interconnection is not known.
\end{displayquote}

Aspect orientation's critics describe similar incompatibilities with existing
best-practices~\cite{przybylek2010wrong,Constantinides04aopconsidered}, as well
as the lack of empirical evidence for the benefits of
obliviousness~\cite{steimann06paradoxical}. Claims about ``better''
aspect-oriented systems being more oblivious should therefore be regarded as
\emph{suggestions} from the literature. While obliviousness and quantification
are useful concepts in discussing research in the field, they also give context
for the research community's perspective that obliviousness and quantification
are design goals for aspect oriented
programmers~\cite{AspectCplusplusDesignImp,kell2008survey,Charfi2006AspectOrientedWL}
(though \citet{leavens2007multiple} suggest they may be best applied in
moderation).

\revnote{Feels like the full literature on AOP is missing here. What is the
  available evidence that it works, if any? What has it been used for? Some sort
  of survey of AOP tech might also be useful for talking about alternatives?}
\reviewresponse{I wonder whether I need to review more literature (possible ---
  I have a few mentioned in comments here and in annotated.bib that I should
  look at if I have time, though I believe I've covered the essentials) or
  whether the tech that's of interest here is reviewed later on. I suspect it's
  included later, because there's a note toward Dynamic Aspect Weaving which
  suggests the tech needs to come before the criticisms review (i.e. there's
  some tooling reviewed, but not where it should be). Mostly folk seem to have
  stuck with AspectJ, and domain-specific / general tooling is reviewed later.
  I'm reticent to move the specific tech reviewed here because I currently have
  a nice separation between the philosophy of the paradigm and its
  implementations; maybe I should make those concepts more clearly separated in
  section headings, or maybe I should move the tech here and make Criticisms its
  own section, rather than a subsec of this introduction.}

\subsection{Criticisms of Aspect Orientation}\label{subsec:aop-criticisms}

The growing aspect-oriented programming research community collected both
proponents and detractors of the paradigm. The opposition to
aspect orientation is of particular relevance to this thesis in two regards. First, this literature review discusses advancements in the field relevant to
the work presented in this thesis. This should be understood within the context
of some perceived weaknesses of aspect orientation, which helps to frame an
understanding of the literature reviewed. Second, the following chapters
introduce contributions which address some criticisms of aspect-oriented
programming, meaning that the criticisms of the paradigm writ large --- and the
properties of work published in awareness of those weaknesses --- motivates some
research presented in later chapters.


\citeauthor{Constantinides04aopconsidered} published an early critique of
aspect-oriented programming~\cite{Constantinides04aopconsidered} which notes
similarities between the paradigm's core concepts and \lstinline{GO-TO}
statements. \citeauthor{Constantinides04aopconsidered} uses the comparison to
demonstrate fundamental issues with the paradigm's philosophy: use of
\lstinline{GO-TO} statements is widely accepted to be bad
practice~\cite{dijkstra1968letters}, and their infamy has become an in-joke for
academics and language enthusiasts~\cite{clark73comefrom}. They note that
\lstinline{GO-TO} statements disorientate a programmer by way of ``destroying
their coordinate system'' --- referring to a developer's technique for
navigating and understanding code --- which results in uncertainty about both a
program's flow of execution and the state of a program at different points of
its flow. \citeauthor{Constantinides04aopconsidered} uses the comparison to
\lstinline{GO-TO} statements to question whether aspect-oriented programs can
have a consistent coordinate system for developers: \lstinline{GO-TO} statements
lack obliviousness as they are visible in disrupted code, wheras aspects are not
represented structurally within a program due to their oblivious design. This
complicates a developer's understanding of where and how flow is interrupted.
They draw comparison between aspects and \lstinline{COME-FROM} statements, an
April Fools' joke where a claimed improvement over \lstinline{GO-TO} is
developed by removing the latter's structural component~\cite{clark73comefrom},
and conclude that existing engineering techniques provide similar benefits
without a trade-off in program legibility. In particular, Dynamic Dispatch is
identified as a preferred alternative.


\citeauthor{steimann06paradoxical} makes a similar but more thorough critique of
aspect-oriented programming~\cite{steimann06paradoxical}. They express concern
that the popularity of aspect-oriented programming --- which was nearly 10 years
old at time of publication --- was founded on the perception that it would solve
real-world engineering problems, yet no proof existed that it was effective in
practice. \citeauthor{steimann06paradoxical} notes that most papers are
theoretical in their discussion on tooling, that examples were typically
repetitive, and that the community's discussion was concerned more with what
aspect orientation could be used for than whether it worked in practice. They
present a comparison between aspect orientation and object-orientation, where
aspect orientation's claimed properties and principles are examined in detail,
and the impact on software engineering is reasoned about from a skeptical
perspective. They compare the paradigms' specific claims that they both support
improved modularity against classic papers on the subjects.\footnote{In
  particular, they compare against literature by \citet{parnas_1972}.}.
\citeauthor{steimann06paradoxical} presents a philosophical examination of
aspect orientation and assesses the paradigm against its purported merits,
discussing whether we should accept the claims made by the aspect-oriented
programming community. Aspect orientation's promise of unprecedented
modularisation is presented as unfulfilled, and
\citeauthor{steimann06paradoxical} reflects critically on the state of
aspect orientation research at the time. However, they conclude by proposing
that the aspect-oriented approach could have a legitimate alternative use-case
--- other than as a \emph{general-purpose} technique for modularising
cross-cutting concerns --- where it could be shown to be effective empirically.

Similar sentiments are shared by \citet{przybylek2010wrong}, who examines
aspect-oriented programming in the context of language designers' quest to
achieve maintainable modularity in system design. They frame the design goals of
aspect orientation as being to represent issues that ``cannot be represented as
first-class entities in the adopted language''. \citeauthor{przybylek2010wrong}
questions whether the modularity offered by aspect orientation can really be
said to make code more modular. In particular, they distinguish between lexical
separation of concerns and the separation of concerns originally discussed by
\citet{djikstra_scientific_thought}. They assess principles of modularity ---
modular reasoning, interface design, and a decrease in coupling --- and find
that the aspect-oriented paradigm can detrimentally impact the expected benefits
of proper modularisation in a program\revnote{Why? Presumably because they use
  the wrong type of modularity (Lexical?) --- if so, make this clear here!}.
They suggest that aspect orientation's claimed benefits are a myth repeated
often enough to be believed.
% \citeauthor{przybylek2010wrong} presents a critical review of aspect
% orientation literature, but marries their critique with mention of others'
% solutions to the problems identified: they point to many papers which suggest
% improvements to the standard AOP approach which might reduce it's negative
% impact or make it more practically viable\revnote{Which papers?!}.

\revnote{This review is arguably unfinished --- there are more papers commented below
  this note which are probably worth
  reading / ensuring I'm citing. Tim has noted that the review on early /
  foundational literature of AOP seems incomplete, and I think the commented
  papers to review below could be good candidates to address this.}

%%  ## References to pick up & review:
%%  
%%  On criticisms of AOP
%%  - Dantas & Walker 2006, from What Is Wrong With AOP
%%  - Leavens & Clifton 2007, from What Is Wrong With AOP
%%  - Filman & Friedman 2001, from What Is Wrong With AOP
%%  - Constantinides, Scotinides & Störzer 2004, from What Is Wrong With AOP
%%  - Tourwe, Brichau & Gybels 2003, from What Is Wrong With AOP
%%  - Wampler 2007, from What Is Wrong With AOP
%%      - "Most AO languages in use today are based on structural information about
%%      join points, such as naming conventions and package structure, rather than
%%      the logical patterns of the software
%%  
%%  On AOP and decreasing coupling
%%  - Yourdon & Constantine 1979, from What is Wrong With AOP
%%      - "The fact that we must know something about another module is a priori
%%      evidence of some degree of interconnection even if the form of the
%%      interconnection is not known" 




\subsection{Alternatives to Aspect Orientation}

Aspect-oriented programming's goal of modularising cross-cutting concerns is
shared by other paradigms. As discussed in
\cref{review_aop_motivations_and_philosophy}, the work first introducing aspect
orientation by \citet{kiczales1997aspect} makes note of similarities to
reflection, metaprogramming \& program transformation, and subject-oriented
programming. They also observe that other disciplines have introduced
``aspectual decomposition'' independently.

\revnote{this needs more explanation. What were Kiczales et al. trying to do,
  what was the context, what did they demonstrate?}\reviewresponse{I covered this
  a fair bit in the earlier parts of this section. Maybe this was unclear? This
  lit is already covered, so fixing earlier notes might be better than adding
  notes here. I've since restructured \& reframed, hopefully enough to make this
clearer, but I'm curious\ldots{}}

\subsubsection{System Diagrams}\label{subsec:system_diagrams_as_aspects}

The example of pre-existing aspectual decomposition by way of diagramming given
by \citet{kiczales1997aspect} is in physical engineering. To give a concrete
example from their description, differing types of diagrams when engineering a
system such as thermal and electric diagrams of a heater are described as
``aspectual'' because of the modular nature of the diagrams; though there might
be many diagrams of different kinds, they compose together to give an overview
of the system being designed.

Similar diagramming techniques have independently arisen in other domains since.
The Obashi dataflow modelling methodology\cite{obashimethodology} by
\citeauthor{obashimethodology} models all possible paths of dataflow through
``B\&IT'' (business and IT) diagrams, where business-specific concerns (people,
locations, groups, and business processes such as payroll, stock-check or
budgeting) are modelled alongside IT concerns such as applications supporting
business processes and the software and hardware infrastructure supporting them.
Modelling dataflows in this way allows for a comprehensive understanding of
assets and business processes. However, in order to understand how data flows
between specific assets within a B\&IT, sub-graphs (``DAVs'', or Dataflow
Analysis Views) denote specific pathways through which data flows between source
and sink assets. Alternatively, a B\&IT can be viewed as a composition of all
possible DAVs within an organisation. Dataflows are therefore broken into
different diagramming techniques and specific business concerns can be described
independently of others, even if these concerns interact in their dataflow
pathways (and, therefore, cutting across each other). Obashi therefore allows
for the aspectual decomposition of business processes, through the description
of an organisation by individual dataflow analysis views, which compose into an
overall model of a system in a B\&IT diagram. Obashi models are an instance of
aspect orientation which were designed for simplicity and
comprehension\cite{obashimethodology,seow2011obashi}, but trade this for
domain-specificity.

\inline{Add notes on Bigraphs here}

\inline{Add notes on Holons here, maybe}

\subsubsection{Metaprogramming}\label{metaprogramming_as_an_aop_alternative}

\revnote{Wouldn't the paragraph below fit better in the section on program mutation?}
\reviewresponse{I'm not sure exactly what section this is, but have reworded the
  below to fit better within this section (BCA is a relative of AOP) and added a
cref to where I \emph{think} it might be added. Unsure though. To discuss and
fix the cref. I haven't added notes on BCA anywhere else. Or, was it intended to
refer to the metaprogramming notes in the next para? If so, I've since collected
these into a subsubsec.}

Metaprogramming is identified as a precursor to aspect orientation by the
original paper on the paradigm~\cite{kiczales1997aspect}. Research into the use
of metaprogramming to simplify the composition of software modules was
undertaken at a similar time by \citet{keller1998binary}, who introduce a
technique they name ``Binary Component Adaptation'' (BCA). Their research into
BCA the difficulties involved in the integration of software components,
particularly considering their evolution over time where components are re-used
with differing requirements. By modifying binaries directly, incompatibilities
between a program and one of that program's dependencies can be resolved by way
of mutating either after compilation. Their implementation defines a
representation for the modification of pre-existing Java class binaries, the
output of which can be verified as also being valid Java class binaries.
\Citeauthor{keller1998binary} claim that BCA allows for dynamic modification of
programs with little overhead. They believe BCA is unique in its combination of
features, which include engineering concerns such as typechecking code which is
subject to adaptation and its obliviousness to source implementation, as well as
guarantees that modifications are valid even for later iterations of the program
subject to adaptation.\footnote{BCA shares concepts with aspect orientation, but
  is also a promising technology for the introduction of process variance; see
  \cref{subsec:variations_in_sm} for a discussion.}

Metaobject protocols describe the properties of an object's class (including,
for example, its position within a class hierarchy) in an adaptable
manner~\cite{kiczales1991art}. \citet{espakaspect} note that this technique can
be used to implement aspect orientation, therefore providing at a minimum the
same functionality, though they achieve this through reflective programming
techniques and are designed with metaprogramming as a primary goal as opposed to
modularisation of cross-cutting
concerns~\cite{kiczales1991art,sullivan2001aspect}.

\subsubsection{Engineering Techniques with Related Aims}
\label{engineering_techniques_as_aop_alternative}

Multiple-dispatch, where methods on objects are chosen to be run based on the
properties of the parameters passed at point of invocation, allows for oblivious
decomposition without the need for a weaver~\cite{dozsa2008lisp}, although this
does not support the goals of aspect orientation in totality. For example, a
programmer might want their program to exhibit differing behaviour when methods
are called with differently-typed arguments, which is supported by multiple
dispatch. However, they might instead want their program to exhibit some
additional behaviour whenever a method is invoked, such as logging, but might
not want to implement logging alongside the rest of their method implementation
for clarity or length reasons. Multiple dispatch therefore offers comparable but
different functionality to a software engineer.

Engineering patterns such as decorators provide similar functionality to
aspects~\cite{friesel2017annotations}, in that cross-cutting concerns can be
separated into their own module, but they differ in their approach to
obliviousness: decorators annotate areas of a codebase they are applied to, and
therefore do not offer obliviousness as aspects do. Decorators allow for
additional functionality to be written as a separate module and applied as a
wrapper around a function definition.\inline{A diagram or Python example might
  help here, as per Tim's suggestion below.} A function with a wrapped
definition is replaced by a function returned by the decorator, which takes the
original definition as an argument~\citep{gof_design_patterns}. Additional logic
can therefore be simply applied before or after a function. While this achieves
a similar effect to aspect orientation --- as logic is added before and/or after
the original logic of a function --- its design principles are different as
annotating the function's definition directly marks it as altered, and so the
original definition cannot be oblivious to the change.

\revnote{Would some illustrative examples of these different approaches help?}
\reviewresponse{Yes, I'm sure they'd help! Diagrams to add here.}

\subsection{Subject-oriented Programming}

Subject-oriented programming is a paradigm which \revnote{Incomplete! Add subject-oriented
  programming here.} \inline{Note: there's an argument that SOP has seen more
  success in practice than AOP: Golang's interfaces are subjective views on a
  struct, and I believe Rust's Traits are too --- confirm this! --- by contrast,
  there's some limited use of AOP in the wild, but related design patterns have
  arguably supplanted them for modularising cross-cutting concerns (I'm thinking
  of Decorators). I'm not aware of any literature which backs this up.}

relevant papers:

\begin{itemize}
\item Harold Ossher, Peri Tarr, William Harrison, Stanley Sutton, N Degrees of
  Separation: Multi-Dimensional Separation of Concerns, Proceedings of 1999
  International Conference on Software Engineering
\item Subject-oriented programming: a critique of pure objects
\item Whatever originally defined SOP
  \item Founding papers for Hyper/J and/or Hyper/C
  \end{itemize}



\section{Dynamic Aspect Weaving}\label{sec:dynamic_aop_review}

\revnote{I think this comes before the critique section perhaps? Need a general
  introduction to AOP techniques and techs, including this?}

\revnote{T: Have I actually explained what a dynamic weaver is? Think not.}

\subsection{Approaches to Dynamic Weaving}
% MARK: dynamicAOchitchyan


The performance issues noted by \citeauthor{popovici2003JITaspects} are explored
\revnote{T: maybe this is a suitable introduction to dynamic weaving, if I add
  an explanatory paragraph at the beginning?} in more detail by
\citet{dynamicAOchitchyan}, who present a review of early dynamic aspect
orientation techniques. They compare AspectWerkz, JBoss, Prose, and Nanning
Aspects through the lens of the authors' prior work on dynamic reconfiguration
of software systems. By comparing different implementations of dynamic
weaving, \citeauthor{dynamicAOchitchyan} contribute a categorisation of the tools'
approaches: \revnote{Need citations to these technologies}

\begin{enumerate}
\item ``Total hook'' weaving, where aspect hooks are woven at all possible
points;
\item ``Actual hook'' weaving, where aspect hooks are woven where required;
\item ``Collective'' weaving, where aspects are woven directly into the executed
code, ``collecting the aspects and base in one unit''.
\end{enumerate}

As \citeauthor{dynamicAOchitchyan} focus on software reconfiguration rather than
the mechanics and design of dynamic aspect weaving, their analysis of the
reviewed tools is of less relevance to the work presented in this thesis than
their generalisation of dynamic weaving. However, their review discusses the
trade-offs of the three approaches identified. \citeauthor{dynamicAOchitchyan}
propose that total hook weaving allows flexibility in the evolution of a
software product, at the expense of the performance of that product; this
contrasts collected weaving, which shifts overhead out of the codebase and into
the maintenance effort. Actual hook weaving is positioned as a compromise
between the two, offering the best approach for none of their criteria but never
compromising so much as to offer the worst, either. This suggests merit in a
tool designed to flexibly offer any weaving approach appropriate for the task at
hand.

\citeauthor{dynamicAOchitchyan} note that one could use many of
the systems they describe in practice. Though the paper is an early publication
in the field, no tool the authors review offers all three dynamic weaving
approaches, and none offers collective weaving alongside either kind of hook
weaving.

\revnote{AspectJ belongs here.}

\subsection{PROSE}

One implementation of dynamic weaving is
PROSE~\cite{popovici2002PROSE,popovici2003JITaspects}, a library which achieves
dynamic weaving by use of a Just-In-Time compiler for Java.
\citeauthor{popovici2002PROSE} motivate the project be identifying aspect
orientation as a possible solution to software's increasing need for
adaptivity.
Mobile devices, for example, could enable a required feature by applying an
aspect as a kind of ``hotfix'', thereby adapting over time to a user's needs.
Other uses of dynamic aspect orientation they identify are in the process of
software development: as aspects are applied to a compiled, live product, the
join points being used can be inspected by a developer to see whether the
correct pointcut is used. If not, a developer could use dynamic weaving to
remove a misapplied aspect, rewrite the pointcut, and weave again without
recompiling and relaunching their project.

The conclusion \citet{popovici2003JITaspects} provide indicates that some
performance issues may prevent dynamic aspect orientation from being useful in
production software, but that it presented opportunities in a prototyping or
debugging context. The PROSE project explores dynamic weaving as it could apply
in a development context, but the authors do not appear to have investigated
dynamic weaving as it could apply to simulation contexts, or others where
aspect-oriented software does not constitute a product.

Their observation that aspect-oriented programming could be
used for the purpose of adaptation or software prototyping instead of
modularisation is an example of an alternative use case for aspect orientation
suggested in \citeauthor{steimann06paradoxical}'s
critique~\cite{steimann06paradoxical}, as discussed in
\cref{subsec:aop-criticisms}.



\subsection{Nu}

\citeauthor{rajan2006nu_towardsao_invocation} propose a new aspect-oriented
invocation mechanism, which they call
``Bind''~\cite{rajan2006nu_towardsao_invocation}. Bind's design is motivated by
opportunities to improve modularity from a design perspective:
\citeauthor{rajan2006nu_towardsao_invocation} assert that ``scattering'' and
``tangling'' can be introduced into a codebase after weaving with some weaver
implementations. This complicates the use of compiled aspect-oriented code, and
the development and execution of unit tests on such a codebase. Bind alleviates
this issue by allowing a developer to choose when aspects are applied. This new
mechanism is presented as an alternative to the weaving of aspect hooks for
load-time registration into target code in the style of
PROSE~\cite{popovici2002PROSE,popovici2003JITaspects} and the direct weaving of
aspect invocations in the style of AspectJ~\cite{aspectj_intro}.

In order to demonstrate Bind's approach to simplifying post-weave codebases,
\citeauthor{rajan2006_towardsao_invocation} also present ``Nu'', an
aspect orientation framework written in .NET supporting Bind. Nu's design is
explained and its implementation presented, which are designed to promote
granularity in join point specification. What results is a flexible model for
aspect orientation which is demonstrated to satisfactorily emulate many other
paradigms and tools, such as aspect orientation in AspectJ, subject-orientation
in HyperJ, and Adaptive Programming.\revnote{I don't define adaptive
  programming. Maybe I should have it as an alternative? End of this section,
  after subject-oriented programming?}

\citeauthor{dyerNUmasters} explain in more depth than in the design
and implementation of the Bind mechanism and the implementation of the Nu
framework~\cite{dyerNUmasters}. A more technical discussion is presented, in
particular on implementation details including optimisation and benchmarking,
largely against AspectJ. Notably, the implementation discussed is a Java
implementation, rather than the .Net implementation presented in
\cite{rajan2006nu}. Many aspect orientation frameworks are language-specific;
the existence of Nu's implementation on multiple platforms highlights the work's
main contribution in the design of the Bind primitive, rather than the
framework itself.

It is noted that it is ``very common in aspect-oriented programming research
literature to provide language extensions to support new properties of
aspect-like constructs''. \citeauthor{rajan2006_towardsao_invocation} look to
provide similar extensions to virtual machines, noting that their mechanism is a
suitable candidate to introduce aspect orientation directly in a language's
virtual machine. They position the project as a general model of aspect-oriented
programming which can flexibly represent a variety of existing approaches. Bind
fulfils this aim by providing a single mechanism which can be used to achieve
many weaving techniques; however, case studies demonstrating that Bind can be
used to make aspect orientation more practically effective for software
developers does not appear to have been published.


\subsection{Dynamic Weaving in Embedded Systems}

\citeauthor{gilani2004family} observe that while there are different approaches
to dynamically weaving aspects, no approach is suitable for an embedded
environment~\cite{gilani2004family}. This is due to these systems' low power and
limited memory. \citeauthor{gilani2004family} propose a framework for these
situations through which weavers can be assessed for their suitability in a
given environment, or generated from a set of desired features.

\citeauthor{gilani2004family} define families of weavers by grouping the the
environments they can be suitably applied to, separating them in particular by
their trade-off between dynamism and resource use. There is an implication that
dynamism and resource use are broadly proportional, presumably because even a
carefully crafted ``actual hook weaver'' or JIT-compiled ``collective weaver''
in the parlance of \citet{dynamicAOchitchyan} carries runtime overhead by virtue
of the weaving mechanism used. A ``collective weaver'' embeds aspect invocations
directly into their join points, which eliminates the need for an intermediary
hook invocation or additional compilation. The embedded systems of interest to
\citeauthor{gilani2004family} have memory in the range of \lstinline{~30kb},
where these overheads could represent significant resource use.

Aspect oriented programming's criticism can often be that it doesn't know what
it ``aims to be good for''\revnote{Where's this quote from? Citation needed.
  Think it could be \citet{przybylek2010wrong} or
  \citet{steimann06paradoxical}.}, and the dynamic weaving of aspects may not be
``good for'' resource-constrained environments at all. As
\citeauthor{gilani2004family} identifies a trade-off between dynamism and
resource economy, a highly constrained environment would not be able to take
advantage of dynamic weaving by definition. Additionally, because the
anticipated benefits of aspect-oriented programs are not observed in practice,
the paradigm may be poorly matched to any use-case where pragmatism is a
concern. From the framework proposed by \citet{gilani2004family}, we might
observe that researchers should seek other contexts in which to apply aspect
oriented programming. Support for this observation can be found in criticisms of
aspect-oriented programming from \citet{steimann06paradoxical}, whose position
that alternative use-cases of aspect orientation should be explored is discussed
in \cref{subsec:aop-criticisms}. Examples of alternative uses include
\citeauthor{popovici2002PROSE}'s proposal of using aspect orientation for
software prototyping, and the proposal presented by \citet{gulyas1999use} that
simulation \& modelling is a more appropriate field. Both of these proposed uses
of aspect orientation involve a different resource economy than that imposed by
embedded systems. A discussion of aspect orientation's use in simulation \&
modelling can be found in \cref{sec:ao_and_modelling}.





\section{Aspect Orientation in Simulation \& Modelling}
\label{sec:ao_and_modelling}

This thesis is concerned with the use of aspect orientation in simulation \&
modelling codebases; it is therefore necessary to review related research in
simulation and modelling. This section reviews literature contributing
aspect-oriented tooling and the philosophy of aspect-oriented experiment design
in the simulation \& modelling community.

\subsection{Suitability of Aspect Orientation in Experimental Codebases}
\label{review_gulyas_use_of_aop_in_research_codebases}

\citet{gulyas1999use} observed that, in the study of complex systems through
software models, the codebase produced serves two purposes: the
experimental subject, and the observational apparatus used to conduct the
experiment itself. They use this framing to identify that the observation of a
program's state constitutes a cross-cutting concern. In order to reduce
scattering and tangling in experimental codebases, \citeauthor{gulyas1999use}
theorise that aspect-oriented programming may separate the logic of observation
from the core logic of a simulation or model.

The approach is demonstrated via their Multi-Agent Modelling Language (MAML). MAML
was designed to enable aspect-oriented simulation of agent-based models and was
implemented using the SWARM simulation system~\citep{hiebeler1994swarm}, a
domain-agnostic framework for agent-based simulation. While SWARM takes the form
of a collection of C libraries, MAML is implemented as a domain-specific
modelling language, allowing it to support aspect-oriented programming as a
language feature.

MAML's aspect orientation effectively makes use of Observer patterns to measure
a simulation's state. This enables a researcher to observe an experiment without
the necessary logic being scattered or tangled in their domain model.
\citeauthor{gulyas1999use} find that aspect-oriented programming provides an
intuitive and straightforward method by which simulated experimental systems can
be composed. They note that MAML's simplicity and its philosophy on modelling
are more ``satisfactory'' than Swarm's standard approach. The team report that
MAML's implementation was more complex than initially conceived: the
\lstinline{patch} unix tool was intended for use as their weaver, though the
team eventually developed a transpiler from MAML to Swarm instead (which they
name \lstinline{xmc}.). The deciding factors for the development of a custom
transpiler are not discussed.

In addition to presenting tooling for aspect oriented simulation,
\citet{gulyas1999use} theorise about the potential benefit of applying aspect
orientation to simulation \& modelling. They observe that there may be benefits
beyond improvements to modularity and a reduction of tangling \& scattering. In
particular, their work discusses specific scenarios in which the \emph{type} of
separation of concerns offered by aspect orientation is desirable, and the
engineering approach to achieving the aim reasonable.\revnote{T: If there's time
  when redrafting, note which scenarios they distinguish between specifically.}
This distinguishes the work in comparison to most other research on aspect
orientation. Many papers describe the expected benefits by simply drawing from
existing literature and the claims made in \citet{kiczales1997aspect}'s first
paper on the subject.\revnote{it seems like this might be an important
  exposition, so perhaps more details?}\reviewresponse{This is covered in the
  criticisms of AOP. Maybe I should just make a reference here, or maybe there's
  room for a paragraph to hammer the point home. Ask Tim which he thinks is more
  appropriate.}

This is a rare example of philosophy on aspect oriented programming's
suitability in a particular use-case. \citeauthor{gulyas1999use}'s work is of
particular importance in this review, because the domain they identify as
particularly suitable is simulation \& modelling, which is the subject of this
thesis. That aspect orientation might be well suited to separating observer and
experiment partially motivates research in later chapters, which investigates
the use of aspects to \emph{modify} simulated behaviour rather than simply
observing it. Further disucssion follows in \cref{sec:lit_discussion}.



\subsection{Aspect Orientation in Discrete Event Simulation}

\subsubsection{Aspect-Orientated Implementations of Simulation Frameworks}

\citet{chibani2013toward} observed that simulation frameworks and the codebases
built upon them can exhibit cross-cutting concerns such as event handling,
resource sharing, and the restoring the state of a simulation run. They
investigated the introduction of aspect orientation to Discrete Event Simulation
(DES) frameworks to address tangling and scattering that may arise from the
cross-cutting concerns they identified. They contribute a discussion of
aspect-oriented programming's potential application to DES codebases, and detail
the avenues available for research in the field. \citet{chibani2013toward}
identify Japrosim~\cite{abdelhabib2008japrosim,belattar2014yet} --- a DES
framework previously developed by the research team, which was designed to model
domains through process interaction in Java --- as an example of an
existing framework which exhibits the cross-cutting concerns they describe.

Later, \citet{chibani2019using} identified opportunities for the use of aspect
orientation in simulation tooling, aiming to increase ``modularity,
understandability, maintainability, reusability, and testability'' by applying
the paradigm~\cite{chibani2019using}. They present a case study of an
application of aspect orientation to simulation tooling by identifying
cross-cutting concerns in Japrosim, a discrete event simulation framework, and
propose an aspect-oriented redesign of the tool using AspectJ.
\citeauthor{chibani2019using} describe Japrosim's existing object-oriented
design, followed by aspect oriented variations of some design elements,
including concurrent process management and in Japrosim's graphical animation
features. A similar survey of areas in which Japrosim's source might benefit
from the application of aspect orientation is presented by
\citeauthor{chibani2014practical} in an earlier
work~\cite{chibani2014practical}. In both cases, the main contribution noted is
the design itself. Counting\revnote{Evaluating?} the main improvements between
the presented aspect-oriented design and the existing object-oriented one is
left to future work\revnote{new sentence here - \citet{chibani2019using}'s later
  work presented an implementation of their proposal and provide a quantitative
  evaluation.}
in the authors' later publication~\cite{chibani2019using}, although a concrete
implementation is linked to and some quantitative evaluation of that
implementation presented in their earlier
publication~\cite{chibani2014practical}. The quantitative evaluation provides
measurements based on \citet{martin1994oo}'s object-oriented design metrics
and demonstrates a greater independence of packages in their aspect oriented
version of Japrosim than in the original. However, the intended aim of aspect
orientation is not to decouple existing packages, but to isolate those packages'
cross-cutting concerns into new ones. It is therefore unclear that their
quantitative evaluation achieves its improvements as a result of aspect
orientation. No further discussion of their results is provided, and it is
possible that the improvement is due to the rewriting necessary in their
maintenance of the Japrosim source, rather than due to their use of aspect
orientation specifically.

Similarly to
\citeauthor{chibani2019using}~\cite{chibani2019using,chibani2013toward,chibani2014practical},
\citeauthor{DEVSaspectorientation2008aksu} observe that there are advantages in
adopting aspect orientation when developing a simulation
framework~\cite{DEVSaspectorientation2008aksu}. Examining the DES framework
Simkit\revnote{grab a citation for Simkit}, they motivate two different
applications of aspect-orientation: a refactoring of the framework itself to
better manage cross-cutting concerns within its codebase; and aspect-oriented
tooling for use by modellers who represent cross-cutting concerns within their
models. Opportunities for improvements in production and development are
discussed, and some implementation notes are detailed, although no concrete
implementation or evaluation is provided; the work instead proposes design
alterations, and the authors ``leave it as a future work \emph{[sic]} to explore
the usability and efficiency'' of aspect orientation used idiomatically
alongside Java's existing reflection offerings. The existence of multiple
attempts to refactor differing simulation packages with aspect orientation
indicates potential for modellers in the use of aspect-oriented patterns, but
the real-world utility of the techniques are omitted. As is common in
aspect-orientation literature, \citeauthor{chibani2019using} and
\citeauthor{DEVSaspectorientation2008aksu} both defer to the general claims that
aspect-orientation improves modularity of cross-cutting concerns and can
eliminate code smells such as tangling and scattering.
\footnote{\citeauthor{chibani2014practical} do present some quantitative
  evaluation, but this is flawed as previously described.}



\subsubsection{Aspect-Orientated Implementations of Simulations \& Models}

Research projects applying aspect-orientation to the implementation of
simulation frameworks~\cite{chibani} fails to provide a case study which evaluates
their technique with real-world examples. However, case studies do exist which
evaluate aspect-orientation's suitability in maintaining model source code.

\citeauthor{ionescu2009aspect} identify an increased demand for computational power in
simulation execution on supercomputers~\cite{ionescu2009aspect}. Existing
known-good models might be unsuitable for the extreme requirements of code
efficiency modellers contend with, but running the code in different
environments requires modifications to make the code suitable in the
environment. These modifications are subject to regulations and introduce risk
of a reduction in quality during maintenance. The authors propose an
aspect-oriented solution to the problem, where aspects modify the simulation
codebase with minimal overhead. An implementation of a real-world model for
disaster prevention is presented, and assessed both by comparison against an
equivalent non-aspect-oriented codebase and by assessment of the aspect-oriented
variant's scalability and reliability in both cluster and multi-cluster
environments. They find that a comparative analysis of generated code and of
their simulations in various configurations both indicate that their
simulation's aspect-oriented implementation is suitable for use in disaster
prevention, implying that aspect orientation could be suitable in scenarios with
comparable requirements.

This work provides evidence that aspect-oriented programming is suitable for
supporting modifications to simulations and models. This contrasts the lack of
evidence for the paradigm's success in improvements to modularity and
maintainability~\cite{przybylek2010wrong,Constantinides04aopconsidered}, and
supports \citet{Steimann06paradoxical}'s suggestion that aspect-orientation
might be suitable for other uses --- as well as the suggestion made by
\citet{gulyas1999use} that the paradigm is well suited to the requirements of
simulation \& modelling codebases.

As it is a rare example of aspect-oriented case studies, the methodology
employed by \citeauthor{ionescu2009aspect} is important to highlight. Their
evaluation measures code quality improvements, which are commonly claimed in
aspect orientation research\revnote{add citations where code quality
  improvements are claimed in AOP research}. Their code
analysis makes use of significant lines of code as a core metric, which doesn't
reliably reflect code quality. As \citet{rosenberg1997some} explains:

\begin{displayquote}
    (\ldots{})~the best use of SLOC is not as a predictor of quality
itself (for such a prediction would simply reduce to a claim about size, not
quality), but rather as a covariate adjusting for size in using another
metric.
\end{displayquote}

Although \citet{ionescu2009aspect} evaluate code quality, the methodology
employed to measure improvements is unreliable.

Improvements in code quality are those which have come under scrutiny by the
critical papers reviewed in \cref{subsec:aop-criticisms}. The results presented
by \citeauthor{ionescu2009aspect} do not satisfy critics' requests for empirical
evidence of improved code quality. This does not impact their aspect-oriented
models' viability: their study demonstrates that their models can be augmented
to support new supercomputing environments without lack of performance. The
models described in this work satisfy that aim: their models are also evaluated
in their performance. Model performance is a priority in supercomputing
contexts, where execution time is financially expensive and energy-intensive.
Quantitative evaluation of their models' execution time shows less than 5\%
slowdown compared to a non-aspect-oriented implementation.
\citeauthor{ionescu2009aspect} deem this a reasonable trade-off for the
engineering improvements they observe.

\citeauthor{ionescu2009aspect}'s application of aspect-orientation to
supercomputing \& disaster prevention simulations meet their performance
requirements and demonstrate a modelling technique which adapts existing models
for use in new environments without directly modifying pre-existing source code.
This result is notable with regards the contributions presented in this thesis, which
similarly aim to augment a pre-existing model without directly modifying its
source code --- though this thesis' targets model reuse and design
simplification rather than for avoiding the regulatory overhead and financial
cost of maintaining models which run on supercomputers.




%%%%%
%%%%% MARK MARK MARK here when redrafting 09-10-2023
%%%%%




\subsection{Aspect Orientation \& Business Process Modelling}\label{subsec:ao_and_bpm_review}


% As business processes models represent a kind of \sociotechnical system, and
% this thesis offers tooling for and results in the modelling of \sociotechnical
% systems, . Additionally,
% related work undertaken before this PhD develops on software engineering
% processes that lend themselves well to the same modelling paradigms as business
% processes (see \cref{chap:prior_work}), and there also exists interest in
% modelling behavioural variance within the business process modelling community
% (see \cref{sec:dynamism_in_sm}). This overlap necessitates a review of related
% literature within the business process modelling field. \inline{Add Charfi \&
% Cappelli's work before Jalali's in this subsec}

Several projects within the business process modelling research community make
use of aspect orientation to design modelling languages which produce less
monolithic business process models~\cite{Cappelli_AOBPM,da2020implementation}
and simplify the composition of models~\cite{charfi2007ao4bpel}. Business
process modelling research is relevant to this thesis' contributions, as
business processes are inherently \sociotechnical and later chapters present
tooling for and results in the modelling and simulation of \sociotechnical
systems using aspect-oriented techniques. In addition, some research conducted
prior to this thesis developed software engineering processes that are
conceptually similar to business process modelling (see \cref{chap:prior_work}).
There also exists interest in modelling behavioural variance within the business
process modelling community (see \cref{sec:dynamism_in_sm}), which is relevant
to this thesis' concern with the aspect-oriented representation of changes to
processes and modelled behaviours.

\revnote{past tense again - the paper has been written/published - foresaw, observed, described...}
\citeauthor{charfi2007ao4bpel} see opportunities in integrating BPEL, an
executable business-process modelling language, with aspect-oriented
concepts~\cite{charfi2007ao4bpel}. This is because when BPEL systems are
composed together the static nature of the logic being composed is not always\revnote{what do you mean by static nature? not clear}
appropriate for BPEL's use cases. The specific use-case examined is web service
definitions, where changes affecting composition of multiple component parts can
affect many areas of a final result, making modification error-prone. They\revnote{Can you give am example}
specifically seek to support dynamic workflow definitions --- ``adaptive
workflows'' --- which BPEL's existing extension mechanisms do not sufficiently
support, but the aspect-oriented literature discusses at length (an overview of
which is presented in \cref{sec:dynamic_aop_review}). Therefore, they look to
construct an aspect-oriented BPEL extension. Using the case study of modelling a
travel agency's web services, they create an aspect-oriented extension by first
defining how such an extension would be represented graphically in BPEL's
workflow diagrams. Further detail is added to arrive at a technical definition
with XML representations, weaving mechanics, and eventually the construction of
a BPEL dialect, AO4BPEL. The authors find that their pointcut system (which
describes join points on both processes and BPEL messages), support for adaptive
workflows, and aspect-oriented approach to workflow process modelling make
AO4BPEL unique at the time of publication, though related AOP implementations
exist in each individual area of their contributions. The work is weakened by
brittle semantics around pointcuts, join points, and the temporal nature of
workflow modelling. For example, they note that defining contingent behaviour
--- only applying an aspect conditionally, based on a trace through a simulation
of a modelled system --- would allow the application of advice only when model
state deems this appropriate.\footnote{The contingent application of model
adaptation is a motivating case for some work presented in this thesis; see
\cref{chap:prior_work} for a discussion.} They also call for more generally
theoretical AOP research, which mirrors the requests some critics of aspect
orientation research make (as noted in \cref{subsec:aop-criticisms}). \revnote{this seems relevant argument so avoid footnotes?}

\revnote{Don't need to state pub type, just name the author}
In a PhD thesis describing AO4BPEL in detail~\cite{Charfi2006AspectOrientedWL}
\citeauthor{Charfi2006AspectOrientedWL} presented a generalisation of the
notation developed for AO4BPEL, which applies to any graphical workflow
modelling language. Accompanying this are some examples of its use building a
framework for enforcing certain requirements of BPEL models, and use of that
framework to develop aspect-oriented frameworks for enforcing security and
reliability within AO4BPEL models.

In later work, \citeauthor{charfi2010AO4BPMN} define a similar aspect-oriented
dialect of BPMN they name AO4BPMN~\cite{charfi2010AO4BPMN}, after asserting that
the concerns addressed by
AO4BPEL~\cite{Charfi2006AspectOrientedWL,charfi2007ao4bpel} in the field of
executable process languages also apply to business-process modelling languages,
and can be solved similarly. The generalised notation of aspectual
workflow models presented in \citeauthor{Charfi2006AspectOrientedWL}'s
thesis~\cite{Charfi2006AspectOrientedWL} are applied to BPMN to produce an
aspect-oriented language specifically for process modelling, as opposed to
executable business process modelling. \revnote{can you say something explicit about the difference and benefits of this?}

\Citeauthor{Cappelli_AOBPM} also note that cross-cutting concerns exist in
business process models, and are specifically motivated by monolithic design
approaches common in business process modelling languages. Like
\citeauthor{kiczales1997aspect}, they claim that a lack of modularity in
business process models leads to cross-cutting concerns scattered throughout a
model~\cite{Cappelli_AOBPM}. To alleviate the issue, they propose a
meta-language, AOPML, which incorporates aspect orientation in a metamodel of
business process modelling languages, and instantiate it within their own
dialect of BPMN. Using a model of a steering committee as a case study, and
separating cross-cutting concerns such as logging, the paper proposes reducing
complexity and repetition graphically, thereby in a manner more in keeping with
the language design philosophies of popular business process modelling
languages, the design and use of which are typically
graphical~\cite{OMG-BPMN-SPEC,opm_original,OMG-UML-SPEC}. They note that this is
in contrast to other applications of aspect orientation in business process
modelling --- specifically AO4BPMN --- where aspect definitions are written in
XML concern not only the advice to be applied but also their relevant join
points, as in general programming aspect orientation implementations such as
AspectJ. In this way, the AOPML exhibits the spirit of business process\revnote{Can you say something more precise than spirit here, e.g. objectives, intentions at least?}
modelling more stringently than does \citeauthor{Charfi2006AspectOrientedWL}'s
notation for aspect-oriented workflow modelling.

The difference between \citeauthor{charfi2010AO4BPMN}'s approach in designing
AO4BPMN~\cite{charfi2010AO4BPMN} and \citeauthor{Cappelli_AOBPM}'s approach in
designing AOPML~\cite{Cappelli_AOBPM} highlights design decisions taken when
introducing aspect orientation in a new domain. There is an opportunity for a
domain-specific aspect orientation framework to align its design with the
traditions and idioms already present in models within that domain, but doing so
may break the traditions and idioms which already exist in aspect-oriented
approaches in other domains.  Comparing the approaches of
\citeauthor{charfi2010AO4BPMN} and \citeauthor{Cappelli_AOBPM} does surface\revnote{demonstrates} that
there may be no clear ``best'' design approach when blending pre-existing
modelling paradigms, such as the graphical modelling languages used in
business-process modelling and the abstract concepts of aspect orientation. The
discussion around whether it is more desirable to adapt existing
design elements of aspect-oriented frameworks to a given domain or adapt that
domain's existing modelling traditions and idioms\revnote{conventions, syntax - again be more precise} to incorporate aspect
orientation as it is used elsewhere is outside the scope of this thesis.

New concepts within the design of aspect orientation frameworks are addressed in
the business process modelling community. \Citeauthor{jalali2012aspect} note
that aspect oriented modelling frameworks often do not explicitly model the
precedence of aspect application~\cite{jalali2012aspect}. They address this
limitation by defining a mechanism to be used in capturing multiple concerns as
aspects, where the invocation of advice must follow a certain precedence. The
aim of the work is not to propose tooling around the precedence of aspect
application so much as to contribute to aspect oriented design theory, providing
a notation for precedence which is broadly applicable. The precedence model is,
put simply, that a mapping exists for each application of advice to join point
such that the mapping defines an ordering on advice for that join point. The
definition defines ``AOBPMN'', a formalised dialect of BPMN supporting aspect
orientation with precedence. A case study is provided where AOBPMN is
instantiated within a coloured Petri net. Their study expands on existing work
by research teams led by Capelli~\cite{Cappelli_AOBPM,da2020implementation} and
by Charfi~\cite{charfi2007ao4bpel}, in that it develops a mature formalism for
and model of aspect orientation as applied to business process modelling.
However, \citeauthor{j alali2012aspect} note that their case study is limited in
scale. No tooling or evaluation of the practical benefit of their approach is
provided.\revnote{add something early on giving an example of how it might be a problem?}


\section{Process Variance in Simulation \& Data Generation}\label{sec:dynamism_in_sm}

Brief intro of the section here~\inline{Brief intro of the process variance
section here}\revnote{Agree, builds well from preceding sections though}
%% Note motivating modelling with variance:
% - representing contingent behaviour
% - generating data with variation, for i.e. process mining.

% First review here should probably be the Aalst-co-written one on executable
% BPMN for log generation with variance

Typically, a simulation concerns a single process.\revnote{Not sure this is true - e.g. MAS?}%
This means that all expected
behaviour must be included within that process; complex or contingent behaviour
must be represented within it. The techniques reviewed here offer separately
including some modification of a process (or represent the modifications of
varied process within the simulated
output~\cite{stocker2013secsy,stocker2014secsy}). The benefit of this approach
is that possible changes to a process can be described once and applied to that
process where appropriate. Process changes might describe attempts to circumvent
security protocols, laziness or confusion in a human actor within the model, or
random ``noise'' so as to produce synthetic log traces containing aberrations
which mimic those found in noisy empirical datasets. In all cases, behavioural
variations can be described as some alteration to a process and applied to
either a model or the product of that model (datasets or log traces) to
represent the same alteration introduced at an arbitrary point of the
simulation.

This decouples the expected behaviour in the original model from simulated
behaviour, which is obtained by composing the model and behavioural variation
using a given technique's method for doing so. This approach to modelling
behavioural variation allows the same altered behaviour, which would otherwise
be described in many disparate points in a model, to instead be written once and
introduced wherever required. The observation that the same variation might
appear in many areas of a model, and that the variation can be separated from
the model and introduced where necessary, frames the modelling of these
variations in the same way as aspect orientation frames cross-cutting concerns.
The work presented in this thesis explicitly applies changes to processes and
simulated behaviour as aspects in the same manner. Therefore, although this
aspectual connection is not made explicit in much of the literature to date, it
is important to review literature on simulation and modelling which modularises
these variations; this section reviews that literature. The work reviewed is
highly relevant to the contributions in this thesis, in particular because the
core motivations of this field are shared by this thesis; the section therefore
leads with a subsection discussing those motivations, and their relationship to
aspect orientation, in detail.


\subsection{Discussion of Variation \& Motivations for Variations in Process Models}\label{subsec:variation_sm_motivations}


\citeauthor{ExecutableBPMNMitsyuk} are motivated by the field of process
mining's requirement for datasets of process logs made from well-understood
process models, defined in a high-level manner~\cite{ExecutableBPMNMitsyuk}.
They demonstrate a technique for generating event logs from BPMN models by
introducing algorithms for the direct simulation of BPMN models and the
collection of traces from those simulations. While their approach does not
support the simulation of all BPMN concepts, notably message passing, they
provide a tool which produces log traces for a BPMN model through PROM, a
standard tool within the process mining community~\cite{van2005prom}. This
results in their technique providing high-level model simulation through
already-standard tooling, meaning adopters of the technique need not rely on
dedicated tooling which may not be compatible with other researchers' process
mining techniques.

The algorithms presented by \citeauthor{ExecutableBPMNMitsyuk} simulate
processes described by BPMN models, but don't include any provisions for
representing variance. However, the technique could plausibly\revnote{remove plausibly - it can or it can't}%
be combined with
aspect orientation techniques for BPMN as discussed in
\cref{ao_and_bpm_review}~\cite{charfi2010AO4BPMN,Cappelli_AOBPM} to represent
alternate behaviour applied contingently. Demonstrating the viability of this
approach is an avenue of research beyond the scope of this thesis. However, the
motivation of the work mirrors that of other research projects reviewed in this
section: a need for synthetic datasets of traces through a process, for use in
scenarios where empirical datasets are difficult to obtain.\revnote{Has this motivation already been explained? If not re-order with below first.}

Difficulties arise when obtaining real-world datasets for many reasons. For
example, large empirical datasets are typically produced by organisations which
would prefer some level of secrecy around their operations, making publishing
those operations for the investigation of research teams unlikely. Researchers
collecting these datasets describe a \textquote{lengthy
process}~\cite{bpi_ten_years_of_datasets} and explain that traces of real-world
processes are hard to obtain because \textquote{higher management [can be]
worried about the risks} of publishing such
datasets~\cite{bpi_ten_years_of_datasets}. Another factor contributing to the
difficulty of collecting empirical datasets is that they often cannot be
collected, either because there is a need to study the process before
implementing it (making synthetic datasets the only option available to
researchers \inline{Find citation for empirical datasets being the only ones
available to researchers, maybe for disaster prevention or similar?}), because
the process is not yet fully understood (making simulation of many variants of
that process useful in aiding understanding~\inline{find a citation for
simulating different systems for finding an optimal one. Arguably Genetic
programming \& hill climbing do similar things?}), or because the dataset itself
is of use to researchers, not the real-world system that produced it (such as in
the case of evaluating process mining
techniques~\cite{van2004process,agrawal1998mining}). \inline{Find some nice way
to round this subsection out, or refactor out subsections}~\inline{Important
to acknowledge somewhere in this subsection that synthetic data generation is a
well-researched field, but that generating logs from simulations with variance
is the specific area relevant to this thesis} \revnote{refactor by re-ordering - explain the motivation first and then intro the approach of charfi}

\subsection{Representing Variations in Process Models and their Outputs}\label{subsec:variations_in_sm}

Research undertaken by \citeauthor{stocker2013secsy}~\cite{stocker2013secsy}
aims to synthesise process logs which are representative of attackers' efforts
to compromise the security of a modelled system. Their research project, named
``SecSY'', is an attempt to address issues arising from the difficulty of
retrieving representative log traces for security-critical systems in which
attacker activity is present. Logs are developed by process simulation through
``well-structured'' models, a mathematical property on which transformations
were previously defined by
\citeauthor{vanhatalo2009refined}~\cite{vanhatalo2009refined}. The authors
develop a tool for the simulation of a process using well-structured process
models, and apply transformations to both the model before execution and the log
it produces through the trace of a simulation. They conclude that their tool is
performant, and verify it can produce logs\revnote{with what particular characteristics? e.g. that plausibly mimic real world logs of real security violations?} representing security violations by
way of analysis through PROM, a popular framework for process mining, and
pre-defined security constraints on their models. They note \revnote{is this a however, i.e. a limitation?} that log traces
cannot be interleaved (due to a lack of parallel simulation of processes), may
be incomplete (missing violations), and that mutated models and traces are not
guaranteed to be sound by construction. However, they see their proposal as a
necessary step in realistic data generation for business processes. A \revnote{further} weakness
of the work is that model and trace modifications \revnote{techniques} are relatively rudimentary \revnote{say limited instead of relatively rudimentary?}:
processes can be added or removed, but complex graph transformations are
presumably \revnote{why only presumably?} only permissible when representable through the composition of the
mutation primitives they provide, on which there are only three for processes:
swapping \lstinline{AND} and \lstinline{XOR} definitions of process gateways, and swapping process
order. Mutations cannot be applied contingent on the state of a simulation run,
for example, representing a decision taken by an attacker based on what had
already happened.\revnote{new paragraph?} In later work, \citeauthor{stocker2014secsy} detail the
technical aspects of SecSY, their tool for implementing the generation of
synthetic logs which use their technique~\cite{stocker2013secsy} to represent
security violations \revnote{of?} security-critical business processes. A Java implementation
of SecSY is described, which simulates well-structured models and applies
mathematically-defined transformations on the model being simulated (before
simulation occurs) and the logs obtained through simulation traces. An
improvement on earlier work is that custom transformers can be written. However,
a limitation of the original work remains, which is that users cannot easily
dictate the degree to which variations are applied.


\citeauthor{pourmasoumi2015business} also address the need for access to
variations on business processes, though for the development of a research
field, ``cross-organisational process mining''~\cite{pourmasoumi2015business}.
Process mining can require many process logs, as does the benchmarking and
evaluation of process mining techniques. Traces from business processes which
are similar but not identical can produce log traces which reflect that
similarity, but also reflect the variations in different instances of those
processes. These log traces exhibiting variation can be used in the training and
analysis of process mining tooling and techniques, which must contend with
natural variation present in the execution of real-world traces. To support the
field, log trace generation from a variety of process models is therefore
required. Such logs are not in adequate supply, as explained in
\cref{subsec:variation_sm_motivations}. The authors' approach to the problem is
to present an algorithm for the mutation of business processes, such that
simulation against variations of the business process can produce process logs
reflecting those variations. Their algorithm makes use of structure tree
representations of processes, which models processes as trees and permits
conversion to BPMN models and Petri nets~\cite{buijs2014flexible}.
\citeauthor{pourmasoumi2015business} make use of this constraint to demonstrate
that their models are block-structured, a mathematical constraint on model
structure which 95\% of models are shown to comply with~\cite{chenthesis}. Their
contribution is a set of transformations on structure trees and block-structured
models, and an algorithm applying these transformations to process models, and a
tool which implements it built on PLG, a process log generation tool. They
conclude that tools such theirs can be used to generate log traces representing
process variation, in such a manner as to satisfy the requirements of the
process mining research community.

\citeauthor{pourmasoumi2015business} describe a list of transformations \revnote{be explicit - on what. Also start with the overall thrust of the work, not the specific contribution} they
explain is \textquote{not intended to be
comprehensive}~\cite{pourmasoumi2015business}, which makes its full potential
unclear. Additionally, processes their tool applies to must be block-structured.
The importance of this requirement is that it limits their technique \revnote{as per first comment on this paragraph - what technique?} to business
process models. It is not demonstrated that models of processes in other domains
satisfy the condition, such as the flow of data~\cite{obashimethodology} or
behaviour of human or technical actors in \sociotechnical
systems~\cite{wallis2018caise}. Finally, the tool is limited by its lack of
capacity to represent variations which are applied contingent on system state. A
use case for modelling behavioural variance is to model changes which are
impossible to anticipate from the vantage of a modeller, such as a
\sociotechnical system's human actors' mistakes, security exceptions and
violations, or corrective actions to mitigate undesirable system state.
\citeauthor{pourmasoumi2015business}'s tool produces variations on a process
model, but modelling behaviour which is expected to vary within iterations of
the \emph{same} model is outwith the scope of their project.
\inline{Shugurov \& Mitsyuk's work should follow this. REVIEW IT.}


%% Shugurov & Mitsyuk here
% Mitsyuk_2016
% shugurov2014generation


\citeauthor{Machado_2011} note that there are operational costs to the
inefficient modelling of business processes. Specifically, they note that
processes can be replicated across automated processes\revnote{unclear here}, and failure to identify
such scenarios give rise to these operational costs. This work's core motivation
is that the representation of variation in process models would allow for the
capture of a replicated process once, with instances of similar processes
described as deviations from that captured blueprint. On this basis,
\citeauthor{Machado_2011} extend BPMN to support the notion of individual
processes as transformations of an underlying process, i.e. that a given process
can sometimes be expressed as a deviation from some pattern, and is therefore
define-able as the composition of that pattern and a variation upon it. Their
approach is illustrated through two small, broadly similar business processes
initially modelled in BPMN and then represented in Haskell, allowing the authors
to demonstrate their representation of variability as process deviation with
realistic examples. While the work presented makes no empirical evaluation of
their technique, \citeauthor{Machado_2011} note that their industrial partner
responded positively to the research presented in this publication and that
further technical improvements are to be made (support for around advice, and
for quantification). They also express an intent to conduct a real-world
evaluation in the HR domain. While we are unaware of any real-world evaluation
undertaken by this research time to date, some formal proofs that the
transformations their tool supports are always well-formed \revnote{have been developed}~\cite{machado2012formal}.


\section{Discussion}\label{sec:lit_discussion}


\revnote{This sentence was salvaged from the simulation \& modelling section.
  It's OK but was waffly there. Maybe it can fit in this discussion
  somewhere?}This thesis draws on the idea that, in response to
\citeauthor{steimann06paradoxical} asking what aspect orientation is good for,
\citeauthor{gulyas1999use} would seem to answer, ``simulation \&
modelling''.\revnote{Worth observing that \citet{ionescu2009aspect} actually
  produce a case studdy of using AOP and their code quality evaluation uses SLOC
(unreliable), but they do find that they can augment models successfully (in
their context, successfully is without significant loss of performance). Backs
up claim in \citet{gulyas1999use} that AOP is useful for simulation \& modelling
and claim in \citet{steimann06paradoxical} that AOP ``doesn't know what it aims
to be good for'' (paraphrased, it's quoted in the ciriticisms section and used
elsewhere too). There's loads of AOP lit but we need to explore using it in new
ways, and actually buildinng models \& running simulations using it can be
motivated by existing literature. I don't believe I make reference to
\citet{ionescu2009aspect} in this section at present.}

% Points to make here:
% \begin{itemize}
% \item Aspect orientation allows for a clean separation of concerns. There has
% been a lot of promising research in the field. 
Aspect orientated programming is designed to permit highly modular software
engineering in scenarios where cross-cutting concerns are identified, by
isolating them as separate modules~\cite{kiczales1997aspect}. The aim in
employing aspect oriented programming is to reduce tangling, where a
cross-cutting concern is intertwined within a program's main concern, and
scattering, where the same cross-cutting concern is re-written at many points in
a program's source. Aspects which modularise that concern can be written once,
separately from a program's main concern, and re-introduced to each point in a
program to which the aspect relates by way of weaving. An aspect orientation
framework must therefore be able to quantify the points at which the aspect
applies~\cite{filman2000aspect}. Aspects specify both advice (the implementation
of its associated cross-cutting concern) and the join point defining where in a
target program that advice should be woven. Aspect orientation therefore implies
that the source aspects are applied to are oblivious to their
application~\cite{filman2000aspect}.

% \item However, we lack empirical results which show the paradigm delivers what
% its original proponents claim.
In theory, the design of the paradigm is such that it should be expected to
increase modularity in the software applying
it~\cite{kiczales1997aspect,filman2000aspect}\inline{more citations}, and its
proponents often claim this modularity as a benefit of the aspect-oriented
approaches of their
research~\cite{gilani2004family,charfi2007ao4bpel,Cappelli_AOBPM,jalali2012aspect,chibani2013toward}.
However, its critics question the reasoning around these benefits, and note that
there is little empirical study into whether aspect oriented programs truly
benefit as a result of this
modularity~\cite{Constantinides04aopconsidered,steimann06paradoxical,przybylek2010wrong}.

% Lead into variance and why there's potential for aop to work, although we'd
% need some kind of empirical proof that it did, partly becaue of aop's track
% record and partly because we'd be automatically rewriting models so they might
% not reflect what their authors intended anymore if we're not careful.
One appropriate application area may be in the representation of behavioural
variation in simulation and modelling. The application of aspects to models is
already well-studied~\cite{DEVSaspectorientation2008aksu,chibani2013toward},
particularly within aspect-oriented business process
modelling~\cite{charfi2007ao4bpel,Cappelli_AOBPM,jalali2012aspect}, where
modelling behavioural variation has also seen some prior
research~\cite{Machado_2011,stocker2013secsy,pourmasoumi2015business,ExecutableBPMNMitsyuk}. 
Outside of business process modelling, aspect orientation would reasonably be
expected to support the development and observation of models
themselves~\cite{gulyas1999use}. 

Research opportunities at the intersection of aspect orientation and the
modelling behavioural variation occur because behavioural variation is an
example of a cross-cutting concern. Changes to expected behaviour such as
laziness, boredom, confusion or misunderstanding can impact many parts of a
process in a \sociotechnical system, but modelling the variation caused by any
one of these requires similar alterations to behaviour in many
disparate parts of the model they occur within. Behavioural variations are
therefore both scattered and tangled, and constitute a cross-cutting concern
which might be well suited to modularisation in aspects.

% NEXT: note the weaknesses in existing work and what needs to be done to show
% that AOP and behavioural variation in S&M do work well together.
Existing aspect orientation techniques and behavioural variation modelling
techniques are ill-equipped to take advantage of this alignment. That behaviour
changes when it varies is tautological; however, changes supported by existing
aspect orientation techniques weave advice before, after, or around their join
points, and therefore do not alter the definition itself. In some systems, some
variations may be representable as additions inserted before and/or after some
other behaviours, but such techniques are unsuitable for representing
modifications of the target behaviour itself, or behaviours which should be
omitted instead of added. Additionally, join points available to an aspectual
programmer may not be granular enough to permit representing the changes they
require in such as system, and aspect orientation's principle of obliviousness
opposes the modification of target code to make new join points available.
Techniques which would directly rewrite target source are typically extremely
low-level, and therefore ill-suited to most modelling
applications~\cite{keller1998binary}. Other techniques which permit defining
changes to processes at a high level may allow a modeller to \emph{describe}
intended changes (such as the high-level annotations supported in
AOPML~\cite{Cappelli_AOBPM}), but such techniques are intended for human
interpretation, not machine execution for simulation purposes. These techniques
permit describing behavioural variation within another process, but only by
virtue of the flexibility of natural-language annotation, and are therefore
unsuitable for simulation and modelling purposes.

Such techniques also lack executable notion of ``state''. Real-world behavioural
variance can often be contingent on the environment an actor exists within.
While variations such as security breaches might be predictable (by identifying
weaknesses in existing processes, for example), variance in \sociotechnical
systems often occurs in the behaviour of human actors. This might be in response
to a degraded mode~\cite{johnson2007degradedmodes}, where behaviour naturally
drifts to a functional --- but undesirable --- state, or due to an individual
making a mistake, forgetting procedure, or being in a state which alters their
behaviour, such as tiredness or drunkenness~\cite{aranTheatreThesis}. A
framework for modelling behavioural variation using aspects should therefore
apply that aspect to a simulated system contingent on the state of that system
at a given point in time. High-level modelling technologies such as BPMN and OPM
are executable~\cite{ExecutableBPMNMitsyuk,opm_original}, but it is not trivially
evident that executable versions of these technologies are compatible with
aspect-oriented modifications of their modelling
language~\cite{charfi2010AO4BPMN,Cappelli_AOBPM}. As noted, low-level program
transformation technologies are also unsuitable. Techniques for applying
variations to models exist, but are unsuitable for simulation (and therefore
cannot represent application based on temporal state)~\cite{stocker2013secsy},
produce individual models representing each possible instance of a
variation~\cite{pourmasoumi2015business}, do not support the dynamic weaving of
aspects for contingent behaviour~\cite{Machado_2011}, or attempt to represent
the changes one would expect in the output of a simulation by executing an
unmodified simulation and amending its output
directly~\cite{shugurov2014generation}. None of these techniques are suitable
for representing a behavioural change which is applied contingent on state.
Incidentally, these techniques for modelling behavioural variation also lack
support for the alteration of a process definition, or changes ``inside'' a
process definition, as discussed earlier, which also makes them unsuitable for
simulated behavioural variation.

\revnote{This is good, but don't be speculative like this.  Draw a conclusion}
Should a tool exist which dynamically weaves definitions of behavioural
variations for contingent application and which is capable of expressing changes
within a join point rather than before or after it, a challenge would arise as
to demonstrating the benefits the tool achieves. Contingent application of
behavioural variation, and the ability to define changes to processes
specifically, would fulfil the opportunity in marrying aspect orientation and
modelling behavioural variation, but the benefit offered by such a tool is
unclear. The introduction of oblivious modification to a model may break its
representation of its real-world analogue, making such a model difficult to
reason about. Added to this is the complexity of the tool's capacity to rewrite
any join point's definition. Though aspect oriented literature often lacks case
studies demonstrating the benefits of the approach, it is particularly important
to investigate whether such a tool could produce realistic models, and whether
the expected benefits of aspect orientation as applied to the model hold in
practice. In particular, would engineers benefit from the modularity afforded
by isolating behavioural variation into an aspect, and would the resulting
aspectual modules be re-usable when modelling other systems?



\subsection{Research Questions}\label{subsec:rqs}

Not sure exactly how to write this. Eeep!


% \item Criticism of aspect orientation notes that there is no clear application
% area for the paradigm (and some areas where it _is_ applied seem unsuitable)
% \item One promising area of application is in modelling, as demonstrated by its
% use in business-process modelling literature.
% \item in addition, in business-process modelling literature we find research
% motivation for --- and research towards --- the modelling of \emph{variations}
% on behaviour. These behavioural variations appear well-suited to modelling as
% aspects, as they typically separate the variation from the domain model in which
% variation is observed, and a given variation might occur in many areas of a
% model. 
% \item However, projects modelling these variations do not support contingently
% applied behavioural variation.
% \item Also lacking in the literature is a method for applying transformations
% \emph{within} a process using aspects. Aspect orientation techniques apply
% alterations before, after, or around their targets.
% \item We therefore see a need for:
%     \begin{enumerate}
%         \item Behavioural variation modelled as aspects
%         \item Tooling supporting a more ``natural'' representation of variation,
%         i.e. a variation of a process applied to that process (rather than
%         additional behaviour appended to its start or finish)
%         \item Contingent behavioural variation, i.e. variations applied during
%         simulation execution
%         \item Empirical evidence that aspect oriented models:
%         \begin{enumerate}
%             \item Allow separation of behavioural variance into aspects
%             \item Allow new modelling techniques, such as
%             compositionally-defined hypothesised behaviour
%             \item Provide the typically defined benefits of aspect orientation,
%             i.e. reusable components in different modelling scenarios (i.e.
%             aspects are ``oblivious'' to the codebase they are applied to)
%         \end{enumerate}
%         \item In addition, we must be certain that a model with variations
%         applied still represents the originally specified model. That is to say,
%         we want to fulfil the opportunities observed in the literature, while
%         demonstrating that applying variation to a model in order to more
%         realistically represent its domain (a model of human behaviour including
%         varied behaviour the mistakes we'd expect to see exhibited by human
%         actors is arguably more ``realistic'' than one without those mistakes)
%         can be done in a ``safe'' way, i.e. modifying the model programmatically
%         does not change the model in a way which makes it \emph{less} reflective
%         of reality by mistake.
%     \end{enumerate}
% \end{itemize}

% The underlying hypothesis is:

% # Aspect orientation is well suited to modelling behavioural variations

% In order to investigate this hypothesis, we must investigate:

% ## Can we modify a model while keeping it at least as plausibly realistic as the
% original (if not more so)?
% ## Can aspect orientation provide the benefits claimed in the paradigm's
% literature in the simulation and modelling domain?

% To support those investigations, we need:

% ## A suitable system to model in which we can assert the ``realism'' of a model
% before and after the application of variance
% ## Aspect oriented tooling which supports application of behavioural variation
% ## Ideally also, aspect oriented tooling which addresses the shortcomings of
% other aspect orientation frameworks (no ``inside'' join point, engineering
% difficulties such as obliviousness' impact on legibility)



% Old research questions were:

% 1. Can we fit model details per-player to get realistic player behaviour?
% 2. Can we cluster based on accuracy of different models for each player?
% 3. Can we generate predictive data for unseen models?

% The contributions we want to demonstrate are:

% 1. Behavioural variation is well-represented by aspect orientation
% 2. Tooling which makes modelling behavioural variation feasible using aspect
%    orientation
% 3. Tooling which improves on aspect orientation's shortcomings (to make AOP
%     more viable to apply to s&m in practice)
% 4. Showing that data produced by aspects is model-specific 


% First draft new research questions:

% 1. Can we demonstrate that a model of human actors can be made realistic
%    through aspect oriented behavioural variation (rather than breaking the
%    model's representation of its domain)?
% 2. Can we demonstrate that our aspects represent real-world variation in
%    actors' behaviour, rather than the emergent change to an overall system?
%    (by clustering, we demonstrate that many people exhibit the _same_ changes,
%    but not everybody does, i.e. the aspect accurately represents a subset of
%    our modelled actors)
% 3. Can aspects applied successfully to one model be expected to see the same
%    success when applied to other models?

% Research questions found in an early .md file detailling the thesis structure:
% - RQ: Can we fit model details per-player to get realistic player behaviour?
% - RQ: Can we cluster based on accuracy of different models for each player?
% - RQ: Can we generate predictive data for unseen models?

%%% Local Variables:
%%% mode: latex
%%% TeX-master: "../thesis"
%%% End:
