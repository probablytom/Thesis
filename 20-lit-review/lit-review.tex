\chapter{Lit review}

% ================================================================================
% A brief outline of what this chapter is trying to get across.
\section{Overview}

Systems-of-systems (SoS) modelling is a deeply interdisciplinary study. This is a blessing and a curse.
\par


% ================================================================================
\section{Extant systems}
These are the systems that exist.


\subsection{Bigraphs}


\subsection{Obashi}


\subsection{Responsibility modelling}


\subsection{Specification modelling}
% Particularly in Z


\subsection{Model checkers}
% Prism, Spin, Uppaal


\subsection{Soft Systems Methodology}


\subsection{Modaf \& Dodaf}


\subsection{Object-oriented models}
% Model ontology represented as objects


\subsection{Functional programming simulations}
% Model ontology represented as data structures and transforms on them
% Maybe mention the possibie directions with session types and HoTT?


\subsection{Actor oriented models}
% Mention that processes can be actors. CCS, π-calc and CSP are all relevant and important here.
%     Particularly, channels in Go/Spin/Erlang vs async&await in Python/Javascript
% Note that workflow modelling is its own section.

% Notes from bearapp:
% * Software engineering
% 	* Object orientation
% 	* Functional programming
% 		* Models & ontologies as data structures
% 	* Processes as actors
% 		* Robin Milner
% 			* π-calc
% 			* CCS
% 		* Tony Hoare
% 			* CSP
% 		* Channel-based communications
% 			* SPIN
% 			* Golang
% 			* Erlang & BEAM
% 		* Await/Async
% 			* Python
% 			* Javascript


\subsection{Graphical Workflow Modelling}
% Lots of models of workflows specifically that are worth mentioning

% Notes from bearapp:
% * Workflow modelling
% 	* YAWL
% 	* Theatre & PyDySoFu & FuzziMoss
% 	* Petri Nets
% 	* Graphical progress languages
% 		* SysML
% 		* Activity diagrams
% 		* OPM
% 		* BPMN
% 			* BPEL


\subsection{Automata}
% For agent-based modelling, even if it's usually relatively simplistic
% Cellular automata. What else?


\subsection{Holons and Property Based Approaches}



\subsection{Systems Theory}
% Notes from bearapp:
% * Systems theory
% 	* Applied systems theory
% 		* & related approaches?
% 		* Relate to property-based / API-based composition a-la holons
% 	* Systems of Systems
% 		* Family of Systems
% 	* Specific writers
% 		* Edward Deming
% 		* Michael Polanyi
% 		* Karl Ludwig von Bertenlaffy
% 		* Eric Trist & Fred Emery


\subsection{Safety Critical Systems}
% detail-oriented but essential for actual engineering. These have similar
% requirements to us!
% What can we learn from them?

% Notes from bearapp: 
% * Safety-critical systems
% 	* Fault trees
% 	* IEC 61508
% 	* FMECA
% 	* HAZOPs
% * OSGI
% 	* Different layers represent a semiotic separation, which is useful in separating different degrees of technical and social/human elements of information, making processing and encoding vastly easier at scale
% 	* A good argument for separating the formal model and its visualisation/editing!

\subsection{Summary of Techniques}
Recap by describing what we've seen in terms of extremely detail-oriented, extremely high-level, and in-between this tradeoff.
  
\subsubsection{Formal methods}\label{sec:review-formal-methods}

Approaches like model checking, and touch on Bigraphical notations
\comment{constraint models?}
\par

% --------------------------------------------------------------------------------
\subsubsection{Engineering approaches}

In engineering --- in any flavour, not just in computing science --- we have to have a rigorous understanding of the systems we build. 
These systems often approach very large scales, however, and grow over time.
This increases the degree of detail requires disproportionately to the decrease in rigour required in comparison to formal approaches as in \ref{sec:review-formal-methods}.
\par

New challenges also arise. Formal methods approaches rarely need to contend with added complexities such requirements engineering, but in a field such as software development this is a vital component of the engineering process, and \emph{must} be included in any useful model of the craft.
\par


% --------------------------------------------------------------------------------
\subsubsection{Informal modelling}
Some modelling approaches are particularly informal; this is useful in scenarios where what is being modelled has to be understood, but is either not required to be captured in a great degree of detail or does not readily offer detail to be modelled at all.
The prime example of this is organisational and modelling, where capturing and representing information is complicated to do with any formal approach.
Detailled approaches fail to contend with the ill-defined, ``fuzzy'' nature of business organisation.
\par

One such ``fuzzy'' method lacking detail might be a holon-based approach\maybe{, similar to dependency injection,} which defines the ``properties'' required for a system to operate and otherwise treats systems as blackboxes. 
\comment{This is the approach implicitly taken in managing TV production. Jena, Blair's girlfriend, discussed with me 03/12/2017 their method for managing scheduling and resourcing, where they defined all of the things they were in charge of, their constraints, and their requirements. They write everything on postits, stick them to a whiteboard, and rearrange to try to get overviews of the various things that need sorted and people's responsibilities in the management of the production. Interestingly, they also do SoS modelling implicitly: sometimes there's a 3rd party that they give responsiiblities to, and this 3rd party is a blackbox which effectively represents a part of *their* system's environment/peers in a SoS or FoS. Their own company is who people like the BBC outsource to, so there's hierarchy (as well as inter-relationships --- a combination of SoS and FoS?...). It would be fascinating to study how they organise their own very ill defined process modelling, and compare to business models, engineering models, and formal models.}
\par

That said, more and more detailled approaches are appearing. One such example would be the growing effort to represent business processes via graphical modelling languages such as OPM, SysML and BPMN.
\par

A slightly more rigorous approach, though more limited in scope, can be found in Obashi's dataflow modelling.
\par


% ================================================================================
\section{Themes in Modelling Paradigms}
Certain themes emerge when we're looking at \emph{what} these formalisms contain.


% --------------------------------------------------------------------------------
\subsection{Composition in SoS}


% --------------------------------------------------------------------------------
\subsection{Systems as Metric Spaces}
Metric spaces in systems like Obashi and object-oriented models. Activity diagrams etc?
\par

Metric spaces are \emph{really} useful for high-level analysis, rather than SPIN-esque verification.
\par

Unfortunately not all models easily adapt to a metric space. Consider bigraphs.
\comment{though this shouldn't be too complex to change, and Michele has played with the idea during his own PhD}
\par

% --------------------------------------------------------------------------------
\subsection{Processes}
Processes quickly complicate a model.


% --------------------------------------------------------------------------------
\subsection{System Entropy}
Processes are the only method we have for \emph{reducing} the entropy of a system.
When we constrain or control a system's change over time by imposing processes on it, we often do so with an eye to limiting how it changes and keeping it locked in a certain state or range of states.
\maybe{consider a garden that one tries to maintain as an example of a system fighting not to stay in a given state, and us constraining its change via the process of gardening.}
How do different systems represent change in entropy? How easily is this analysed?
Entropy change seems like a very useful metric of emergent complexity in a system, so the ability to represent this in SoS is pretty vital.
Also, even a little growth in entropy can quickly cause problems for modelling particuarly large systems of systems, because it can cause state explosion which is intractable to represent with the methods shown above. \comment{consider state explosion in formal methods, for example, where a small increase in entropy causes their state space to increase exponentially, which is what makes verification in something like PRISM NP-hard. Entropy really does matter!} 
\par

Entropy is important for engineering reasons, too. Consider software entropy.
discuss the link between software entropy and badly separated concerns / poor modularisation.
High entropy -> difficulties in maintaining and architectural debt.
How is this represented in software engineering? Business engineering? Elsewhere?
\par

\maybe{discuss ``occam's architectural razor''? System components should be
  exactly as simple as they can be without sacrificing functionality to prevent
  unncessessary increase in system entropy}

  

% ================================================================================
\section{Comparisons}

Some methods fail to properly represent a high-level, human-understandable overview of a system and its architecture --- essential for SoS modelling.
This is necessary for both the process of building the model as a human engineer, and understanding it as a layman or somebody who needs to understand the \emph{implications} of the model.
\par

Other methods fail to properly represent sufficient detail to analyse their model, or to verify properties --- essential for predicting behaviour with certainty.
This is particularly vital for understanding whether a system will or won't have certain properties; making sure that there's a low chance of failure over a certain length of time, for example, or resilience to certain threats.
\par
