\chapter{A chapter title here for the experiment moving aspects to new systems,
  or systems with some changes}
\label{chap:exp2_old_aspects_new_systems}

This is a relatively short chapter; a lot of the building blocks for it exist in
the previous chapter, so there's less ground to cover. If it ends up quite
lop-sided, I'd chop the earlier chapter in two rather than artificially making
this chapter beefier; I think it'd flow better.

\section{(Reword) Experimental motivations / motivation of research question}
This section should describe why it's interesting to move an aspect trained on
some actors' behaviours to a new system, and discuss what investigating this can
teach us that we \emph{don't} already know from the previous experimental
chapter. We've got aspects which represent behavioural variance. If the system
changes, can we expect that these aspects still apply to the new system? Is the
representation of behavioural variance in this model separable from the system
the behaviour occurs within?

\subsection{Coupling of Model and Behaviour}
Explaining the issues of behaviour coupled to models. Highly related to some of
the reviewed literature --- I forget exactly what --- which discussed whether
aspects which were designed to be applied to one system could feasibly be
transferred to other systems, or whether they're inherently aware of the system
they're originally designed for. The core concept is that to be making changes
to some underlying codebase, you probably have to know what that codebase is, in
most cases at least.

\subsection{(Reword) Gaps/opportunities left by previous experiment}
In the previous experiment we demonstrated that behavioural variance can be
plausibly realistic. Are those variations separable from their underlying model,
or --- when trained i.e. made realistic --- do they suffer from the coupling
discussed in the previous subsection?

\subsection{Research Question}
Whatever the specific research question's wording for this was. Something about
decoupling realistic aspects from a given model maybe? It should be in an
earlier chapter somewhere.

\section{Experimental Design}
We took old player data and trained aspects on them, and in the previous
experiment we found they were statistically significantly accurate. We ran a
second season of RPGLite with slightly different parameters on player data. We
modelled how players learned and variations on their learning patterns, so in
theory, we should be able to apply the new learning patterns to the other system
too. Do we have to re-train the aspects? How portable are they? This section
lays out the design of this experiment.


\subsection{Changes to RPGLite}
What changes did we make to our system?


\subsection{Applying Behavioural Variations to New System}
Layout of new experiment, how it'll work, what's measured, why it should answer
the RQ.


\section{Applying Aspects from a Control System to a New System}
Our implementation \& results from the experiment described above. Discusses how
the above experiment was realised, lays out the results we found, and relates
those results to the research question we started with.

\subsection{Implementation}
Implementation of the experiment

\subsection{Results}
Presentation of the results of the experiment, analysis, some discussion (more
in next section too)

\section{Discussion}
Some notes discussing the outcome of this research question (that trained
aspects representing behavioural variance aren't separable from the system
they're trained around)

