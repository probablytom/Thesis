\chapter{A chapter title here for the experiment moving aspects to new systems,
  or systems with some changes}
\label{chap:exp2_old_aspects_new_systems}
\label{chap:experimental_results}
\label{sec:optimisation_with_aspects_experimental_results}



The naive model of RPGLite and the \aspectoriented models of learning \&
confidence which are described in \cref{chap:experiment_setup} are designed to
answer the following research questions:

\begin{researchquestion}
  \begin{itemize}
\item \rqtwo{}
\item \rqthree{}
\item \rqfour{}
  \end{itemize}
\end{researchquestion}

% === Keeping this here for now --- think I might want to re-use it for talking
% === about the prior distribution model.
%To investigate these, the naive model of RPGLite play is augmented using the
%aspect-oriented models of learning outlined in
%\cref{sec:optimisation_with_aspects_aspectsdeveloped}. The datasets produced by
%executing both the unaltered naive model and the naive model with aspects
%applied are compared against the real-world dataset described in
%\cref{chap:rpglite}. Comparisons are made by quantifying the similarity of
%the character pair preferences found in each dataset as defined in
%\cref{measuring_charpair_similarity}. Whichever synthetic dataset is most
%similar to the empirically sourced one can be said to be the most realistic.
%This experiment has the null hypothesis that introducing aspects has no impact
%on model realism. If no change in model behaviour can be measured when learning
%models are applied, then it would be possible to \emph{represent} model changes
%as advice as demonstrated in
%\cref{sec:optimisation_with_aspects_aspectsdeveloped} but not possible to
%\emph{use} those changes in their aspect-oriented form.

To investigate each research question, relevant advice is woven into the naive
model, and datasets are generated of recorded simulated gameplay. To answer the
proposed research questions, these datasets are compared against the empirically
sourced datasets described in \cref{chap:rpglite}. Different experiments require
different comparisons and yield different contributions, but all make use of the
same foundations described in \cref{chap:experiment_setup}.

This chapter explores the results of the experiments which are enabled by the
previous chapter's foundations.
The first section describes how synthetic datasets are interpreted to yield
``optimal'' parameters when simulating a given player.
The second section explores an experiment answering the third research question,
concerning the use of advice to introduce new parameters and behaviours to a model.
The third section explores an experiment answering the fourth research question,
concerning the portability of advice as individual modules to new systems, or to
changed instances of the same system.
\inline{
  Add crefs to the sections mentioned here once they exist.
}

\inline{
  TODO replace following sections to match the above.
}


\subsection{Identifying Statistically Significant Results in Datasets}

\inline{
  A discussion of how datasets are evaluated. We produce \emph{many} datasets
  which have annealed to the highest stat and P val we could find, but they're
  split across many folds. How do we determine whether a set of parameters works
  ``enough''? We find the highest correlation val and lowest pval which work
  across a majority of folds, then test against the complete dataset to be
  absolutely certain that correlation doesn't drop. This gives us the best
  parameters for a player: the parameters were reliably annealed to across many
  folds to give statistically significant correlation, retained their
  correlation against a testing set, and further retained their correlation
  against the entire dataset.
}

\inline{
  Is there a pattern in players getting reliable correlation and having played
  many games? Maybe we just don't have enough games to do this for most players.
  Check this; could be worth a note toward the end, in a discussion or similar.
}

\inline{
  This is an explanation of the analysis script, not the process f annealing; we
  should already have that in the experimental design section (currently 6.5, at
  time of writing).
}





\subsection{Selecting Character Pairs from a Known Distribution}

The second research question presented in \cref{subsec:rqs} is:

\begin{researchquestion}
  Can a model be made more realistic by applying aspect-oriented improvements?
\end{researchquestion}

This question can be answered directly by weaving advice which modifies player
behaviour so that character pairs are selected from a distribution calculated
from existing player data. If the dataset produced with this advice woven
correlates with the pre-calculated distribution of character pair preferences,
then the simulated behaviour must have been affected by the aspects in the
manner the advice modelled, and so made the model more realistic. If the dataset
correlates with that produced by the naive model, it must have had no effect.
And, if it correlates with neither, then the weaving of advice modelling a
specific character pair selection did impact player behaviour, but did not
produce expected results.\footnote{In this scenario, there would likely be an
error in implementation as the aspects demonstrably made a difference to
behaviour, but not what was intended. Rather than confirming the hypothesis that
advice can be used to augment models of behaviour or the null hypothesis that
weaving such advice has no effect, it would indicate a flawed experiment.}

Datasets representing every real-world player who completed more than 100 games
in RPGLite's first season were generated with the advice described in
\cref{sec:optimisation_with_aspects_aspectsdeveloped} woven. There were 21 such
players. The \tau correlation statistic (and corresponding p-values) of these
datasets compared to their players'
real-world datasets and the naive model's dataset is presented in
\cref{fig:known_distribution_results_s1}.

\begin{figure}[h]
\begin{center}
  \begin{tabular}{c|c|c}
    \emph{Simulated Player} & \emph{P-Value} & \emph{Correlation Statistic} \\ \hline\hline
    Data1 & data2 & data3 \\
    Data1 & data2 & data3 \\
  \end{tabular}
  \caption{Correlation statistics generated by weaving aspects into the naive
    model which select character pairs with the same distribution as a specific
    real-world player}
  \label{fig:known_distribution_results_s1}
\end{center}
\end{figure}



\inline{TODO: add notes explaining what an acceptable pval and correlation stat
  are. This will be useful to refer back to in the next subsection, where not
  all players are simulated by each model.}


\inline{Explain the results \& relate them to the final RQ.}



\subsection{Selecting Character Pairs using Confidence-Based Learning}

\inline{I suspect I'll move this to the discussion of the next experiment, so
  that the aspects using the known distribution are presented as an
  investigation of the first research question, and the aspects using the
  confidence-based model of learning are presented as an investigation of the
  second.}

Selecting character pairs based on a known distribution demonstrates that advice
can alter the behaviour of actors in a model, improving the model's accuracy.

\inline{I'm increasingly realising that the aspect drawing from a known
  distribution is both more accurate and more portable than our models of
  learning: we can apply it to S2 and just draw from season-specific data to get
  the correct results. It's actually way better than our models of learning.
  However, the models of learning let us investigate specific players'
  behaviour, so they let us use the naive model to investigate the bahviour of
  real-world players, which is still quite valuable, but might want a different
  approach to the write-up. I'm pretty lost with this chapter --- need Tim's
  input, I think.}



%This is a relatively short chapter; a lot of the building blocks for it exist in
%the previous chapter, so there's less ground to cover. If it ends up quite
%lop-sided, I'd chop the earlier chapter in two rather than artificially making
%this chapter beefier; I think it'd flow better.

This needs rewriting (and re-titling!) --- it's now a chapter on the results for
all the experiments, including the three RQs:

\begin{itemize}
  \item \rqtwo{}
  \item \rqthree{}
  \item \rqfour{}
\end{itemize}

\section{(Reword) Experimental motivations / motivation of research question}
This section should describe why it's interesting to move an aspect trained on
some actors' behaviours to a new system, and discuss what investigating this can
teach us that we \emph{don't} already know from the previous experimental
chapter. We've got aspects which represent behavioural variance. If the system
changes, can we expect that these aspects still apply to the new system? Is the
representation of behavioural variance in this model separable from the system
the behaviour occurs within?

\subsection{Coupling of Model and Behaviour}
Explaining the issues of behaviour coupled to models. Highly related to some of
the reviewed literature --- I forget exactly what --- which discussed whether
aspects which were designed to be applied to one system could feasibly be
transferred to other systems, or whether they're inherently aware of the system
they're originally designed for. The core concept is that to be making changes
to some underlying codebase, you probably have to know what that codebase is, in
most cases at least.

\subsection{(Reword) Gaps/opportunities left by previous experiment}
In the previous experiment we demonstrated that behavioural variance can be
plausibly realistic. Are those variations separable from their underlying model,
or --- when trained i.e. made realistic --- do they suffer from the coupling
discussed in the previous subsection?

\subsection{Research Question}
Whatever the specific research question's wording for this was. Something about
decoupling realistic aspects from a given model maybe? It should be in an
earlier chapter somewhere.

\section{Experimental Design}
We took old player data and trained aspects on them, and in the previous
experiment we found they were statistically significantly accurate. We ran a
second season of RPGLite with slightly different parameters on player data. We
modelled how players learned and variations on their learning patterns, so in
theory, we should be able to apply the new learning patterns to the other system
too. Do we have to re-train the aspects? How portable are they? This section
lays out the design of this experiment.


\subsection{Changes to RPGLite}
What changes did we make to our system?


\subsection{Applying Behavioural Variations to New System}
Layout of new experiment, how it'll work, what's measured, why it should answer
the RQ.


\section{Applying Aspects from a Control System to a New System}
Our implementation \& results from the experiment described above. Discusses how
the above experiment was realised, lays out the results we found, and relates
those results to the research question we started with.

\subsection{Implementation}
Implementation of the experiment

\subsection{Results}
Presentation of the results of the experiment, analysis, some discussion (more
in next section too)

\section{Discussion}
Some notes discussing the outcome of this research question (that trained
aspects representing behavioural variance aren't separable from the system
they're trained around)

