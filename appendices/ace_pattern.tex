\chapter{Details of Pattern Used when Designing RPGLite's Naive Model}
\label{appendix_ace_pattern}

This appendix contains additional details on the actor, context, and environment
variables used as arguments to the functions implementing RPGLite's workflow
steps. Each simulation step receives these three arguments at a minimum. Because
steps of the model are functions, and therefore valid join points, aspects
applied to these have access to the entire state of the simulation.


\begin{description}
  \item[Actor ---] allows the function to identify the actor performing the activity
    defined by the function. This argument is any object uniquely identifying an
    actor. While RPGLite's naive model did not use this
    variable to store actor-specific data, the object identifying the actor
    could be used to encapsulate actor-specific information across different
    workflows or different invocations of the same workflow.
  \item[Context ---] allows a workflow step to determine details of the current
    thread of work being undertaken by the actor. This is necessary because in
    some simulations, the same actor might pause and resume multiple occurrences
    of the same activity --- for example, they might concurrently play three
    different matches in RPGLite. As a result, it is necessary to understand the
    context of the action being performed by the actor in question. This
    argument can be any object uniquely identifying the context of a piece of
    work, but should be mutable (such as a class or dictionary-like object) to
    permit the communication of information across invocations of different
    action-representing functions.
  \item[Environment ---] an actor's actions can be determined by the global
    environment they act within. There may be ancillary details to the actor's
    actions and the context of their particular thread of work which they are
    undertaken within which are used to determine behaviour, such as a landscape
    they traverse or other actors they might choose to interact with. An actor
    may also alter the state of that environment. Because all actors share
    access to a global environment, it can also function as an area of memory
    used for message passing or a space where actors can set values and flags to
    alter the environment of other actors.
    
    This concept is related to
    environments in some other simulation frameworks, such as
    SimPy\cite{simpy_documentation}, where the environment controls scheduling
    and execution. However, other frameworks' environments impose model
    properties such as a model of time which may not be of interest to a
    researcher; this structure imposes no such constraints and its flexibility
    offers a general-case design pattern rather than the structure of a
    pre-existing framework, as it requires nothing more than a mutable
    function argument common throughout a model's implementation.
\end{description}

Each simulation step receives these three arguments at a minimum. Because steps
of the model are functions, and therefore valid join points, aspects applied to
these have access to the entire state of the simulation. This happens to be a
useful property for \aspectoriented{} simulation \& modelling, though not a
necessary one.