\chapter{Prior Work}\label{chap:prior_work}

% To cover PDSF, CAISE paper, and anything else we've done previously that's
% related to this work (e.g. that masters thesis that models covid).

% Some notes on where PDSF was before the PHD work, edited from the first draft
% of the thesis.

% \pdsf is a Python library~\cite{pdsf_repo}
% built for making changes to the source code of a Python function as it is
% called, and before it is executed, while the original function definition
% remains oblivious to the changes being made. It was originally developed as an
% honours-level dissertation, which was built upon and detailled in a subsequent
% paper~\cite{wallis2018caise}. This thesis furthers that original work. To be
% clear about the work this thesis contains, the state of the project
% \emph{before} this work began is briefly discussed here.

\pdsf is a Python library~\cite{pdsf_repo} which implements a dynamically-woven
aspect orientation framework with the capacity to weave changes within a
specified join point. It is designed primarily to address the needs discussed in
\cref{sec:lit_discussion}, particularly those of aspect-oriented simulation and
modelling.

Work on PyDySoFu predates the work presented in this thesis: an early version
was developed for the modelling of behavioural variation. This prototype was
redeveloped and improved to provide more suitable tooling for more complex
experiments, and to better address the requirements of such a tool and
opportunities found in the existing literature. \pdsf{}'s improved design and
implementation is discussed in \cref{chap:pdsf_rewrite}. 

In this chapter, prior work on \pdsf{} and the research conducted with it is
reviewed, so as to isolate the contributions presented in this thesis from
previously published research.

\subsection{Motivations for \pdsf{}'s Original Development}

The original incarnation of \pdsf{} was developed with different motivations
than those outlined in \cref{chap:lit_review}. It is therefore important to
elucidate the context in which it was designed and developed.

\pdsf{} was originally developed for the representation of behavioural variance
in sociotechnical systems, and was first produced as a proof-of-concept; the
core contribution was one of tooling. It was developed for use in Python
codebases, by virtue of the language's widespread use and its flexibility in its
modelling of data and process.

The original version of the tool was to be applied to models of behaviour in
\sociotechnical systems, where individual actions were represented as functions.
Actions which could be decomposed further into more granular actions were to be
defined as a functions calling their more granular counterpart functions.
Invocations of low-level behaviours would implement some change to an
environment in the model which its modelled behaviour would be expected to
incur. Invocations of high-level behaviours, containing the invocations of
lower-level behaviours they compose in the model, would therefore apply the
combined effect of the collected behaviours they represent. A benefit of this
approach to modelling behaviour was that high-level behaviours could implement
the ``flow'' of a behaviour. For example, a behaviour which would be modelled in
a flowchart as having some loop could be modelled analogously in the method
described through use of primitive control flow operators in Python, such as
\lstinline{for} and \lstinline{while} loops.

Another benefit of this approach is that the behaviours modelled have a
predictable structure which is amenable to metaprogramming. A low-level
behaviour's affect could be changed by changing the function definition; more
structural changes could be made by altering the flow of less granular
behaviours. A simple high-level behaviour containing a series of function
invocations (modelling an ordered list of steps in the \sociotechnical system)
can be represented as a literal list of function calls. The contents of such a
list is trivially modifiable. Removing an item from a list or truncating it at a
certain length, for example, are both achievable in a trivial manner using
high-level languages such as Python. Notably, many behaviours can be conceived
of which could be represented as high-level behaviours but would not be amenable
to a simple list of more granular behaviours, such as a behaviour with a looping
quality. 

With a mechanism to rewrite either an implementation of a behaviour or a
collection of behaviours (in the less granular functions mentioned), the

\subsection{\pdsf{}'s Implementation and Features}

% The original implementation of \pdsf patched Python classes with
% additional functionality. Attributes of Python objects are usually retrieved using
% dot notation (i.e. \lstinline{object_id.attr_id}), which evaluates internally to
% a call to \lstinline{object_class.__getattribute__(``attr_id'')}. PyDySoFu replaces a
% class' built-in \lstinline{__getattribute__()} method with a new one, which
% calls the original to acquire the required attribute. 

Core aspect orientation concepts include join points, aspects, and advice.
\pdsf was not originally designed with aspect orientation in mind --- though its
design aligned with aspect oriented design in many ways --- and implemented
analogues of these concepts using fundamental Python concepts. In particular, it
used Python's decorators to specify join points (and the advice to be applied to
them).
The intended use of \pdsf{}'s original implementation was that a decorator,
\lstinline{@fuzz()}, could be used to annotate a function which might undergo
variation at runtime.

% The variation to be applied was represented as a function
% which took a list of ``steps'' and returned a new list of those steps, represent

In the case where the required attribute is not callable, the value is returned
as normal. Callable attributes are modified, however. In this case, the
replacement \lstinline{__getattribute__()} also checks for a set of
manipulations to make to the original code. These can be applied before or after
the original code is run, as well as around it. A new function is returned
containing a reference to the originally sought attribute, but which will search
for these additional pieces of work before executing it, and can execute this
work before or after the call (or both). These pieces of work are referred to as
``advice'', adopting aspect orientation terminology.

As discussed further in \cref{subsec:pdsf_aop}, this approach is effectively an
implementation of a traditional aspect orientation framework. However, unlike
existing frameworks, PyDySoFu also supports a special kind of ``around'' advice:
before a function is called, it can be rewritten. This is done by applying
``before'' advice which retrieves the abstract syntax tree of the target
callable attribute using Python's \lstinline{inspect} module (its built-in
reflection), applying arbitrary transformations to the tree, and recompiling it
into a Python \lstinline{code} object (its representation of its
internal bytecode). At this point, many things are possible: the transformation
can be cached for later use, can replace the original callable's
\lstinline{code} object to make the transformation persistent, or can be
discarded after use. This transformed code is run in lieu of the original,
effectively enabling aspect orientation which can make adaptations \emph{inside}
a procedure as well as before and after its execution.

This approach also had some limitations:

\begin{itemize}
    \item Traditional pointcuts cannot target points inside a procedure, meaning
    that an aspect applied ``inside'' its target must manage the points where
    its transformation is applied manually.
    \item Importantly, a callable object's internal bytecode cannot be replaced
    in Python3, leading to a rewrite discussed in \cref{chap:pdsf_rewrite}.
    \item This method is significantly slower than other aspect orientation
    approaches, as rewriting a class' \lstinline{__getattribute__} method means
    that \emph{every} resolution of an object's attributes --- whether they are
    methods or values, and including a class' built-in ``magic'' methods ---
    incurs an overhead from the replaced \lstinline{__getattribute__}
    implementation. However slight this overhead can be made, affecting Python's
    built-in methods on classes means that rewriting the
    \lstinline{__getattribute__} method is unavoidably expensive due to the
    scale of these methods' use.
\end{itemize}

However, the goal of the original research was to develop a flexible
``proof-of-concept'' of aspect orientation adapting procedure definition at
runtime, which was successfully
achieved\cite{wallis2018caise,wallis2018genetic}. 

\subsection{Aspect Orientation \& PyDySoFu}\label{subsec:pdsf_aop}

The goals of ``changing a function's behaviour'' and maintaining
``obliviousness'' in the original definition of that function speak to the goals
of the aspect oriented programming paradigm\cite{kiczales1997aspect}. Quoting
their original definitions:

\begin{displayquote}
    Components are properties of a system, for which the implementation can be
    cleanly encapsulated in a generalized procedure. Aspects are properties
    for which the implementation cannot be cleanly encapsulated in a
    generalized procedure. Aspects and cross-cut components cross-cut each other
    in a system’s implementation.
    [ \ldots{} ]
    The key difference between
    AOP and other approaches is that AOP provides component and aspect languages
    with different abstraction and composition mechanisms.
\end{displayquote}

Generally, aspect orientation is percieved to be a technique for separation of
concerns. Any cross-cutting concerns can be separated from their components into
aspects applied where that concern arises. The strength of aspect orientation
lies in its compositional nature: developers can write short, maintainable
implementations of a procedure's core purpose (for example, business logic) and
ancillary concerns such as logging or security can be woven into this
implementation as preprocessing, compilation, or at runtime. This compositional
nature is what gives rise to aspect orientation's ``obliviousness'', as the
procedure targetted by a piece of advice is written without regard to that fact.

The original PyDySoFu implementation was an aspect orientation library focusing
on separating a function's definition from \emph{potential changes to it}. This
was used to model ``contingent behaviour'' --- behaviour sensitive to some
condition --- as an original, ``idealised'' definition of that behaviour, plus
some possible alterations. These changes might apply to many different
behaviours in the same manner, and therefore represent concerns which separate
cleanly into an aspect. An example would be the behaviour of a worker whose job
requires focus on allocated tasks. A lack of focus could be represented as steps
of the worker's tasks being executed in duplicate, out-of-order, or skipped.
Assuming aspects as described by \citeauthor{kiczales1997aspect} are able to
edit the definition or execution of a procedure\footnote{As opposed to simply
wrapping it with additional behaviour before and/or after execution}, such
contingent behaviours are well modelled as aspects.

To achieve this, a model was presented in \cite{wallis2018caise} wherein aspects
were developed which could change function \emph{definitions} on each invocation
of that function, contingent on program state. This allowed behavioural
adaptation to be simulated in an aspect-oriented fashion. In addition, a library
of behavioural adaptations called FUZZI-MOSS\inline{CITECITECITE} was developed
which implemented many cross-cutting, contingent behaviours in procedural
simulations of \sociotechnical systems.

One important contribution of this work is that PDSF aspects are effectively
able to operate \emph{inside} a target. In typical aspect orientation frameworks
such as AspectJ~\cite{aspectj_intro}, aspects operate by effectively prepending
or appending work to a target, referred to as ``before'' or ``after'' pointcuts
respectively. To do both is referred to as ``around''. By manipulating
procedures within Python directly, PDSF is able to manipulate its target from a
new perspective, adding (or removing) work during the target's
execution\footnote{Similarly to \cref{subsec:bca}, but in an aspect oriented
manner.}. Moreover, because weaving is performed dynamically, every execution of
a function may perform different operations.

\subsection{Opportunities presented by PyDySoFu}

PyDySoFu presented several oppportunities for future research. Some salient
properties of the original work include:

\begin{itemize}
    \item It provided an aspect orientation library which could weave and
    unweave aspects during program execution, without relying on anything other
    than Python's built-in language features. As discussed in
    \cref{sec:dynamic_aop_review}, this is supported by some early aspect
    orientation frameworks also, but AspectJ dominates in the world of aspect
    orientation frameworks and does not support weaving during program execution.
    \item It provided the capacity to weave aspects \emph{inside} targets, as
    opposed to around them, or at either end of their execution. So far as we
    are aware, no aspect orientation framework in research or industry has
    offered this feature, and its applications and potential are yet to be
    explored.
    \item Relatedly, PyDySoFu was used in the context of simulating behaviour which may change
    over time. Contingent behaviour being a cross-cutting concern is an
    innovation of the early research which suggests aspect orientation may have
    strong applications in \sociotechnical simulation \& modelling.
\end{itemize}

\inline{Do we need a brief explainer of what aspect orientation is before
jumping into outside lit? Or will this go in the introduction? already a little
in the earlier litrev subsections.}

The amount of potential investigation which can be done into the dynamic weaving
of target-altering / ``inside'' aspects in simulation \& modelling applications
is vast. While literature on the complete topic is absent, each individual
component of this research angle is well-studied on its own. These opportunities
might be related to existing literature through the following questions:

\begin{itemize}
    \item How does PyDySoFu compare to existing aspect orientation frameworks,
    particularly those with a focus on dynamic weaving? Related frameworks are
    summarised and compared in \cref{sec:dynamic_aop_review}.

    \item What is the use of aspect orientation in simulation \& modelling? How
    does the approach taken in \pdsf's prior work relate to existing
    approaches? This will be discussed for simulation in
    \cref{sec:ao_and_simulation}, and for modelling in
    \cref{sec:ao_and_modelling}.
    
    \item Variability is important to capture in any \sociotechnical model or
    simulation. How is variability treated in existing literature, and how does
    this relate to \pdsf's approach? This will be explored in
    \cref{sec:dynamism_in_sm}.
\end{itemize}


