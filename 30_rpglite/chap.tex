\chapter{Testing Synthetic Data Realism with RPGLite}

\section{The RPGLite Model}

The model for RPGLite is composed of a base model describing the game rules and
random player behaviour. To this, we add four aspect-oriented components of the
model, in an attempt to adapt the base model's random behaviour into something
``realistic''.

\subsection{Base Model}
RPGLite's base model consists of a set of classes, representing playable
characters, and a set of functions representing aspects of gameplay. Each
function can be thought of as a discrete action an actor might take in a
discrete-event agent-based simulation. These functions all adhere to an
architecture which allows them to access and manipulate pertinent details of the
simulation:

\begin{description}
  \item[actor — ] allows the function to identify the actor performing the activity
    defined by the function. This argument is any object uniquely identifying an
    actor.
    \item[context — ] allows the function to determine details of the current
      thread of work being undertaken by the actor. This is necessary because in
      some simulations, the same actor might pause and resume multiple
      occurrences of the same activity --- for example, they might concurrently
      play three different matches in RPGLite. As a result, it is necessary to
      understand the context of the action being performed by the actor in
      question. This argument can be any object uniquely identifying the context
      of a piece of work, but is most likely a class or dictionary-like object
      to permit the communication of information across invocations of different
      action-representing functions.
      \item[environment — ] an actor's actions might involve 
\end{description}





% ========================================================================
\section{Experiment 1: Correlating Synthetic and Real-World Character Choices}

