\documentclass[draft]{article}

\title{Plausibly Realistic Sociotechnical Simulation with Aspect Orientation}
% Alternative titles:
%   Exploring the efficacy of aspect orientation in sociotechnical simulation
%   The use of runtime aspect weaving in simulation
\author{Tom Wallis}
\date{Forthcoming 2021}
\degree{Doctor of Philosophy}

% I heard you like ~~~CITATIONS~~~
\usepackage[
  backend=biber,
  bibencoding=utf8,
  style=ieee %% draft probably wants changing...
]{biblatex}
\addbibresource{refs.bib}


% Packages I for sure need
\usepackage{float}
\usepackage{graphicx}
\usepackage{titletoc}
\usepackage{enumitem}
\usepackage{csquotes}
\usepackage{xstring}

% Fonts
\usepackage[no-math]{fontspec}
\setmainfont{Valkyrie A}
\setsansfont{Concourse 4}
\setmonofont{Triplicate A Code}
\newfontfamily{\advocate}{Advocate55WideMed}
\newfontfamily{\concourse}{Concourse 4}
\newfontfamily{\concourseindex}{Concourse Index}

\usepackage{sectsty}
\allsectionsfont{\concourse}

% Listings config
\lstset{language=Python,
frame=L,
numbers=left,
stepnumber=1,
numberstyle=\footnotesize,
float,
floatplacement=H,
basicstyle=\normalsize\ttfamily}

% Extra colours for listings
\definecolor{backcolour}{gray}{0.95}

% Inline listings text size correction from https://tex.stackexchange.com/a/575053
\makeatletter
\lst@AddToHook{TextStyle}{\let\lst@basicstyle\ttfamily}
\makeatother



% For inline todos...nicked from https://tex.stackexchange.com/a/219354
\makeatletter
\usepackage{regexpatch}
\xpatchcmd{\@todo}{\setkeys{todonotes}{#1}}{\setkeys{todonotes}{inline,#1}}{}{}
\makeatother

% List formatting changes
\setitemize[1]{label={\concourseindex\normalsize\symbol{"2022}}}
\setitemize[2]{label={\concourseindex\small\symbol{"2022}}}
\setitemize[3]{label={\concourseindex\footnotesize\symbol{"2022}}}
\setitemize[4]{label={\concourseindex\scriptsize\symbol{"2022}}}
\setitemize[5]{label={\concourseindex\scriptsize\symbol{"2022}}} % Technically could be \tiny
\setenumerate[1]{font={\concourseindex\normalsize},label=\arabic*}
\setenumerate[2]{font={\concourseindex\small},label=\arabic*}
\setenumerate[3]{font={\concourseindex\footnotesize},label=\arabic*}
\setenumerate[4]{font={\concourseindex\scriptsize},label=\arabic*}
\setenumerate[5]{font={\concourseindex\scriptsize},label=\arabic*} % Technically could be \tiny

% Some quote styling changes, modified from https://tex.stackexchange.com/a/468294
\usepackage[%
linewidth=1pt,
middlelinecolor= black,
middlelinewidth=0.4pt,
roundcorner=10pt,
topline = false,
rightline = false,
bottomline = false,
rightmargin=0pt,
skipabove=0pt,
skipbelow=0pt,
leftmargin=-1cm,
innerleftmargin=1cm,
innerrightmargin=0pt,
innertopmargin=0pt,
innerbottommargin=0pt,
]{mdframed}
\makeatletter
%Take the original environment definition and change the leftmargin to 1cm
\renewenvironment*{displayquote}
  {\begingroup\setlength{\leftmargini}{1cm}\csq@getcargs{\csq@bdquote{}{}}}
  {\csq@edquote\endgroup}
\makeatother
%Hooks
%Use single spacing, set 10pt font, set quote style curly quotes, and beginning quotes
\renewcommand{\mkbegdispquote}
    {\begin{mdframed}\fontsize{12pt}{12pt}\setstretch{1.25}\setquotestyle{quote}\textooquote}
%End displayquote environment with ending quotes
\renewcommand{\mkenddispquote}{\textcoquote\end{mdframed}}


% Macros
% A todo system that is turned off by enabling final mode.
\makeatletter
\@ifclasswith{good_template}{final}
{\newcommand{\inline}[1]{}}
{\newcommand{\inline}[1]{{\color{red}{#1}}\addcontentsline{tdo}{todo}{#1}{}}}
\makeatother

\newcommand{\labelledsec}[2]{\section{#1}\label{sec:#2}}
\newcommand{\labelledsubsec}[2]{\subsection{#1}\label{subsec:#2}}
% a macro for indicating a number in an enumerated list using the concourseindex
% font (for consistency)
\newcommand{\pointno}[1]{{\concourseindex{}#1}}
% a macro to stop me switching between hyphenation/spacing/neither
\newcommand{\sociotechnical}{socio-technical }
% To prevent me absent-mindedly shortening pydysofu to pdsf, the macro is handy.
\newcommand{\pdsf}{PyDySoFu }


\newenvironment{researchquestion}{
  \begin{displayquote}\itshape
}{
  \end{displayquote}
}

\begin{document}

\maketitle

\begin{abstract}
  What am I trying to show?\todo{Actually write an abstract.}
\end{abstract}

% ==============================================================================
\section{Introduction}
\label{sec:introduction}
Systems\todo{should I be acknowledging OBASHI in this report?} modelling is a
complicated field, largely born of its highly interdisciplinary nature. This is
both a blessing and a curse.\par


On one hand, the breadth of the field gives a range of diverse problems to work
on, and many interesting opportunities spring forth as a result. At the same
time, this scope breeds a host of different philosophies regarding systems
modelling, often muddying the waters with regards the field's literature. this
can make identifying small-scale, incremental improvements on the existing
literature hard to identify. Also --- as will be discussed --- systems science
is notoriously difficult to verify with quantitative data.\footnote{As this
  point is pertinent to the research at hand, it will be discussed in
  \emph{SECTION}}\todo{identify section!}.\par

Fortunately, the Cambrian Explosion that is modern systems ressearch providees
an opportunity to do reearch which can make tangible, genione improvements to
everyday life. The potential here easily justifies the friction involved in
performing rigorous research in the field!\par

This work bgan in a somewhat aimless fashion, and arrived at exciting research
opportunities toward the end of this meandering. The structure of that journey
of discovery is reflected in the structure of this report: beginning without a
clear end in sight; splintering in different directions, as possibilities are
felt out; finally, converting on a well-founded conclusion by the end. This
end-result would be hard to achieve without the wandering involved, so this
report should be read with the patient optimism that things \emph{do} converge
by the end.\par

\todo{Write a summary of the chapters, once they're actually written.}




\part{Model Transformations}
% ==============================================================================
\section{Systems Modelling Literature}
\label{sec:literature}
The diversity of system modelling requirements results in a variety of
philosophies regarding what models should look like. A useful distinction
between these philosophies is their take on the traditional input, process,
output contept; different modelling paradigms \emph{capture} information in
different ways, allow varying degrees of anaysis to be performed on this
captured information, and lend themselves to different methods for the
\emph{exposition} of the results of this analysis.\par

The distinction matters. A person undertaking a modelling task must capture all
pertinent data --- many systems exhibit chaotic behaviour, so minute details
within captured information in the model can make a significant difference to
the modelling effort's end result --- persumably, the task is being carried out
so the modeller can learn something about the system in question! --- the kinds
of analysis that are possible are therefore of great importance, as available
analysis methods will have an impact on what can be inferred from the model ---
moreover, though, the modelling effort would be for nought if the recipient of
the outcome of the analysis found it illegible, meaning that the methods for
displaying the \emph{output} of that analysis is just as important as the other
factors!\par

While this may seem like a trivial observation, modelling paradigms in the wild
rarely succeed on all three counts, and many different approaches to these three
factors exist. UML, for example, is much more concerned with understanding the
structure of a model than performing a detailled analysis of that structure and
coming to quantitative, verifiable results. It's graphical nature belies the
fact that often, seeing the \emph{structure} of captured information is the most
valuable output of the modelling process in the paradigm. Even in mostly
graphical approaches, though, differences in philosophy can arise. To contrast:
these same systems are sometimes modelled using process calculi such as
\picalculus. Here, a satisfying visual representation is hard to find, but the
intent of the modelling effort is a detailled formal analysis. Difficulty in
capturing information, or in the legibility of the results, is a friction
sometimes justified by the detail of analysis a model permits. \par


With this in mind, comparing modelling approaches can help to explore both the
philosophies modellers enter into the practice of model construction with, and
the state of the art in actually constructing those models.\todo{Revisit this
  when writing about pdsf! We're targeting modellers who enter into the
  practice with a certain perspective, too. It's a nice callback to the related
  lit, and it limits our own scope in useful ways and gives us an ``out'', so
  we can justify some of pdsf's limitations.}\par


\subsection{A Note on Informal Modelling Paradigms}
\label{subsec:informal_model_lit}
Many\todo{consider deleting} informal modelling methods have appeared in an effort to fulfil
requirements in a few areas:

\begin{itemize}
\item Modelling hard-to-define systems\\
  Many systems are difficult to achieve low-resultion understandings of, because
  --- usually --- they are socio-technical in nature. The can sometimes have
  aspects\footnote{See Applied Systems Theory, in section \emph{FIND SECTION
      HERE}.}\todo{cref the applied systems theory section here.} of their
  systems captured, but the details required to produce a complete more are
  effectively infinite in scale.
\item Modelling black boxes\\
  It can be difficult-to-impossible to infer low-level detail from a system
  which is either obfuscated or intractable to break apart. Here, ``black
  boxes'' are characterised by their inputs and outputs where possible.
\item System scope\\
  Often, modelling is undertaken at a large scale. It is intractable to gather
  finely grained information in these situations for most modellers, leading to
  high-level modelling paradigms.
\end{itemize}

Systems with these properties are often assessed from a high level, and due to
their nature, are often either heterogeneous or partly human in nature (such as
business or military systems).\par

\subsection{UML, SysML \& OPM}
One approach to modelling these sorts of systems is the diagrammatic approach
taken by UML\cite{uml_citations}, which is the de-facto modelling framework for
software architectures. UML has been refined gradually over \~20 years of use,
and has grown to model a variety of systems. The variety of UML's diagramming
capabilities is well-summarised in a class diagram from its own
specifications:\par

\todo{include class diagram of UML diagram types, example from early spec found
  on wikipedia}

UML's origins are in mapping the structure of software, but it can now be used
to model a variety of activities via SysML\cite{sysml_citations}. A comparison
between this and its early competitor, OPM\cite{opm_citations}, which also
models a variety of systems, can be enlightening.\par

Where UML's expressiveness comes from an array of diagram types and extensions
to its ordinary diagrams --- such as SysML --- OPM uses a more curated set of
building blocks to build models from, gaining expressiveness with a simpler
modelling system. These building blocks are structured objects of information,
and processes that the information is transformed by --- concepts which prove to
be rather universal, as sen in OPM's cross-disciplinary
appeal\cite{opm_rna_research}. OPM's native focus on processes and simple
fundamental units delivers the useful feature that OPM models are generally
\emph{executable} for analysis, allowing deeper insights into the subject of
the model.\par

This simplicity is often cited as a benefit of
OPM: there is less for a modeller to learn, and the core concepts building the
framework are powerful. However, while both are recognised standards (UML is a
standard supported by the Object Management Group, where OPM is supported by
ISO), UML seems to see much wider industrial use. I\todo{can I use ``I'', or is
  that bad form in a progression report?} posit that this is actually
\emph{because} of the additional diagram classes in UML: each diagram class
becomes slightly simpler than an OPM diagram, and the result is that each
diagram comes with its own visual identity. This imparts lots of context about
what the model is displaying to a viewer at a glance, and reduces the amount of
friction involved in the modelling process.\par

The amount of context available at a glance turns out to be especially important
for UML, because UML can rarely be processed for analysis. Instead, results of
the UML modelling process are typically done by a visual analysis of a diagram.
Some tools exist for analysing he contents of a model\cite{eclipse_papyrus},
but these are typically limited in scope and too clunky to use --- often, this
friction in use overwhelms the potential gains from the tool.\par

UML is capable of dealing with a variety of socio-technical, hard-to-define
systems via its different diagram types and can work at a variety of resolutions
of detail, making it adept at handling ``black boxes''. However, as scale
increases, so does the number of diagrams --- UML therefore relies on the
quality of tooling available to cope with the difficulties of visual analysis at
scale.\par

% ===== OBASHI
\subsection{OBASHI}
\label{subsec:obashi}
A modelling platform which takes pains to fix this problem of visual analysis at
scale is OBASHI\cite{obashi_methodology}. OBASHI sees wide industrial use for
the specific task of modelling dataflow in socio-technical systems. Like UML,
OBASHI's modelling paradigm is entirely graphical. Models have simple
fundamental components --- ``elements'' and six kinds of ``relationships'' ---
which are composed together according to a series of rules. Elements are
separated into ``layers'', and an ordering is imposed which groups social and
technical kinds of elements separately, mediating dataflow between them via
business processes.\par

The modelling system is proprietary, and a sophisticated tool is available from
the company developing the methodology. OBASHI's limited scope, focusing on
dataflow, means that this tool can perform some automated analysis on an
otherwise graphical system model, such as impact analysis of element failure or
the generation of different diagrams representing different potential pathways
for dataflow between elements.\par

OBASHI takes an interesting perspective on the difficulties of modelling
hard-to-define systems, as the design of its diagrams is specifically
constructed to cater to non-technical recipients of diagrams. As the outputs of
analysis are often more diagrams, the diagrams that are produced are laid out in
such a way that stakeholders in a diagram who are unfamiliar with the nuances of
their subject can still understand the implications of an analysis. This is done
by utilising some of OBASHI's rules (specifically, element and relationship
persistence) to safely construct subgraphs which represent meaningful analyses.
OBASHI's ability to safely construct subgraphs eliminates the diagram scale
issues that OPM and UML/SysML have. OBASHI also caters to black-box scenarios
and socio-technical, hard-to-define systems by use of its rules. this is
possible, however, because OBASHI is capable of modelling only systems which
permit the flow of information or data --- this lmitation in scope allows OBASHI
to successfully address many of the difficulties of informal modelling, at the
cost of flexibility.\par

% ===== BPMN
\subsection{BPMN \& BPEL}
\label{subsec:bpmn}
A popular alternative for business modelling is the informal modelling solutions
found in BPMN\cite{bpmn_sources}, which has seen success in modelling business
processes. BPMN is generally a graphical modelling framework, with a language to
support executable process definitions via its companion language,
BPEL\cite{BPEL_SOURCES}. Not unlike OBASHI, BPMN aims to demonstrate concepts
graphically in a simple enough format that non-technical recipients of diagrams
can understand their meaning, while supporting more complex analyses where
necessary. BPMN can be considered the business equivalent of UML, and is also
supported by the Object Management Group\cite{source_for_this_maybe}.\par

BPMN has had unusual academic interest, in that some low-level concepts have
been applied to this otherwise very high-level notation. For example, BPMN
models can be transformed to DEVS models\cite{bazoun2014business}, a timing
model for agent-based systems built on an extension of Moore machines, a kind of
finite-state machine\cite{DEVS}.\todo{Write a little more on BPMN and how it
  relates to other modelling approaches, as well as what we can do with it.}\par


% ===== Applied Systems Theory
\subsection{Applied Systems Theory}



\todo{Anything else worth mentioning here? SSM? VSM?}





% \subsection{Formal Modelling Paradigms}
% \label{subsec:formal_model_lit}

% ===== Bigraphs
\subsection{Bigraphs}
\label{subsubsec:bigraphs}
A convenient segue from informal modelling techniques to formal ones is
Bigraphs, in that, to a degree, Bigraphs support both formal \emph{and} informal
paradigms. This is because Bigraphs are defined in three ways:

\begin{enumerate}
\item As a graphical notation
\item As an algebra
\item As a category
\end{enumerate}

As a result of a carefully constructed mathematical definition, Bigraphs have
the capacity to cope with often complicated features, such as safely composing
systems together, or the formal manipulation of a system's structure. This
mathematical foundation can make bigraphs complex to use for a non-technical
modeller. Fortunately, the graphical notation for birgaphs is both simple and
expressive.\par

Bigraphs are often discussed as having both \emph{space} and
\emph{motion}\cite{milner2009space}. The \emph{space} of a system is defined by
Milner (the original designer of the bigraph formalism) as the placements of
elements relative to each other --- this is represented, mathematically, as a
forest of nodes, where parental relationships in a tree in the forest represents
containment of one node inside another.\footnote{This definition is purely
  mathematical, and what that containment represents in the real world is the
  choice of the modeller.}\par

The \emph{motion} of a model is defined as how the model changes over time.
Milner represents system behaviour as change according to ``reaction rules'':
patterns which match on a subgraph of a complete bigraph, and dictate how the
matching subgraph looks in a future iteration of the bigraph. Reaction rules
turn out to be rather powerful, and can represent any system within a category
of systems called a \emph{bigraphical reactive
  system}\cite{milner_early_brs_definition} (BRSes). BRSes can represent a
number of calculi, including \picalculus and the ambient
calculus\cite{bigraphs_and_transitions_milner_jensen}.\par

While literature on the
subject is elusive, I\todo{Can I use ``I'' in a report?} infer that BRSes might
be represented as a temporal hypergraph, where a regular bigraph is represented
by a hypergraph, and the application of reaction rules modifies the hypergraph
as an atomic change at each unit of time.\todo{Consider deleting.}\par

Bigraphs seem to see little industrial success, but have inspired a number of
variants\cite{bigraphs_with_sharing,directed_bigraphs} and plenty of academic
interest\cite{impalas_stevegnani,bigraph_model_checking,bigraph_languages}.
Emergence has been studied in the context of Bigraphical Reactive Systems\cite{bigraph_emergence}, but
this is not one of the field's main active research areas at present.


% ===== Petri Nets
\subsection{Petri Nets}
\label{subsubsec:petrinets}
As an alternative to bigraphs, system modelling can be done formally via methods
such as Petri Nets\cite{petri_net_seminal}. A Petri Net is a directed graph of
states and processes those states can undergo to reach future states. Petri Nets
were born initially out of chemistry literature\cite{petri_net_seminal}, but
have seen applications in areas as diverse as workflow
modelling\cite{petri_net_workflow_modelling},
concurrency\cite{petri_net_concurrency}, and molecular networks in systems
biology\cite{petri_nets_for_biology}.\par

Petri nets have similarities with OPM, as they are both fundamentally concerned
with state (arbitrary state in the case of Petri Nets, and a more specific state
of structured data in the case of OPM) and processes that their states go
through. Unsurprisingly, some literature exists on the combination of petri nets
and OPM, which could eventually lead to a high-level modelling system with a
powerful mathematical framework underlying it.\par

\todo{Feels like I can write more here, but frankly, I'm not sure exactly what.
  Petri nets always seemed a little boring to me...}



% ===== CAS modelling via pdes, cellular automata...
\subsection{Complex Adaptive Systems Modelling}
\todo{write up cas modelling via cellular automata, pdes, etc}



% ===== Models in code: LogoNet, bespoke models like jagora
\subsection{Code and Bespoke Models}
\todo{Write up models in code, logonet being the big example but smaller ones
  too like PyCX., Jagora is an example of a bespoke model of a problem domain,
  which is worth covering because the practice of making those is pertinent to
  what we're getting at later on.}


\subsection{PyDySoFu}




\subsection{Findings after Evaluating the Literature}
% Why we didn't do model transformation, in the end.
% However, PydySoFu turns out to be \emmph{quite} promising!


\part{Exploratory Development}
% ==============================================================================
\section{PyDySoFu's initial state}
\label{sec:pydysofu}







% ==============================================================================
\section{Improvements to PyDySoFu}
\label{sec:pdsf_improvements}








\section{Segue: Genetic Programming}
% ReaLX submission stuff








\part{Resilience}
% ==============================================================================
\section{Resilience and its Literature}
\label{sec:resilience}
\todo{read absolutely any literature on resilience other than the couple of
  Hollnagel papers I've got}


\section{PyDySoFu for Resilience Modelling}




% ==============================================================================
\section{Thesis Statement}
\label{sec:thsis_statement}







% ==============================================================================
\section{Future Plan}
\label{sec:future_plan}








% ==============================================================================
\section{Risks and Mitigating Actions}
\label{sec:risks}







% ==============================================================================
\section{Conclusion}
\label{sec:concluion}














\bibliography{lib}

\end{document}