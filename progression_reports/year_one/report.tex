\documentclass[draft,12pt]{article}

\title{Plausibly Realistic Sociotechnical Simulation with Aspect Orientation}
% Alternative titles:
%   Exploring the efficacy of aspect orientation in sociotechnical simulation
%   The use of runtime aspect weaving in simulation
\author{Tom Wallis}
\date{Forthcoming 2021}
\degree{Doctor of Philosophy}

% I heard you like ~~~CITATIONS~~~
\usepackage[
  backend=biber,
  bibencoding=utf8,
  style=ieee %% draft probably wants changing...
]{biblatex}
\addbibresource{refs.bib}


% Packages I for sure need
\usepackage{float}
\usepackage{graphicx}
\usepackage{titletoc}
\usepackage{enumitem}
\usepackage{csquotes}
\usepackage{xstring}

% Fonts
\usepackage[no-math]{fontspec}
\setmainfont{Valkyrie A}
\setsansfont{Concourse 4}
\setmonofont{Triplicate A Code}
\newfontfamily{\advocate}{Advocate55WideMed}
\newfontfamily{\concourse}{Concourse 4}
\newfontfamily{\concourseindex}{Concourse Index}

\usepackage{sectsty}
\allsectionsfont{\concourse}

% Listings config
\lstset{language=Python,
frame=L,
numbers=left,
stepnumber=1,
numberstyle=\footnotesize,
float,
floatplacement=H,
basicstyle=\normalsize\ttfamily}

% Extra colours for listings
\definecolor{backcolour}{gray}{0.95}

% Inline listings text size correction from https://tex.stackexchange.com/a/575053
\makeatletter
\lst@AddToHook{TextStyle}{\let\lst@basicstyle\ttfamily}
\makeatother



% For inline todos...nicked from https://tex.stackexchange.com/a/219354
\makeatletter
\usepackage{regexpatch}
\xpatchcmd{\@todo}{\setkeys{todonotes}{#1}}{\setkeys{todonotes}{inline,#1}}{}{}
\makeatother

% List formatting changes
\setitemize[1]{label={\concourseindex\normalsize\symbol{"2022}}}
\setitemize[2]{label={\concourseindex\small\symbol{"2022}}}
\setitemize[3]{label={\concourseindex\footnotesize\symbol{"2022}}}
\setitemize[4]{label={\concourseindex\scriptsize\symbol{"2022}}}
\setitemize[5]{label={\concourseindex\scriptsize\symbol{"2022}}} % Technically could be \tiny
\setenumerate[1]{font={\concourseindex\normalsize},label=\arabic*}
\setenumerate[2]{font={\concourseindex\small},label=\arabic*}
\setenumerate[3]{font={\concourseindex\footnotesize},label=\arabic*}
\setenumerate[4]{font={\concourseindex\scriptsize},label=\arabic*}
\setenumerate[5]{font={\concourseindex\scriptsize},label=\arabic*} % Technically could be \tiny

% Some quote styling changes, modified from https://tex.stackexchange.com/a/468294
\usepackage[%
linewidth=1pt,
middlelinecolor= black,
middlelinewidth=0.4pt,
roundcorner=10pt,
topline = false,
rightline = false,
bottomline = false,
rightmargin=0pt,
skipabove=0pt,
skipbelow=0pt,
leftmargin=-1cm,
innerleftmargin=1cm,
innerrightmargin=0pt,
innertopmargin=0pt,
innerbottommargin=0pt,
]{mdframed}
\makeatletter
%Take the original environment definition and change the leftmargin to 1cm
\renewenvironment*{displayquote}
  {\begingroup\setlength{\leftmargini}{1cm}\csq@getcargs{\csq@bdquote{}{}}}
  {\csq@edquote\endgroup}
\makeatother
%Hooks
%Use single spacing, set 10pt font, set quote style curly quotes, and beginning quotes
\renewcommand{\mkbegdispquote}
    {\begin{mdframed}\fontsize{12pt}{12pt}\setstretch{1.25}\setquotestyle{quote}\textooquote}
%End displayquote environment with ending quotes
\renewcommand{\mkenddispquote}{\textcoquote\end{mdframed}}


% Macros
% A todo system that is turned off by enabling final mode.
\makeatletter
\@ifclasswith{good_template}{final}
{\newcommand{\inline}[1]{}}
{\newcommand{\inline}[1]{{\color{red}{#1}}\addcontentsline{tdo}{todo}{#1}{}}}
\makeatother

\newcommand{\labelledsec}[2]{\section{#1}\label{sec:#2}}
\newcommand{\labelledsubsec}[2]{\subsection{#1}\label{subsec:#2}}
% a macro for indicating a number in an enumerated list using the concourseindex
% font (for consistency)
\newcommand{\pointno}[1]{{\concourseindex{}#1}}
% a macro to stop me switching between hyphenation/spacing/neither
\newcommand{\sociotechnical}{socio-technical }
% To prevent me absent-mindedly shortening pydysofu to pdsf, the macro is handy.
\newcommand{\pdsf}{PyDySoFu }


\newenvironment{researchquestion}{
  \begin{displayquote}\itshape
}{
  \end{displayquote}
}

% Bibliography management
\usepackage{natbib}
\bibliographystyle{plain}

\begin{document}
\maketitle

\begin{abstract}
  What am I trying to show?
  
\end{abstract}

\section{Introduction}
\label{sec:introduction}
Modeling is the process of observing the world, selecting detail from it,
and using that detail to build a replica. In a way, the method we use to build
these models doesn't really matter: whether it's programming code, a
mathematical equation written on paper, a model train set, or a lego house built
by a child.

What separates modern system modeling from children playing with lego is our
ability to \emph{infer} useful information from our models of the world. this
requirement is surprisingly difficult to fulfill, and many different
philosophies regarding system modeling have surfaced in an attempt to satisfy it.


\section{Relevant Literature}
% Things I can mention here, ordered from high-level to low-level:
% BPMN
% Obashi
% UML
% Viable Systems Model
% Soft Systems Methodology
% Applied Systems Theory
% BPEL
% Workflow Modeling
% Agent-Based Models   - (these can sort of go anywhere in this list!)
% Software models
% Description logics etc
% Bigraphs
% Process calculi

% Agent-based models can fit in lots of places here!

As a result of the various different philosophies toward system modeling in the
field, it is important not only to have a deep understanding of the literature,
but an appreciation of its breadth; each approach to modeling has its strengths
and weaknesses, and often a weakness of one approach is made evident by the
strength of another.\par

An example: UML\citep{rumbaugh2017unified}, the de-facto standard for modeling
software systems --- at least object-oriented systems --- takes a graphical
approach which models the structure of a software system, and attempts to help
people to reason about architecture in a number of ways. (The UML standard
supports a variety of model types, such as activities, class structures, and
state machines.) This is in contrast to, for example, a formal model used when
modeling something like a Complex Adaptive
System\citep{groscomplex,petri1962kommunikation}, in which the graphical
representation of a system's structure is less important than a communication of
what, precisely, is contained within the model.

\subsection{A Brief Taxonomy of Modeling Paradigms}
% Talk a little about the need for capture, analysis, exposition like we have
% input, process, output, and discuss the fact that different philosophies exist
% to target different parts of this process specifically; doing all three is
% surprisingly elusive, as we'll see in the following review of some approaches
% from the literature.
Reading the literature, it seems that the focuses of different models can
actually be broken into three different stages --- similar to the ``input,
process, output'' common when discussing programming --- which I will refer to
as \emph{capture}, \emph{analysis} and \emph{exposition}.\par

First, system models have to model \emph{something}; information about the
subject of a model is therefore stored in the model as part of its construction.
Sometimes, this information storage is implicit: partial differential equations
can be a useful modeling tool, but aside from some insights into the
relationships between different aspects of system behaviour, it can be difficult
to infer a wealth of information about a model's subject from the equations.
This is in contrast to modeling via something like Obashi, where the intent of the
model is to make as clear as possible what has been captured in the model, so
the model's architect can begin to draw insights regarding what this structure
might mean.
This is \emph{capture}.\par

Second, simply capturing information about a system can only tell us so much
about it, and often model architects build their models not just for clarity
about the world, but so as to \emph{analyze} the captured information. This can
help the model architect to draw insights that aren't immediately obvious from a
visual analysis of the structure of a system, which tools such as Obashi or UML
typically provide. Models built from agent-based
modeling~\citep{tisue2004netlogo} or from cellular
automata~\citep{wolfram1984cellular} can be a useful way to investigate the
behaviour that a system exhibits, by observing how traits change over time. This
is one way to spot emergence in a system\citep{miller2006modeling}. Emergent
properties of systems can be difficult to spot, but their discovery can be an
essential aim when constructing a model. Understanding in what way emergence
arises from simple phenomena remains an important problem in system modeling,
particularly in complex adaptive system modeling~\citep{gell1994complex}.
The capacity of a model to be assessed, whether algorithmically, by human
effort, or by some mix of the two where a computer aids human assessment, is a
model's capacity for \emph{analysis}.\par

Third, there is little use in analyzing a model if results of this analysis
cannot be communicated effectively to an end-user. While clearly fundamental,
there is a wealth of different approaches to communicating the results of
analysis to a model's architect. NetLogo\citep{tisue2004netlogo}, for example,
employs a friendly graphical approach, where the results of analysis can be
exposed from the model itself --- all models are agent-based, and activity is
represented graphically during the model's execution --- or from graphs which
can be produced if written as part of the model. Solutions such as
OBASHI\citep{obashimethodology} employ a similar approach, where specific
classes of diagrams can be constructed to explain the outcome of a piece of
analysis. Interestingly, OBASHI constructs these models so that the results of
analysis can be understood not only be a technically adept model architect, but
also by non-technical stake-holders in insights drawn from the model. In
industry, a model architect is rarely the sole stake-holder of their model.
While this detail feels self-evident, many modeling platforms are ill-suited to
industry use because results drawn from model analysis must then be communicated
by the model architect to other stake-holders in the model, meaning that the
architect of the model must be adept in both modeling and communication. Other
modeling platforms, such as UML\citep{rumbaugh2017unified} or
bigraphs\citep{milner2009space} also represent models in specific ways to aid
non-technical understanding.
Methods for expressing the outcome of an analysis
--- to a model architect and, on occasion, to non-technical users of the model
--- are the modeling paradigm's solutions to \emph{exposition}.\par

With these three properties of a modeling framework in mind, it is helpful to
understand different approaches to system modeling, through the lens of these
simple input, process, and output concepts.


\subsection{BPMN}





\section{Working toward alternative models}

\subsection{Behavioral Variance via Process Fuzzing}

\subsection{A detour: genetic programming?}

\subsection{Back on track: security policies}



\section{The Road Ahead}  % Where do we go from here?

\subsection{Thesis Statement}

\subsection{Tooling}

\subsection{Experiment 1}

\subsection{Experiment 2}



\section{Risks Involved}

\subsection{Validating and Verifying Models}

\subsection{Breadth}
% The potential impact is kind of broad!


\newpage  % TODO: consider deleting
\bibliography{lib}

\end{document}