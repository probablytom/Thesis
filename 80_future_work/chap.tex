\chapter{Future Work}\label{chap:future_work}

The focus of this thesis is in developing a state-of-the-art aspect-oriented 
framework, producing a suitably constrained experimental environment to
demonstrate its effectiveness, and using that environment to investigate whether
aspect orientation\inline{is suitable for simulation purposes? It's about
showing that we \emph{can} use aspect orientation appropriately in simulation
environments, and that aspect orientation can also lead us to realistic and
nuanced simulations, too. Go back through the pdsf and lit review chapters to
make this argument properly.}. As we have found that aspect orientation is
appropriate in this context, successfully produced this well-constrained
environment for simulation, and produced a novel aspect orientation framework
which demonstrates novel and powerful weaving concepts, lots of opportunities
for research outwith this thesis' scope present themselves.

This chapter describes some possibilities for the presented research to be
extended in the future.

\inline{Rewrite the chunks in each future work section for this chapter, into their own
subsecs with proper explorations, citations, and so on. This chapter is
currently a scapbook of ideas!}

\labelledsec{Future Work pertaining to PyDySoFu}{future_work_pdsf}

\labelledsubsec{Aspect-Oriented Metaprogramming}{aspect_oriented_metaprogramming}

The combination of metaprogramming and aspect-orientation introduces powerful
new possibilities in the realm of aspect-orientation. In traditional
aspect-oriented work, aspects treat their targets as black boxes. This leads to
some limitations:

\begin{itemize}
    \item Traditional aspects cannot add their behavioural modifications
    interspersed within the work being done by their target. The ``textbook''
    use-case for aspect-orientation is logging: aspects can separate logging
    from the business logic they are applied to. However, a programmer in more
    mainstream programming styles may wish to insert logging behaviour
    \emph{within} their business logic, rather than \emph{around} it.
    Aspect-oriented metaprogramming makes this possible, as the target can have
    logging logic interspersed through otherwise decoupled business logic when
    woven. As even in aspect-orientation's most famous example there are
    benefits to the introduction of aspects woven within their targets, a study
    of the broader utility of the aspect-oriented metaprogramming approach
    should be conducted.
    \item Traditional aspects cannot make decisions based on reflection on
    specific properties of the code they are being applied to, such as
    calculations made, data accessed, computational complexity, and so on. There
    are many scenarios where one can anticipate this reflective behaviour to be
    useful. For example, compilation of Python code for efficient performance on
    a GPU (redirected from the CPU), as in \inline{cite the thesis here of
    Jeremy's student who worked in Python. He was very nice --- can't remember
    his name for the life of me.}, seems to decouple from the rewritten logic
    nicely in concept, but relies on an examination of iteration logic and
    specifics of the code being recompiled. Aspects with metaprogramming
    directly support reflection as access to the target's AST is trivial to
    achieve. There are many possible applications for this technology, and
    likely in a diverse set of domains; there is therefore research to be done
    to demonstrate the utility of this new approach.
    \item \inline{Would be nice to have a third reason as to why \pdsf
    introduces new research possibilities.}
\end{itemize}


\labelledsec{Future Work pertaining to RPGLite}{future_work_rpglite}

RPGLite's dataset was analysed for the purposes it was collected for in this
thesis: to aid in the realistic simulation of a well-controlled \sociotechnical
system. However, many analyses are yet to be explored:

\begin{itemize}
    \item Why were games abandoned? Are there patterns that can be identified
    which lead players to abandon games?
    \item Players likely formed cliques, where they would play against people
    they knew (perhaps in person) rather than relying on RPGLite's matchmaking
    features to find new opponents. The existence of cliques of players may have
    implications for the playstyles of players, how they learned ``better''
    strategies over time, and the players' general dedication to playing RPGLite
    (and therefore producing a greater wealth of data for analysis and
    dissemination to the community)
    \item RPGLite's dataset contains information about players' interactions
    with the application itself; as the game made available some features
    typical of modern games (leaderboards, matchmaking, achievements, graphical
    customisation), an analysis of the features most commonly used can shed
    light on the more effective aspects of modern game design in both the
    general playerbase and more dedicated players\footnote{For example, are some
    features heavily used, but only by a dedicated subset? Do all players use
    other features a moderate amount, showing mild but general appeal?}
    \item RPGLite's playerbase was recruited informally and there is scope for a
    larger and longer-term data collection effort to be made. A re-release of
    the application in major mobile app stores with a concerted effort to
    release new seasons of the game and maintain player interest for an extended
    period of time
    --- perhaps with additional features, such as in-game chat, favourites lists
    --of
    previous opponents, or match replay and analysis (with suggestions for
    improved play backed by the formal methods inherant in RPGLite's design)
    would enable a richer analysis, and broader utility to the games research
    community.
\end{itemize}

While investigations into these questions warrant further study, they remain
outwith the scope of this thesis, which focuses on simulation technologies more
than it does game design. There are many opportunities available for the game
design research community to investigate. Publications in the field from
co-creators of RPGLite reflect further on the design and future improvements of
the game; see \inline{cite William's PhD here}.


\labelledsec{Future Work pertaining to Aspect-Oriented Simulation}{future_work_simulations}

Aspect-orientation's goal of separation of concerns, and the possibility of
using its trait of obliviousness to augment naive models in ways the original
creator did not anticipate, presents research opportunities that are also
outwith the scope of this thesis.

\begin{itemize}
    \item Myriad models exist which have been provided accurate results in past
    research, but could not account for unforseen modern situations. For
    example, models of world health over time could not account for the Covid19
    pandemic, and models of the world economy could not incorporate real-world
    data from the recession caused by responses to the pandemic, or the 2008
    financial crisis. The World3 model is an example of one which has provided
    accurate predictions for decades, but could not account for incidents in
    modern times when constructed. Models such as these work from prior data which
    requires some adjustment as simulated time progresses to account for events
    of a large enough scale to disrupt their simulated system (here, global
    population, industry, food, resources, and pollution). An alternative to
    adjusting the models directly --- adding cases at the relevant points in
    time to introduce ``blips'' in simulated data --- is to construct aspects
    which represent global events such as pandemics, economic crises, and others
    such as war or famine. These can be modelled on real-world data, which we
    have shown in this research to produce realistic simulations. A
    proof-of-concept of the approach as applied to pre-existing models would
    start this work, and the augmentation of existing models to improve their
    accuracy can follow.
    \item Relatedly, aspects can represent anticipated future states so as to
    model their potential impact without modifying a known-good model of the
    world today. Future health and economic crises can be constructed as
    prospective changes to a model in an aspect, and applied to investigate the
    possible effects. A potential benefit of this approach as opposed to the
    simple modification of an existing model would be that many potential crises
    can be applied in any combination. For example, 10 aspects representing
    unpredictable future events yield 1024 possible combinations: there are
    \(2^{10}\) possible combinations of these aspects being applied or omitted
    from an execution of a simulation. Work to develop aspect-oriented models of
    speculative futures therefore gives an exponential number of predicted
    futures, which one could analyse to predict possible future trends. With a
    successful proof-of-concept of the augmentation of existing models to
    represent past events, this further step could anticipate future events and
    take advantage of aspect orientation's unique properties as a tool for
    simulation and modelling.
    \item Combinations of traits in human modelling where a player might not fit
    to one specific trait well, but fits to a combination applied weakly.
\end{itemize}


\labelledsec{Testing Frameworks to detect Unrealistic Behavioural Variances}{sciunits_for_unrealistic_states}

Given we don't know the impact of variances exactly, something like sciunits
could give us "bounds" on realism in our model, i.e. the sciunit should encode
limits on what the real-world system does, and let us know whether those limits
are broken when applying variances (or combinations of them)

\labelledsec{Standards for Model Features}{standard_aosm_techniques}

How can we make a more "principled" approach to adopting features for modelling?
PDSF's approach to adding things to a model is powerful and flexible. A
technique that would work for broad applications could allow for a standard in
the RSE community for simulations, at least for sociotechnical simulation. What
are the best practices around the modification of scientific models in this way,
that could encourage collaboration and the sharing of models across groups?

\labelledsec{Optimisation of multiple models}{many_aspectual_models_to_optimise}

An explanation of the future work in \cref{subsubsec:ensure_best_move}, where we
suggest that it'd be interesting future work for somebody to anneal to multiple
models of aspectually applied behavioural variance. 

The work to do on this point is relatively trivial --- just a grid search on
many dimensions really --- but we've not done it and it'd risk detracting from
our goal anyway, which is to show that we can optimise a model (so we should
keep things from being unnecessarily complicated!) so worth leaving for an
honours / masters dissertation.

\labelledsec{Discussion}{future_work_conclusion}

This section is not intentionally left blank.

