\chapter{Future Work}\label{chap:future_work}

The focus of this thesis is to develop a state-of-the-art aspect-oriented
framework, to produce a suitable experimental environment to demonstrate its
effectiveness, and to use that environment to investigate whether aspect
orientation can be used to augment models with behavioural variance. \inline{is
suitable for simulation purposes? It's about showing that we \emph{can} use
aspect orientation appropriately in simulation environments, and that aspect
orientation can also lead us to realistic and nuanced simulations, too. Go back
through the pdsf and lit review chapters to make this argument properly.}. We
have found that aspect orientation can be used to create realistic and nuanced
simulations, have successfully produced a well-constrained environment to
simulate for testing purposes, and have produced a novel aspect orientation
framework which demonstrates novel and powerful weaving concepts; lots of
opportunities for research outwith this thesis' scope present themselves.

This chapter describes some possibilities for the presented research to be
extended in the future.

\inline{Rewrite the chunks in each future work section for this chapter, into their own
subsecs with proper explorations, citations, and so on. This chapter is
currently a scrapbook of ideas!}

\subsection{Aspect-Oriented Metaprogramming in real-world Software Engineering}
\label{subsec:aspect_oriented_metaprogramming}

The combination of metaprogramming and aspect-orientation introduces powerful
new possibilities in the realm of aspect-orientation. In traditional
aspect-oriented work, aspects treat their targets as black boxes; this leads to
some limitations, which aspect-oriented metaprogramming is well positioned to address.

Traditional aspects cannot add their behavioural modifications interspersed
within the work being done by their target. The ``textbook'' use-case for
aspect-orientation is logging: aspects can separate logging from the business
logic they are applied to. However, a programmer in mainstream programming
paradigms may wish to insert logging behaviour \emph{within} their business
logic, rather than \emph{around} it. Aspect-oriented metaprogramming makes this
possible, as the target can have logging logic interspersed through otherwise
decoupled business logic when woven. To achieve this end goal without
``within''-style aspect application would require a refactoring of business
logic to create join points which traditional aspects could apply against; this
breaks the ``obliviousness'' property of the aspect orientated philosophy as
discussed in \cref{chap:lit_review}.


\inline{These were originally written as separate points, but they shouldn't be. Rework the preceding and following paragraphs so they read like a series of points as to the opportunities aomp \& \pdsf make available.}\inline{Maybe this whole thing's just one subsection? Maybe it should live under a single [section] heading?}    

Another example of this technique's applicability in a variety of real-world
situations is that traditional aspects cannot make decisions based on reflection
on specific properties of the code they are being applied to. This could include
calculations made, data accessed, or computational complexity as examples. A
concrete example would be the compilation of Python code for efficient
performance on a GPU (redirected from the CPU)~\cite{dejice2020thesis}, which is
achieved via a JIT compiler which optimises looping logic for parallel execution
on a GPU. Similar optimisations could be made through the analysis of common
codepaths and profiling of expensive parts of those codepaths, and the automated
optimisation of them. Research projects are also enabled which do not require
any optimisation heuristics or logic: as \pdsf makes both the target of aspect
application and the arguments passed to that target visible, it enables
instrumentation of programs\footnote{In a similar manner to the instrumentation
of RPGLite mdoels in \cref{subsec:aspects_instrumenting_model}.} to identify ---
for example --- whether arguments with certain properties cause slowdowns on a
codebase. In a hypothetical situation where a codebase was extremely performant
on a dataset except when processing frames of the dataset containing prime
numbers, isolating data causing worst-case performance and either automatically
analysing it or presenting it for a researcher's analysis could be a useful
feature for resource-constrained RSE teams, or as a built-in feature of
profilers.

Aspects with metaprogramming directly support reflection, as access to the
target's AST is provided as a feature of the technique, and allow transformed
code to be compiled just-in-time, which directly enables research projects such
as these. A study of the utility of the aspect-oriented metaprogramming approach
would support the development of these tools and techniques.


\section{Future Work pertaining to Aspect-Oriented Simulation}\label{sec:future_work_simulations}

This thesis has described novel techniques augmenting models with new features
and behaviours, and does so in an aspect-oriented manner. The potential for
improving simulations and models using aspect orientation has already been
discussed (most notably by \citeauthor{gulyas1999use}~\cite{gulyas1999use} as
discussed in \cref{ao_and_modelling}); however, \pdsf and aspect-oriented
metaprogramming present new opportunities in the field. This section discussed
avenues for future research which \pdsf enables.


\subsection{Augmentation of pre-existing models}

Myriad models exist which have been provided accurate results in past research,
but where unexpected events such as financial collapses, pandemics, or
unpredictable weather events cause one-off shifts in real-world data which
cannot be accounted for. For example, models of world health over time could not
account for the Covid19 pandemic, and models of the world economy could not
incorporate real-world data from the recession caused by responses to the
pandemic, or the 2008 financial crisis. The World3 model is an example of one
which has provided accurate predictions for decades\inline{Find citations for
World3 model}, but could not account for unexpected incidents when constructed.

Models are often initialised with data describing a system's initial state,
which may lead to correct predictions prior to a large event. Future predictions
could only represent the impact of a large, unpredictable event within the
system as changes to data or process; changing either for the entire model risks
compromising a model's predictive quality prior to an event. The modification
must therefore be represented as a special case within the model, which
distracts from the model's core logic in a manner not unlike the scattering and
tangling discussed in \cref{lit_review_AOP_explainer}. It therefore has similar
properties to a cross-cutting concern~\cite{kiczales1997aspect} and may be
suitable for factoring out of a model's logic and into advice.

The resource constraints and stringent requirements for accuracy in research
codebases also present challenges which augmentation through aspects could
assist with. Compensating for a special case in a pre-existing model would
require maintenance of the codebase, which takes time and could inadvertently
alter its behaviour. Time is a scarce resource in research environments, and
undesired changes to a model's behaviour can invalidate research results. An
alternative to adjusting the models directly is to construct aspects which
represent large and unpredictable events in real-world systems such as
pandemics, economic crises, war and famine. These can be modelled on real-world
data for accuracy, which can produce realistic simulations as this thesis
demonstrates. This use of aspect orientation in simulation and modelling can be
investigated by creating a proof-of-concept of the approach as applied to
pre-existing models.\footnote{This approach is similar to \pdsf{}'s initial
proof-of-concept study~\cite{wallis2018caise} as discussed in
\cref{chap:prior_work}.} Studies can also be conducted to investigate
whether an aspectually-augmented model is quicker to construct and easier to
maintain in future than a codebase with ``patches'' written into its original
logic.

Researchers investigating this technique's application to existing models could
also investigate the difficulties of augmenting a model created with no
intention of weaving advice in the future. The construction of advice requires
appropriate join points to be specified, and codebases which are structured in a
way which doesn't yield convenient join points might be more complex to augment
aspectually. These cases raise another use-case of \pdsf{}'s ``within''-style
weaving through runtime metaprogramming: where other aspect orientation
frameworks force aspects to treat the targets they are invoked on as
black-boxes, \pdsf can make modifications within them. \pdsf{}'s new kinds of
aspects may therefore be more useful than those in other frameworks for
researchers responsible for maintaining codebases written with no consideration
for join-point specification, and would therefore be more universally
applicable; this possibility requires future investigation.

If the research described successfully shows that aspectually-augmented
simulations are easier and quicker for an RSE to maintain and deliver than
direct maintenance of the model's codebase, then augmentation of existing models
to improve their accuracy can follow in the community.
\revnote{If I've got time, I'd love to push my codebase augmenting a World3
model to achieve just this with predictive implications for the 2008 financial
crisis and the covid pandemic.}
    








% ======= mark: above written, below yet to write (only stubs).
% ======= 230806







\subsection{Aspects to make Predictions with Uncertain System Dynamics}

% Relatedly, aspects can represent anticipated future states\inline{rework; it's
% not anticipated future states, it's making long-term predictions based on
% small but discrete changes to existing dynamics, which can be quantised as
% aspects.} to model their potential impact without modifying a known-good model
% of the world today. Future health and economic crises can be constructed as
% prospective changes to a model in an aspect, and applied to investigate the
% possible effects. A potential benefit of this approach as opposed to the
% simple modification of an existing model would be that many potential crises
% can be applied in any combination. For example, 10 aspects representing
% unpredictable future events yield 1024 possible combinations: there are
% \(2^{10}\) possible combinations of these aspects being applied or omitted
% from an execution of a simulation. Work to develop aspect-oriented models of
% speculative futures therefore gives an exponential number of predicted
% futures, which one could analyse to predict possible future trends. With a
% successful proof-of-concept of the augmentation of existing models to
% represent past events, this further step could anticipate future events and
% take advantage of aspect orientation's unique properties as a tool for
% simulation and modelling.

%% TODO: start the below describing the _problem_, then move to the pdsf
%        solution

Similarly to augmenting models with aspects representing unpredictable events,
aspects might be used as hypotheses as to parts of a system's dynamics which are
unclear to researchers at time of modelling. 


\subsection{FuzziMoss-style libraries of sociotechnical
fuzzers}\label{future_work_revive_fuzzimoss}

We could try to revive FuzziMoss and implement a broader range of behaviourally
varying aspects. The final result of this research might be mis-interpreted as
indicating that this is a non-starter, but standard operations could still be
useful \emph{if not fitted to real-world data and re-used}, providing to
researchers standard variations which are fitted to specific use-cases, or
treated as building blocks to produce more complex models of behavioural
variation, either to apply to a system under study or to validate a theory about
the variation itself.


\subsection{Standards for Model Features}\label{subsec:standard_aosm_techniques}

How can we make a more "principled" approach to adopting features for modelling?
PDSF's approach to adding things to a model is powerful and flexible. A
technique that would work for broad applications could allow for a standard in
the RSE community for simulations, at least for sociotechnical simulation. What
are the best practices around the modification of scientific models in this way,
that could encourage collaboration and the sharing of models across groups?


\subsection{Testing Frameworks to detect Unrealistic Behavioural Variances}\label{sciunits_for_unrealistic_states}

Given we don't know the impact of variances exactly, something like sciunits
could give us "bounds" on realism in our model, i.e. the sciunit should encode
limits on what the real-world system does, and let us know whether those limits
are broken when applying variances (or combinations of them)


\subsection{Optimisation of multiple models}\label{many_aspectual_models_to_optimise}

An explanation of the future work in \cref{subsubsec:ensure_best_move}, where we
suggest that it'd be interesting future work for somebody to anneal to multiple
models of aspectually applied behavioural variance. 

The work to do on this point is relatively trivial --- just a grid search on
many dimensions really --- but we've not done it and it'd risk detracting from
our goal anyway, which is to show that we can optimise a model (so we should
keep things from being unnecessarily complicated!) so worth leaving for an
honours / masters dissertation.



\section{Future Work pertaining to Aspect-Oriented Simulation}\label{sec:future_work_simulations}

Aspect-orientation's goal of separation of concerns, and the possibility of
using its trait of obliviousness to augment naive models in ways the original
creator did not anticipate, presents research opportunities that are also
outwith the scope of this thesis.

\subsection{Augmentation of pre-existing models}

Myriad models exist which have been provided accurate results in past research,
but could not account for unforseen modern situations. For example, models of
world health over time could not account for the Covid19 pandemic, and models of
the world economy could not incorporate real-world data from the recession
caused by responses to the pandemic, or the 2008 financial crisis. The World3
model is an example of one which has provided accurate predictions for decades,
but could not account for incidents in modern times when constructed. Models
such as these work from prior data which requires some adjustment as simulated
time progresses to account for events of a large enough scale to disrupt their
simulated system (here, global population, industry, food, resources, and
pollution). An alternative to adjusting the models directly --- adding cases at
the relevant points in time to introduce ``blips'' in simulated data --- is to
construct aspects which represent global events such as pandemics, economic
crises, and others such as war or famine. These can be modelled on real-world
data, which we have shown in this research to produce realistic simulations. A
proof-of-concept of the approach as applied to pre-existing models would start
this work, and the augmentation of existing models to improve their accuracy can
follow.
    

\subsection{Aspects to make Predictions with Uncertain System Dynamics}

Relatedly, aspects can represent anticipated future states\inline{rework; it's
not anticipated future states, it's making long-term predictions based on small
but discrete changes to existing dynamics, which can be quantised as aspects.}
to model their potential impact without modifying a known-good model of the
world today. Future health and economic crises can be constructed as prospective
changes to a model in an aspect, and applied to investigate the possible
effects. A potential benefit of this approach as opposed to the simple
modification of an existing model would be that many potential crises can be
applied in any combination. For example, 10 aspects representing unpredictable
future events yield 1024 possible combinations: there are \(2^{10}\) possible
combinations of these aspects being applied or omitted from an execution of a
simulation. Work to develop aspect-oriented models of speculative futures
therefore gives an exponential number of predicted futures, which one could
analyse to predict possible future trends. With a successful proof-of-concept of
the augmentation of existing models to represent past events, this further step
could anticipate future events and take advantage of aspect orientation's unique
properties as a tool for simulation and modelling.


\subsection{FuzziMoss-style libraries of sociotechnical
fuzzers}\label{future_work_revive_fuzzimoss}

We could try to revive FuzziMoss and implement a broader range of behaviourally
varying aspects. The final result of this research might be mis-interpreted as
indicating that this is a non-starter, but standard operations could still be
useful \emph{if not fitted to real-world data and re-used}, providing to
researchers standard variations which are fitted to specific use-cases, or
treated as building blocks to produce more complex models of behavioural
variation, either to apply to a system under study or to validate a theory about
the variation itself.


\subsection{Standards for Model Features}\label{subsec:standard_aosm_techniques}

How can we make a more "principled" approach to adopting features for modelling?
PDSF's approach to adding things to a model is powerful and flexible. A
technique that would work for broad applications could allow for a standard in
the RSE community for simulations, at least for sociotechnical simulation. What
are the best practices around the modification of scientific models in this way,
that could encourage collaboration and the sharing of models across groups?


\subsection{Testing Frameworks to detect Unrealistic Behavioural Variances}\label{sciunits_for_unrealistic_states}

Given we don't know the impact of variances exactly, something like sciunits
could give us "bounds" on realism in our model, i.e. the sciunit should encode
limits on what the real-world system does, and let us know whether those limits
are broken when applying variances (or combinations of them)


\subsection{Optimisation of multiple models}\label{many_aspectual_models_to_optimise}

An explanation of the future work in \cref{subsubsec:ensure_best_move}, where we
suggest that it'd be interesting future work for somebody to anneal to multiple
models of aspectually applied behavioural variance. 

The work to do on this point is relatively trivial --- just a grid search on
many dimensions really --- but we've not done it and it'd risk detracting from
our goal anyway, which is to show that we can optimise a model (so we should
keep things from being unnecessarily complicated!) so worth leaving for an
honours / masters dissertation.




\section{Future Work pertaining to RPGLite}\label{sec:future_work_rpglite}
RPGLite's dataset was analysed for the purposes it was collected for in this
thesis: to aid in the realistic simulation of a well-controlled \sociotechnical
system. However, many analyses are yet to be explored:

\subsection{Causes of Game Abandonment}
Why were games abandoned? Are there patterns that can be identified
which lead players to abandon games?
    
\subsection{Patterns of Play within Cliques of Players}
Players likely formed cliques, where they would play against people they knew
(perhaps in person) rather than relying on RPGLite's matchmaking features to
find new opponents. The existence of cliques of players may have implications
for the playstyles of players, how they learned ``better'' strategies over time,
and the players' general dedication to playing RPGLite (and therefore producing
a greater wealth of data for analysis and dissemination to the community)
    
\subsection{Additional Game Features Driving Engagement}
RPGLite's dataset contains information about players' interactions with the
application itself; as the game made available some features typical of modern
games (leaderboards, matchmaking, achievements, graphical customisation), an
analysis of the features most commonly used can shed light on the more effective
aspects of modern game design in both the general playerbase and more dedicated
players\footnote{For example, are some features heavily used, but only by a
dedicated subset? Do all players use other features a moderate amount, showing
mild but general appeal?}\footnote{Find related research on engagement in mobile
games, which must be \emph{plentiful}.}.
    

\subsection{Larger-scale Data Collection}
RPGLite's playerbase was recruited informally and there is scope for a larger
and longer-term data collection effort to be made. A re-release of the
application in major mobile app stores with a concerted effort to release new
seasons of the game and maintain player interest for an extended period of time
--- perhaps with additional features, such as in-game chat, favourites lists
--of previous opponents, or match replay and analysis (with suggestions for
improved play backed by the formal methods inherant in RPGLite's design) would
enable a richer analysis, and broader utility to the games research community.

While investigations into these questions warrant further study, they remain
outwith the scope of this thesis, which focuses on simulation technologies more
than it does game design. There are many opportunities available for the game
design research community to investigate. Publications in the field from
co-creators of RPGLite reflect further on the design and future improvements of
the game; see \inline{cite William's PhD here}.


\section{Discussion}\label{sec:future_work_conclusion}

This section is not intentionally left blank.

