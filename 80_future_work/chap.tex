\chapter{Future Work}\label{chap:future_work}

The focus of this thesis is to develop a state-of-the-art aspect-oriented
framework, to produce a suitable experimental environment to demonstrate its
effectiveness, and to use that environment to investigate whether
\aspectorientation{} is a suitable tool for the augmentation of simulation \&
modelling codebases. \Cref{chap:experiment_setup} and
\cref{chap:experimental_results} showed that \aspectorientation{} can be used to
create a realistic and nuanced model using \pdsfthree{} to model RPGLite. Having
successfully achieved the aims of the project, these contributions enable
opportunities which are outwith this thesis' scope.

This chapter describes some of those opportunities. It illustrates both the
possibilities to extend the research presented in this thesis, and the other
research projects adjacent to this one which are enabled by this thesis'
contributions.

\section{Aspect-Oriented Metaprogramming in Real-World Software Engineering}
\label{future_work_aspect_oriented_metaprogramming}

The combination of metaprogramming and \aspectorientation introduces new
possibilities in \aop{}. In traditional \aspectorientation frameworks, aspects
treat their targets as black boxes. This leads to limitations which
\aspectoriented{} metaprogramming such as \pdsfthree{}'s fuzzers can address.

Traditional aspects cannot intersperse their behavioural modifications with the
work being done by their target, as they apply their logic before or after the
target's execution. The ``textbook'' use case for aspect-orientation is logging:
aspects can separate logging from the business logic they are applied to.
However, a programmer in mainstream programming paradigms may wish to insert
logging behaviour \emph{within} their business logic rather than \emph{around}
it. Aspect-oriented metaprogramming enables this as the target can have logging
logic woven within business logic, which remains decoupled within the codebase.
To achieve this end goal without ``within''-style aspect application would
require a refactoring of business logic to create join points which traditional
aspects could apply against; this somewhat weakens the ``obliviousness'' property of the
\aspectoriented{} philosophy discussed in
\cref{review_aop_motivations_and_philosophy} in that target code must be modified
to apply advice. If the code must be modified anyway, that modification could
introduce new logic rather than refactoring what already exists so that an
aspect can do the same. 

To demonstrate the improved pragmatism of \aop{} in real-world software
engineering, existing codebases should be augmented using aspects. This work
could take many forms. An industrial partner could adopt \aop{} in an existing
project with consultation from a research team, who discuss the industrial
team's experiences after some period of \aspectoriented development is
completed. Alternatively, pull-requests in open-source projects could be
reimplemented by a research team using \aspectorientation{}, and the two
implementations could be compared by the project's maintainers, who could be
interviewed by the research team to discern whether \aop{} could be of practical
use to them. The team may see \aop{} as a means improve the codebase's
modularity and maintainability, or to deploy hotfixes in their code and features
hidden behind feature flags which can be disabled during runtime by making use
of dynamic aspect weaving. The research team may make use of
\aspectoriented{}
metaprogramming such as \pdsfthree{}'s fuzzers to better support the team's
application of \aop{} to an existing industrial codebase. Some promising related
work has been undertaken by \citet{przybylek2018empirical}, who compared student
comprehension of \aspectoriented{} programs but did not evaluate the paradigm's
appeal to industrial teams or its effectiveness as perceived by engineers in
industry.  

In either of the research projects suggested, the academic team seek to answer a
research question such as:

\begin{researchquestion}
Can \aop{} be used in existing industrial codebases to improve modularity and
maintainability of the project as perceived by that project's team of maintainers?
\end{researchquestion}


% ==== These were originally written as separate points, but they shouldn't be.
%      Rework the preceding and following paragraphs so they read like a series
%      of points as to the opportunities aomp \& \pdsf make
%      available.}\inline{Maybe this whole thing's just one subsection? Maybe it
%      should live under a single [section] heading?

% Another example of this technique's applicability in a variety of real-world
% situations is that traditional aspects cannot make decisions based on reflection
% on specific properties of the code they are being applied to. This could include
% calculations made, data accessed, or computational complexity as examples. A
% concrete example would be the compilation of Python code for efficient
% performance on a GPU (redirected from the CPU)~\cite{dejice2020thesis}, which is
% achieved via a JIT compiler which optimises looping logic for parallel execution
% on a GPU. Similar optimisations could be made through the analysis of common
% codepaths and profiling of expensive parts of those codepaths, and the automated
% optimisation of them. Research projects are also enabled which do not require
% any optimisation heuristics or logic: as \pdsf makes both the target of aspect
% application and the arguments passed to that target visible, it enables
% instrumentation of programs\footnote{In a similar manner to the instrumentation
% of RPGLite mdoels in \cref{subsec:aspects_instrumenting_model}.} to identify ---
% for example --- whether arguments with certain properties cause slowdowns on a
% codebase. In a hypothetical situation where a codebase was extremely performant
% on a dataset except when processing frames of the dataset containing prime
% numbers, isolating data causing worst-case performance and either automatically
% analysing it or presenting it for a researcher's analysis could be a useful
% feature for resource-constrained RSE teams, or as a built-in feature of
% profilers.

% Aspects with metaprogramming directly support reflection, as access to the
% target's AST is provided as a feature of the technique, and allow transformed
% code to be compiled just-in-time, which directly enables research projects such
% as these. A study of the utility of the aspect-oriented metaprogramming approach
% would support the development of these tools and techniques.






% \section{Future Work pertaining to Aspect-Oriented
% Simulation}\label{future_work_simulations}

% This thesis has described novel techniques augmenting models with new features
% and behaviours, and does so in an aspect-oriented manner. The potential for
% improving simulations and models using aspect orientation has already been
% discussed (most notably by \citeauthor{gulyas1999use}~\cite{gulyas1999use} as
% discussed in \cref{ao_and_modelling}); however, \pdsf and aspect-oriented
% metaprogramming present new opportunities in the field. This section discussed
% avenues for future research which \pdsf enables.






\section{Augmentation of Pre-Existing Models}
\label{future_work_nudge_model_state}
\label{future_work_using_aspects_to_correct_simulated_state}

Models which predict the future state of some system may be accurate in the
general case,
but cannot account for unforseeable events such as financial collapses,
pandemics, or unpredictable weather events. Pre-existing models cannot account
for these one-off shifts in system state due to the random nature of these events. For
example, models of world health over time could not account for the Covid19
pandemic, and models of local or global economies could not incorporate
real-world data from the recession resulting from the pandemic. The 2008
financial crisis also impacted local and global economies, but could not be
predicted before the event. An example of a model predicting both health and
economic outcomes is the World3 model, which has
provided accurate high-level predictions of the state of various global systems
over many decades~\cite{branderhorst2020update}, but predictions for
recent years fail to account for these unforseeable events~\cite{nebelrecalibration}.

If the modification is represented as a special case within the model, a
research software engineer must introduce their corrective code alongside the
model's core logic, thereby tangling\footnote{For an overview on the tangling of
cross-cutting concerns, see \cref{review_aop_motivations_and_philosophy}.} the
two. It therefore has the properties of a cross-cutting
concern~\cite{kiczales1997aspect,filman2000aspect}, and can be factored into an
aspect as such. 

Research investigating the use of aspects to correct for one-off events could
aim to show that an aspect can ``nudge'' a system's state in-line with
real-world data when exceptional circumstances affect a modelled system but
cannot be elegantly represented within the model's logic. 
If this technique proves successful, it may also lead to other novel ways of
simulating systems. For example, future unforeseeable events could be
compensated for by predicting them at different points in a simulated system's
future and observing how their impact changes the simulation over time. This is
discussed further in \cref{future_work_hypothesising_system_states}.

While there is potential for lots of related research in the engineering of
models, a first study should be conducted to demonstrate that the concept is
sound and that it is useful in practice. A research question
this early study could answer is:

\begin{researchquestion}
Can aspect orientation be used to introduce special cases to real-world systems
to correct a model's predictions in when a system under research is altered by
unforeseeable ``freak events''?
\end{researchquestion}




\section{Aspect Orientation's Utility for Research Software Engineers}
\label{future_work_aop_for_rses}

A corollary of the research opportunities for aspect orientation's use in
software engineering discussed in
\cref{future_work_aspect_oriented_metaprogramming} is that there is an
opportunity for research software engineers to benefit from the adoption of
aspect oriented programming. However, while aspect orientation's use in
industrial software engineering has drawn
criticism~\cite{steimann06paradoxical,przybylek2010wrong,Constantinides04aopconsidered}
its use within research codebases is a special case where it may be more
suitable. The results presented in \cref{chap:experimental_results} show that
\aop{} \emph{can} be successfully employed in research codebases to represent
changes to models. Related suggestions for the use of aspect orientation in
research codebases were proposed by \citeauthor{gulyas1999use} (as discussed in
\cref{review_gulyas_use_of_aop_in_research_codebases}). However, the benefits
proposed in earlier work concerned the design of the software itself. Aside from
aspect-orientation's utility in software design in a research context, there are
potential benefits for the \emph{practice} of developing these codebases which
may be of professional interest to research software engineers (``RSEs'').


The resource constraints and stringent requirements for accuracy in research
codebases present challenges which aspect-orientation could assist with.
Compensating for a special case in a pre-existing model would require
maintenance of the codebase, which takes time. Care must be taken not to
inadvertently alter its behaviour. Time is a scarce resource in research
environments, and undesired changes to a model's behaviour can invalidate
research results. An alternative to adjusting the models directly is to
construct aspects which represent large and unpredictable events in real-world
systems such as pandemics, economic crises, war and famine. These can be
modelled on real-world data for accuracy, which can produce realistic
simulations as \cref{chap:experimental_results} demonstrates. The use of aspect
orientation in simulation and modelling can be further investigated by creating
a proof-of-concept of the approach as applied to pre-existing
models.\footnote{This approach is similar to \pydysofu{}'s initial proof-of-concept
study~\cite{wallis2018caise} as discussed in \cref{chap:prior_work}.} Studies
can also be conducted to investigate whether an aspectually-augmented model is
quicker to construct and easier to maintain in future than a codebase with
``patches'' written into its original logic.

Researchers investigating this technique's application to existing models could
also investigate the difficulties of augmenting a model originally implemented
without any intention of weaving advice in the future. The construction of
advice requires appropriate join points to be specified, and codebases which are
structured in a way which doesn't yield convenient join points might be more
complex to augment aspectually. These cases raise another use-case of \pdsfthree{}'s
``within''-style weaving through runtime metaprogramming: where other aspect
orientation frameworks force aspects to treat the targets they are invoked on as
black boxes, \pdsfthree can make modifications within them. \pdsfthree{}'s fuzzers may
make \aop{} more useful than the aspects offered by other frameworks for
maintainers of research codebases. This possibility requires future
investigation.

If the research described successfully shows that aspectually-augmented
simulations are easier and quicker for a research software engineer to maintain
and deliver than direct maintenance of the model's codebase, then augmentation
of existing models to improve their accuracy can follow in the community.
 \revnote{If I've got time, I'd love to push my codebase augmenting a World3
model to achieve just this with predictive implications for the 2008 financial
crisis and the covid pandemic.}% 
Researchers investigating this may address research questions such as:

\begin{researchquestion}
    \begin{enumerate}
\item Can existing models developed \emph{without} aspectual augmentation in
mind be made more accurate through \aspectoriented{} ``patches''?
\item Can research software engineers use \aop{} to more easily maintain
existing codebases without making invasive changes to experimental source code
in the process?
    \end{enumerate}
\end{researchquestion}

This is differentiated from the contributions in \cref{chap:experiment_setup,chap:experimental_results}
by its application to pre-existing
codebases. It is also related to the proposed benefits to research software
engineers discussed in \cref{future_work_nudge_model_state}.
    
% ==== also wrote the below in another section and realised it's acutally about
% this point. I'd already finished this subsec, but leaving below in case I want
% to use it later.


% Relatedly, model features can be added as extensions through aspect
% orientation with the possibility of lower maintenance cost and increased
% flexibility. By way of illustration, consider researchers investigating a
% car's structural integrity in a collision, and suppose that they can create
% models of physical objects interacting. A scenario might arise where the
% precise mechanics of a car's structural change under impact may be poorly
% understood; many hypotheses may exist as how crumple zones made of a certain
% material absorb an impact and morph under its force. In this scenario, the
% task of the researchers might be to investigate different designs of their car
% to optimise crumple zone performance. Many models could be produced; one for
% each crumple zone. A model could also be constructed with parameterised
% properties of the car, allowing one codebase to consolidate several models.
% However, these parameters are likely to relate to the research at hand; to
% parameterise unnecessary components of the model adds complexity and spends
% development time for no clear gain. Future research projects may investigate
% model components which were not parameterised in the model's initial design.
% This raises a problem: how can the development team work to re-use the model
% while minimising maintenance cost?

% A solution is to add additional parameters as aspects. As aspect orientation
% frameworks supporting within-style changes can make arbitrary changes to the
% codebase, modifications can be made which leave the original model intact, but
% produce an additional model by using aspect orientation as to ``patch'' the
% original. If, in the future, extensions are required for either the original
% model or its extended version, aspects can be added to either. The logical
% conclusion of the design is that all model features could be implemented as
% sets of aspects adding functionality to a simple base model, and a model
% suitable for any piece of work can be constructed by composing relevant model
% components (weaving relevant aspects), extending functionality by adding a new
% behaviour to the set of available models if required. 
    



%% === Explainer for future Tom: I removed the below because it's included in 
%%     \cref{future_work_nudge_model_state}, but left the below here on the
%      offchance I wanted to come back to it when editing. If you don't care,
%      just delete it.
% \subsection{Aspects to Nudge System State}
% \label{future_work_using_aspects_to_correct_simulated_state}
% \revnote{Title too informal? Can't find a good title for this\ldots{}}

% Relatedly, aspects can represent anticipated future states\inline{rework; it's
% not anticipated future states, it's making long-term predictions based on
% small but discrete changes to existing dynamics, which can be quantised as
% aspects.} to model their potential impact without modifying a known-good model
% of the world today. Future health and economic crises can be constructed as
% prospective changes to a model in an aspect, and applied to investigate the
% possible effects. A potential benefit of this approach as opposed to the
% simple modification of an existing model would be that many potential crises
% can be applied in any combination. For example, 10 aspects representing
% unpredictable future events yield 1024 possible combinations: there are
% \(2^{10}\) possible combinations of these aspects being applied or omitted
% from an execution of a simulation. Work to develop aspect-oriented models of
% speculative futures therefore gives an exponential number of predicted
% futures, which one could analyse to predict possible future trends. With a
% successful proof-of-concept of the augmentation of existing models to
% represent past events, this further step could anticipate future events and
% take advantage of aspect orientation's unique properties as a tool for
% simulation and modelling.

%% TODO: start the below describing the _problem_, then move to the pdsf
%        solution



\section{Hypothesising Possible System Dynamics via Aspects}
\label{future_work_hypothesising_system_states}

% In the above section we claim that we can augment models to keep them relevant
% in the face of unpredictable events; we can also investigate uncertain present
% properties of systems by modelling them in a naive manner and encoding
% hypothesised behaviour as aspects; we can correlate real-world data to data
% produced by aspectually agumented simulations to discover which hypothesised
% system is most similar to the real world. We can also use this as a benchmark
% to judge improvements made by improved hypotheses in the future.

Unpredictable events can cause discrepancies between simulated system state at a
given time and the real-world system it models, as discussed in
\cref{future_work_using_aspects_to_correct_simulated_state}. While this
technique presents promising research opportunities, researchers face other kind
of model uncertainty which aspect orientation could also counter. In some
scenarios it is difficult or impossible to make predictions about a system's
future states, because its dynamics are actively being researched. Standard
scientific practice is to create a model to create synthetic datasets which
indicate accuracy if their predictions align with what is empirically observed~\cite{popper1972theoryevolution}.
Some aspects of the system under study may be well-understood.

Rather than creating models of an existing system which encode its hypothesised
behaviour, a naive model can be created which operates as the scientific
consensus understands it. Hypothesised behaviour takes the form of aspects
altering the model in any manner the hypothesis requires. Within-style aspects
enable arbitrary modification to simulated behaviour, which increases \aop{}'s
potential in this use-case. Data produced by each model can be compared to
empirically sourced data, and their similarity quantified, as demonstrated in
\cref{chap:exp1_simulation_optimisation} and
\cref{chap:exp2_old_aspects_new_systems}. An experiment's null hypothesis would
be that the naive model's similarity to empirically sourced data is greater than
that of the aspectually augmented model; the experiment's hypothesis would be
that the behavioural change applied as aspects is more representative of the
system under study than that of the community consensus (encoded in the naive model).

This technique has a satisfying\revnote{Maybe too informal, but I don't know,
maybe live a little\ldots{}?} property: the hypothesis in a given experiment is
completely encoded by its aspectual representation, and if an experiment is
successful then the augmented model can be adopted in future research. In this
way, experimental design and scientific process are directly represented by the
structure of the codebase, and the community's progression to increasingly
accurate models of a system is represented by the progressive adoption of
aspects as ``patches'' to an original theory.

Hypotheses can also be created compositionally in this model. Researchers might
develop a series of potential system properties or behaviours, but are unable to
investigate all reasonable combinations in a timely manner. However, sets of
aspects representing each can be composed to produce $2^N$ models of potentially
realistic behaviours from $N$ hypothesised behaviours by applying each possible
combination. The results produced by each can be compared to an empirical
dataset to identify which combination of behaviours most closely resembles that
of the real-world system under study.

Finally: hypothesised properties of a system might interact, meaning their
discovery could involve the fitting of multiple models of possible behaviours to
discover the properties of a real-world system. The use of aspects to encode ---
and discover the presence of --- hypothesised behaviour is discussed in
\cref{future_work_discovering_emergent_properties} as the RPGLite dataset
presents convenient opportunities for doing so. The optimisation of multiple
aspectual models of behaviour is discussed in
\cref{future_work_many_aspectual_models_to_optimise}.




\section{\AspectOriented Models to Support the Investigation of Scientific Progress}

The technique discussed in \cref{future_work_hypothesising_system_states} for
developing experimental model codebases has another desirable property: it
simultaneously reflects different philosophies of the scientific process in its
encoding of hypotheses and the research community's acceptance of a perspective
in their fields. These philosophies are those of \citeauthor{kuhn2012structure}
and \citeauthor{popper1972theoryevolution}. \citet{kuhn2012structure} explains
the scientific process as inherently social: it starts with a paradigm which is
accepted as broadly true, and accumulates an increasing number of exceptions
until the paradigm itself is deemed unfit, and a new basis for a field's
research is adopted.
\citeauthor{popper1972theoryevolution}~\cite{popper1972theoryevolution} explains
the scientific process as an approximation towards truth, with incremental
progress made with each result achieved by a research community. The proposed
technique for developing experimental model codebases demonstrates features of
both.

To illustrate this, consider the original model a paradigm initially selected by
community consensus. The successful application of aspects would be equivalent
to Popper's incremental movements toward truth as successful experiments are
conducted; Kuhn's exceptions to the agreed model can be identified as
experiments which demonstrate weaknesses in the original model. In this case,
each successive new model adopted by a community is adopted in a Popplarian
manner: improvements are objectively measured, incremental, and would trend
towards truth as a model's behaviour fits empirical observations increasingly
closely. However: over a sufficient period of time, the incremental patching of
an original model would produce an accepted community model which contains a
relatively large amount of discovery and complexity encoded in aspects, as
compared to the original model they are applied to. One would expect the
research community to rewrite the base model to simplify future aspect
application and to more elegantly encode recent research findings; effectively
discarding the original paradigm in favour of a new one. This process is Kuhn's
``paradigm shift'', where paradigms are dropped once a generation of researchers
determine that an originally accepted theory on a topic is unfit for purpose as
evidenced by mounting exceptions in the literature; a new paradigm is to be
accepted by the community, as a new base model would have to be written and
adopted.

The relevant philosophy of science is more nuanced than its brief explanation
here, and the suitability of the approach for the development of research
codebases is to be investigated; the work involved is outwith the scope of this
thesis, but is suggested as future work. A basis for the incremental improvement
of models via aspects is effectively demonstrated in
\cref{chap:exp1_simulation_optimisation} and
\cref{chap:exp2_old_aspects_new_systems}, but the feasibility of the approach as
a basis of a community's scientific process and relation to philosophy of
science warrants further investigation.


\section{Standards for Aspect Orientation in Research Codebases}
\label{future_work_standard_aosm_techniques}
% The above makes the case that aspects might be shared between different
% research groups' experiments, and also makes the case that models with
% different aspects applied should be comparable. To do that, and to encourage a
% community to blossom, there should be guidelines and/or stnadards around
% developing aspectual model features. 


% How can we make a more ``principled'' approach to adopting features for
% modelling? PDSF's approach to adding things to a model is powerful and
% flexible. A technique that would work for broad applications could allow for a
% standard in the RSE community for simulations, at least for sociotechnical
% simulation. What are the best practices around the modification of scientific
% models in this way, that could encourage collaboration and the sharing of
% models across groups?


The possibility of a research community sharing their research as
\aspectoriented{} changes to model involves model logic written as
advice. The community developing these aspects have the responsibilities of
maintaining a codebase as well as the added complexity that research software
engineering introduces. These codebases may be used for many years, and may be
iterated on in a series of future experiments. The legibility of these codebases
and their long-term maintenance are areas of criticism in the software
engineering
community~\cite{Constantinides04aopconsidered,steimann06paradoxical,przybylek2010wrong}.
The research community must therefore mitigate these weaknesses of the
aspect-oriented paradigm when adopting the techniques discussed in this chapter
for simulation and modelling.

To address the concern of the visibility of advice being woven, researchers may
already take advantage of improvements to tooling produced by the aspect
orientation research community, including IDE
integration~\cite{clement2003eclipseAJDT} and runtime
inspection~\cite{mehner2002towards}, should make clear to engineers what advice
is being woven in a codebase and assist with debugging aspect-oriented programs
respectively. In addition, \aspectorientation{} frameworks specifically designed
to clarify to an engineer the aspects being woven should allow for less friction
on the part of a maintainer who inherits a codebase from another developer and
must reason about its behaviour. This is particularly important if the
maintainer aims to weave more aspects into the codebase, and so must understand
its existing behaviour before augmenting it further. Adopting weaving patterns
such as import hook weaving --- described in \cref{chap:pdsf_rewrite} --- should
make a program clearer to a developer regardless of the tooling they have access
to.

The impact of framework design on a codebase's maintenance should assist a
developer even in the absence of tooling, but the success of import hook weaving
in this regard is untested. An appropriate research question which arises is
therefore:

\begin{researchquestion}
Do aspect orientation frameworks with weaving techniques designed to simplify a
developer's understanding of a program affect a codebase's long-term
maintainability?
\end{researchquestion}


\section{Standard Aspect-Oriented Model Features}
\label{standard_aosm_model_features}
\label{future_work_revive_fuzzimoss}
% The above makes the case for creating standards for model features, motivated
% by the possibility that aspects can be shared between research groups or the
% application of different aspects applied. A step toward creating that standard
% could be to create a fuzzimoss-style library of either fundamental fuzzers,
% tooling to simplify common fuzzing operations, or…??

Researchers who build aspect-oriented models and extend others' aspect-oriented
codebases must be able to collaborate at least as easily as they currently do in
a culture without aspect orientation. One way aspect orientation might improve
researchers' ease of collaboration is with standardised libraries for aspect
construction. Similar libraries were developed when developing a case study for
\pdsfthree{}'s viability~\cite{wallis2018caise,fuzzimoss_repo}, but additional
opportunities to complete and expand the tool remain as discussed in
\cref{subsec:prior_work_fm}.

The original library with this aim, Fuzzi-Moss~\cite{fuzzimoss_repo}, was
originally designed to provide standardised aspects to represent behavioural
variance in \sociotechnical systems~\cite{wallis2018caise}. The aspects
developed in Fuzzi-Moss were not developed with a notion of being fitted to
real-world data. Instead, they are simple models of behavioural variances such
as distraction, which are parameterised to allow users of the library to use
these simple models in whatever manner is appropriate for their use-case. A
broader collection of these behavioural variations could simplify the use of
aspect-oriented behavioural variation in the research community writ large, by
removing researchers' burden to develop these themselves. Early construction of
a library with Fuzzi-Moss' goals would also support researchers in sharing
models or extending others', as they would be familiar with a common set of
tools. The development of such a library and the production of case studies
demonstrating its effectiveness would answer the research question:

\begin{researchquestion}
    Can researchers using aspect-oriented behavioural variance share a common
    set of tools to support and simplify the use of the technique in their
    codebases?
\end{researchquestion}

Other tooling to support researchers in the development of aspectually augmented
simulations and models could also be developed. For example, a library of
fuzzers which make changes to an abstract syntax tree could be constructed. Such
a library would not model specific behaviours, but would allow researchers to
build models performing within-style aspect weaving without writing code which
contained no metaprogramming logic. Instead, this logic would be encapsulated in
utility functions provided by the proposed library. This would also reduce the
work required of researchers looking to use the techniques demonstrated in this
thesis. The library could also support the development of a Fuzzi-Moss-like set
of \sociotechnical behavioural variances. However, such a library does not
currently exist. A summary of the contribution to the research community which
the proposed library would make is:

\begin{researchquestion}
    The development of a library of metaprogramming operations which simplify
    the construction of within-style aspects, supporting its research use and
    demonstrated in case studies.
\end{researchquestion}

The proposed library need not be a separate codebase to the \aspectorientation{}
framework it is designed alongside. Instead, it could be developed with
interoperability in mind to permit reuse across different implementations of
\aop{}. This would enable researchers who develop their models as aspects to
share their work with others who use other frameworks. Precedent for
interoperability between frameworks has already been set by AspectJ, for which
the accompanying DSL has been adopted by SpringBoot's implementation of
\aop{}~\cite{introducing_spring_aop_chapter_integration_with_aspectj}. To
achieve this, some standardisation around the design of \aspectorientation{}
frameworks would be required, so that aspects can be developed against a common
foundation.





\section{Testing Frameworks to Detect Model Incorrectness}
\label{sciunits_for_unrealistic_states}

Many of the uses of aspect orientation in simulation \& modelling research
discussed in this chapter affect the construction and maintenance of a codebase.
However, aspect orientation also has potential in the instrumentation of an
experiment. This was theorised by \citet{gulyas1999use} and put into practice in
this thesis to instrument the naive model of RPGLite to observe its state (see
\cref{subsec:aspects_instrumenting_model} for a description of the relevant
aspects). Instrumentation has potential to be used in aspects of research
software engineering for purposes other than the observation of system state:
for example, aspects may be useful to alert researchers to impossible states
visited by a simulation by instrumenting them to observe a simulation's state
and assert that it is within expected bounds.

Many models construct a version of a real-world system which cannot achieve
certain states. For example, a model of collisions might never be expected to
see an increase in the total energy present within the model;
simulations of \sociotechnical systems might expect bounds on the hours worked
by a simulated workforce, or the amount of output the workforce creates;
researchers might study a system to observe an emergent property which is
bounded by laws around its growth (for example, a linear increase in some
property over time as opposed to exponential growth). It is important that any
model of a real-world system accurately reflects the system's physical limits
and underlying mechanics. Models may fail to reflect these mechanics for many
reasons, including:

\begin{itemize}
    \item The model may be developed with bugs which are not detected, i.e. it is
    conceptually incorrect.
    \item The model may be verified and trusted, but later maintenance introduces
    undetected errors, i.e. it contains bugs.
    \item The model may be conceptually correct, but given unexpected inputs.
    \item The model may be conceptually correct, but a bug may exist in a
    dependency, on certain hardware, or in some other detail specific to the
    environment it runs in as opposed to its implementation.
    \item The model may be conceptually correct when initially implemented, but
    fail to meet researchers' expectations at a later date due to a shifting
    perspective within the research community. This may not be apparent when
    reading the model's source code, but may be an emergent property arising
    from interactions within the system. Non-deterministic interactions leading
    to emergent properties would be difficult to verify through static analysis.
\end{itemize}

Software engineering practices such as unit testing are common practice and may
catch these bugs through identifying incorrect behaviour in model components.
Thorough test suites may also include integration tests which observe the
behaviour of many model components working together. However, there are
limitations of both testing techniques:

\begin{itemize}
    \item Test suites as described may fail to identify non-deterministic errors
    in a model, which arise only in a small number of cases;
    \item Traditional test suites also may fail to identify emergent state
    errors, where individual components operate correctly but their repeated
    integration results in an incorrect emergent property of the system;
    \footnote{Emergent properties are often the state under study, as in \pydysofu{}'s original
    case study~\cite{wallis2018caise}.}
    \item Test suites often fix values such as random seeds or input data to
    remove non-deterministic behaviour. This may hide errors arising from
    non-determinism in a modelled system or may not exhibit the properties of
    real-world data, thus exhibiting different behaviour under test than when an
    experiment is conducted, making it difficult to verify in conditions which
    differ from those of an experimental run.
\end{itemize}

\Cref{chap:experiment_setup,chap:experimental_results} showed that aspects can
be used to instrument models to assert a system state within expected bounds,
both as components of a test suite and as assurance that a model was
conceptually correct at any time it is used as part of an experiment. All of
these limitations can be alleviated by a test suite which is able to assert the
correctness of model state at different points in a program's execution. Aspects
can be used to instrument a model and observe its behaviour while a simulation
executes as part of an experimental run, ensuring that the results of the
experiment were not impacted by any incorrectness tested for. These states may
also be deterministically caused by experimental input data which is not
reflected by the inputs used by a test suite. Testing for erroneous state during
an experimental run ensures that incorrect states were not reached through
unexpected inputs, meaning they could not affect results and so alleviating this
issue. Emergent properties of a system may also be tracked over time, and
aspects may measure emergent properties while a simulation executes to ensure
they are valid at all times. For example, these aspects could ensure expected
growth curves for the property in question and check that values are within
physically possible bounds.

Aspects can also be used as regression tests to detect the recurrence of bugs
after they have been identified and fixed. If a simulated system enters an
incorrect state during development, and the underlying bug is fixed, an aspect
observing the system's erroneous property can be constructed which alerts
developers if the bug is re-introduced.

Aspects instrumenting a model throughout its use in an experiment would also
serve to guarantee peer reviewers of a study that the model was correct with
regards the properties instrumented by aspects. Functionally speaking, a suite
of aspects which observe bounds on and characteristics of system properties is a
declarative set of known-good behaviours of a model. Peer reviewers who sought
to check that a model was correct should be able to extend this list of
properties and observe any assertions when reproducing a study's findings.

Within-style aspects are particularly appropriate for this task, as the
measurement desired of a program might exist \emph{inside} an existing function or
block of code. It is conceivable that observing system state before and after a
function executes would not be sufficiently granular to assert correct system
state in some cases, particularly if the model's codebase does not include
suitable join-points to weave onto. Aspectual instrumentation of research codebases is
therefore feasible for many more models using \pdsfthree{}'s within-style fuzzing.

To demonstrate the efficacy of this approach, a library of tools to observe
boundaries on system states could be developed, similarly to the library of
tools suggested in \cref{standard_aosm_model_features}. Case studies augmenting
research codebases could then be constructed. These should include new
codebases, to demonstrate the technique's utility during development, but should
also include the instrumentation of old models and the detection of any
irregularities in existing, published work. Such a project would answer the
research question:

\begin{researchquestion}
    Can aspect orientation be used to instrument a model and assert correctness
    by observing that the model's properties remain within expected bounds?
\end{researchquestion}



% \section{Discovery of System Features by Fitting Aspect Parameters}
% \label{future_work_feature_discovery}

% In the experiments in this thesis we build some aspects of behaviours we know
% exist in the system (such as players learning to play better / changing play
% styles with long-term exposure to RPGLite). Could we build aspects as
% hypothesised system behaviour, attempt to fit them, and so discover things
% about the real-world system?

% This might be better related to sections above on hypothesised behaviour being
% tested by RPGLite, and it's well explored in an RPGLite future work section
% below, so maybe this section isn't required. I've commented it out as I don't
% think it adds anything that \cref{future_work_hypothesising_system_states}
% doens't already add.



\section{Optimisation of Multiple
Aspects}\label{future_work_many_aspectual_models_to_optimise}

Experiments in \cref{chap:exp1_simulation_optimisation} and
\cref{chap:exp2_old_aspects_new_systems} made use of models which were fitted to
real-world data. However, they only fitted parameters of learning models, and
only some parameters were fitted. This was partly due to the
computational cost involved in fitting these parameters and time constraints on
this research.

Optimisation is an independent research field with a broad range
of techniques outwith the scope of this thesis to explore; however, a study of
optimisation algorithms and technologies as applied to the fitting of aspects to
real-world system models is a topic for future research. Models of real-world
systems can be computationally expensive to complete, meaning that measuring the
accuracy of a model with a variety of parameters could be computationally
expensive. A survey of optimisation techniques suitable for aspectually
augmented simulations and models would answer the research question:

\begin{researchquestion}
    Which optimisation techniques are best suited to fitting parameters of
    aspects woven into models of real-world systems, with regards the
    accuracy of the aspectually augmented model produced?
\end{researchquestion}



\section{Future Work pertaining to RPGLite}\label{sec:future_work_rpglite}
RPGLite's dataset was analysed for the purposes it was collected for in this
thesis: to aid in the realistic simulation of a well-controlled \sociotechnical
system. However, some pieces of analysis have not been completed. This section
discusses future work enabled by the dataset of real-world play which this
thesis contributes.

\subsection{Causes of Game Abandonment}\label{future_work_rpglite_abandonment_reasons}

The games included in the dataset produced by RPGLite finish in varying states:
some are incomplete (as some games were ongoing when the snapshot of the game's
database was taken), and others are finished. A third category of games are ones
where a player opted to abandon a game they were participating in. This feature
of RPGLite was developed to allow players who were playing inactive opponents to
free a game slot. It also allowed players who felt they had no hope of winning ---
or were bored with a particular game --- to forfeit and start a new one.

As many data points are included in the published dataset, it is possible to
observe how active players were (for example, by measuring the frequency with
which they would open RPGLite), and engagement with the app might be quantified
by players' exploration or frequent use of leaderboards and customisation
features. Correlations might exist between players' activity, engagement, or
experience with their propensity to abandon a game in different states. One
hypothesis could be that particularly active players frequently forfeited games
against inactive players out of boredom, or that players who were unlikely to
win a game (or were inexperienced players of RPGLite paired against competent
players) might be more likely to forfeit. These results might be of interest to
the broader game design community in industry and research. Research
investigating reasons for abandonment, for example, could answer the research
question:

\begin{researchquestion}
    Can players who are likely to forfeit a game be identified through patterns
    of RPGLite play?
\end{researchquestion}





\subsection{Patterns of Play within Cliques of Players}
\label{future_work_rpglite_playerbase_clique_discovery}

RPGLite's application implemented matchmaking systems which allowed players to
advertise themselves as open to new games, challenge players discovered on a
leaderboard, or rematch against players they had previously played against
through a game history. However, two factors might have impacted how players
chose their opponents:

\begin{enumerate}
    \item The recruiting methods used mean that people might have known each
    other before playing RPGLite and so might be biased toward playing each
    other rather than other members of the community,
    \item Players who find others they are well-matched against might repeatedly
    re-match rather than continuing to discover players they might have less
    satisfying games with.
\end{enumerate}

These factors would imply the formation of cliques of players: small groups who
often created games against each other much more frequently than they did with
the game's broader community. If so, this could have implications for uses of
the PRGLite play dataset. For example, in the experiments discussed in
\cref{chap:exp1_simulation_optimisation} and
\cref{chap:exp2_old_aspects_new_systems} no structure is imposed on the matching
of simulated players, meaning they are are uniformly exposed to their simulated
community. As these experiments simulate learning from the outcomes of games,
exposure to the entire community would expose players to learnings from all
games played; cliques of players have a limited set of games to learn from, and
so could affect what is learned by that clique. The broader community would also
be insulated from what was learned within the clique, meaning that a loss of
information affects all players.

Future research making use of the RPGLite dataset should be aware of any
underlying patterns which could affect the dataset's emergent properties, such
as learning styles. Clique formation is an example of such a pattern; others
might yet present themselves. Future research making use of the dataset of
RPGLite play may look to address research questions such as:

\begin{researchquestion}
    Are patterns such as cliques of players present in the RPGLite dataset? If
    so, what patterns exist, and what is likely to have driven the formation of
    those patterns?
\end{researchquestion}

    

\subsection{Large-Scale Data
Collection}
\label{future_work_rpglite_large_scale_data_collection}

There is scope for a larger and longer-term RPGLite data collection effort to be
made than this thesis had scope for. RPGLite's player base was recruited
informally, only two seasons of the game were run, and it was not heavily
advertised to prospective players once it was clear that sufficient data would
be collected for the research that required it. A re-release of the application
in mobile app stores with a concerted effort to release new seasons of the game
and maintain player interest for an extended period of time would enable a
richer analysis, and broader utility to the games research community. New
RPGLite data collection efforts could also take the opportunity to expand the
game's implementation with additional features, such as in-game chat, favourites
lists of previous opponents, or match replay and analysis with suggestions for
improved play backed by the formal methods inherent in RPGLite's design

RPGLite's dataset contains information about players' interactions with the
application itself. As the game made available some features typical of modern
games such as leaderboards, matchmaking, achievements, and graphical
customisation, player data containing interaction data on features which are
used commercially could be used to shed light on the more effective aspects of
modern game design for engagement purposes. There are many opportunities
available for the game design research community to investigate. Unfortunately,
they remain outwith the scope of this thesis, which focuses on simulation
technologies more than it does game design. However, Some already-published work
reflects further on the design and possible future improvements of the
game~\cite{kavanagh2019balancing,RPGLiteLessonsLearned,kavanagh2020,kavanagh2021thesis,kavanagh2021gameplay}.

A large-scale user study of RPGLite could also be used in pursuit of other
research questions suggested in this chapter. Players could be classified by
their likelihood of abandoning games using the patterns identified in
\cref{future_work_rpglite_abandonment_reasons} to verify the patterns
empirically. A larger dataset might be useful in testing the discovery of system
behaviours using hypothesised behaviours encoded in aspects. This could be used
for the aspectual encoding of an experimental setup, as discussed in
\cref{future_work_hypothesising_system_states}, the discovery of features of
player behaviour suggested in
\cref{future_work_discovering_emergent_properties}, the discovery of dataset
properties such as player cliques including the real-time prediction of clique
formation to during data collection for experimental verification of a clique
classification process discussed in
\cref{future_work_rpglite_playerbase_clique_discovery}, or the fitting of
multiple aspects to models as described in
\cref{future_work_many_aspectual_models_to_optimise}. Relevant research
questions are discussed in their respective sections of this chapter.



\section{\AspectOriented{} Discovery of System Properties}
\label{future_work_discovering_emergent_properties}

The RPGLite gameplay dataset presents opportunities for research in aspectually
augmented simulation \& modelling which is related to the future work described
in \cref{future_work_hypothesising_system_states}, which discussed the use of
aspect orientation to encode hypothesised behaviour. Researchers may be able to
use aspects to discover unknown properties of a real-world system. This would be
achieved by writing aspects which alter a simulation to discern some
pattern in it, or in the behaviours which produced a dataset that simulation
uses. Rather than writing aspects to encode a scientific hypothesis, they could
also be used as an investigative tool.

The existence of some biases which could affect RPGLite player behaviour is
unknown. However, these may be discernable by encoding the behaviours as aspects
and observing whether player behaviour is made realistic when they are woven.
Examples of possible biases include: whether cliques were formed by RPGLite
players; the behavioural impact of player interaction with ancillary game
features such as leaderboards or profile customisation; whether players' early
aptitude for RPGLite may cause them to lose interest relatively quickly due to a
lack of challenge or spur them to climb leaderboards or continue to win games;
whether players more likely to challenge others with more experience learn
RPGLite quickly and be more likely to continue playing, or quickly lose interest
due to repeated loss. These may be identified by writing aspects which encode
them and observing whether simulated play produces realistic data.

While these behaviours can be hypothesised, many might have an impact on each
other. Players might find experienced peers through the leaderboard, so their
progression in or engagement with the game could be due to their propensity to
explore (by using the leaderboard) or their exposure to more advanced play
(through games with experienced peers). This can be investigated through aspects
fitted to real-world data, with parameters describing the strength of their
effect.\footnote{Research into relevant techniques is proposed in
\cref{future_work_many_aspectual_models_to_optimise}.} The \emph{relative}
importance of each concern could be discovered by determining accurate values
for the parameters controlling the strength of every one, similarly to the
experimental design used in \cref{chap:experimental_results}. This would discern
the existence and relative impact of real-world behaviours.

This proposal is an extension of the research discussed in
\cref{future_work_hypothesising_system_states}, but the RPGLite dataset is a
suitable testing ground for the technique which already exists. Larger-scale
data collection as described in
\cref{future_work_rpglite_large_scale_data_collection} could provide
opportunities to test this more rigorously, as the discovery of emergent
properties of RPGLite play might require a larger dataset than this thesis had
the scope to contribute. Future work demonstrating this technique could address
the research question:

\begin{researchquestion}
    Can potentially interacting emergent properties in a real-world system be
    identified through the fitting of aspectual models representing hypothesised
    properties to empirical datasets which may exhibit them?
\end{researchquestion}


\revnote{
    Add a section on \pdsfthree{} improvements: primitives to make it easier to write
    fuzzers, such as standard AST modifications and easy, high-level
    specification of where those modifications should go. Doesn't belong here,
    move it near a section on tooling above.
}

\section{Discussion}\label{sec:future_work_conclusion}

% Kuhn notes that research results aren't just facts in the landscape in which
% researchers operate, but shed light on a field in a manner which can influence
% the field and the manner in which research is created. \pdsf{} and the
% aspectually augmented modelling it has been used to demonstrate are effective
% in modelling concerns in a model in a cross-cutting manner, but the
% research and tooling created to conduct that research offer alternative ways
% to conduct simulation and modelling research, and demonstrate a use case of
% aspect oriented programming which has seen some tooling development but little
% real-world use. The benefits of the technology's use in research software
% engineering and in experimental design and development requires further study,
% and there exists potential for myriad use cases. Tooling can be improved, the
% potential capabilities of the technology can be explored, and its use in
% future simulation and modelling research can be explored in a variety of
% fields.

The contributions presented in earlier chapters motivate more research efforts
spanning multiple fields, and has implications for experimental design as well
as software engineering and simulation \& modelling techniques. This chapter
contributes an investigation into this future work by identifying specific
opportunities yielded by earlier contributions and proposing research questions to
precisely communicate the future work enabled by those contributions.

It is important to examine new research results. Contributions can impact
existing results, and the theories they are founded upon.
\citet{kuhn2012structure} notes (emphasis added):

\begin{displayquote}
   \textelp{}~new inventions of theory [are not] the only scientific events that
   have revolutionary impact upon the specialists in whose domain they occur.
   The commitments that govern normal science specify not only what sorts of
   entities the universe does contain, but also, by implication, those that it
   does not. It follows \textelp{} that a discovery like that of oxygen or
   X-rays does not simply add one more item to the population of the scientist's
   world. Ultimately it has that effect, but not until the professional
   community has re-evaluated traditional experimental procedures, altered its
   conception of entities with which it has long been familiar, and, in the
   process, shifted the network of theory through which it deals with the world.
   \emph{Scientific fact and theory are not categorically separable, except perhaps
   within a single tradition of normal-scientific practice. That is why the
   unexpected discovery is not simply factual in its import and why the
   scientist's world is qualitatively transformed as well as quantitatively
   enriched by fundamental novelties of either fact or theory.}
\end{displayquote}

The ideas presented in this thesis --- in particular the aspectual augmentation
of simulations \& models --- are not new ``theories'' in the sense that they
contribute any profound or revolutionary new paradigms. However, they do
contribute a new approach to simulation \& modelling, and do so through new
aspect orientation techniques. There are therefore contributions in both fields
in which the ``scientific fact and theory'' Kuhn refers to are affected: in both
fields new research opportunities and experimental practices are made feasible.
These contributions concern both research \emph{facts} such aspect-oriented
runtime metaprogramming, and \emph{theories} forming the foundation of those
facts such as the use of new aspect orientation technologies in experimental
design.

New capabilities of aspect orientation frameworks are suggested, built, and
demonstrated in experimental case studies through \pdsfthree{}. Their design and the
extension of the tooling around these concepts present opportunities for future
work as discussed in \cref{future_work_standard_aosm_techniques} and in
\cref{standard_aosm_model_features}, but the implications of their use in
software engineering requires further study, as discussed in
\cref{future_work_aspect_oriented_metaprogramming}. There are specific software
engineering implications for research software engineering, as discussed in
\cref{future_work_nudge_model_state,future_work_aop_for_rses,sciunits_for_unrealistic_states,future_work_many_aspectual_models_to_optimise},
Further, the methodologies demonstrated here yield metascientific research
opportunities, as discussed in
\cref{future_work_hypothesising_system_states,sciunits_for_unrealistic_states,future_work_many_aspectual_models_to_optimise,future_work_discovering_emergent_properties}.
This thesis contributes not only the ideas presented in earlier chapters, but a
discussion of their implication for Kuhn's ``scientific fact[s] and theor[ies]''
in this chapter for the relevant fields.
\revnote{
    Maybe go into more detail on what was mentioned in specific cross-referenced
    sections here? A long list of sections this chapter contains without any
    extra explanation isn't very satisfying, even if it's useful within this
    framing of exploring what Kuhn meant and how it relates to this thesis.
}


The dataset produced by RPGLite also offers opportunities for further study.
Research opportunities presented by the game design and formal methods are
explored by \citeauthor{kavanagh2021thesis}~\cite{kavanagh2021thesis}, but the
dataset this thesis helped to contribute also offers opportunities in the
mining of player behaviour and the dataset's use in future aspectually
augmented modelling research, particularly if a large-scale data collection
study is undertaken. These possibilities were discussed in
\cref{sec:future_work_rpglite}.


