% Broad   ---  Narrow  --- Broad
% Context ---  Content --- Conclusion

\vspace{-2\baselineskip}
\chapter*{Abstract}
% \begin{abstract}

% The context must get across the gap that the paper will fill.
% The first sentence orients the reader by introducing the broader field in
% which the particular research is situated.
% Then this context is narrowed until it lands on the open question that the research answered.
% A successful context section sets the stage for distinguishing the paper's contributions from
% the current state of the art by communicating what is missing in the literature (i.e., the specific gap)
% and why that matters (i.e., the connection between the specific gap and the broader context that the paper opened with).

    % Scientific models are a core component of many fields' research, and the
    % development of those models over time is an important contribution of
    % researchers' work. However, developing these models can be challenging.
    % Models can change over time as scientific consensus shifts, can require
    % reengineering as bugs are discovered, and are subject to requirements other
    % codebases are not: they must meet scientific standards and pass peer review,
    % as well as their outputs. 

    % Aspect-oriented programming is a field (paradigm) where parts of a program
    % which are present in many modules but do not subit to the reqirements of
    % typical modularisation techniques are factored into ``aspects''. Aspects
    % allow cross-cutting concerns to be modularised and ``woven'' into a target
    % codebase. While aspect-oriented programming has seen some academic interest
    % since its original incarnation, it has seen lukewarm industrial adoption and
    % has not been extensively studied as to whether it achieves its claimed
    % benefits in practice. 
    

    % ---

    % \Aop{} is a paradigm which (programming) modularises parts of a program
    % which repeat across many modules but are difficult to refactor into their
    % own module using other techniques. Aspects describe a change (addition) to
    % make to a program and where that change should be introduced; these changes
    % are ``woven'' into the program, in different ways. (Just something to add
    % --- pdsf introduces the idea of a change.) The paradigm has seen some
    % academic interest, but has not achieved widespread industrial adoption, and
    % its claimed benefits to a codebase (originally claimed benefits of improved
    % modularisation and maintainability as a result) have not been demonstrated
    % conclusively in academic literature to date. While the paradigm has been
    % widely studied, its effective use in different scenarios has not been well
    % proven. (Use-cases)

    % This thesis suggests the use of \aop{} in research software engineering
    % (proposes) to take advantage of its ability to add logic to a program,
    % rather than its ability to modularise that program. Research software
    % engineers are tasked with the development of scientific models \&
    % simulations which are subject to requierments most codebases are not: they
    % must produce output which passes peer review, rigorously implement a
    % scientificly interesting (of scientific interest) theory or model, and
    % maintain scientific relevance over time if part of a long-term research
    % project or in the case of community adoption. As a result, these codebases
    % must be maintained over time and demonstrated to work to a research
    % community. We propose that \aop{} fulfils the need of research software
    % engineers to produce models which can be improved over time and maintained
    % in the face of changing research needs by augmenting those codebases with
    % aspects to ``update'' them and retain their relevance.
    
    % ---

    % % AOP can describe changes to points in a program, which was designed to
    % % separate out ``concerns'' of that program but recieved only limited
    % % mainstream / industrial adoption. 
    
    % \Aop{} is a programming paradigm which is designed to separate parts of a
    % program into modules, when those parts do not produce modules [are not
    % easily separated using other means]. The technique describes these parts as
    % aspects: changes to a program (advice) combined with a place to apply that
    % change (a join-point). However, while \aop{} has seen some interest from the
    % academic community [academic interest], it has received lukewarm adoption
    % from industry and its benefits in real-world engineering contexts are not
    % well demonstrated.

    % % We think AOP might be well suited to use in scientific codebases because
    % % models are often used for different things (modelling different scenarios
    % % or hypothetical systems) and are often updated over time (if the research
    % % community's consensus changes or if a model is reused in a new context).
    % % These changes might be well-described as advice. However, people haven't
    % % researched the application of AOP to developing scientific models before.
    % % We want to figure out whether research software engineers can take
    % % advantage of AOP to produce realistic models, by representing changes to
    % % models as advice.
    
    % An alternative use for \aop{} is in producing and maintaining codebases for
    % research purposes. Lots of modern research is conducted with the aid of
    % software models. These models often represent different scenarios in the
    % real world, or require maintenance over time in light of changing consensus
    % in a research community, or are updated to model a new hypothesis when
    % reused in future research projects. In all cases, research software
    % engineers look to describe a change to some foundational model. This raises
    % the possibility that these changes might be well described as aspects.
    % Augmenting models using aspects would allow researchers to concisely
    % describe variations of a model, compare different hypothesised systems, or
    % maintain that model over time without rewriting a delicate, peer-reviewed
    % codebase. However, the application of \aop{} to developing models \&
    % simulations has not been studied to date.
    
    % ---

    % \Aop{} is a programming paradigm used to modularise parts of a program which
    % are difficult to separate using other means. The technique describes these
    % parts as changes to a program, applied to specific parts of that program.
    % However, while \aop{} has received some academic interest, it has seen
    % lukewarm adoption from industry, and its benefits in real-world engineering
    % contents are not well demonstrated. Particular use-cases where \aop{} is of
    % particular utility are not common in its literature. 
    
    % One possible use for \aop{} is in producing and maintaining codebases for
    % research purposes. In many fields, modern research is often conducted with
    % the aid of software models. These models often represent different possible
    % scenarios, require maintenance over time in light of changing scientific
    % consensus, or are updated to represent a new hypothesis when reused in
    % future research projects. In all cases, research software engineers must
    % describe a change to some foundational model. This raises the possibility
    % that these changes might be well described as aspects. Augmenting models
    % using aspects would allow researchers to concisely describe variations of a
    % model, compare different hypothesised systems, or maintain their model over
    % time without rewriting a delicate, peer-reviewed codebase. However, \aop{}'s
    % application to models' codebases has not been studied to date.
    
    % ---

    \Aop{} is a software engineering paradigm used to modularise parts of a
    program which are difficult to separate using other means. It does this
    using aspects: combinations of program modifications and places they should
    be applied. While it has received some academic interest, \aspectorientation
    has seen lukewarm adoption from industry and its practical benefits are not
    well demonstrated. The paradigm lacks a use-case for which aspects are
    uniquely well-suited.

    One such use-case may be in producing and maintaining codebases for research
    purposes. In many fields, research is conducted with the aid of software
    models. Changes to these models are delicate: they may invalidate results,
    add complexity to a codebase, or absorb researchers' time. These changes
    \emph{could} be represented as aspects. Doing so may avoid these issues, but
    the paradigm is yet to be applied to codebases for scientific models. We
    propose that aspects are well-suited to describe changes to models and that
    \aspectoriented{} modelling would ease model maintenance.



% The content (“Here we”) first describes the novel method or approach that you
% used to fill the gap/question. Then you present the meat—your executive
% summary of the results.

% Methods --- what was done?

    This thesis investigates the viability of \aspectoriented modelling. An
    \aop{} framework is implemented which circumvents criticisms of the
    paradigm, and contributes new kinds of aspects which are useful for
    describing changes to models. With this tool, a case study of
    \aspectoriented modelling is constructed using a model of a real-world
    mobile game and its players' activity. This forms the foundation of three
    experiments. They conclude that aspects can be used to successfully augment
    models, can add new behaviours and parameters to models, and can be reused
    across different models in some cases. As these contributions invite new
    research opportunities in many fields, the thesis also enumerates the
    possibilities enabled for others researching \aop{}, simulation \&
    modelling, and research software engineering, as well as the methodological
    implications for researchers whose hypotheses are encoded within software
    models.

% Results (contributions etc --- this is important, don't compromise length or
%          quality.)
    
% contributions to mention:
% - RPGLite (implementation, dataset)
% - pdsf
% - RQ 2,3,4 as above
% - future work
% - 

% the conclusion interprets the results to answer the question that was posed at the end of the context section.
% 2nd part "which highlights how this conclusion moves the broader field forward (i.e., ‘broader significance’)."

% Conclusion (take-home message, additional contributions of importance,
%             perspective [summary?])
    




% \end{abstract}
