\message{ !name(lit-review.tex)}
\message{ !name(lit-review.tex) !offset(-2) }
\chapter{Lit review}

% ================================================================================
% A brief outline of what this chapter is trying to get across.
\section{Overview}

Systems-of-systems (SoS) modelling is a deeply interdisciplinary study. This is a blessing and a curse.
\par


% ================================================================================
\section{Extant Modelling Frameworks}
These are the systems that exist.


\subsection{Bigraphs}

The seminal work on Bigraph modelling is \cite{milner2009space}, which defines
bigraphs, shows how they can be used, and demonstrates how different
communication calculi can be represented in a bigraphical notation. The book
defines, among other things, bigraphs' category theoretic underpinning, and an
algebra used to translate between this detailled mathematical foundation and the
graphical representation. It has also been used to define some programming
languages and
logics~\cite{Perrone:2012:MCB:2245276.2231985,sevegnani2015bigraphs}.
\par

Bigraphs are effectively the combination of two graphs: a \emph{place} graph,
which is a forest showing how a system is structured, and a \emph{link} graph, a
graph showing how nodes in the place graph communicate. Bigraphs' triplicate
definition in category theory, graphical notation and algebra make them
unusually adept at expressing high-level concepts as well as low-level detail.
\todo{mention signatures etc as a way of translating high-level details}
\par

Bigraphs represent the ``motion'' of a system as a series of ``reaction rules'',
which, together with a standard bigraph, define a \emph{bigraphical reactive
  system} (BRS). A BRS can be used for the purposes of model
checking~\cite{sevegnani2015bigraphs,perrone2012model}. A BRS defines a host of
changes which a system can undergo, and identifies whether a change can take
place by matching a pattern to a subgraph of the bigraph. This allows for the
automatic generation of model checking code, meaning that a fourth
representation of systems, formal models, can be added to the algebraic,
graphical and mathematical representations discussed. BRSes have been studied in
the context of games~\cite{benford2016lions}, protocol
modelling~\cite{calder2014modelling}, and algebras such as mobile
ambients and the \picalculus{}~\cite{milner2001bigraphical}.
\par

Unfortunately, standard bigraphical notation has some limitations. These include
a lack of ability to represent node sharing, where ``ownership'' of a node on
the place graph can be represented in multiple logically-defined places, and a
lack of ability to natively encode a metric space, important for system analysis.
\par

\subsubsection{Bigraphs with Sharing}

Some model components do not cleanly decompose into a tree of nested places.
This is because, while trees neatly encode physical place, they do not encode a
``logical'' entity which can exist in multiple places at once (on account of it
being incorporeal) cleanly. Node sharing was introduced to the standard
bigraphical notation for simpler and more elegant representation of a
non-physical entity being represented in a bigraph. 
\par

Two examples of their use are in Calder and Stevegnani's work on modelling parts
of the 802.11 protocol~\cite{calder2014modelling} and
games~\cite{benford2016lions}. 

\subsubsection{Directed Bigraphs}
\label{sec:review-directed-bigraphs}

There's a formalism for directed bigraphs. \todo{write this}

\subsubsection{Bigraphs as Metric Spaces}

No published work has appeared which expressly uses a bigraph for the purposes
of modelling a metric space. One reason for this might be that there is no
obvious way to combine arbitrary metrics into a useful calculation, as bigraphs
do not encode space in a regular, euclidean way.
\par

This concern is well warranted but does not excuse the lack of literature in the
area. A metric-space bigraph could be used, for example, to represent the time
taken to communicate between different nodes on a network where a network
topology was represented bigraphically, or perhaps to represent the financial
cost of travelling between destinations where the edges of the bigraph were
weighted by the minimum cost of transport from one place to another.
\par

Metric spaces are important to system analysis. For example, impact analysis or
risk assessment require quantitative analysis of a system. Ideally, one should
be able to take an ``aspect'' of a system (to borrow jargon from Applied Systems
Theory~\ref{sec:review-applied-systems-theory}) and assess the system's
operations according to that particular aspect, drawing numerical calculations
about the analysis performed. This is difficult to do without encoding a metric
space into a bigraph.
\par

One supposes that the metric space would take little more than edge labelling
with some dictionary of properties, and supplying a function with the link graph
to define how to combine metrics. This may become slightly more complicated in
the case of a Directed Bigraph~\ref{sec:review-directed-bigraphs}.
\par

\subsection{Obashi}

Obashi is a dataflow modelling methodology which specifically targets modelling
business and IT systems (B\&IT systems). It has seen success as a method for
representing how components are interconnected and has a strong emphasis on the
representation of a B\&IT system as a metric space, which they encourage analysis
on.
\par

System visualisation is a fundamental component of the Obashi methodology, where communication about a system's composition and operation is
one of the key takeaways of their model's use. This means that Obashi-certified
systems engineers can model a system, bring Obashi diagrams to their
non-certified coworkers, and have their models be understood. To this end,
Obashi permits system \emph{decomposition} via the specification of subsystems
or specific formats of subsystems they refer to as ``dataflows''.
\par

Obashi sees particular success in supplying tooling for its systems
modelling methodology, which can be unusual in business-oriented modelling
frameworks.
\par

Interestingly, Obashi's B\&IT systems are socio-technical systems. To represent
a sociotechnical system within their framework, they separate the social and
technical components, and link these via business processes. Their contention
that different types of system component can be categorised and can communicate
only with specific others gives them their six categories: Organisation, for the
social components, Business Processes for the communication between social and
technical components, and Applications which Business Processes communicate
with. Applications then communicate with Systems, which communicate with
Hardware, leading to the final category, Infrastructure. These six ``layers''
are responsible for Obashi's name, an acronym of the six in the order they
communicate in.
\par

\subsection{PRINCE2 \& ITIL}


\subsection{Responsibility Modelling}
Responsibility modelling is a paradigm where socio-technical systems are
modelled as a collection of actor roles with particular goals which they must
ensure are achieved. ``Actor roles'' here means that, rather than representing
specific instances of real-world actors, responsibility modelling models actors
in the abstract, and information about different \emph{types} of actors is
collected instead.
\par

This method is adept at representing the dependancies between actors in a
system. An example would be modelling a water system, where a tap is responsible
for delivering water when a wheel is turned; this is dependent on connected
pipes fulfilling their responsibility to deliver water from a water production
facility; this in turn is dependent on the water production facility properly
producing and supplying water, which are its own responsibilities.
\par

Some extensions to responsibility modelling permit advanced model details such
as uncertainty in system decomposition~\cite{simpson2017formalised}, and models
have the advantage of being simply represented in a convenient graphical
format~\cite{storer2008modelling}.

\subsection{Specification Modelling}
% Particularly in Z


\subsection{Model Checking}
% Prism, Spin, Uppaal


\subsection{Soft Systems Methodology}


\maybe{\subsection{Viable Systems Modelling}}


\subsection{MoDaF, DoDaF and NaF}


\subsection{Object Oriented Models}
% Model ontology represented as objects


\subsection{Functional Programming Simulations}
% Model ontology represented as data structures and transforms on them
% Maybe mention the possibie directions with session types and HoTT?


\subsection{Actor Oriented Models}
% Mention that processes can be actors. CCS, π-calc and CSP are all relevant and important here.
%     Particularly, channels in Go/Spin/Erlang vs async&await in Python/Javascript
% Note that workflow modelling is its own section.

% Notes from bearapp:
% * Software engineering
% 	* Object orientation
% 	* Functional programming
% 		* Models & ontologies as data structures
% 	* Processes as actors
% 		* Robin Milner
% 			* π-calc
% 			* CCS
% 		* Tony Hoare
% 			* CSP
% 		* Channel-based communications
% 			* SPIN
% 			* Golang
% 			* Erlang & BEAM
% 		* Await/Async
% 			* Python
% 			* Javascript


\subsection{Graphical Workflow Modelling}
% Lots of models of workflows specifically that are worth mentioning

% Notes from bearapp:
% * Workflow modelling
% 	* YAWL
% 	* Theatre & PyDySoFu & FuzziMoss
% 	* Petri Nets
% 	* Graphical progress languages
% 		* SysML
% 		* Activity diagrams
% 		* OPM
% 		* BPMN
% 			* BPEL


\subsection{Automata}
% For agent-based modelling, even if it's usually relatively simplistic
% Cellular automata. What else?


\subsection{Holons and Property Based Approaches}
The Holon-oriented philosophy toward system composition is neatly summarised
in~\cite{blair2015holons}. The philosophy's name is a reference to the holistic
approach of General Systems Theory, which supposes that the entirety of a system
cannot be simply represented by its more fundamental components.
\par

A holonic approach to system composition proposes that interfaces between
components (holons) making a system define ``properties'' they require from, and
can offer to, their peers at composition time (``rprops'' and ``pprops''
respectively). The combination of holons creates a system which has its own
properties defined, thereby constructing a higher-level holon which can be used
to construct systems-of-systems of indefinite size.
\par

This approach has some relations in software engineering, where component-based
architectures are not uncommon. Software is often developed via the use of
libraries, for example. Different software systems interact via APIs with
published ``endpoints'' --- these are very similar to the holon philosophy's
pprops. Architectural patterns such as Dependancy Injection can abstract
different components that a piece of software might rely on, and ensure that the
abstracted dependancies provide all necessary properties via techniques such as
Java's interfaces.\todo{get citations for dependancy injection}
\par

\subsection{Systems Theory}

\subsubsection{Applied Systems Theory}\label{sec:review-applied-systems-theory}
% Notes from bearapp:
% * Systems theory
% 	* Applied systems theory
% 		* & related approaches?
% 		* Relate to property-based / API-based composition a-la holons
% 	* Systems of Systems
% 		* Family of Systems
% 	* Specific writers
% 		* Edward Deming
% 		* Michael Polanyi
% 		* Karl Ludwig von Bertenlaffy
% 		* Eric Trist & Fred Emery


\subsection{Safety Critical Systems}
% detail-oriented but essential for actual engineering. These have similar
% requirements to us!
% What can we learn from them?

% Notes from bearapp: 
% * Safety-critical systems
% 	* Fault trees
% 	* IEC 61508
% 	* FMECA
% 	* HAZOPs
% * OSGI
% 	* Different layers represent a semiotic separation, which is useful in separating different degrees of technical and social/human elements of information, making processing and encoding vastly easier at scale
% 	* A good argument for separating the formal model and its visualisation/editing!

\subsection{Summary of Techniques}
Recap by describing what we've seen in terms of extremely detail-oriented, extremely high-level, and in-between this tradeoff.
  
\subsubsection{Formal Methods}\label{sec:review-formal-methods}

Approaches like model checking, and touch on Bigraphical notations
\comment{constraint models?}
\par

% --------------------------------------------------------------------------------
\subsubsection{Engineering Approaches}

In engineering --- in any flavour, not just in computing science --- we have to have a rigorous understanding of the systems we build. 
These systems often approach very large scales, however, and grow over time.
This increases the degree of detail requires disproportionately to the decrease in rigour required in comparison to formal approaches as in \ref{sec:review-formal-methods}.
\par

New challenges also arise. Formal methods approaches rarely need to contend with added complexities such requirements engineering, but in a field such as software development this is a vital component of the engineering process, and \emph{must} be included in any useful model of the craft.
\par


% --------------------------------------------------------------------------------
\subsubsection{Informal Modelling}
Some modelling approaches are particularly informal; this is useful in scenarios where what is being modelled has to be understood, but is either not required to be captured in a great degree of detail or does not readily offer detail to be modelled at all.
The prime example of this is organisational and modelling, where capturing and representing information is complicated to do with any formal approach.
Detailled approaches fail to contend with the ill-defined, ``fuzzy'' nature of business organisation.
\par

One such ``fuzzy'' method lacking detail might be a holon-based approach\maybe{, similar to dependency injection,} which defines the ``properties'' required for a system to operate and otherwise treats systems as blackboxes. 
\comment{This is the approach implicitly taken in managing TV production. Jena, Blair's girlfriend, discussed with me 03/12/2017 their method for managing scheduling and resourcing, where they defined all of the things they were in charge of, their constraints, and their requirements. They write everything on postits, stick them to a whiteboard, and rearrange to try to get overviews of the various things that need sorted and people's responsibilities in the management of the production. Interestingly, they also do SoS modelling implicitly: sometimes there's a 3rd party that they give responsiiblities to, and this 3rd party is a blackbox which effectively represents a part of *their* system's environment/peers in a SoS or FoS. Their own company is who people like the BBC outsource to, so there's hierarchy (as well as inter-relationships --- a combination of SoS and FoS?...). It would be fascinating to study how they organise their own very ill defined process modelling, and compare to business models, engineering models, and formal models.}
\par

That said, more and more detailled approaches are appearing. One such example would be the growing effort to represent business processes via graphical modelling languages such as OPM, SysML and BPMN.
\par

A slightly more rigorous approach, though more limited in scope, can be found in Obashi's dataflow modelling.
\par


% ================================================================================
\section{Themes in Modelling Paradigms}
Certain themes emerge when we're looking at \emph{what} these formalisms contain.


% --------------------------------------------------------------------------------
\subsection{Composition in SoS}


% --------------------------------------------------------------------------------
\subsection{Replacing Detail with Uncertainty}


% --------------------------------------------------------------------------------
\subsection{Systems as Metric Spaces}
Metric spaces in systems like Obashi and object-oriented models. Activity diagrams etc?
\par

Metric spaces are \emph{really} useful for high-level analysis, rather than SPIN-esque verification.
\par

Unfortunately not all models easily adapt to a metric space. Consider bigraphs.
\comment{though this shouldn't be too complex to change, and Michele has played with the idea during his own PhD}
\par

% --------------------------------------------------------------------------------
\subsection{Processes}
Processes quickly complicate a model.


% --------------------------------------------------------------------------------
\subsection{System Entropy}
Processes are the only method we have for \emph{reducing} the entropy of a system.
When we constrain or control a system's change over time by imposing processes on it, we often do so with an eye to limiting how it changes and keeping it locked in a certain state or range of states.
\maybe{consider a garden that one tries to maintain as an example of a system fighting not to stay in a given state, and us constraining its change via the process of gardening.}
How do different systems represent change in entropy? How easily is this analysed?
Entropy change seems like a very useful metric of emergent complexity in a system, so the ability to represent this in SoS is pretty vital.
Also, even a little growth in entropy can quickly cause problems for modelling particuarly large systems of systems, because it can cause state explosion which is intractable to represent with the methods shown above. \comment{consider state explosion in formal methods, for example, where a small increase in entropy causes their state space to increase exponentially, which is what makes verification in something like PRISM NP-hard. Entropy really does matter!} 
\par

Entropy is important for engineering reasons, too. Consider software entropy.
discuss the link between software entropy and badly separated concerns / poor modularisation.
High entropy -> difficulties in maintaining and architectural debt.
How is this represented in software engineering? Business engineering? Elsewhere?
\par

\maybe{discuss ``occam's architectural razor''? System components should be
  exactly as simple as they can be without sacrificing functionality to prevent
  unncessessary increase in system entropy}

  

% ================================================================================
\section{Comparisons}

Some methods fail to properly represent a high-level, human-understandable overview of a system and its architecture --- essential for SoS modelling.
This is necessary for both the process of building the model as a human engineer, and understanding it as a layman or somebody who needs to understand the \emph{implications} of the model.
\par

Other methods fail to properly represent sufficient detail to analyse their model, or to verify properties --- essential for predicting behaviour with certainty.
This is particularly vital for understanding whether a system will or won't have certain properties; making sure that there's a low chance of failure over a certain length of time, for example, or resilience to certain threats.
\par

\message{ !name(lit-review.tex) !offset(-399) }
