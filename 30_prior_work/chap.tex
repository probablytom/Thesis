\chapter{Prior Work}\label{chap:prior_work}

% % To cover PDSF, CAISE paper, and anything else we've done previously that's
% % related to this work (e.g. that masters thesis that models covid).

% % Some notes on where PDSF was before the PHD work, edited from the first draft
% % of the thesis.

% % \pdsf is a Python library~\cite{pdsf_repo}
% % built for making changes to the source code of a Python function as it is
% % called, and before it is executed, while the original function definition
% % remains oblivious to the changes being made. It was originally developed as an
% % honours-level dissertation, which was built upon and detailled in a subsequent
% % paper~\cite{wallis2018caise}. This thesis furthers that original work. To be
% % clear about the work this thesis contains, the state of the project
% % \emph{before} this work began is briefly discussed here.

% \pdsf is a Python library~\cite{pdsf_repo} which implements a dynamically-woven
% aspect orientation framework with the capacity to weave changes within a
% specified join point. It is designed primarily to address the needs discussed in
% \cref{sec:lit_discussion}, particularly those of aspect-oriented simulation and
% modelling.

% Work on PyDySoFu predates the work presented in this thesis: an early version
% was developed for the modelling of behavioural variation. This prototype was
% redeveloped and improved to provide more suitable tooling for more complex
% experiments, and to better address the requirements of such a tool and
% opportunities found in the existing literature. \pdsf{}'s improved design and
% implementation is discussed in \cref{chap:pdsf_rewrite}. 

% In this chapter, prior work on \pdsf{} and the research conducted with it is
% reviewed, so as to isolate the contributions presented in this thesis from
% previously published research.

% \subsection{Motivations for \pdsf{}'s Original Development}

% The original incarnation of \pdsf{} was developed with different motivations
% than those outlined in \cref{chap:lit_review}. It is therefore important to
% elucidate the context in which it was designed and developed.

% \pdsf{} was originally developed for the representation of behavioural variance
% in sociotechnical systems, and was first produced as a proof-of-concept; the
% core contribution was one of tooling. It was developed for use in Python
% codebases, by virtue of the language's widespread use and its flexibility in its
% modelling of data and process.

% The original version of the tool was to be applied to models of behaviour in
% \sociotechnical systems, where individual actions were represented as functions.
% Actions which could be decomposed further into more granular actions were to be
% defined as a functions calling their more granular counterpart functions.
% Invocations of low-level behaviours would implement some change to an
% environment in the model which its modelled behaviour would be expected to
% incur. Invocations of high-level behaviours, containing the invocations of
% lower-level behaviours they compose in the model, would therefore apply the
% combined effect of the collected behaviours they represent. A benefit of this
% approach to modelling behaviour was that high-level behaviours could implement
% the ``flow'' of a behaviour. For example, a behaviour which would be modelled in
% a flowchart as having some loop could be modelled analogously in the method
% described through use of primitive control flow operators in Python, such as
% \lstinline{for} and \lstinline{while} loops.

% Another benefit of this approach is that the behaviours modelled have a
% predictable structure which is amenable to metaprogramming. A low-level
% behaviour's affect could be changed by changing the function definition; more
% structural changes could be made by altering the flow of less granular
% behaviours. A simple high-level behaviour containing a series of function
% invocations (modelling an ordered list of steps in the \sociotechnical system)
% can be represented as a literal list of function calls. The contents of such a
% list is trivially modifiable. Removing an item from a list or truncating it at a
% certain length, for example, are both achievable in a trivial manner using
% high-level languages such as Python. Notably, many behaviours can be conceived
% of which could be represented as high-level behaviours but would not be amenable
% to a simple list of more granular behaviours, such as a behaviour with a looping
% quality. 

% With a mechanism to rewrite either an implementation of a behaviour or a
% collection of behaviours (in the less granular functions mentioned), modelling
% in such a fashion could therefore lend itself to semantically simple
% metaprogramming that could represent real-world variations in behaviour.
% However, largely for reasons discussed in \cref{sec:variation_sm_motivations},
% metaprogramming for representing realistic behavioural variations in
% \sociotechnical simulations should be able to take advantage of system state.
% Many real-world behaviours are contingent on environmental state. Real-world
% actors in \sociotechnical systems might become tired after lots of work, or
% proportionally to time of day within a simulation. Therefore, the
% metaprogramming as described should be performed during runtime, for which no
% suitable candidate was available. \pdsf was developed to fulfil this
% requirement, so that behavioural variance in \sociotechnical simulation could be
% modelled as described and subsequently studied.

% \section{\pdsf{}'s Implementation and Features}

% % The original implementation of \pdsf patched Python classes with
% % additional functionality. Attributes of Python objects are usually retrieved using
% % dot notation (i.e. \lstinline{object_id.attr_id}), which evaluates internally to
% % a call to \lstinline{object_class.__getattribute__(``attr_id'')}. PyDySoFu replaces a
% % class' built-in \lstinline{__getattribute__()} method with a new one, which
% % calls the original to acquire the required attribute. 

% Aspect orientation's core concepts include join points, aspects, and advice.
% \pdsf was not originally designed with aspect orientation in mind --- though its
% design aligned with aspect oriented design in many ways --- and implemented
% \emph{analogues} of these concepts using fundamental Python concepts. In
% particular, it used Python's decorators to specify join points (and the advice
% to be applied to them). The intended use of \pdsf{}'s original implementation
% was that a decorator, called \lstinline{@fuzz()}, could be used to annotate a
% function which might undergo variation at runtime.

% \revnote{Add some diagrams here for a high-level description of the procedure that happens when a variation is applied.}

% \labelledsubsec{Python Decorators}{decorators_explained}

% Python decorators annotate function definitions. They are also
% functions\footnote{Decorators can also be defined in other ways, such as
% callable classes, but this is unimportant for the explanation at hand. While it
% is a simplification of the technical details of Python decorators, for the
% purpose of a clear example we will consider decorators to be functions
% specifically.}, which accept a function as an argument and return a function.
% Decorators transform the functions they annotate: any function with a decorator
% applied to its definition is passed to that decorator as an argument immediately
% after compilation. The function returned by the decorator is used as the final
% value of the original function's compilation, i.e. when compiling a function
% \lstinline{f} annotated by a decorator \lstinline{d}, the value referenced by
% the symbol ``\lstinline{f}'' is \lstinline{d(f)}.

% \begin{figure}[h]\label{decorator_simple_explanation}
%     \begin{center}
%         \begin{lstlisting}
% def example_decorator(annotated_func):
%     def _wrapped_func(*args, **kwargs):
%         print("Running a decorated function")
%         to_return = annotated_func(*args, **kwargs)
%         print("Decorated function has returned")
%         return to_return
%     return _wrapped_func

% @example_decorator
% def add(a, b):
%     return a + b

% print(add(5, 6))
%         \end{lstlisting}
%     \end{center}
%     \caption{Python code implementing a simple decorator.}
% \end{figure}


% \Cref{decorator_simple_explanation} contains a simple example, where the
% function \lstinline{add} is annotated by a decorator, called
% \lstinline{example_decorator}. The relevant compilation and execution steps of
% running the code outlined in the \cref{decorator_simple_explanation} is as
% follows:

% \begin{enumerate}
%     \item The first function definition, \lstinline{example_decorator}, is
%     compiled and the resulting function object is placed in the variable
%     namespace as ``\lstinline{example_decorator}''.
%     \item \lstinline{add} is compiled, producing a function object. However,
%     this function object is not the object which is eventually available in the
%     variable namespace as the symbol, ``\lstinline{add}'', because its
%     definition is annotated by \lstinline{@example_decorator}.
%     \item \lstinline{example_decorator} is invoked, and the argument it is
%     passed is the function object compiled in step \pointno{2}.
%     \item Executing \lstinline{example_decorator} produces another function
%     object, named ``\lstinline{_wrapped_func}'' in the local namespace, and
%     returned. When executed, this function will print
%     ``\lstinline{Running a decorated function}'',
%     executes the function ``\lstinline{annotated_func}''
%     (which is \lstinline{add} in this invocation), and finally prints
%     ``\lstinline{Decorated function has returned}'', preserving and returning
%     the values returned by the invocation to \lstinline{annotated_func} (here
%     \lstinline{add}). Note that all arguments to \lstinline{_wrapped_func} are
%     passed to \lstinline{annotated_func} (here \lstinline{add}), meaning that
%     the function object produced in this process is agnostic as to the arguments
%     it is invoked with.
%     \item When the decorator returns, the compilation of \lstinline{add} is
%     complete. The function object returned by the decorator,
%     \lstinline{_wrapped_func}, is added to the namespace as the symbol
%     ``\lstinline{add}''. Future references to ``\lstinline{add}'' will resolve
%     to this function object.
%     \item In the final line, when \lstinline{add} is called, the function
%     object \lstinline{_wrapped_func} is invoked with the arguments \lstinline{5}
%     and \lstinline{6}. It outputs:\inline{Fix blockquote formatting} \begin{blockquote}
%     Running a decorated function
%     Decorated function has returned
%     11
%     \end{blockquote}
% \end{enumerate}

% Python decorators can be used to append functionality to the beginning or end of
% a function invocation by adding business logic in place of the first and second
% print statements within \lstinline{_wrapped_func} respectively. However, if the
% function provided as an argument could be rewritten, and the rewritten function
% returned, then the returned function would literally be a variation of the one
% the decorator was applied to. Using this mechanism, the original incarnation of
% \pdsf can introduce \emph{procedural} variance into a program. When applied to a
% model of behaviour as previously described, procedures model behaviours, making
% procedural variance \emph{behavioural} variance also. 

% \labelledsubsec{The \atfuzz decorator}{early-pdsf-fuzz-decorator}

% \revnote{\atfuzz isn't the version used in the caise paper, so this is pretty irrelevant. Replace with something more substantial around weaving with e.g. fuzz\_clazz.}
% To introduce behavioural variance with an effect that is contingent on model
% state at runtime, \atfuzz works with two more complex details about Python
% decorators.

% \inline{expand}
% First, \atfuzz is a function which \emph{generates} a decorator.

% \inline{expand}
% Second, the parameter \atfuzz accepts represents the variance to be applied, as
% previously mentioned.

% \inline{expand}
% The combination of these two factors allows for the introduction of procedural
% variance contingent on model state. Because this procedural variance can be
% applied to procedures representing behaviour in a structured and predictable
% manner, \atfuzz can also be used to represent behavioural variance in a model.


% % \labelledsec{FuzziMoss: a library of behavioural variances in sociotechnical
% % % systems}{priorworkfuzzimoss}
% % \labelledsec{FuzziMoss: A Library of Behavioural Variances in \sociotechnical systems}{priorworkfuzzimoss}

% \inline{expand}
% \atfuzz can accept any function with the correct signature as its representation
% of variance. A library of such variances is a good and cool thing to have.

% \inline{expand}
% Here are some examples of the variances included in FuzziMoss.

% \labelledsec{Theatre_ag: controlling the percieved time of \sociotechnical agents}{prior_work_theatreag}

% \expand
% The simulation and modelling environment \pdsf and its related software stack
% was built around involves representing tasks as procedures, but agents in
% \sociotechnical systems complete tasks concurrently in real-world time.

% \expand
% Without mechanisms for controlling the concurrent activity of such agents, race
% conditions between their actions might occur.

% \expand
% To demonstrate that behavioural variance was being introduced to a model in a
% controlled manner, and to experimentally demonstrate that it was having the
% expected effect on simulated behaviour, controlling for these race conditions
% was necessary.

% \expand
% Theatre_AG controls simulated behaviour by introducing the notion of a clock,
% where simulated behaviour is enacted in quantised units of time.

% \labelledsec{Demonstrating behavioural variations in \sociotechnical models}

% \expand
% To demonstrate that use of the \atfuzz decorator, making use of various
% introduced by Fuzzi-Moss and in a controlled environment through use of Theatre_AG, 

% \labelledsec{Shared concepts with aspect-oriented programming}{aop-and-early-pdsf}


% \labelledsec{Weaknesses of the approach}{prior_work_weaknesses}


% %% ==== MARK
% %% Everything below this is from the original draft of the thesis. Leaving it
% % here to assist in redrafting. 



% % The variation to be applied was represented as a function
% % which took a list of ``steps'' and returned a new list of those steps, represent

% In the case where the required attribute is not callable, the value is returned
% as normal. Callable attributes are modified, however. In this case, the
% replacement \lstinline{__getattribute__()} also checks for a set of
% manipulations to make to the original code. These can be applied before or after
% the original code is run, as well as around it. A new function is returned
% containing a reference to the originally sought attribute, but which will search
% for these additional pieces of work before executing it, and can execute this
% work before or after the call (or both). These pieces of work are referred to as
% ``advice'', adopting aspect orientation terminology.

% As discussed further in \cref{subsec:pdsf_aop}, this approach is effectively an
% implementation of a traditional aspect orientation framework. However, unlike
% existing frameworks, PyDySoFu also supports a special kind of ``around'' advice:
% before a function is called, it can be rewritten. This is done by applying
% ``before'' advice which retrieves the abstract syntax tree of the target
% callable attribute using Python's \lstinline{inspect} module (its built-in
% reflection), applying arbitrary transformations to the tree, and recompiling it
% into a Python \lstinline{code} object (its representation of its
% internal bytecode). At this point, many things are possible: the transformation
% can be cached for later use, can replace the original callable's
% \lstinline{code} object to make the transformation persistent, or can be
% discarded after use. This transformed code is run in lieu of the original,
% effectively enabling aspect orientation which can make adaptations \emph{inside}
% a procedure as well as before and after its execution.

% This approach also had some limitations:

% \begin{itemize}
%     \item Traditional pointcuts cannot target points inside a procedure, meaning
%     that an aspect applied ``inside'' its target must manage the points where
%     its transformation is applied manually.
%     \item Importantly, a callable object's internal bytecode cannot be replaced
%     in Python3, leading to a rewrite discussed in \cref{chap:pdsf_rewrite}.
%     \item This method is significantly slower than other aspect orientation
%     approaches, as rewriting a class' \lstinline{__getattribute__} method means
%     that \emph{every} resolution of an object's attributes --- whether they are
%     methods or values, and including a class' built-in ``magic'' methods ---
%     incurs an overhead from the replaced \lstinline{__getattribute__}
%     implementation. However slight this overhead can be made, affecting Python's
%     built-in methods on classes means that rewriting the
%     \lstinline{__getattribute__} method is unavoidably expensive due to the
%     scale of these methods' use.
% \end{itemize}

% However, the goal of the original research was to develop a flexible
% ``proof-of-concept'' of aspect orientation adapting procedure definition at
% runtime, which was successfully
% achieved\cite{wallis2018caise,wallis2018genetic}. 

% \subsection{Aspect Orientation \& PyDySoFu}\label{subsec:pdsf_aop}

% The goals of ``changing a function's behaviour'' and maintaining
% ``obliviousness'' in the original definition of that function speak to the goals
% of the aspect oriented programming paradigm\cite{kiczales1997aspect}. Quoting
% their original definitions:

% \begin{displayquote}
%     Components are properties of a system, for which the implementation can be
%     cleanly encapsulated in a generalized procedure. Aspects are properties
%     for which the implementation cannot be cleanly encapsulated in a
%     generalized procedure. Aspects and cross-cut components cross-cut each other
%     in a system’s implementation.
%     [ \ldots{} ]
%     The key difference between
%     AOP and other approaches is that AOP provides component and aspect languages
%     with different abstraction and composition mechanisms.
% \end{displayquote}

% Generally, aspect orientation is percieved to be a technique for separation of
% concerns. Any cross-cutting concerns can be separated from their components into
% aspects applied where that concern arises. The strength of aspect orientation
% lies in its compositional nature: developers can write short, maintainable
% implementations of a procedure's core purpose (for example, business logic) and
% ancillary concerns such as logging or security can be woven into this
% implementation as preprocessing, compilation, or at runtime. This compositional
% nature is what gives rise to aspect orientation's ``obliviousness'', as the
% procedure targetted by a piece of advice is written without regard to that fact.

% The original PyDySoFu implementation was an aspect orientation library focusing
% on separating a function's definition from \emph{potential changes to it}. This
% was used to model ``contingent behaviour'' --- behaviour sensitive to some
% condition --- as an original, ``idealised'' definition of that behaviour, plus
% some possible alterations. These changes might apply to many different
% behaviours in the same manner, and therefore represent concerns which separate
% cleanly into an aspect. An example would be the behaviour of a worker whose job
% requires focus on allocated tasks. A lack of focus could be represented as steps
% of the worker's tasks being executed in duplicate, out-of-order, or skipped.
% Assuming aspects as described by \citeauthor{kiczales1997aspect} are able to
% edit the definition or execution of a procedure\footnote{As opposed to simply
% wrapping it with additional behaviour before and/or after execution}, such
% contingent behaviours are well modelled as aspects.

% To achieve this, a model was presented in \cite{wallis2018caise} wherein aspects
% were developed which could change function \emph{definitions} on each invocation
% of that function, contingent on program state. This allowed behavioural
% adaptation to be simulated in an aspect-oriented fashion. In addition, a library
% of behavioural adaptations called FUZZI-MOSS\inline{CITECITECITE} was developed
% which implemented many cross-cutting, contingent behaviours in procedural
% simulations of \sociotechnical systems.

% One important contribution of this work is that PDSF aspects are effectively
% able to operate \emph{inside} a target. In typical aspect orientation frameworks
% such as AspectJ~\cite{aspectj_intro}, aspects operate by effectively prepending
% or appending work to a target, referred to as ``before'' or ``after'' pointcuts
% respectively. To do both is referred to as ``around''. By manipulating
% procedures within Python directly, PDSF is able to manipulate its target from a
% new perspective, adding (or removing) work during the target's
% execution\footnote{Similarly to \cref{subsec:bca}, but in an aspect oriented
% manner.}. Moreover, because weaving is performed dynamically, every execution of
% a function may perform different operations.

% % change to a comparison between this and the lit, what are the limitations of
% % this approach, etc.
% \subsection{Opportunities presented by PyDySoFu}

% PyDySoFu presented several oppportunities for future research. Some salient
% properties of the original work include:

% \begin{itemize}
%     \item It provided an aspect orientation library which could weave and
%     unweave aspects during program execution, without relying on anything other
%     than Python's built-in language features. As discussed in
%     \cref{sec:dynamic_aop_review}, this is supported by some early aspect
%     orientation frameworks also, but AspectJ dominates in the world of aspect
%     orientation frameworks and does not support weaving during program execution.
%     \item It provided the capacity to weave aspects \emph{inside} targets, as
%     opposed to around them, or at either end of their execution. So far as we
%     are aware, no aspect orientation framework in research or industry has
%     offered this feature, and its applications and potential are yet to be
%     explored.
%     \item Relatedly, PyDySoFu was used in the context of simulating behaviour which may change
%     over time. Contingent behaviour being a cross-cutting concern is an
%     innovation of the early research which suggests aspect orientation may have
%     strong applications in \sociotechnical simulation \& modelling.
% \end{itemize}

% \inline{Do we need a brief explainer of what aspect orientation is before
% jumping into outside lit? Or will this go in the introduction? already a little
% in the earlier litrev subsections.}

% The amount of potential investigation which can be done into the dynamic weaving
% of target-altering / ``inside'' aspects in simulation \& modelling applications
% is vast. While literature on the complete topic is absent, each individual
% component of this research angle is well-studied on its own. These opportunities
% might be related to existing literature through the following questions:

% \begin{itemize}
%     \item How does PyDySoFu compare to existing aspect orientation frameworks,
%     particularly those with a focus on dynamic weaving? Related frameworks are
%     summarised and compared in \cref{sec:dynamic_aop_review}.

%     \item What is the use of aspect orientation in simulation \& modelling? How
%     does the approach taken in \pdsf's prior work relate to existing
%     approaches? This will be discussed for simulation in
%     \cref{sec:ao_and_simulation}, and for modelling in
%     \cref{sec:ao_and_modelling}.
    
%     \item Variability is important to capture in any \sociotechnical model or
%     simulation. How is variability treated in existing literature, and how does
%     this relate to \pdsf's approach? This will be explored in
%     \cref{sec:dynamism_in_sm}.
% \end{itemize}




% % ==== MARK Second rework, with caise-version of pdsf.

% \pdsf is a Python library~\cite{pdsf_repo} which implements a dynamically-woven
% aspect orientation framework with the capacity to weave changes within a
% specified join point. It is designed primarily to address the needs discussed in
% \cref{sec:lit_discussion}, particularly those of aspect-oriented simulation and
% modelling.

% Work on PyDySoFu predates the work presented in this thesis: an early version
% was developed for the modelling of behavioural variation. This prototype was
% redeveloped and improved to provide more suitable tooling for more complex
% experiments, and to better address the requirements of such a tool and
% opportunities found in the existing literature. \pdsf{}'s improved design and
% implementation is discussed in \cref{chap:pdsf_rewrite}. 

% In this chapter, prior work on \pdsf{} and the research conducted with it is
% reviewed, so as to isolate the contributions presented in this thesis from
% previously published research.

An implementation of the main tool developed and used in this thesis, ``\pdsf'',
predates this thesis. As context for the contributions in this thesis, this
chapter will describe the state of the project before the presented research was
undertaken. Motivations for the tool's original development are described in
\cref{sec:pdsf_motivations}, which are followed by its design and implementation
in \cref{sec:prior_work_pdsf}, and that of related tooling for experiments and
simulations in \cref{sec:prior_work_machinery}. The chapter concludes with a
description of the research undertaken using these tools in
\cref{sec:caise_paper}, as some results in the representation of behavioural
variance using aspect orientation were found using these tools which predate
this thesis and offer important background for the research undertaken in it.

\section{Motivations in originally implementing \pdsf}\label{sec:pdsf_motivations}

The original incarnation of \pdsf{} was developed with different motivations
than those outlined in \cref{chap:lit_review}. It is therefore important to
elucidate the context in which it was designed and developed.

\pdsf{} was originally developed for the representation of behavioural variance
in sociotechnical systems, and was first produced as a proof-of-concept; the
core contribution was one of tooling. It was developed for use in Python
codebases, by virtue of the language's widespread use and its flexibility in its
modelling of data and process.

The original version of the tool was to be applied to models of behaviour in
\sociotechnical systems, where individual actions were represented as functions.
Actions which could be decomposed further into more granular actions were to be
defined as functions with calls to invoke their more granular counterpart
functions. Invocations of low-level behaviours would implement some change to an
environment in the model which its modelled behaviour would be expected to
incur. Invocations of high-level behaviours, containing the invocations of
lower-level behaviours they compose in the model, would therefore apply the
combined effect of the collected behaviours they represent. A benefit of this
approach to modelling behaviour was that high-level behaviours could implement
the ``flow'' of a behaviour. For example, a behaviour which would be modelled in
a flowchart as having some loop could be modelled analogously in the method
described through use of primitive control flow operators in Python, such as
\lstinline{for} and \lstinline{while} loops.

Another benefit of this approach is that the behaviours modelled have a
predictable structure which is amenable to metaprogramming. A low-level
behaviour's affect could be changed by changing the function definition; more
structural changes could be made by altering the flow of less granular
behaviours. A simple high-level behaviour containing a series of function
invocations (modelling an ordered list of steps in the \sociotechnical system)
can be represented as a literal list of function calls. The contents of such a
list is trivially modifiable. Removing an item from a list or truncating it at a
certain length, for example, are both achievable in a trivial manner using
high-level languages such as Python. Notably, many behaviours can be conceived
of which could be represented as high-level behaviours but would not be amenable
to a simple list of more granular behaviours, such as a behaviour with a looping
quality. 

With a mechanism to rewrite either an implementation of a behaviour or a
collection of behaviours (in the less granular functions mentioned), modelling
in such a fashion could therefore lend itself to semantically simple
metaprogramming that could represent real-world variations in behaviour.
However, largely for reasons discussed in \cref{sec:variation_sm_motivations},
metaprogramming for representing realistic behavioural variations in
\sociotechnical simulations should be able to take advantage of system state.
Many real-world behaviours are contingent on environmental state. Real-world
actors in \sociotechnical systems might become tired after lots of work, or
proportionally to time of day within a simulation. Therefore, the
metaprogramming as described should be performed during runtime, for which no
suitable candidate was available. \pdsf was developed to fulfil this
requirement, so that behavioural variance in \sociotechnical simulation could be
modelled as described and subsequently studied.



\section{Original \pdsf Implementation}\label{sec:prior_work_pdsf}

To disambiguate the improvements made to the original \pdsf implementation
in the tooling contributions of this thesis --- and to explain the fundamental
concepts involved in both implementations' approaches to aspect orientation and
the application of behavioural variance --- the implementation of the original
\pdsf tool, which predates the work presented in this thesis, is described here.



\subsection{Weaving mechanism}\label{subsec:prior_work_weaving}
The original \pdsf implementation made use of a weaving mechanism which could be
categorised as ``total weaving'' in the parlance of
\citeauthor{dynamicAOchitchyan}~\cite{dynamicAOchitchyan}: hooks to apply advice
are woven into every possible join point. The library was implemented in Python,
which offers a flexible object model \pdsf is able to take advantage of in order
to weave its hooks.

Python's object model has three key properties which the original implementation
of \pdsf takes advantage of:

\begin{enumerate}
    \item Everything in Python is an object, including types, functions, and
    classes
    \item Objects are, in essence, implemented as a dictionary\footnote{Python's
    name for what other languages might call a \lstinline{map} or
    \lstinline{hashmap}.} with string keys. All attributes of an object --- such
    as a method on an instance of a class --- are values in this dictionary, and
    their identifiers are the string keys of the dictionary. In essence,
    \lstinline{someObject.val} is notionally equivalent to
    \lstinline{someObject.__dict__['val']}, though the subtleties of this
    mechanism will be explained later.
    \item Operations on objects are handled by ``magic methods'', which are
    specifically named methods on objects which Python calls for fundamental
    behaviour in the language. For example, \lstinline{objA == objB} is
    interpreted by Python as \lstinline{objA.__eq__(objB)}. Other built-in
    behaviours in Python have similar reserved method names which represent the
    implementation of some behaviour. These methods can be overridden, or
    specified by a programmer if they wish.
\end{enumerate}

\pdsf weaves aspect hooks into classes by taking advantage of these three
properties of Python. At a high level, \pdsf operates by replacing the
\lstinline{__getattribute__()} method of a class object with a custom one.
\lstinline{__getattribute__()} is the magic method responsible for performing
lookups in an object's underlying dictionary. The replacement
\lstinline{__getattribute__()} also looks up attributes in the relevant object's
underlying dictionary, but it also searches for advice to be applied when
performing these lookups, and applies any advice it finds. The replacement
\lstinline{__getattribute__()} method represents the aspect hooks woven by
\pdsf{}.

Hooks are applied to a class by way of an invocation to a function,
\lstinline{fuzz_clazz}, which takes a class as a parameter and weaves aspect
hooks into that class. \inline{Add citation for pdsf, asp}
\lstinline{fuzz_clazz} replaces the \lstinline{__getattribute__()} method of the
class with a new function object which it constructs. The replacement function
object operates \inline{This sentence was left half-finished --- no idea why,
but the para needs wrapping up.}

More specifically, the replacement \lstinline{__getattribute__()} method makes a
call to the class' original \lstinline{__getattribute__()} method to retrieve an
attribute when required. If this attribute is not a function or method, it is
returned by the woven \lstinline{__getattribute__()} function and the program
affected class behaves as if it was never altered. However: if an attribute to
be retrieved is a method or function, a new function is constructed and
returned. This function looks up and applies advice to the attribute originally
retrieved by the program, and contains a reference to the original aspect to
enable its execution. This wrapping function is therefore the core of the
weaving process: a replaced \lstinline{__getattribute__()} method injects the
aspect hooks which constitute the aspect-orientation framework. Advice to be run
before, around, and after a join point were implemented as invocations of advice
functions before a target was executed for ``before'' advice, after a target was
executed for ``after'' advice, and with an around function being run with a
reference to the advice's to invoke at its own discretion.

\subsection{Applying Process Mutations}\label{subsec:prior_work_mutations}
The development of \pdsf{} was intended to support simulation \& modelling
research by providing a way of applying program modifications at runtime. It
sought to fulfil the needs of a program which determined that a (potentially
non-deterministic) change to the behaviour it modelled was required, and enable
the program to apply that change mid-process in an aspect-oriented fashion.
Aspect orientation libraries typically support advice woven before, around, or
after a join point; modifying the join point itself essentially allows changes
inside its definition, introducing a fourth type of weaving. \pdsf achieves this
through a special type of ``before''-style aspect, which it calls a ``fuzzer''.

Fuzzers implement transformations on abstract syntax trees. They are implemented
as functions which receive a list of AST objects representing the definition of
a function which satisfies a fuzzing advice's join point, and return another
list of AST objects, which replace the target function's definition. Any
transformation resulting in a valid AST is permitted. A code snippet
demonstrating the implementation of this process is shown in
\cref{fig:v1_pdsf_fuzzing_impl_codesnippet}.

\begin{figure}
\begin{lstlisting}
class FuzzingAspect(IdentityAspect):

    def __init__(self, fuzzing_advice):
        self.fuzzing_advice = fuzzing_advice

    def prelude(self, attribute, context, *args, **kwargs):
        self.apply_fuzzing(attribute, context)

    def apply_fuzzing(self, attribute, context):
        # Ensure that advice key is unbound method for instance methods.
        if inspect.ismethod(attribute):
            reference_function = attribute.im_func
            advice_key = getattr(attribute.im_class, attribute.func_name)
        else:
            reference_function = attribute
            advice_key = reference_function

        fuzzer = self.fuzzing_advice.get(advice_key, identity)
        fuzz_function(reference_function, fuzzer, context)


def fuzz_function(reference_function, fuzzer=identity, context=None):
    reference_syntax_tree = get_reference_syntax_tree(reference_function)

    fuzzed_syntax_tree = copy.deepcopy(reference_syntax_tree)
    workflow_transformer = WorkflowTransformer(fuzzer=fuzzer, context=context)
    workflow_transformer.visit(fuzzed_syntax_tree)

    # Compile the newly mutated function into a module, extract the mutated function code object and replace the
    # reference function's code object for this call.
    compiled_module = compile(fuzzed_syntax_tree, inspect.getsourcefile(reference_function), 'exec')

    reference_function.func_code = compiled_module.co_consts[0]
\end{lstlisting}
\caption{A code snippet from the original \pdsf implementation, implementing
\lstinline{Fuzzing} aspects and applying fuzzing to a function definition.}
\label{fig:v1_pdsf_fuzzing_impl_codesnippet}
\end{figure}

As can be seen in \cref{fig:v1_pdsf_fuzzing_impl_codesnippet}, fuzzing aspects
are implemented by way of ``prelude'' advice (\pdsf{} nomenclature for advice
run \emph{before} a target invocation, inspired by early work on TheatreAG,
which is discussed in \cref{subsec:prior_work_theatre}). Prior to the target
being run, the fuzzer replaces its underlying code object with one produced by
compiling a fuzzing aspect's returned AST objects. Replacing the underlying code
object with a modified one allows the fuzzer to define arbitrary modifications
to the definition of the function, thereby achieving runtime metaprogramming.


\subsection{Limitations}\label{subsec:prior_work_pdsf_limitations}

The original implementation of \pdsf was intended to demonstrate the feasibility
of runtime metaprogramming and its potential in simulation and modelling.
However, its design presents limitations.

There is an overhead involved in running
the wrapped function for every invocation of \lstinline{__getattribute__()}, and
also in running aspect hooks for all possible join points, even when those hooks
are not targets of advice at a given moment.
When an attribute is the target of advice, aspects are discovered and applied.
However, aspects are discovered by lookup within the scope of the function
creating a replacement \lstinline{__getattribute__()} method. This design
requires multiple instances of advice weaving to create multiple replacement
\lstinline{__getattribute__()} calls, all of which are invoked on any attribute
lookup. A single target of advice which has multiple pieces of advice woven
therefore incurs a performance penalty for every piece of advice applied, which
must be incurred when any attribute is looked up on the target's class. If the
join point defining the target may apply to many classes, each class must incur
the same penalty, even if none of their attributes are targets of advice in
practice. The original library was developed to demonstrate the feasibility of
the idea underlying \pdsf{} (runtime metaprogramming), but the weaving mechanism
implemented left room for improvement. A robust implementation with attention
paid to reducing this overhead is introduced in \cref{chap:pdsf_rewrite}.

Similarly, an additional overhead is incurred by a lack of caching of the
modifications fuzzers make to function definitions. One can envisage a need for
runtime metaprogramming which produces different function definitions at
different times: an example could be modelling different degrees of degraded
modes introduced to an actor's behaviour in safety-critical systems
research~\cite{johnson2007degradedmodes}. One can also envisage no such need: an
example could be minor temporary modifications otherwise permanently made in
program maintenance, such as to constants within a function definition, to the
format of a function's return value, or adding control flow which exits a
function early on a termination condition. The requirement is a product of the
tool's use case in different scenarios. In scenarios where the same modification
is to be made every invocation, a fuzzer need only be run once; optimisations
enabling the caching of fuzzing aspects' effects would provide better
performance in use cases where such a feature is appropriate.

Other aspect orientation frameworks offer support for other types of advice.
Handi-wrap and AspectJ both support features related to the processing of
exceptions thrown by a program~\cite{Baker_2002,aspectj_intro} and these
features have inspired work into improved exception handling in object-oriented
systems~\cite{millham2011aopandoopsecurity}. However, this version of \pdsf
offers no direct support for exception handling. Opportunities to support the
feature were therefore available for future revisions of the library to
capitalise on; such a revision is also presented in \cref{chap:pdsf_rewrite}.

A final limitation is that the weaving technique this early of \pdsf used is
incompatible with Python3, as replacing \lstinline{__getattribute__()} is not
possible in Python's newer version. It was determined that a tool which was of
practical use to the simulation and modelling community should be produced which
would remain useful to future researchers making modern models; as the existing
weaving technique lacked performance, an opportunity presented itself for a
complete redesign. The resulting new design is presented in
\cref{chap:pdsf_rewrite} which makes use of a new weaving technique.



\section{Additional Simulation Machinery}\label{sec:prior_work_machinery}

Other related projects developed tooling for sociotechnical simulation \&
modelling. Fuzzi-Moss was a project collecting a library of standardised
behavioural modifications for use in sociotechnical simulation \& modelling;
Theatre\_AG was a project offering a model of time against which actors within
sociotechnical simulations \& models could act. While these projects ultimately
were not used in producing the contributions of this thesis, they are outlined
here as relevant to the original \pdsf project as they were originally
developed as a suite of simulation \& modelling tools to be employed together.

\subsection{Fuzzi-Moss}\label{subsec:prior_work_fm}

Fuzzi-Moss\footnote{A backronym for ``Fuzzing Models of sociotechnical
simulations''} was a library of standard behavioural variations written as
fuzzers to be applied by \pdsf{}~\cite{fuzzimoss_repo}. It was primarily created
for use in a model of the impact of inconsistency in teams' executions of
software engineering methodology, which is discussed further in
\cref{sec:caise_paper}.

\begin{figure}
    \begin{lstlisting}
def missed_target(random, pmf=default_distracted_pmf(2)):
    """
    Creates a fuzzer that causes a workflow containing a while loop to be prematurely terminated before the condition
    in the reference function is satisfied.  The binary probability distribution for continuing work is a function of
    the duration of the workflow, as measured by the supplied turn based clock.
    :param random: a random value source.
    :param pmf: a function that accepts a duration and returns a probability threshold for an
    actor to be distracted from a target.
    :return: the insufficient effort fuzz function.
    """

    def _insufficient_effort(steps, context):

        break_insertion = \
            'if not self.is_distracted() : break'

        context.is_distracted = IsDistracted(context.actor.clock, random, pmf)

        fuzzer = \
            recurse_into_nested_steps(
                fuzzer=filter_steps(
                    fuzz_filter=include_control_structures(target={ast.While}),
                    fuzzer=recurse_into_nested_steps(
                        target_structures={ast.While},
                        fuzzer=insert_steps(0, break_insertion),
                        min_depth=1
                    )
                ),
                min_depth=1
            )

        return fuzzer(steps, context)

    return _insufficient_effort
    \end{lstlisting}
\end{figure}

Fuzzi-Moss contained utilities and fuzzers such as:

\begin{itemize}
    \item Several probability mass functions representing chance of
    unexpected behaviour given a length of time and an actor's\ldots{}:
    \begin{itemize}
        \item conscientiousness, a lack of which would increase chance of behavioural adaptation due to lack
    of effort;
        \item concentration, a lack of which would increase chance of behavioural
    adaptation due to distraction
    \end{itemize}
    \item A \lstinline{missed_target} fuzzer, which terminated a
    \lstinline{while} loop early if activated via a probability mass function of
    an actor's propensity for negligence. As an example of a Fuzzi-Moss fuzzer,
    this is shown in \cref{fig:fuzzimoss_missed_target_fuzzer}.\revnote{Consider
    moving some code snippets like those of Fuzzi-Moss, which aren't necessarily
    that important in the grand scope of the thesis, to an appendix.}
    \item An \lstinline{incomplete_procedure} fuzzer, which truncated the
    steps\footnote{Represented by lines of code} taken by an actor if activated
    via a probability mass function representing an actor's propensity for
    distraction
\end{itemize}

Plans were also made for fuzzers representing an actor ``becoming
muddled''\footnote{A behavioural variation caused by confusion.} and making
mistakes in decision-making, but neither have been completed at time of writing.
A discussion around the revival of this project in the context of the \pdsf
rewrite presented in \cref{chap:pdsf_rewrite} is given in
\cref{sec:future_work_reviving_fuzzimoss}. 


\subsection{Theatre\_Ag}\label{subsec:prior_work_theatre}

Theatre\_Ag (``Theatre'') is a project defining a model of time against which actors in
sociotechnical models \& simulations can act~\cite{theatre_ag_repo}. In the
project's overview, it describes itself as\ldots{}:

\begin{blockquote}
Theatre\_Ag is a workflow oriented agent based simulation environment.
Theatre\_Ag is designed to enable experimenters to specify readable workflows
directly as collections of related methods organised into Plain Old Python
Classes that are executed by the agents in the simulation. All other simulation
machinery (critically task duration and clock synchronization) is handled
internally by the simulation environment.
\end{blockquote}

The central metaphor underlying Theatre's model of timing is theatrical: actors
in a simulation or model are members of a ``cast'' (a collection of actors) who
enact a ``workflow'' (simulation steps) in a ``scene'' (domain model within
which the actors interact). Central to the library is its clock: tasks are given
durations, and a clock which synchronises all agents' position in time ticks to
complete different tasks. The theatrical model theatre introduces is the context
for \pdsf{}'s nomenclature for its types of advice: ``prelude'' advice happens
before a task and ``encore'' advice is invoked afterward, as a prelude and
encore would be in a literal theatre.

Theatre has been used as the environment in models of TCP/IP, algorithmic
trading, the spread of disease~\cite{aranTheatreThesis}, and the impact of
behavioural variation in software engineering methodologies as described in
\cref{sec:caise_paper}.


\section{Example Studies using \pdsf for Behavioural Simulation}\label{sec:caise_paper}

The viability of encoding behavioural variations as aspects using \pdsf has been
demonstrated in earlier studies~\cite{wallis2018caise,aranTheatreThesis}.
\revnote{Maybe I should include something on Aran's masters dissertation too,
rather than just citing it?} The study modelled software engineers working to
different methodologies of software engineering: waterfall, in which
requirements are gathered, software is developed to meet requirements, quality
assurance steps are undertaken, and the resulting software is delivered to
customers; and TDD, where the development of tests for quality assurance
precedes the development of features. The study sought to investigate whether,
when software engineers were working suboptimally, there was a difference in the
rate of bugs introduced to a program developed under each methodology.

The study began with a ``naive'' model of software engineers following each
paradigm, devleoped in Python using \pdsf, Theatre, and Fuzzi-Moss. Engineers
would produce ``chunks'' of code, which could contain bugs. In quality
assurance, engineers were modelled as attempting to identify bugs in different
areas of the codebase, fixing them if they were discovered. Developers could
commit chunks of code toward features identified through requirement
engineering, which were eventually completed, but could potentially contain
undiscovered bugs within chunks of code.

This model was then augmented aspectually using \pdsf. Distraction was
represented through the truncation of functions representing workflow steps.
Developers were modelled with different levels of distraction, affecting a
probability mass function (PMF) which would activate when a developer was
modelled as being distracted in a given moment. If the PMF activated, the
workflow step invoked at that moment was truncated using \pdsf. The model showed
that developers following the TDD methodology could successfully complete a
larger number of features on average than those following waterfall, concurring
with the prevailing consensus on the two methodologies.\revnote{Add a citation
for TDD vs agile! Citations are missing from the copy of the CAiSE paper I can
find, but I'd like to use the same one.}

In replicating the community understanding of the model, the paper demonstrated
the feasibility of aspectually augmenting modelled behaviour: the simulation
took a naive model with no capacity for analysing errors, and introduced new
features of the model supporting an avenue of investigation otherwise
impossible with the methodologies represented by the naive model. That the
resulting simulation matched the expectations existing within the community gave
confidence that the tool could be used to build realistic simulations where
some features of a model could be separated from its core codebase.

\section{Discussion}\label{sec:prior_work_discussion}

The existing case study employing \pdsf demonstrated that realistic model
features could be separated from their core codebase, and gave credence to
\pdsf{}'s use as a tool for aspectually augmenting models with behavioural
variance. It also left many research questions unanswered and tooling
flaws unsatisfied, however:\revnote{Does this want to be a list? Consider
reworking to a paragraph.}

\begin{itemize}
    \item Aspects were believed to be ``realistic'' as they represented the
    expected outcomes of the simulation. However, no real-world data was used to
    corroborate the claim, and it was unclear that aspectually augmented
    behaviour could capture the variations present in real-world human
    behaviour.
    \item These aspects also demonstrated variations which were identifiable in
    the emergent properties of a system (for example, mean time to failure of a
    software system under development, or successfully completed features). The
    variations applied to individual developers might have poorly modelled
    individual behaviour, but produced accurate emergent properties of the
    system individual developers acted within.
    \item As discussed in \cref{subsec:prior_work_pdsf_limitations}, \pdsf's
    implementation at the time was a proof-of-concept which, while successfully
    demonstrating the potential of aspect-oriented runtime metaprogramming, was
    also inefficient, feature-incomplete, and lacked compatibility with modern
    software engineering tooling.
    \item Models of distraction were adopted from the common library provided by
    Fuzzi-Moss. However, this model was not applied to other codebases. It
    remains unclear that Fuzzi-Moss' model of distraction is broadly applicable
    in other projects: different models of distraction might be required by
    different researchers. Further, a model of distraction which realistically
    represents the behaviour of an individual (rather than the emergent
    properties of the system that individual acts within) might not apply to
    other systems the individual acts within. Briefly put, The portability of
    aspectually modelled behavioural simulation has not been investigated, and
    literature within the aspect orientation community lacks evidence to support
    a belief in their portability~\cite{przybylek2010wrong,Constantinides04aopconsidered,steimann06paradoxical}.
\end{itemize}

This left opportunities to improve both the tooling offered for aspect-oriented
runtime metaprogramming, and the evidence supporting its use to encode
behavioural variations in sociotechnical systems. Improvements to the tooling
follow in \cref{chap:pdsf_rewrite}; later chapters propose --- and discuss the
implementation of --- real-world systems which are suitable for modelling using
\pdsf{} in \cref{chap:rpglite}, and a study of those systems using aspectually
augmented models in \cref{chap:exp1_simulation_optimisation} and \cref{chap:exp2_old_aspects_new_systems}.
