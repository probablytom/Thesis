\chapter{Closing Discussion}\label{chap:closing_discussion}

\inline{
  Closing discussion chapter is currently just — to be fleshed out
}

This thesis presented research into the application of aspect-oriented
programming to the development of models \& simulations, with particular
interest in the technique's use in a research setting. 

Core to the approach is the notion that parts of a model expressing a model's
contingent behaviour, introducing new parameters to it, or other changes a
researcher might make to a codebase to investigate hypothesised behaviour can be
understood as cross-cutting concerns.

We investigated the technique as applied to \sociotechnical systems, but it
needn't be limited to models of only one kind of system. This technique is
general and shows promise as a novel approach to designing software for research
purposes. It may also yield new approaches to collaboration in research.





\section{Summary of Contributions}

We introduced a new version of \pdsf which addresses criticisms of
aspect-oriented programming, in particular its principle of obliviousness. The
tool is later demonstrated as appropriate for use in the simulation \& modelling
context.

We produced a dataset of player interaction with a game designed for researchers
to study, RPGLite. RPGLite's design was introduced and its implementation
explained. We found the dataset useful in our own studies and anticipate that
future researchers will also benefit from a well-described \sociotechnical
system for which there exists a rich dataset of human interactions, well-defined
mathematical properties, and a detailled account of its implementation and
distribution.

We demonstrate that aspects can be applied to alter model behaviour by
altering a model of RPGLite to generate player-specific character choice data.

We demonstrate that a model can be extended using aspects to encapsulate more
complex behaviours. Models of learning with additional parameters are created
and also used to create player-specific models of gameplay. We demonstrate that
different individuals' learning patterns can be characterised by their accurate
simulation using this model of learning, thereby identifying properties of those
players which is not apparent from the dataset alone. For players whose learning
is well modelled by these aspects, we demonstrate that the models can be tuned
to reflect an individual's gameplay with statistical significance.

We also investigate the portability of aspects applied to different models.
\inline{
  Finish stub (/section) concluding the contributions of our research on  aspect
  portability once the chapter on aspect porting is completed.
}





