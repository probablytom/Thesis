\chapter{Closing Discussion}\label{chap:closing_discussion}

The discussion in this chapter summarises the findings made in earlier
chapters, and provides some concluding remarks.

First, the contributions made in earlier chapters are summarised to provide an
overview of the research as a whole. Next, some limitations of the findings and
methodology are acknowledged in the spirit of scientific integrity. The chapter
then ends by reviewing the core concept of the thesis: that parts of scientific
models can be written as --- or refactored to be --- advice, using
aspect-oriented programming. This concept is examined through the lens of
earlier chapters' investigations \& results.




\section{Summary of Contributions}
\label{conclusion:contributions}

Several contributions are made in earlier chapters. This section summarises them
individually to provide an overview of the presented research and its
conclusions. These contributions were made in response to opportunities found in
\cref{chap:lit_review}'s literature review, and are distinct from the
contributions of earlier work as outlined in \cref{chap:prior_work}.

\Needspace{3\baselineskip}
\begin{description}

  \item[A new aspect weaving technique aiming to improve aspect-oriented code's legibility]
    The principle of obliviousness which forms part of aspect-oriented
    programming's design
    philosophy~\cite{filman2000aspect,kell2008survey,Charfi2006AspectOrientedWL},
    but critics note that it makes the intended behaviour of a program difficult
    to ascertain, and should therefore be used in
    moderation~\cite{leavens2007multiple} or not at
    all~\cite{przybylek2010wrong,Constantinides04aopconsidered}. \pdsf{} applies
    aspect hooks to modules as they are imported, meaning that modules which
    \emph{can} be altered by aspects are identified not only when weaving is
    applied, but when any module is used. As this design preserves
    obliviousness, function implementations still contain no reference to any
    potentially applied aspects --- but any other module which invokes that function
    must import it with hooks applied.

    Advice may still be woven anywhere within the module calling the function,
    but the entire module does not need to be searched for a call to \pdsf{}'s
    weaver: developers can simply find the module's import to check whether it
    can be influenced by aspects. This reduces the possible parts of the
    codebase one must check to identify whether arbitrary changes could be made
    to a function's behaviour from anywhere at all to a single, well-defined
    point in the program, addressing the limitation of obliviousness raised by
    critics such as \citet{leavens2007multiple} or
    \citet{Constantinides04aopconsidered} without sacrificing the design
    principle itself.

  % \item[A new type of advice which operates \emph{within} its join point, rather
  %   than before and/or after it] \inline{
  %     I think this really falls within prior work, and I shouldn't count it as a
  %     contribution, but I'm leaving the stub here while I consider this. We
  %     _do_ make improvements to the technique.
  %   }

  % RQ1
  \item[Development of a tool which is suitable for use in aspect-oriented
    modelling \& simulation] The application of advice to existing codebases
    requires some technique to overcome a lack of useful join points in a
    pre-existing model. In addition, \aop{} has been subject to criticism that
    obliviousness limits program legibility, an important quality of software
    developed for research purposes. To address this, a redesign of \pdsf{} was
    developed which provided compatibility with modern versions of Python and
    improved on the design of its hook weaver to address criticisms of \aop{}'s
    legibility by adding aspect hooks at import time, thus preserving
    obliviousness while improving legibility.

  \item[Demonstration that a model's behaviour can be augmented to accurately
    reflect a system under study using advice] \revnote{This description list
    item label is too big and doesn't fit; think the environment needs some
    styling, using the enumitem package.}The review of relevant literature
    in \cref{chap:lit_review} observes that many research projects in
    \aop{} produce tooling, but do not confirm that
    tooling's hypothetical contribution empirically, and few case studies exist
    which demonstrate the benefits of \aop{} empirically.
    It is therefore particularly important that \pdsf{} is demonstrated to
    successfully represent changes to a model as advice.

    \pdsf{} was used to apply advice which augmented a naive model of RPGLite
    play; character selection was altered to reflect the characters selected by
    real-world individuals, as observed in the RPGLite gameplay
    dataset~\cite{rpglite_dataset}. This advice successfully produced synthetic
    gameplay datasets which correlated with the individuals being modelled by
    the advice. Advice is therefore demonstrated to be a viable mechanism for
    the encoding of changes to a model, producing the expected datasets when
    applied. \pdsf{} is also shown to be used successfully in a simulation \&
    modelling setting.


  \item[A model of RPGLite play, and corresponding dataset of player
    interactions] Supporting the earlier contributions is a model of RPGLite
    play\inline{Cite repo for RPGLite model}, paired with a dataset of RPGLite
    player interactions~\cite{rpglite_dataset}. Both are publicly available in
    support of other researchers' future work.

  \item[An exploration of the possibilities which aspect-oriented modelling
    yields] The demonstration that one can successfully represent changes to
    models as aspects yields novel research opportunities which are largely
    undocumented in the literature. Having demonstrated that model behaviours
    can be woven as advice rather than calcifying the changes in a model's
    source code, a large body of future work can be produced investigating more
    specific uses of the technique and applying it in novel, ways.
    
    The future work illustrated in \cref{chap:future_work} follows the example
    of \citet{marsh1994formalising}, whose thesis introduces a formalism of
    trust and also yields a large number of research opportunities.
    \citeauthor{marsh1994formalising} notes that there exists such a large
    number of avenues of research which existing literature does not identify in
    any fashion that their thorough discussion of now-feasible opportunities
    constitutes a contribution in its own right; ``future work'' in the context
    of their results doesn't only refer to improvements on their own findings,
    nor to more advanced versions of their own formalism, but to entirely new
    research projects across a variety of fields. Similarly, the application of
    aspect orientation to simulation \& modelling supports novel research in
    applying the technique to existing codebases, using the technique to develop
    rigorous methodologies for the acceptance of changes to models, alternative
    forms of collaboration for research teams, and other ``future work''
    possibilities. These possibilities are enumerated in \cref{chap:future_work}
    as a thorough exploration of earlier contributions' significance. Where
    possible, all future work identified is accompanied by a specific research
    question, to qualify the particular contribution that work would yield and
    to simplify other researchers' engagement with the topic.
\end{description}



\section{Limitations}

\revnote{
  Check with Tim: is it worth acknowledging some limitations in the conclusion?
  I know it's a little unconventional, but I think it also shows some awareness
  / humility. Hoping it makes questioning in the viva easier, because I can
  point to areas where I openly acknowledge what could be improved in the
  future; I'm following the scientific process, and aware of what ``good'' work
  looks like. At least, that's what I'm trying to demonstrate here.
}
As with all research, the contributions summarised in
\cref{conclusion:contributions} have some limitations. These are acknowledged
here in the spirit of openness, and because identifying weaknesses in any piece
of work helps to improve anything building on it.

\Needspace{3\baselineskip}
\begin{description}

  \item[Import Hook Weaving does not solve all problems with Obliviousness]
    A weakness of aspect-oriented programming which its critics
    identify~\cite{steimann06paradoxical,Constantinides04aopconsidered,przybylek2010wrong}
    is that aspect-orientation's principle of obliviousness makes a program more
    difficult to reason about. Obliviousness --- that the join-point of some
    advice is unaware that a weaver might change it --- complicates reasoning
    about a program, because it's unclear from reading its source code that
    additional logic may be included when it is run, where that logic is to be
    woven, and what it is to do. \pdsf{}'s ability to weave changes
    \emph{within} its join-point as opposed to before or after it introduces new
    ways for a program to exhibit unexpected logic. The only way to identify
    whether a program been affected by \pdsf{} is to identify how it is used, by
    observing where it is imported and whether \pdsf{} is utilised there. Other
    aspect-orientation frameworks provide tooling to facilitate easier
    inspection of aspect-oriented programs; at present, this is not provided for
    \pdsf{}.

  \item[Only one instance of aspect-oriented modelling is investigated]
    The experiments described in \cref{chap:experiment_setup} and
    \cref{chap:experimental_results} investigate the use of \aop{} in a
    simulation \& modelling context, but only apply \aspectorientation{} to one
    model, and specifically one which is \sociotechnical{} in nature. As advice
    is a feasible mechanism to represent changes to models, changes which do not
    relate to behavioural variance should be considered also. Promising related
    work includes research by \citet{Cieslak_2011} which represents details of
    plant growth in \aspectoriented{} L-systems. The technique could be used to
    model more \sociotechnical{} systems such as degraded
    modes~\cite{johnson2007degradedmodes} and
    epidemiology~\cite{aranTheatreThesis} or tackling modelling problems in
    other fields, such as diverging behaviours in astronomical
    models~\cite{comparison_of_galaxy_formation_models,comparison_of_radiative_models_of_galaxies}
    or cross-cutting concerns in business-process
    modelling~\cite{Cappelli_AOBPM,da2020implementation} and
    botany~\cite{Cieslak_2011}. Future work should prove \aop{}'s utility in
    a broader array of contexts.

  \item[Further research is required into the portability of aspects across
  models] Results for the third experiment, in \cref{sec:rq3}, investigated the
  portability of aspects across models and found mixed results. The model of
  learning appears to successfully model players in season 2 of RPGLite but is
  unable to model as many players as it could for season 1 regardless of how the
  model is used. One explanation of this result is that biased player and
  assumptions within the learning model impacted the aspect's suitability to
  modelling a second season of RPGLite. Aspects implementing the prior
  distribution model are successful in simulating players across seasons;
  therefore, the approach seems viable, but future research should clarify the
  weaknesses of the learning model as applied to season 2 of RPGLite. 

\end{description}





\section{\thesistitle}


This thesis presents research into the application of aspect-oriented
programming to the development of models and simulations in a research setting. The use of aspects to
represent changes to a model is novel. Core to the approach is the notion that
parts of a model such as behavioural traits, additional parameters (and their
impact on a model), and behaviour contingent on environmental factors can be
suitably modularised from it as cross-cutting concerns. Similarly, minor changes
to a model such as altering existing model properties can be represented as
advice rather than as a change to the model's source, making the change easily
enabled or disabled without adding complexity to the model itself.

Verification that the technique is viable --- and a demonstration of it ---
required the development of new tooling and the application of that tooling to
investigate the benefits augmenting a model using aspects. \pdsf{} was rewritten
to support Python 3 and to weave aspect hooks at import time, allowing for
dynamic and flexible weaving of advice. A system suitable for simulation ---
RPGLite --- was designed, implemented as a mobile game, and released to collect
data to support modelling efforts. A model of RPGLite was constructed, and a
formalism of learning and confidence was defined, implemented, and used as the
foundation of experiments investigating \aspectoriented{} modelling. Three
experiments were designed and conducted to investigate whether \aop{} could
be used to change models, to compose a model with more complex behaviours, and
to build single units of model change which are applicable across multiple
models. Positive results from these experiments indicate that the technique may
apply to other fields and use cases, and the breadth of these opportunities was
explored in its own chapter.

% Curiously, aspect-oriented programming as implemented by \pdsf can be used to
% augment codebases which were not designed to support the weaving of aspects. No
% change to their codebase is required to support aspect-oriented programming, so
% somebody looking to make modifications can design those as advice without
% requiring anything of the original codebase. In addition, a project using \pdsf
% to weave advice makes no changes to the language and requires no specialist
% tooling; it can be adopted in any other software project without the other
% project needing to support aspect-oriented programming or be built with the
% paradigm in mind. In this way, the choice to use of aspect-oriented programming
% when developing models affects only the team developing that model; their code
% requires nothing of anything it uses or is derived from, and their model can
% be used by other research teams as if it were any other software
% project.\inline{
%   Other AOP systems like AspectJ or nu require special compilers etc and
%   graphical systems like AOBPM introduce requirements for specialised tooling
%   too --- confirm and cite.
% }

\Aop{} is shown to successfully modularise changes to models. Lots more can be
done to explore how it can be used and to identify its limitations; the work
presented shows that it is feasible and promising.





