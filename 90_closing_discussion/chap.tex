\chapter{Closing Discussion}\label{chap:closing_discussion}

The discussion in this chapter summarises the findings made in earlier
chapters, and provides some concluding remarks.

First, the contributions made in earlier chapters are summarised to provide an
overview of the research as a whole. Next, some limitations of the findings and
methodology are acknowledged in the spirit of scientific integrity. The chapter
then ends by reviewing the core concept of the thesis: that parts of scientific
models can be written as --- or refactored to be --- advice, using
aspect-oriented programming. This concept is examined through the lens of
earlier chapters' investigations \& results.




\section{Summary of Contributions}
\label{conclusion:contributions}

Several contributions are made in earlier chapters. This section summarises them
individually to provide an overview of the presented research and its
conclusions. These contributions were made in response to opportunities found in
\cref{chap:lit_review}'s literature review, and are distinct from the
contributions of earlier work as outlined in \cref{chap:prior_work}.

\begin{description}

  \item[A new aspect weaving technique aiming to improve aspect-oriented code's legibility]
    The principle of obliviousness which forms part of aspect-oriented
    programming's design
    philosophy~\cite{filman2000aspect,kell2008survey,Charfi2006AspectOrientedWL},
    but critics note that it makes the intended behaviour of a program difficult
    to ascertain, and should therefore be used in
    moderation~\cite{leavens2007multiple} or not at
    all~\cite{przybylek2010wrong,Constantinides04aopconsidered}. \pdsf{} applies
    aspect hooks to modules as they are imported, meaning that modules which
    \emph{can} be altered by aspects are identified not only when weaving is
    applied, but when any module is used. As this design preserves
    obliviousness, function implementations still contain no reference to any
    potentially applied aspects --- but any other module which invokes that function
    must import it with hooks applied.

    Advice may still be woven anywhere within the module calling the function,
    but the entire module does not need to be searched for a call to \pdsf{}'s
    weaver: developers can simply find the module's import to check whether it
    can be influenced by aspects. This reduces the possible parts of the
    codebase one must check to identify whether arbitrary changes could be made
    to a function's behaviour from anywhere at all to a single, well-defined
    point in the program, addressing the limitation of obliviousness raised by
    critics such as \citet{leavens2007multiple} or
    \citet{Constantinides04aopconsidered} without sacrificing the design
    principle itself.

  % \item[A new type of advice which operates \emph{within} its join point, rather
  %   than before and/or after it] \inline{
  %     I think this really falls within prior work, and I shouldn't count it as a
  %     contribution, but I'm leaving the stub here while I consider this. We
  %     _do_ make improvements to the technique.
  %   }

  % RQ1
  \item[Development of a tool which is suitable for use in aspect-oriented
    modelling \& simulation] Having established through \cref{chap:lit_review}
    that no tool suitable for use in aspect-oriented modelling \& simulation
    exists, \cref{subsec:rqs} proposed the research question: \emph{
      is an aspect orientation tool which is appropriate for use in simulation
      \& modelling feasible to design?
    } To address this, a redesign of \pdsf{}\footnote{Note that an early version
    of \pdsf{} predates this thesis, as discussed in \cref{chap:prior_work}.}
    was developed which provided compatibility with modern versions of Python
    % , introduced optimisations for within-style aspect weaving, 
    and improved on the design of its hook weaver to address criticisms of
    aspect-oriented programming's legibility by adding aspect hooks at import
    time --- thus preserving obliviousness while improving legibility.

  \item[Demonstration that a model's behaviour can be augmented to accurately
    reflect a system under study using advice] The review of relevant literature
    in \cref{chap:lit_review} observes that many research projects in
    aspect-oriented programming produce tooling, but do not confirm that
    tooling's hypothetical contribution empirically, and few case studies exist
    which demonstrate the benefits of aspect-oriented programming empirically.
    It is therefore particularly important that \pdsf{} is demonstrated to
    successfully represent changes to a model as advice.

    \pdsf{} was used to apply advice which augmented a naive model of RPGLite
    play; character selection was altered to reflect the characters selected by
    real-world individuals, as observed in the RPGLite gameplay
    dataset~\cite{rpglite_dataset}. This advice successfully produced synthetic
    gameplay datasets which correlated with the individuals being modelled by
    the advice. Advice is therefore demonstrated to be a viable mechanism for
    the encoding of changes to a model, producing the expected datasets when
    applied. \pdsf{} is also shown to be used successfully in a simulation \&
    modelling setting.

  \item[Demonstration of introducing new behaviours into a model using advice]
    \inline{
      TODO write about using AOP to introduce new behaviours to a model
    }

  \item[Investigation of the portability of advice from one model to another]
    \inline{
      TODO: write about investigating aspect portability (they're modules, so in
      theory they can be portable\ldots{} in practice, obliviousness makes this
      tricky.
    }

  \item[A model of RPGLite play, and corresponding dataset of player
    interactions] Supporting the earlier contributions is a model of RPGLite
    play\inline{Cite repo for RPGLite model}, paired with a dataset of RPGLite
    player interactions~\cite{rpglite_dataset}. Both are publicly available in
    support of other researchers' future work.

  \item[An exploration of the possibilities which aspect-oriented modelling
    yields] The technique investigated in earlier contributions --- the
    application of aspect-oriented programming to model implementations ---
    yields novel research opportunities which are undocumented in the literature
    of relevant fields, although some researchers (notably
    \citet{gulyas1999use}) hypothesise some opportunities. Having demonstrated
    that model behaviours can be woven as advice rather than calcifying the
    changes in a model's source code, we anticipate that a large body of future
    work can be produced investigating more specific uses of the technique and
    applying it in novel, domain-specific ways.\revnote{
      This contribution in particular is explained in a very wordy way. Can we
      be more concise here?
    }
    
    The future work illustrated in \cref{chap:future_work} follows the example
    of \citet{marsh1994formalising}, a thesis which introduces a formalism of
    trust and also yields a large quantity of research opportunities.
    \citeauthor{marsh1994formalising} notes that there exists such a large
    quantity of research opportunities which existing literature does not
    identify in any fashion that their thorough discussion of now-feasible
    research opportunities constitutes a contribution in its own right; ``future
    work'' in the context of their results doesn't refer to improvements on
    their own findings, nor to more advanced versions of their own formalism,
    but to entirely new research projects across a variety of fields. Similarly,
    the application of aspect orientation to modelling \& simulation supports
    novel research in applying the technique to existing codebases, using the
    technique to develop rigorous methodologies for the acceptance of changes to
    models, alternative forms of collaboration for research teams, and other
    ``future work'' possibilities. These possibilities are enumerated in
    \cref{chap:future_work} as a thorough exploration of earlier contributions'
    significance, and is a contribution in its own right in the same manner as
    in the work of \citet{marsh1994formalising}.

\end{description}



\section{Limitations}

\inline{
  Check with Tim: is it worth acknowledging some limitations in the conclusion?
  I know it's a little unconventional, but I think it also shows some awareness
  / humility. Hoping it makes questioning in the viva easier, because I can
  point to areas where I openly acknowledge what could be improved in the
  future; I'm following the scientific process, and aware of what ``good'' work
  looks like. At least, that's what I'm trying to demonstrate here.
}
As with all research, the contributions summarised in
\cref{conclusion:contributions} have some limitations. These are acknowledged
here in the spirit of openness, and because identifying weaknesses in any piece
of work helps to improve anything building on it. \inline{
  Find one or two more limitations to write here if this section is going to be
  kept.
}

\begin{description}

  \item[\pdsf{} improves on problems with obliviousness, but does not entirely
    solve them]
    A weakness of aspect-oriented
    programming which its critics
    identify~\cite{steimann06paradoxical,Constantinides04aopconsidered,przybylek2010wrong}
    is that aspect-orientation's principle of obliviousness makes a program more
    difficult to reason about. Obliviousness --- that the join-point of some
    advice is unaware that a weaver might change it --- complicates reasoning
    about a program, because it's unclear from reading its source code that
    additional logic may be included when it is run, where that logic is to be
    woven, and what it is to do. \pdsf{}'s ability to weave changes
    \emph{within} its join-point as opposed to before or after it introduces new
    ways for a program to exhibit unexpected logic. The only way to identify
    whether a program been affected by \pdsf{} is to identify how it is used, by
    observing where it is imported and whether \pdsf{} is utilised there. Other
    aspect-orientation frameworks provide tooling to facilitate easier
    inspection of aspect-oriented programs; at present, this is not provided for
    \pdsf{}.
  \item[Only one instance of aspect-oriented modelling is investigated]
    \inline{this is a stub, complete me} We only investigate a model of RPGLite;
    it'd be nice to apply AOP to more than just a model of a \sociotechnical
    system, or show that it can be applied in many contexts.
\end{description}





\section{Using Aspects to Augment Models}

\inline{
  This was originally written as an introduction to the chapter, but I've moved
  it to the end, because what it's getting at --- the central concept of using
  aspect-oriented programming when developing scientific models --- is actually
  a good thing to wrap the entire thesis up with, \emph{and} it just fits more
  cleanly at the end and gives this chapter a nicer flow.\\HOWEVER I haven't rewritten it since moving it to the bottom. It needs editing
  to fit here, I expect. Also, it's a little unfocused. What is it trying to
  say, exactly\ldots{}?
}

This thesis presents research into the application of aspect-oriented
programming to the development of models \& simulations, with a particular
interest in the technique's use in a research setting. The use of aspects to
represent changes to a model appears entirely novel, though some references to
it exist in the literature, in particular that of \citet{gulyas1999use}. Core to
the approach is the notion that parts of a model such as behavioural traits,
additional parameters (and their impact on a model), and behaviour contingent on
environmental factors can be suitably modularised from it as cross-cutting
concerns. 

Verification that the technique is viable --- and a demonstration of it ---
required the development of new tooling, and the application of that tooling to
investigate the benefits augmenting a model using aspects.

The work presented investigates the technique as applied to RPGLite as an
example \sociotechnical system. However, the application of aspect-oriented
programming needn't be limited to models of one type of system. In theory, the
technique can be viable in any scenario where a model developed for research
purposes contains cross-cutting concerns. Where future research projects make
significant changes to the behaviour a model exhibits, these changes may also be
suitably represented as aspects woven into the original model, rather than a
fork of it.

Curiously, aspect-oriented programming as implemented by \pdsf can be used to
augment codebases which were not designed to support the weaving of aspects. No
change to their codebase is required to support aspect-oriented programming, so
somebody looking to make modifications can design those as advice without
requiring anything of the original codebase. In addition, a project using \pdsf
to weave advice makes no changes to the language and requires no specialist
tooling; it can be adopted in any other software project without the other
project needing to support aspect-oriented programming or be built with the
paradigm in mind. In this way, the choice to use of aspect-oriented programming
when developing models affects only the team developing that model; their code
requires nothing of anything it uses or is derived from, and their model can
be used by other research teams as if it were any other software
project.\inline{
  Other AOP systems like AspectJ or nu require special compilers etc and
  graphical systems like AOBPM introduce requirements for specialised tooling
  too --- confirm and cite.
}






