\chapter{Closing Discussion}\label{chap:closing_discussion}

The discussion in this chapter summarises the findings made in earlier
chapters, and provides some concluding remarks.

First, the contributions made in earlier chapters are summarised, as an overview
of the research as a whole. Next, some limitations of the findings and the
methodology are acknowledged in the spirit of scientific integrity. The chapter
then ends by reviewing the core concept of the thesis: that parts of scientific
models can be written as --- or refactored to be --- advice, using
aspect-oriented programming. This notion is examined through the lens of the
thesis' contributions.
\revnote{
  I'm a little unfocused and distracted at time of writing --- moving on, but I
  wonder if this can be phrased better, particularly the bit about the last
  section reviewing AOP \& modelling as a whole.
}



\section{Summary of Contributions}
\label{conclusion:contributions}

% TODO: rewrite the below with references to the chapters making the
% contributions. Perhaps it can be a description list of individual
% contributions and the chapters (in order) which make them.

We introduced a new version of \pdsf which addresses criticisms of
aspect-oriented programming, in particular its principle of obliviousness. The
tool is later demonstrated as appropriate for use in the simulation \& modelling
context.

We produced a dataset of player interaction with a game designed for researchers
to study, RPGLite. RPGLite's design was introduced and its implementation
explained. We found the dataset useful in our own studies and anticipate that
future researchers will also benefit from a well-described \sociotechnical
system for which there exists a rich dataset of human interactions, well-defined
mathematical properties, and a detailled account of its implementation and
distribution.

We demonstrate that aspects can be applied to alter model behaviour by
altering a model of RPGLite to generate player-specific character choice data.

We demonstrate that a model can be extended using aspects to encapsulate more
complex behaviours. Models of learning with additional parameters are created
and also used to create player-specific models of gameplay. We demonstrate that
different individuals' learning patterns can be characterised by their accurate
simulation using this model of learning, thereby identifying properties of those
players which is not apparent from the dataset alone. For players whose learning
is well modelled by these aspects, we demonstrate that the models can be tuned
to reflect an individual's gameplay with statistical significance.

We also investigate the portability of aspects applied to different models.
\inline{
  Finish stub (/section) concluding the contributions of our research on  aspect
  portability once the chapter on aspect porting is completed.
}



\section{Limitations}

\inline{
  Check with Tim: is it worth acknowledging some limitations in the conclusion?
  I know it's a little unconventional, but I think it also shows some awareness
  / humility. Hoping it makes questioning in the viva easier, because I can
  point to areas where I openly acknowledge what could be improved in the
  future; I'm following the scientific process, and aware of what ``good'' work
  looks like. At least, that's what I'm trying to demonstrate here.
}
As with all research, the contributions summarised in
\cref{conclusion:contributions} have some limitations. These are acknowledged
here in the spirit of openness, and because identifying weaknesses in any piece
of work helps to improve anything building on it.

\begin{description}

  \item[\pdsf{} improves on problems with obliviousness, but does not entirely
    solve them]
    A weakness of aspect-oriented
    programming which its critics
    identify~\cite{steimann06paradoxical,Constantinides04aopconsidered,przybylek2010wrong}
    is that aspect-orientation's principle of obliviousness makes a program more
    difficult to reason about. Obliviousness --- that the join-point of some
    advice is unaware that a weaver might change it --- complicates reasoning
    about a program, because it's unclear from reading its source code that
    additional logic may be included when it is run, where that logic is to be
    woven, and what it is to do. \pdsf{}'s ability to weave changes
    \emph{within} its join-point as opposed to before or after it introduces new
    ways for a program to exhibit unexpected logic. The only way to identify
    whether a program been affected by \pdsf{} is to identify how it is used, by
    observing where it is imported and whether \pdsf{} is utilised there. Other
    aspect-orientation frameworks provide tooling to facilitate easier
    inspection of aspect-oriented programs; at present, this is not provided for
    \pdsf{}.
  \item[Only one instance of aspect-oriented modelling is investigated]
    \inline{this is a stub, complete me} We only investigate a model of RPGLite;
    it'd be nice to apply AOP to more than just a model of a \sociotechnical
    system, or show that it can be applied in many contexts.

\end{description}





\section{Using Aspects to Augment Models}

\inline{
  This was originally written as an introduction to the chapter, but I've moved
  it to the end, because what it's getting at --- the central concept of using
  aspect-oriented programming when developing scientific models --- is actually
  a good thing to wrap the entire thesis up with, \emph{and} it just fits more
  cleanly at the end and gives this chapter a nicer flow.\\HOWEVER I haven't rewritten it since moving it to the bottom. It needs editing
  to fit here, I expect. Also, it's a little unfocused. What is it trying to
  say, exactly\ldots{}?
}

This thesis presents research into the application of aspect-oriented
programming to the development of models \& simulations, with a particular
interest in the technique's use in a research setting. The use of aspects to
represent changes to a model appears entirely novel, though some references to
it exist in the literature, in particular that of \citet{gulyas1999use}. Core to
the approach is the notion that parts of a model such as behavioural traits,
additional parameters (and their impact on a model), and behaviour contingent on
environmental factors can be suitably modularised from it as cross-cutting
concerns. 

Verification that the technique is viable --- and a demonstration of it ---
required the development of new tooling, and the application of that tooling to
investigate the benefits augmenting a model using aspects.

The work presented investigates the technique as applied to RPGLite as an
example \sociotechnical system. However, the application of aspect-oriented
programming needn't be limited to models of one type of system. In theory, the
technique can be viable in any scenario where a model developed for research
purposes contains cross-cutting concerns. Where future research projects make
significant changes to the behaviour a model exhibits, these changes may also be
suitably represented as aspects woven into the original model, rather than a
fork of it.

Curiously, aspect-oriented programming as implemented by \pdsf can be used to
augment codebases which were not designed to support the weaving of aspects. No
change to their codebase is required to support aspect-oriented programming, so
somebody looking to make modifications can design those as advice without
requiring anything of the original codebase. In addition, a project using \pdsf
to weave advice makes no changes to the language and requires no specialist
tooling; it can be adopted in any other software project without the other
project needing to support aspect-oriented programming or be built with the
paradigm in mind. In this way, the choice to use of aspect-oriented programming
when developing models affects only the team developing that model; their code
requires nothing of anything it uses or is derived from, and their model can
be used by other research teams as if it were any other software
project.\inline{
  Other AOP systems like AspectJ or nu require special compilers etc and
  graphical systems like AOBPM introduce requirements for specialised tooling
  too --- confirm and cite.
}






