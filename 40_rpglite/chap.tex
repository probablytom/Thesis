\chapter{Testing Synthetic Data Realism with RPGLite}

\section{The RPGLite Model}

The model for RPGLite is composed of a base model describing the game rules and
random player behaviour. To this, we add four aspect-oriented components of the
model, in an attempt to adapt the base model's random behaviour into something
``realistic''.

\subsection{Base Model}
RPGLite's base model consists of a set of classes, representing playable
characters, and a set of functions representing aspects of gameplay. Each
function can be thought of as a discrete action an actor might take in a
discrete-event agent-based simulation. 
\par

The basic concept of the model is, given two players and an environment they
play within, to simulate a game in this environment between these players. This
is invoked via a \lstinline{play_game} method, which constructs a ``game
document'' to store game details in, selects a player to go first, and then
plays the game via a series of small functions intended to represent the
discrete actions a player takes when playing RPGLite. Broadly, these involve
choosing characters and choosing moves, as RPGLite is a simple game.
\par

It should be noted, however, that all behaviour in this base model is strictly
random. Any decision, such as the choice of characters to play or moves to make,
are made at random in a uniform manner. As such, the base model effectively
represents ``naive'' play: the equivalent play in the real world would be
represented by players tapping randomly on areas of a phone screen until a game
concluded, with no discernable strategy.

\subsection{Aspects applied}

Four aspects are applied to the base model in order to impose a model of
``realistic'' play over the original naive model. Our hypothesis is:

\todo{Rewrite --- our original wording of the hypothesis clearly no longer works.}
\begin{displayquote}
  Is it possible to take a model which poorly represents some real-world system,
  and have that model represent emergent properties of said system by way of
  composing additional model details?
\end{displayquote}

The nature of the comparison between the real-world system --- here RPGLite ---
and the model --- here our aspect-applied `naive' model --- will be detailed in
\cref{exp:rpglite-simple}. Briefly, the experiment correlates synthetic data
with real-world data to assess whether real-world character choices correlate to
those chosen in the synthetic dataset. We are therefore interested in applying
aspects which produce a model where character pairs are chosen for a given game
in a manner reflecting the real world. Here we describe the additional model
details composed via aspect-oriented programming so as to produce a more
``realistic'' model.
\par

The four aspects applied perform the following functions:

\begin{itemize}
\item \ldots{}adding to the model a record of players' experience over time
  \item \ldots{}players making ideal moves according to
    \cite{kavanagh2020}\todo{find citation for William's journal paper submitted
    after game-on presentation.}
  \item \ldots{}players ``remembering'' the outcomes of games they have played,
    in particular the winning character pair
    \item \ldots{}``experienced'' players choosing to play with characters they
      have seen perform well in earlier games
\end{itemize}

The first of these aspects tracks player experience according to a logistic
curve model. It is applied to a \lstinline{play_game} pointcut to update a point on a
logistic curve when a game is completed. The logistic curve was chosen for its
simplicity and relatively close shape to the desired slope of player
experience\footnote{A sigmoid was desirable because, with RPGLite being a
  relatively simple game, we expected that players would take a few games to
  become familiar with the concepts and strategies involved and, once passing
  familiarity was achieved, there was little else to learn; hence a single
  ``phase transition''-style slope on the experience curve.}. At the end of each
game, players move a small degree on the curve as dictated by the curve's
relative growth rate: a parameter for optimisation in the experiments as
detailled in \cref{exp:rpglite-simple}.
\par

The second of these aspects applies to a \lstinline{get_moves_from_table} pointcut and
intercepts the return value of the target function, which randomly generates a
move to play. The move is replaced by looking up an ideal move in
\cite{kavanagh2020}\todo{find citation for William's journal paper submitted
after game-on presentation.}, generating a move object for the game in progress
according to \citeauthor{kavanagh2020}'s calculated optimal move, and returning this
in lieu of the original random move. In so doing we inject optimal moves played
into the game, meaning that our simulated players can ``see'' the optimal move
in a given scenario and are unlikely to make mistakes. In an analysis of moves
made by players it can be seen that players are likely to play the ``right''
move, and so we generalise to all modelled players playing optimally.\todo{back
  this up with a graph or something, I'm certain William has this already.}

The third aspect applies to a \lstinline{}
\todo{\textbf{finish this!} I need to actually look at the code and make sure
  the aspects do what I'm saying they do, I think there's a bit of a mismatch.
  It's not super important anyway.}



% ========================================================================
\label{sec:experiment1}
\label{exp:rpglite-simple}
\section{Experiment 1: Correlating Synthetic and Real-World Character Choices}

