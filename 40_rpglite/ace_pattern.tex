\todo{Rewrite a little --- I should really actually define this well and call it
the ``ace pattern''! It's a nice structure and probably works well as an elegant
meta model if I can work out how to sell it right.}
These functions all adhere to an
architecture which allows them to access and manipulate pertinent details of the
simulation:

\begin{description}
  \item[actor ---] allows the function to identify the actor performing the activity
    defined by the function. This argument is any object uniquely identifying an
    actor.
  \item[context ---] allows the function to determine details of the current
    thread of work being undertaken by the actor. This is necessary because in
    some simulations, the same actor might pause and resume multiple
    occurrences of the same activity --- for example, they might concurrently
    play three different matches in RPGLite. As a result, it is necessary to
    understand the context of the action being performed by the actor in
    question. This argument can be any object uniquely identifying the context
    of a piece of work, but is most likely a class or dictionary-like object
    to permit the communication of information across invocations of different
    action-representing functions.
  \item[environment ---] an actor's actions are often determined by the global
    environment they act within. There may be ancillary details to the
    actor's actions and the context of their particular thread of work which
    they are undertaken within which are used to determine behaviour, such
    as a landscape they traverse or other actors they might choose to
    interact with. Because all actors share access to a global environment,
    this also provides a message passing space, or a space where actors can
    set values and flags other actors might look to, should those details be
    more general than their specific thread of work at a given point in time.
\end{description}

This architecture was designed because the flexibility of raw source code was
necessary for integrating existing Python tools into our model, meaning that
bespoke agent-based modelling platforms were unsuitable for our model
development.
\par

In particular, the properties this architecture satisfies are our need for a
modelling structure which permitted simple procedural code for modelling (as
used in \cite{pycx}) as well as object-oriented modelling where the above
functions can be understood as methods instead (as used in
\cite{wallis2018caise}). The format also had to be flexible, permitting a simple
single-threaded model or concurrent, possibly multi-threaded models as in
\cite{theatre_codebase}\todo{make a theatre citation via joss}. This
architecture permits this by the property of obliviousness: the types of each
argument are unimportant, meaning that thread-safe structures could be provided
where needed, the complexity of data structures used could be determined by the
needs of the model itself, and communication could be handled in any way that
made sense to a given model. In iterations of various models through this work,
this obliviousness lended itself to communication via  shared mutable data structures
available to all actions (such as the use of a python dictionary for the model
environment) or communication via a dedicated structure for message passing
(such as more complicated actor classes with message inboxes and actor sets in an
environment providing broadcast lists).

While there were technical reasons why such a general approach to model
architecture was taken, the general pattern provided by the architecture is also
useful for reasons of compatibility. This pattern permits translation between
any two languages which provide function calls with arguments of a model's
activities and the way they break down into individual procedures. In this way,
a model written in Python according to this pattern could more easily be
translated (assuming no additional constraints) into other languages, such as
Golang, Erlang, Julia, or any other language of interest in the current
scientific zietgeist. This was of interest in the case of potential studies of
usability and comprehension of our models, particularly with the added
complexity of an aspect-oriented approach to model composition.


