\chapter{Introduction}



\section{A primer on aspect orientation}

\inline{Explain aspect orientation briefly.} Some useful notes for the
explanation:

\begin{itemize}
    \item AO originated in xerox parc, first described in
    \cite{kiczales1997aspect}. There are lots of weaving mechanisms for regular,
    static aspect orientation, and there's a good early survey of them all (and
    implementation in a custom OO language specifically for this!) in
    \cite{masuhara2003modeling}.
    \item AO has some forebears: metaobject protocols, subject-oriented
    programming, adaptive programming, composition filters. The latter three are
    described by \cite{chibani2019using} as being alternative \emph{kinds} of AO
    --- I disagree, but they're certainly attempting similar things.
    \item The original and still most widely used AO implementation is AspectJ,
    which comes with its own aspect language. It's grown over the years and is
    used sometimes in industry [citation needed...]. A smaller alternative would
    be Spring AOP\inline{find a citation for spring AOP}
\end{itemize}



\labelledsec{Prior Work}{priorwork}

\inline{Write a section for the introduction describing the work done on pdsf
before this, to delineate where we're starting from and avoid any claims of
plagiarism. This can be short, the first sec of the lit review is a proper
discussion, but the tool should be mentioned here. See
\cref{sec:pdsf_early_work} for what already exists.}



\labelledsec{Terms \& definitions}{glossary}

\inline{Complete the glossary in \cref{sec:glossary}.}

\inline{Decide whether terms like BPMN, simulation \& modelling, etc also belong
in the glossary\cref{sec:glossary}.}

\begin{description}
    \item[Aspect]  
    \item[Advice] 
    \item[Joinpoint]
    \item[Pointcut]  
    \item[Weaving] 
    \item[AspectJ] The original aspect orientation framework, with language
    extensions to describe pointcuts and aspects.  
    \item[Target] The procedure an aspect is applied to via a join point, to
    affect advice.
    \item[PyDySoFu] 
    \item[]
\end{description}