\chapter{Introduction}

\revnote{This introductory bit should be half its current length\ldots{}}
This thesis is about two distinct areas of study and what happens when they are
married together. First, it is concerned with \aop{}: a software engineering
paradigm which has received some academic interest, but comparatively little
industry use or practical demonstration of its benefits. It applies some extra
program logic (called ``advice'') to places where that program logic is required
(called ``join-points''); the combination of each is an ``aspect''. Integral to
the paradigm's design is its property of obliviousness: the implementation of a
join-point does not indicate whether an aspect applies to it, which simplifies
those parts of the codebase. \Aspectorientation has been well researched, but a
specific context for which it is better suited than its competitors is yet to be
found. Its tooling and theory are well established, but it isn't the ``tool of
choice'' for any particular problem. Critics cite obliviousness as a potential
reason for this: reasoning about a program is difficult if it could be altered
by any other part of the program. However, a use-case where \aop{} \emph{is}
well-suited would allow the existing theory and tooling to be put to use.

Second, it is concerned with how researchers write programs to support their
work. Many fields rely on software models when conducting some investigation or
study. This software is subject to unique requirements: the need to publish
reproducible results extremely strict; the representation of hypothetical
systems or behaviours which, by definition, have not been represented before;
and the knowledge that future studies (potentially conducted by other research
teams) may look to extend that software in ways which are hard to predict, as
they encode novel ideas for research.\revnote{
   Maybe get rid of this list / rework to something snappier? Don't love it…
}
The design and maintenance of
these codebases is unique as a result of these requirements.

Later chapters motivate and demonstrate that \aop{} is well-suited to be used as
a tool for maintaining software models. We suggest that aspects can represent
units of change to an existing model. This allows research software engineers to
develop their experimental models as descriptions of the differences between one
version of a system and their hypothesised one, by encoding those differences as
aspects. As later chapters discuss, the properties which make the paradigm
difficult to apply in industrial codebases can be a strength in the context of
research software engineering: as a model is oblivious to the application of
aspects, removing those aspects is trivial. One can imagine that different
models could be constructed by composing a ``base'' model with different sets of
aspects. Failed experiments would not incur technical debt, or necessitate the
careful removal of code related to that experiment.

\Aspectoriented{} modelling may be the application for which \aop{} is uniquely
well-suited, because obliviousness is a unique property of the paradigm.
However, aspects have not been used to date to represent units of change to
scientific models, and so the viability of the technique is unknown. This thesis
demonstrates that \aspectoriented{} modelling is feasible by showing that
aspects can be used to represent change to a modelled system, that they can be
used to change that system significantly by adding new parameters and modelled
behaviours, and that aspects as units of change can be successfully applied to
different systems, where appropriate.

This introductory chapter proposes the problem to be solved in research software
engineering more thoroughly in \cref{intro_problem_to_solve}, and proposes an
\aspectoriented{} solution in \cref{intro_proposed_solution}. The latter section
also describes \aop{} in more detail, and narrows the scope of the types of
models within the scope of this thesis. \Cref{intro_contributions} notes the
research questions and key contributions found in later chapters, and the
structure of the rest of the thesis is described in
\cref{intro_thesis_structure}.



\section{The Problem to Solve}
\label{intro_problem_to_solve}

Consider a research software engineer, responsible for the development and
maintenance of a well-adopted model of some \sociotechnical system such as a
public transport system, the spread of a disease in a pandemic, or a software
development team's day-to-day work. As the model's use spreads and users'
requirements broaden, how might our engineer manage the maintenance burden of
their codebase? What tools and techniques might they make use of to ease their
task? Their tool might have originally modelled the interactions of individual
developers in a team (to extend the latter example), and offered parameters for
development methodologies or team sizes; subsequent researchers might look to
run simulations of the team with biased behaviour such as illness or tiredness,
or different experience levels in the team, or communication breakdowns between
individuals, or the impact of a change in management or project direction. What
then?

Typically, if a required change would alter the model significantly, researchers
looking to adopt the tool for new purposes would fork it (assuming the source is
available to them under a permissive enough licence) and would make whichever
modifications suited their needs. Other researchers might make their own forks.
If the change is small, or if the new behaviour can be enabled in configuration,
then it might be merged into the original tool. The contribution is thus
disseminated to the research community who adopted the original tool without the
need to migrate to a fork. However, if a future team wanted to research a
combination of factors --- say, simulating the impact of a change of management
on teams with different levels of experience and communication quality --- they
would repeat the process, producing another fork with their own implementation
of this particular combination. As modifications are built on top of each other,
the logic for each possible behaviour is interwoven with that of the original
model.

Such a codebase would be increasingly difficult to maintain. For example, the
abstractions used to modularise it may have elegantly separated different
concerns into different modules at first, but abstractions are often
domain-specific. As the model becomes increasingly general-purpose, the
abstractions used to separate concerns accumulate technical debt in the face of
broader use-cases. In addition, the behaviours added by different teams may not
make sense when enabled at the same time. In the example of a model of a
software development team, a change representing remote work and a change
representing the spread of a virus in different office settings are at odds
conceptually and both are unlikely to be required in most cases. As a result,
the tool's source code risks becoming confusing and unwieldy.

There is therefore an opportunity to improve the design of models and their
codebases. We identify a need for a technique which allows a model to be altered
in any arbitrary fashion so that the changes somebody could make to the tool can
be represented as separate, optional modules --- even when that change
introduces or alters logic in the middle of an existing process. It should also
permit a developer to select the changes they need for their study, effectively
composing a model with whichever behaviours they require.



\section{A Proposed Solution}
\label{intro_proposed_solution}
We identify \aop{} as a tool which is well-suited to the
needs of research software engineers who build models as described earlier. This
section gives a high-level introduction to aspect-oriented programming concepts.
A more detailed discussion of the aspect-oriented
paradigm and its literature follows in \cref{chap:lit_review}.

The systems being modelled throughout this thesis are \sociotechnical systems,
which involve the interaction human actors have with technology and business
processes. We expect this technique is broadly applicable to simulations and
models generally. However, to limit the scope of the research undertaken in this
thesis, the scope within which it is assessed is reduced and it is examined as
applied to \sociotechnical models in particular. These are also briefly
introduced.

Having introduced \aop{} and \sociotechnical systems, the
section concludes by examining the technique's possible application in the
context of the problem described.





\subsection{What is \AOP{}?}

\Aop{} originated in \citeauthor{kiczales1997aspect}'s work
at Xerox Parc~\cite{kiczales1997aspect}. The motivation for the paradigm was the
modularisation of ``cross-cutting concerns'': parts of a program which are not
directly related to the fulfilment of its requirements but are common throughout
it. Cross-cutting concerns generally don't relate to their neighbouring logic
semantically. Rather, they fulfil an ancillary task, such as logging. This
suggests it is a suitable candidate for modularisation: the single
responsibility principle~\cite{martin2003singleresponsibility} suggests that
software with multiple concerns should be refactored into separate modules.
However, traditional techniques are unsuitable for the creation of separate
modules in these situations.

Aspect-oriented programming introduces a new pattern which modularises
cross-cutting concerns. This is achieved by separating them from a program's
core logic and rewriting them as modules (``advice''), the combination of some
implementation (an ``aspect'') and a place to apply it (a ``join-point'')
\revnote{Maybe introduce pointcuts here too --- although I don't think it would
add much to the discussion} --- which are later ``woven'' back into their
intended positions by ensuring the aspect's logic is included somehow at its
corresponding join point. This allows high-level behaviours such as logging,
synchronisation, or memory safety to be factored away from a program's logic and
into separate units. Unlike traditional modules, which are included by the code
which requires them, a ``weaver'' composes advice and the code they should be
applied to, with neither requiring knowledge of the other. This separation is
core to aspect-oriented programming's design, and is referred to as
``obliviousness''.

Aspect-orientation allows for existing codebases to be structured in new ways,
but its notion of modularising cross-cutting concerns is general and can be
applied in other fields too. For example, business process modelling languages
have been adapted to accommodate behaviours or processes modelled using
advice~\cite{Cappelli_AOBPM,da2020implementation,charfi2007ao4bpel}. Some
researchers suggest using aspect-oriented programming to separate concepts in
experimental software, such as experiment and observation, to mirror the setup
of a traditional experiment in research-specific codebases~\cite{gulyas1999use}.
Concepts originating in \aop{} have also seen use outside of software
engineering~\cite{Cieslak_2011,Cappelli_AOBPM,da2020implementation,charfi2007ao4bpel}.

\revnote{
  This can be reworked to refer to variance less in line with the approach to
  discussing the research I now take in \cref{chap:experiment_setup} and
  \cref{chap:experimental_results}. However it's not essential, so I'll revisit
  \emph{if time permits}.
}This thesis explores modelling changes to behaviour using advice, and is applied
to the modelling of \sociotechnical systems in particular. \Sociotechnical
systems partly consist of the behaviour of human agents within a system, which
can exhibit variations contingent on environmental factors and other states.
This variance might be expressed differently in different agents in the system.
In later chapters, we demonstrate that these behavioural variations can be
separated from a generic model into advice, with the advantage that the generic
model's codebase is simplified and any particular behaviour of interest in a
particular study can be modelled by weaving those behaviours at runtime. Advice,
as demonstrated in later chapters, is a flexible and useful tool for composing
models.


\subsection{Challenges in Aspect-Orientation Today}

The \aspectoriented{} paradigm has some drawbacks. Identifying these means they
can be addressed when applying \aop{} to a new domain, and generally
contextualise the research in later chapters. While they are discussed in detail
in \cref{subsec:aop-criticisms}, a quick summary is given here as background for
the ease of reading later chapters and for the discussion of the contributions
made in this thesis, which follows in \cref{intro_contributions}.

The paradigm's philosophy of obliviousness can make codebases difficult to
comprehend. When reading the implementation of a function or feature, it's not
possible to know whether advice is being applied to it elsewhere by inspecting
the relevant source code in isolation. If aspects are applied to it, the
behaviour of that code might be altered. Accurately understanding a program's
control flow can therefore be challenging as a consequence of aspect-oriented
programming's fundamental design choices.

Other limitations of aspect-oriented programming concern aspects themselves.
\Aspectorientation{} tooling typically allows for advice to be woven before
join-points, after them, or around them (in effect, both before and after).
However, this requires aspects to treat their join-points as black boxes, with
the result that logic cannot be added within them. Many codebases are not
designed with the intention of applying aspects, meaning that there may be no
suitable join-points to add the required logic to.

Lastly, while a variety of toolsets exist for aspect-oriented programming, few
studies appear to have been undertaken confirming that its expected benefits
actually materialise~\cite{przybylek2018empirical}. Some scepticism around its
practical benefits have been
raised~\cite{steimann06paradoxical,przybylek2010wrong,Constantinides04aopconsidered}.
An application of aspect orientation to a new domain ought to demonstrate that
the technique is of practical benefit, rather than a theoretical curiosity.



\subsection{What are \SocioTechnical Systems?}

The research in this thesis applies aspect-oriented programming to the modelling
\& simulation of \sociotechnical systems. A \sociotechnical system is one
composed of people, technology, and the interactions between the two. The
term originates in the study of work and organisations undertaken by
\citet{trist1951sociotechnical}. \Sociotechnical systems research continues to
focus on organisations and
workplaces~\cite{pasmore2019reflections,baxter2011socio}, though the interaction
of people and technology is also studied more broadly in areas such as degraded
modes~\cite{johnson2007degradedmodes}, resilience
engineering~\cite{hollnagel2006resilience}, and responsibility
modelling~\cite{lock2009responsibility}.

The involvement of human behaviour in the system's dynamics makes these systems
useful subjects for \aspectoriented{} modelling: variations in the behaviour of
an individual or an entire group is may be considered a concern which cuts
across different actors (or actors in models of different systems if the
behaviour is exhibited in more than one model). To this end, models of learning
are developed in \cref{chap:exp1_simulation_optimisation} which represent
learning as a cross-cutting concern. In addition, datasets of users'
interactions with computer systems can be collected to empirically verify
\sociotechnical{} models of their behaviour. This is also useful for the
research in later chapters: \cref{chap:rpglite} describes the design,
implementation, and data collected from the release of a mobile game, another
example of a \sociotechnical system.

The technique of creating models using \aspectorientation to modularise
particular behaviours or dynamics is not exclusively \sociotechnical in nature.
We suspect that it can be generalised to the behaviours and dynamics
of an arbitrary system in an arbitrary field. However, the paradigm is
demonstrated as a \sociotechnical modelling technique for the purposes of
setting bounds on our aims. It is not feasible to verify the
technique's appropriateness in \emph{every} type of system, but it is feasible to
investigate the technique as applied to a \emph{particular} type of system.
Also, aspect-orientation has received attention in the business process
modelling research
community~\cite{charfi2007ao4bpel,Cappelli_AOBPM,Charfi2006AspectOrientedWL},
setting a precedent for its broader use in modelling other \sociotechnical
systems too.


\subsection{Simulations \& Models}

The field of simulation \& modelling concerns the building of models of some
system and the simulation of that system using the model. We have found
(anecdotally) that the difference between the two can be vague; some definitions
are provided to avoid confusion.

In this thesis,``model'' will be used to refer to some representation of a
system or subject of study, or an abstraction of it. In this sense, a model can
be a concept, a physical model, a diagram, and so on --- in this thesis in
particular, ``model'' will be used with the meaning: \emph{``a software
  representation of a system or other subject of study''}. The term
``simulation'' also requires definition, and throughout this thesis will
generally refer to the execution of a model. More formally put, ``simulation''
is used to refer to \emph{``the emulation of the processes within a system or
  other subject of study''}. A simulation typically generates data, such as a
record of system states over time or a log of actions taken by some actor within
that system.

As the research in later chapters is primarily concerned with the simulation \&
modelling of \sociotechnical systems, the subject of a simulation or model
should be assumed to be a \sociotechnical system unless otherwise specified.


\subsection{Possible Benefits of Aspect-Oriented Modelling}
Some specific benefits we anticipate from an aspect-oriented approach to
building and maintaining models include:

\Needspace{3\baselineskip}
\begin{description}
  \item[Advice as Units of Model Change] Scientific models which already exist
    are difficult to modify. They can be adopted by many research groups,
    meaning breaking changes impact the broader community; they can be the basis
    of published results, so changing the source code might invalidate the
    relationship between ongoing work and published work; and they can be
    brittle, as the incentive when writing software for research purposes is to
    achieve results worth publishing, rather than to produce a high-quality and
    maintainable codebase. This differs from a commercial software engineering
    team's incentives, which are typically to produce software which they can
    continue to produce in the future with minimal overhead imposed by code
    quality. For these reasons, scientific codebases have special requirements
    which discourage direct modification, particularly for different use-cases.
    Aspect-oriented programming allows updates to codebases to be written
    without direct modification to the source code. In the case of research
    software specifically, this has been shown to achieve positive results in
    previous case studies~\cite{ionescu2009aspect}.

  \item[Advice as Tools for Instrumenting Scientific Codebases]
    Models of a system are ideally concerned with the logic required to
    accurately model the system itself, and do not contain additional logic to
    instrument the model for the purposes of a particular experiment. This is
    desirable because it allows the model to be re-used for many experiments, as
    the instrumentation to make observations for a particular purpose are not
    woven throughout its logic. This also makes the codebase easier to read: a
    researcher interested in it only needs to read the logic which was required
    to implement a model, and doesn't also have to identify the parts of the
    program which don't model a system at all (but observe a simulation of it
    instead). Finally, the separation of observational apparatus from a model
    mirrors the design of a traditional experiment, where observations are
    carefully made so as not to bias results. This potential use-case of aspect
    orientation has been suggested in the community~\cite{gulyas1999use}.

  \item[Advice as Hypothesised System Behaviour] If advice is woven into a
  model, side effects of the advice could be used to alter the model. In this
  scenario, advice could be written which changes a model to have a desired
  effect, such as correcting an error in the model or updating it in light of
  new research. Moreover, an established model may serve as a ``proving ground''
  for future research: hypothesised behaviours in a system may be written as
  updates to a proven model of that system, and its predictive quality compared
  to the model executed without advice woven. A simple experimental structure is
  yielded by this process. Given advice may be useful as a unit of model change,
  these units may be proposed, tested, and combined to produce a variety of
  models to meet different research groups' needs without requiring onerous
  maintenance efforts.
\end{description}




\section{Contributions}
\label{intro_contributions}

\subsection{Research Questions}
\label{intro_rqs}

The research questions investigated in the following chapters are: 

\begin{researchquestion}
\begin{description}
\item[RQ1] \rqtwo{}
\item[RQ2] \rqthree{}
\item[RQ3] \rqfour{}
\end{description}
\end{researchquestion}

These are motivated following a review of relevant literature in \cref{subsec:rqs}.

\subsection{Summary of Contributions}

In developing tooling for aspect-oriented modelling and investigating the
technique's feasibility, this thesis makes multiple contributions to the
relevant research communities, including but not limited to answering the above
research questions. These are summarised as:

\Needspace{3\baselineskip}
\begin{description}
  \item[Tooling for \AspectOriented{} Modelling] This thesis presents a redesigned
    and re-implemented version of a tool, \pdsf{}, originally
    prototyped for some prior work,\footnote{See \cref{chap:prior_work} for a
    discussion on related research undertaken prior to starting this PhD.} which contributes both a tool which can
    be employed for aspect-oriented modelling and a weaving technique designed
    for legibility when applying aspects to models. It can be applied to
    existing Python code with no modification to target code, and introduces
    minimal dependencies when added to a project.
  \item[Dataset describing \SocioTechnical System Interaction] A dataset
    describing 370 players' interactions with a mobile game released for
    research purposes called RPGLite. The design, implementation, and data
    obtained from the game is presented, and the dataset collected via RPGLite
    is used to design experiments presented in this thesis.
  \item[Demonstration of Aspect-Oriented Model Enhancement] Aspects are used to
    augment a model of RPGLite gameplay and demonstrate that aspect-oriented
    enhancements can alter a model to improve it (according to some metric). In
    particular, we show that a model can be augmented to synthesise data with
    the properties of the real-world dataset.
  \item[Using Aspect-Oriented Enhancements to Identify Hypothesised Player
    Behaviour] A model of learning is presented with multiple parameters. We
    demonstrate that this can be tuned to specific real-world players' behaviour
    using the dataset collected, resulting in player-specific models of
    learning. Properties of learning of specific real-world players are
    identified by fitting a learning model to their RPGLite gameplay data.
  \item[Investigation into Aspect Portability] We investigate whether
    aspect-oriented changes to a model can be ported from one system to another,
    taking advantage of aspect-orientation's modular nature.
  \item[Exploration of Opportunities Enabled by Aspect-Oriented Modelling]
    New research opportunities are yielded by this novel technique in simulation
    \& modelling. As in other PhD theses where contributions carry potential for
    many pieces of future work~\cite{marsh1994formalising}, the avenues for
    future research are broad enough that identifying and enumerating them is a
    substantial piece of work and constitutes an additional contribution.
\end{description}


% \section{Terms \& definitions}\label{sec:glossary}

% \inline{Complete the glossary in \cref{sec:glossary}.}

% \inline{Decide whether terms like BPMN, simulation \& modelling, etc also belong
% in the glossary\cref{sec:glossary}.}

% \inline{Dejice's thesis has a similar glossary; how did he structure it? Wasn't
%   an appendix. Was it a section of the intro or something else?}

% \begin{description}
%   \item[\AOP{}]
%   \item[Cross-Cutting Concern]
%     \item[Aspect] 
%     \item[Join Point]
%     \item[Advice] 
%     \item[Point Cut] 
%     \item[Weaving] 
%     % \item[AspectJ] The original aspect orientation framework, with language
%     % extensions to describe pointcuts and aspects.
%     \item[Target] The procedure an aspect is applied to via a join point, to
%     affect advice.
%     \item[\pdsf{}] 
%     \item[]
% \end{description}



\section{Thesis Structure}
\label{intro_thesis_structure}

The rest of the thesis is structured as follows.

\Cref{chap:lit_review} surveys the project's relevant literature and identifies
specific research questions in the field which the thesis addresses. Some
earlier work precedes the research presented in this thesis. To delineate this
from the contributions presented in later chapters, \cref{chap:prior_work}
surveys the state of the research project before this PhD began.

Having surveyed the literature, identified research questions, and established
the starting point of the research, \cref{chap:pdsf_rewrite} describes the
re-design and implementation of an aspect-oriented modelling tool, \pdsf{}.
Other technical contributions follow in \cref{chap:rpglite}, which describes the
design and implementation of RPGLite, a mobile game developed for research
purposes, as well as the data collected from it.

Later chapters explore the application of \aop{} to simulation \& modelling
codebases. Three experiments are constructed using a model of RPGLite to answer
the research questions mentioned in \cref{intro_rqs}. The experiments involved
share the majority of their technical foundations and methodology, but are used
in different ways to research different facets of \aop{} as applied to
simulation \& modelling. For this reason, their common foundations and
methodology are explained in \cref{chap:experiment_setup}. The specifics of each
experiment --- and the results of those experiments --- are described in their
own chapter, \cref{chap:experimental_results}, alongside answers to the research
questions they address.

Finally, the encoding of behaviours and model properties as aspects yields novel
research approaches, a discussion of which is omitted in all relevant literature
reviewed to date. We therefore investigate the possibilities \aspectoriented{}
modelling enabled in \cref{chap:future_work} thoroughly, in the same vein as
other theses making similar contributions~\cite{marsh1994formalising}.







