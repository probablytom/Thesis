\chapter{Introduction}

Software's structure is an important part of its design. Software engineering
paradigms and techniques are often concerned with this structure:
object-oriented programming attempts to combine information and its related
processes; functional programming has a focus on the composition of processes
from small, general ones; design patterns such as factories, observers, or
adaptors offer a jargon for common structures used when solving common problems.
These paradigms and techniques are tools, each suited to addressing different
problems in different scenarios or domains.

This thesis investigates a paradigm
being applied to a new domain, where it can be used for new purposes and in new
ways. Our paradigm is aspect-oriented programming; our domain is \sociotechnical
simulation \& modelling. 

Aspect-oriented programming is concerned with a program's modularity. It creates
a new kind of module --- ``advice'' --- which, informally described, expresses
a change to existing code. The research we present in later chapters uses this
novel structure to define new models by composing a model with desired modifications.

This is the ``elevator pitch'' for the research in this thesis. Its high-level
motivation is
presented in a later section in more detail, but
some background knowledge of aspect orientation and sociotechnical modelling is necessary. 
This introduction starts with a brief overview\footnote{A more detailled
  background to the research is given
in the literature review in \cref{chap:lit_review}; the aim of this introduction is to familiarise
the reader with the concepts and aims of later chapters, and so opts for a light touch on the 
relevant topics. Rigorous discussions follow in the appropriate chapters.} of aspect orientation and of
sociotechnical modelling, and moves on to explain the high-level aims of the thesis more
concretely. An overview of the work required to achieve these aims is given, and an outline
of the contributions in later chapters follows. To aid the reader in later sections,
this introduction concludes with a brief glossary of important terms and an outline of
the rest of the thesis.


% === 231026 The below was a first draft from today on the introduction but
% didn't make the cut; instead I think it can be the start to an intro to AOP
% (with some citations added of course)


% Software's structure is an important part of its design. Software engineering
% paradigms and techniques are often concerned with this structure:
% object-oriented programming attempts to combine information and its related
% processes; functional programming has a focus on the composition of processes
% from small, general ones; design patterns such as factories, observers, or
% adaptors offer a jargon for common structures used when solving common problems.
% These paradigms and techniques are tools, each suited to addressing different
% problems in different scenarios or domains. This thesis investigates a paradigm
% being applied to a new domain, where it can be used for new purposes and in new
% ways. Our paradigm is aspect-oriented programming; our domain is \sociotechnical
% simulation.

% % Briefly described, aspect-orientation was
% % originally proposed as a solution to the problem of structuring the parts of a
% % program which are not related to its core logic but are common throughout it,
% % called ``cross-cutting concerns''.

% Aspect-oriented programming offers a program structure designed for improved
% modularisation of program logic which repeats frequently and doesn't relate to
% its neighbouring logic, but does not yield modules using other techniques. An
% example of this is logging. Logging is not typically required for a program to
% achieve its aims, but code which logs pieces of information appears in many
% software components and generally amid program logic which has nothing to do
% with logging at all. These are termed ``cross-cutting concerns''. The goal of
% aspect-oriented programming is to modularise these cross-cutting concerns. This
% is achieved by separating them from a program's core logic and rewriting them as
% modules --- termed ``aspects'' --- which are later ``woven'' back into their
% intended positions. This allows high-level behaviours such as logging,
% synchronisation, or memory safety to be factored away from a program's logic and
% into a program's 

% The concept of separating a program's behaviour into a separate module so as to
% be re-introduced in whatever manner is 





% === 231026 the below probably isn't what I want as my first page, but I do
% think it motivates the problem well. It takes too long to get to the point (I
% guess? It _is_ similar to Steve Marsh's thesis intro IMO) but too casual to go
% in the middle of the chapter, so I think it's appropriate as the start of the
% first section, with a brief para or two heading up the chap.




% === stubs for what comes next...:

% We suggest that changes like these are actually well encapsulated as aspects.
% By weaving different aspects into the original tool, different models can be
% constructed by composition.

% However, there are some challenges with this approach. How does one know
% whether the resulting model is valid? Can this be used practically? Can models
% be made more realistic using the technique, or can new models with different
% behaviours be produced through composition? Are the resulting modules
% portable, so that a single module representing a behaviour can be woven into different codebases?



\section{A Primer on Aspect Orientation \& \SocioTechnical Systems}

As the aims of this research are concerned with aspect orientation, a discussion of
those aims or the relevant research questions requires a familiarity with the concepts
aspect-oriented programming concerns itself with. To that end, this 
section contains a brief summary of aspect-orientation to familiarise the reader
with some general concepts. A more detailled discussion of the aspect-oriented
paradigm and its literature follows in \cref{chap:lit_review}.



\subsection{What is Aspect-Oriented Programming?}

Aspect-oriented programming originated in \citeauthor{kiczales1997aspect}'s work 
in Xerox Parc~\cite{kiczales1997aspect}. The motivation for the paradigm was the 
modularisation of ``cross-cutting concerns'': parts of a program which are not
directly related to the fulfilment of its requirements but are common throughout it.
Cross-cutting concerns generally doesn't relate to
their neighbouring logic: instead they fulfil an ancilary task, such as logging.
This suggests it is a suitable candidate for modularisation: the single responsibility
principle\inline{Cite Robert Martin} suggests that software with multiple concerns
should be refactored into separate modules. However, traditional techniques are unsuitable
for the creation of separate modules in these situations.

Aspect-oriented programming introduces a new pattern which modularises cross-cutting concerns. This
is achieved by separating them from a program's core logic and rewriting them as
modules --- termed ``advice'', the combination of some implementation (an
``aspect'') and a place to apply it (a ``join point'') --- which are later ``woven'' back into their
intended positions by ensuring the aspect's logic is included somehow at its
corresponding join point. This allows high-level behaviours such as logging,
synchronisation, or memory safety to be factored away from a program's logic and
into separate units. Unlike traditional modules, which are included by the code which
requires them, a weaver composes advice and the code they should be applied to, with
neither requiring knowledge of the other. This separation is core to aspect-oriented
programming's design, and is referred to as ``obliviousness''.

Aspect-orientation allows for existing codebases to be structured in new ways,
but its notion of modularising cross-cutting concerns is general and can be
applied in other fields too. For example, business process modelling languages
have been adapted to accomodate behaviours or processes modelled using
advice~\cite{Cappelli_AOBPM,da2020implementation,charfi2007ao4bpel}. Some
researchers suggest using aspect-oriented programming to separate concepts in
experimental software, such as experiment and observation, to mirror the setup
of a traditional experiment in research-specific codebases~\cite{gulyas1999use}.

This thesis explores modelling changes to behaviour using advice, as applied to
the modelling of \sociotechnical systems. These models partly consist of the
behaviour of human agents within a system, which can exhibit variations
contingent on environmental factors and other states, and might exhibit variance
between actors. In later chapters, we demonstrate that these behavioural
variations can be separated from a generic model into advice, with the advantage
that the generic model's codebase is simplified and any particular behaviour of
interest in a particular study can be modelled by weaving those behaviours at
runtime. Advice, as demonstrated in later chapters, is a flexible and useful
tool for composing models.


\subsection{Challenges in Aspect-Orientation Today}

%STUB


\begin{itemize}
  \item AOP's ``obliviousness'' makes it difficult to know what a program will
    do by reading its source code. Program flow is effectively non-linear again,
    so predictions as to what a program will ``do'' are difficult to make.
  \item There's lots of AOP research out there, but it mostly builds tooling or
    describes languages for specifying advice. AOP's claimed benefits aren't
    well tested empirically.
  \item Advice always treats its target as a blackbox.
\end{itemize}

The aspect-oriented paradigm has some drawbacks. Identifying these means they
can be addressed when applying aspect-oriented programming to a new domain where
possible, and generally contextualise the research in later chapters. While they
are discussed in detail in \cref{subsec:aop-criticisms}, a quick summary is
given here as background for this chapter's later explanation of this thesis'
aims and contributions (see \cref{intro_high_level_vision} and \cref{intro_contributions}).

The paradigm's philosophy that advice and the code it is applied to are
oblivious to one another produces codebases which can be tricky to
comprehend. When reading the implementation of some function or
feature, it's not possible to know whether advice is being applied to it
elsewhere. If it is, the behaviour of that code might be altered. Accurately
understanding a program's control flow can therefore be challenging as a
consequence of aspect-oriented programming's fundamental design choices.

Other limitations of aspect-oriented programming concern advice itself. Aspect
orientation tooling typically allows for aspects to be woven before their join
points, after them, or around them (in effect, both before and after). However,
this requires aspects to treat their join points as blackboxes: while
introspection might reveal some details about them, logic cannot be added within
them. Many codebases are not designed with the intention of applying aspects,
meaning that there may be no suitable join points to alter behaviour in the way
a developer intends.

Lastly, while a variety of toolsets exist for aspect-oriented programming, few
studies appear to have been undertaken confirming that its expected benefits
actually materialise. Some skepticism around its practical benefits have been
raised~\cite{steimann06paradoxical,przybylek2010wrong,Constantinides04aopconsidered}.
An application of aspect orientation to a new domain ought to demonstrate
that the technique is of practical benefit, rather than a theoretical curiosity.



\subsection{What are \SocioTechnical Systems?}

The research in this thesis applies aspect-oriented programming to the modelling
\& simulation of \sociotechnical systems. 
A \sociotechnical system is one constituted of people, technology, and the interactions
between the two. The term originates in the study of work and organisations
undertaken by \citet{trist1951sociotechnical}. \Sociotechnical systems research
continues to focus on organisations and
workplaces~\cite{pasmore2019reflections,baxter2011socio}, though the interaction
of people and technology is also studied more
broadly in areas such as degraded modes~\cref{johnson2007degradedmodes},
resilience engineering~\cite{hollnagel2006resilience}, and responsibility modelling~\cite{lock2009responsibility}.

The involvement of
human behaviour in the system's dynamics makes these systems a useful subject
for aspect-oriented modelling, as variations in the behaviour of an individual
or an entire group is a candidate for cross-cutting concern. To this end, models of learning as
cross-cutting concerns are developed in
\cref{chap:exp1_simulation_optimisation}. In addition,
datasets of users' interactions with computer systems can be collected to
empirically verify models of their behaviour. This is also useful for the
research in later chapters: \cref{chap:rpglite} describes the design,
implementation, and data collected from the release of a mobile game, another
example of a \sociotechnical system.

The technique of creating models using aspect-orientation to modularise
particular behaviours or dynamics is not exclusively \sociotechnical in nature.
We suspect that the technique can be generalised to the behaviours and dynamics
of an arbitrary system in an arbitrary field; however, the paradigm is
demonstrated as a \sociotechnical modelling technique for the purposes of
setting bounds on our aims. It is not feasible to verify the
technique's appropriateness in \emph{every} type of system, but it is feasible to
investigate the technique as applied to a \emph{particular} type of system.
Also, aspect-orientation has received attention in the business process
modelling research
community~\cref{charfi2007ao4bpel,Cappelli_AOBPM,Charfi2006AspectOrientedWL},
setting a precedent for its broader use in modelling other \sociotechnical
systems too.


\subsection{Simulations \& Models}

The field of simulation \& modelling concerns the building of models of some
system and the simulation of that system using the model. We have found
(anecdotally) that the difference between the two can be vague; some definitions
are provided to avoid confusion.

In this thesis,``model'' will be used to refer to some representation of a
system or subject of study, or an abstraction of it. In this sense, a model can
be a concept, a physical model, a diagram, and so on --- in this thesis in
particular, ``model'' will be used with the meaning: \emph{``a software
  representation of a system or other subject of study''}. The term
``simulation'' also requires definition, and throughout this thesis will
generally refer to the execution of a model. More formally put, ``simulation''
is used to refer to \emph{``the emulation of the processes within a system or
  other subject of study''}. A simulation typically generates data, such as a
record of system states over time or a log of actions taken by some actor within
that system.

As the research in later chapters is primarily concerned with the simulation \&
modelling of \sociotechnical systems, the subject of a simulation or model
should be assumed to be a \sociotechnical system unless otherwise specified.




\section{High-Level Vision for a \SocioTechnical Modelling Technique}
\label{intro_high_level_vision}

Consider a research software engineer, responsible for the development and
maintenance of a well-adopted model of some \sociotechnical system --- perhaps a
public transport system, the spread of a disease in a pandemic, or a software
development team's day-to-day work. As the model's use spreads and users'
requirements broaden, how might our engineer manage their maintenance burden?
What tools and techniques might they make use of to ease their task? Their tool
might have originally modelled the interactions of individual developers in a
team (to extend the latter example), and offered parameters for development
methodologies or team sizes; subsequent researchers might look to run
simulations of the team with biased behaviour such as illness or tiredness, or
different experience levels in the team, or communication breakdowns between
individuals, or the impact of a change in management or project direction. What
then?

Typically, if a required change would alter the model significantly, researchers
looking to adopt the tool for new purposes would fork it (assuming the source is
available to them under a permissive enough license) and would make whichever
modifications suited their needs. Other researchers might make their own forks.
If the change was small, or if the new behaviour was enabled in configuration,
then it might be merged into the original tool. The contribution is thus
disseminated to the research community who adopted the original tool without the
need to migrate to a fork. However, if a future team wanted to research a
combination of factors --- say, simulating the impact of a change of management
on teams with different levels of experience and communication quality --- they
would repeat the process, producing another fork with their own implementation
of this particular combination. As modifications are built on top of each other,
the logic for each possible behaviour is interwoven with that of the original
model.

Such a codebase would be increasingly difficult to maintain. For example, the
abstractions used to modularise it may have elegantly separated different
concerns into different modules at first, but abstractions are often
domain-specific; as the model becomes increasingly general-purpose, the
abstractions used to separate concerns accumulate technical debt. In addition,
the behaviours added by different teams may not make sense when enabled at the
same time. In the example of a model of a software development team, a change
representing remote work and a change representing 
the spread of a virus in different office settings are at odds conceptually and
both are unlikely to be required in most cases. As a result, the tool's source
code risks becoming confusing and unwieldy.

To resolve this without onerous maintenance burdens or several forked tools, we
propose that additional behaviours, model parameters, or observational
instrumentation be 
modularised into aspects. If they are, then each individual research group can
share a common model without directly modifying it for their own ends:
aspect-oriented programming's principle of obliviousness means that no changes
should require direct modification of the original model's source, because it
should be ``unaware'' that it is the subject of aspect weaving at all.

With this approach, individual research teams can create a model for their own purposes by weaving
their desired changes into the original model. Different teams can share
aspects, standardising future research. The addition of another modification is
the construction of another stand-alone module. The creation of a new model
is the specification of a new set of pieces of advice to weave. This allows for
all research teams to create the models required for their own work, without
creating several forked tools or a single tool with many cross-cutting concerns
and a complex architecture.

Some specific benefits we anticipate from such an architecture include:

\begin{description}
  \item[Advice as Units of Model Change] Scientific models which already exist
    are difficult to modify. They can be 
    adopted by many research groups, meaning breaking changes impact the broader
    community; they can be the basis of published results, so changing the source
    code might invalidate the relationship between ongoing work and published work;
    and they can be brittle, as the incentive when writing software for research
    purposes is to achieve results worth publishing, rather than to produce a
    high-quality and maintainable codebase.\footnote{As opposed to a commercial
      software engineering team's incentive, which is typically to produce software
      which they can continue to produce in the future with minimal overhead imposed
    by code quality.} For these reasons, scientific codebases have special
    requirements which discourage direct modification, particularly for
    different use-cases. Aspect-oriented programming allows updates to codebases to be
    written without direct modification to the source code. In the case of
    research software specifically, this has been shown to achieve positive
    results in previous case studies~\cite{ionescu2009aspect}.

  \item[Advice as Tools for Instrumenting Scientific Codebases] Models of a
    system are ideally concerned with the logic required to accurately
    model the system itself, and do not contain additional logic to instrument the
    model for the purposes of a particular experiment. This is desirable because it
    allows the model to be re-used for many experiments, as the instrumentation to
    make observations for a particular purpose are not woven throughout its logic.
    This also makes the codebase easier to read: a researcher interested in it
    must only read the logic required to implement a model, and doesn't
    also have to identify the parts of the program which don't model a system at
    all, but produce observations from a simulation based on the model. Finally,
    the separation of observational apparatus from a model mirrors the design of
    a traditional experiment, where observations are carefully made so as not to
    bias results. This potential use-case of aspect orientation has been
    suggested in the community~\cite{gulyas1999use}.

  \item[Advice as Hypothesised System Behaviour] \inline{Putting a pin in this
      --- this section's getting really long, and I don't think the introduction
      needs to be such a mammoth. This section in particular is pushing it a
    bit, 3 pages at time of writing I think.}
\end{description}





%  % === 231106 Planning on replacing this with the contributions section, which
%  % should explain each contribution and also explain how it relates to the
%  % high-level vision. Better than a separate section, then a contributions
%  % section too...
% \section{Achieving the High-Level Vision}
% \label{intro_achieving_the_thesis_aims}
% 
% 
% \subsection{Non-Blackbox Aspects}
% 
% Adaptations of existing research codebases will have to account for software
% which doesn't have useful join-points to weave onto, as models are written today
% without aspect orientation as a design goal. In addition, the requirements of
% future studies may not be apparent to the original implementers of a tool. As
% aspect-oriented programming frameworks offer join points which specify that
% aspects should be woven before, after, or around their target, any modifications
% which must happen \emph{within} a join point are impossible to represent with
% existing tooling. 
% 
% A method for weaving aspects \emph{into} their join points is therefore required
% to fulfil the requirements of aspect-oriented simulation \& model development.
% 
% 
% \subsection{Legible Aspect-Oriented Programs}
% 
% Software used in a research setting carries its own unique requirements. In
% particular, they must be easily peer-reviewed and verified by a community, and
% their maintenance is important for the purposes of a team's future work.
% If aspect-oriented programming were to negatively impact a codebase's
% legibility, this would run against the requirements of research software
% engineers. Yet, aspect-oriented programming's principle of obliviousness is
% argued to limit developers' comprehension of their
% codebases~\cite{Constantinides04aopconsidered}. An aspect-oriented approach to
% developing simulations and models should therefore accommodate program
% legibility, either in framework design or in some supporting tooling. 



\section{List of Contributions}
\label{intro_contributions}

% ALL STUB S
Not sure whether these should be subsections but I'm leaning towards not. Here
they are in description-list style, because i think description lists with
paragraph-length definitions look really good in Ben's thesis. Maybe Tim has a
different take?

\begin{description}
  \item[PyDySoFu] Tooling for Aspect Orientation with within-style aspects and a
    new weaving technique which is familiar to other AOP designs but makes the
    intended use of a package a little clearer, particularly when used in the
    intended way (i.e. in the main entrypoint to an experimental codebase).
  \item[RPGLite Gameplay Data] provides an implementation of mobile game design,
    RPGLite, and published a dataset produced by hundreds of players'
    interactions with the game \inline{How many people played RPGLite exactly?}
  \item[Demonstration of aspect-oriented model enhancement] Aspects are used to
    augment a model of gameplay and demonstrate that aspect-oriented
    enhancements can change a model and make it ``realistic''. We show that a
    model can be augmented to synthesise data with the properties of the
    real-world dataset. Depending on the discussion currently being had with
    Tim, we might not claim this as a contribution --- maybe it's just too weak
    (I think it might be, but would like a good way to sell the
    prior-distribution model).
  \item[Tuning a Complex Aspect-Oriented enhancement for hyper-realistic
    modelling] A model of learning is presented which can be tuned to specific
    real-world players in order to maximise its realism. Properties of learning
    of specific real-world players are identified by fitting a learning model to
    their RPGLite gameplay data.
  \item[Investigation into Aspect Portability] We investigate whether
    aspect-oriented changes to a model can be ported from one system to another,
    taking advantage of aspect-orientation's modular nature. \inline{Revisit.
      Does this work? Experiments need re-running. Should work for
      prior-distribution, maybe not for models of learning\ldots{}}
\end{description}


\section{Terms \& definitions}\label{sec:glossary}

\inline{Complete the glossary in \cref{sec:glossary}.}

\inline{Decide whether terms like BPMN, simulation \& modelling, etc also belong
in the glossary\cref{sec:glossary}.}

\inline{Dejice's thesis has a similar glossary; how did he structure it? Wasn't
  an appendix. Was it a section of the intro or something else?}

\begin{description}
  \item[Aspect-Oriented Programming]
  \item[Cross-Cutting Concern]
    \item[Aspect] 
    \item[Advice] 
    \item[Joinpoint]
    \item[Pointcut] 
    \item[Weaving] 
    \item[AspectJ] The original aspect orientation framework, with language
    extensions to describe pointcuts and aspects.
    \item[Target] The procedure an aspect is applied to via a join point, to
    affect advice.
    \item[PyDySoFu] 
    \item[]
\end{description}


\section{Thesis Structure}

The rest of the thesis is structured as follows. \inline{Write the thesis structure!}
