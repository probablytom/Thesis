% Typeset a scientific notation number of the form 1.23e-5 => 1.23 x 10^-5
\newcommand{\scientific}[1]{\num[{scientific-notation = true, separate-uncertainty = true}]{#1}}

% Old macros for ensuring all secs and subsecs had labels.
% Don't use these! Kept because I think they're still used _somewhere_...
\newcommand{\labelledsec}[2]{\section{#1}\label{sec:#2}}
\newcommand{\labelledsubsec}[2]{\subsection{#1}\label{subsec:#2}}

% a macro to stop me switching between hyphenation/spacing/neither
\newcommand{\sociotechnical}{socio-technical\xspace}
\newcommand{\Sociotechnical}{Socio-technical\xspace}
\newcommand{\SocioTechnical}{Socio-Technical\xspace}

% macros for consistency with hypenation for aspect orientation and
% aspect-oriented programming.
\newcommand{\aspectorientation}{aspect orientation\xspace}
\newcommand{\Aspectorientation}{Aspect orientation\xspace}
\newcommand{\AspectOrientation}{Aspect Orientation\xspace}
\newcommand{\aspectoriented}{aspect-oriented\xspace}
\newcommand{\Aspectoriented}{Aspect-oriented\xspace}
\newcommand{\AspectOriented}{Aspect-Oriented\xspace}
\newcommand{\aop}{\aspectoriented programming\xspace}
\newcommand{\Aop}{\Aspectoriented programming\xspace}
\newcommand{\AOP}{\AspectOriented Programming\xspace}

% To prevent me absent-mindedly shortening pydysofu to pdsf, the macro is handy.
\newcommand{\pdsf}{PyDySoFu\xspace}

\newcommand{\theatreag}{Theatre\_AG}

% Utility for fuzz
\newcommand{\atfuzz}{\lstinline{@fuzz}\xspace}

% Make sure final section is same as title
% \newcommand{\thesistitle}{Aspectually Augmented Models}
% \newcommand{\thesistitle}{Aspects as Units of Model Change}
% \newcommand{\thesistitle}{Simulation \& Modelling using \aop{}}
% \newcommand{\thesistitle}{\AspectOriented{} Simulations \& Models}
% \newcommand{\thesistitle}{\AspectOriented Models for Simulation}
% \newcommand{\thesistitle}{\AspectOriented Simulation Models}
% \newcommand{\thesistitle}{Composing Simulation Models using Aspects}
% \newcommand{\thesistitle}{On the Composition of Simulation Models}
\newcommand{\thesistitle}{\AspectOriented Modelling}


% If a paragraph hasn't been written yet, it's annotated with EXPAND as a note
% to come back and finish it later.
\newcommand{\expand}{\inline{EXPAND}\xspace}

%% Research questions
% \newcommand{\rqone}{Is an \aspectorientation{} tool which is appropriate for use in simulation \& modelling feasible to design?}
\newcommand{\rqone}{Can models of systems more accurately reflect their subjects by weaving aspects which represent improvements?}
\newcommand{\rqtwo}{Can advice be used to faithfully introduce behaviours or parameters into a model which were not originally present in it?}
\newcommand{\rqthree}{Can advice be used as a portable module, such that
\aspectoriented{} improvements to one model can be woven into another without
loss of performance?}

% For text within circles
\newcommand*\circled[1]{\tikz[baseline=(char.base)]{
            \node[shape=circle,draw,inner sep=2pt] (char) {#1};}}
            

\ExplSyntaxOn
    \cs_new_eq:NN \calc \fp_eval:n
\ExplSyntaxOff
\newcommand*{\roundToDPs}[2][]{\num[%
  round-mode=places,% Round output to specified number of places
  round-precision=3,% Round-precision is 3
  output-decimal-marker={.},% Use . as decimal marker
  #1% Other options
  ]{\calc{#2}}}

% To clarify that I go by both Tom and William in the references, in response to
% corrections
\newcommand{\clarifymynames}{\emph{NB: The authors ``Tom Wallis'' and ``William Wallis'' are the same person.}}