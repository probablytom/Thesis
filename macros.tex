% Typeset a scientific notation number of the form 1.23e-5 => 1.23 x 10^-5
\newcommand{\scientific}[1]{\num[{scientific-notation = true, separate-uncertainty = true}]{#1}}

% Old macros for ensuring all secs and subsecs had labels.
% Don't use these! Kept because I think they're still used _somewhere_...
\newcommand{\labelledsec}[2]{\section{#1}\label{sec:#2}}
\newcommand{\labelledsubsec}[2]{\subsection{#1}\label{subsec:#2}}

% a macro to stop me switching between hyphenation/spacing/neither
\newcommand{\sociotechnical}{socio-technical\xspace}
\newcommand{\Sociotechnical}{Socio-technical\xspace}
\newcommand{\SocioTechnical}{Socio-Technical\xspace}

% macros for consistency with hypenation for aspect orientation and
% aspect-oriented programming.
\newcommand{\aspectorientation}{aspect orientation\xspace}
\newcommand{\Aspectorientation}{Aspect orientation\xspace}
\newcommand{\AspectOrientation}{Aspect Orientation\xspace}
\newcommand{\aspectoriented}{aspect-oriented\xspace}
\newcommand{\Aspectoriented}{Aspect-oriented\xspace}
\newcommand{\AspectOriented}{Aspect-Oriented\xspace}
\newcommand{\aop}{\aspectoriented programming\xspace}
\newcommand{\Aop}{\Aspectoriented programming\xspace}
\newcommand{\AOP}{\AspectOriented Programming\xspace}

% To prevent me absent-mindedly shortening pydysofu to pdsf, the macro is handy.
\newcommand{\pdsf}{PyDySoFu\xspace}

% Utility for fuzz
\newcommand{\atfuzz}{\lstinline{@fuzz}\xspace}

% If a paragraph hasn't been written yet, it's annotated with EXPAND as a note
% to come back and finish it later.
\newcommand{\expand}{\inline{EXPAND}\xspace}

%% Research questions
\newcommand{\rqone}{Is an \aspectorientation{} tool which is appropriate for use in simulation \& modelling feasible to design?}
\newcommand{\rqtwo}{Can models of systems more accurately reflect their subjects through advice-based improvements?}
\newcommand{\rqthree}{Can advice be used to accurately introduce behaviours or parameters into a model which were not originally present in it?}
\newcommand{\rqfour}{Can advice be used as a portable module, such that \aspectoriented{} improvements to one model can be woven into another without loss of performance?}