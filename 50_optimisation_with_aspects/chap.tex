\chapter{Simulation Optimisation with Aspect Orientation}\label{chap:exp1_simulation_optimisation}

With a game deployed to experiment participants and a dataset of empirical play
collected, it was possible to determine optimal play in any game state. This
entirely separate body of work is documented in another student's PhD
thesis\inline{Cite William's PhD thesis}. This dataset leads to
further research. If we understand how players \emph{should} play, and we have
data to indicate how they \emph{do} play, we can investigate how real-world
players might be modelled. 

\labelledsec{Aims}{aop_simulation_optimisation_aims}

Aspect orientation's use in previous simulation and modelling efforts have
typically focused on the use of aspects to compose model or simulation
details\inline{There must be tons of good citations for aspects being used to
compose together a simulation / model}. Critics of aspect orientation note that
the act of process composition makes visually understanding codebases difficult,
and so ensuring that a simulation properly models real-world behaviour is made
trickier with the introduction of aspect orientation. However, aspect
orientation might instead be used to \emph{augment} an existing model, by
rethinking what aspects are used to represent.

An alternative use of aspects would be to first build a non-aspect-oriented
model of \emph{expected} behaviour, and separately build aspects which describe
deviations from this. For example, one might more realistically simulate safety
procedures by first producing an idealised, ``naive'' model of what employees
are expected to do, and separately model alterations to prescribed behaviour as
an employee's boredom, expectation that checks and balances are unneccesary
wastes of their time, and so on --- effectively, separating out models of
degraded modes\cite{johnson2007degradedmodes}.

Previous research on the use of aspect orientation to model degraded modes
adopted the traditionally claimed benefit of aspect orientation: separation of
cross-cutting concerns, allowing for a greater reusability of
codebases\cite{wallis2018caise}. A repository of cross-cutting concerns in
socio-technical simulation such as boredom was developed as a library to be
applied to any future models\cite{fuzzimoss_repo}. However, aspects used in
simulation have no intrinsic need to represent concerns that are cross-cutting.
Indeed, whether they can be accurately used to represent cross-cutting concerns
in simulation is the topic addressed in \inline{Add a cross-reference to the
chapter on cross-cutting concern simulation accuracy when it exists}. Aspects
might instead be used to represent \emph{amendments to processes} which deviate
from an expected norm, in this case represented by the idealised model aspects
are applied to.

To more concretely relate this to the experiment at hand: play of RPGLite can be
modelled as players matchmaking, picking characters, and then mutually taking
turns until one player's characters are entirely expired. Once a player's
characters are dead, new matches can be made. This can continue indefinitely.
Lacking a heuristic to select next moves or characters, players might be
modelled as picking random moves. However, heuristics for move selection can be
added to the naive model of play by way of augmenting the processes already
defined through aspects. This approach can be of significant utility in both
modelling player behaviour and accurately modelling different players:

\begin{enumerate}
    \item Different players might use their own unique heuristics to model play.
    Each player's behaviour is therefore well described by separating what play
    ``looks like'' to what makes a given player play differently to their peers.
    \item Different players might lean more heavily on different heuristics, or
    mixes thereof. Play might be characterised by reliance on experience, on
    recent games, on knowledge of an opponent, and so on; these different
    variables can be expected to be weighted differently by each player, adding
    complexity to the code which models this individualised play.
    \item A modeller might discover a new idea for a heuristic long after
    developing an original concept for a model. The easiest methods for amending
    the original model should require the least rewriting of original code. Due
    to the impact of \pointno{2}, ideal architectures for an approach such as
    this should require these heuristics to be defined entirely separately to
    the base model.
\end{enumerate}

Considering \pointno{1}, \pointno{2}, and \pointno{3}, architectures and
paradigms which enable separation of concerns are well-suited to defining
alternative approaches to play. Some architectural approaches such as mixins or
plugin design patterns might support this structure well, but they typically
rely on language features (in the case of mixins) or knowledge of software
engineering (in the case of design patterns). Aspect orientation is typically
provided to developers as a framework or runtime in a language (such as
AspectJ\cite{aspectj_intro} or PROSE\cite{popovici2002PROSE}) and can require
minimal architectural understanding to use: concepts are simple, and the effort
of composition is alleviated by the supporting framework or runtime.

The approach makes little use of aspect orientation's significant contribution
--- cross-cutting concerns --- as whether behaviour cross-cuts different parts
of a codebase is not of interest in this use case. Instead, aspect orientation
is treated as a composition mechanism with a reasonably low degree of technical
knowledge required.

\labelledsubsec{PyDySoFu Suitability}{optimisation_with_aspects_usingpdsf}
Some aspect orientation frameworks do not adequately achieve this requirement.
For example, the most influential framework, AspectJ, requires the use of
language extensions to define integrate aspect
orientation\cite{AspectJLanguageAndTools}, and similar additional complexity is
added in seemingly every alternative framework, through the use of bespoke
virtual machines, compilers, translators, or
languages\cite{rajan2006nu_towardsAO_invocation,popovici2003JITaspects,AspectCplusplusDesignImpl,baker2002maya}.

PyDySoFu, however, requires very little additional knowledge to use. Its design
prioritises simplicity and a shallow learning curve that makes its adoption by
researchers without a software engineering background feasible: \inline{maybe
cut this list of reasons PyDySoFu is fantastic...}

\begin{itemize}
    \item PyDySoFu is implemented as a pure-python library, meaning that it can
    be installed through Python's package manager (pip) and imported like any
    other Python library. No additional supporting infrastructure is required.
    \item Aspects in PyDySoFu are simple functions which take as arguments
    whichever pieces of information are pertinent for the function's use as an
    aspect\footnote{For example, an ``encore'' aspect which is woven after a
    target procedure returns will be provided that target's return value.}.
    \item To weave a PyDySoFu aspect requires only a method call, which returns a
    \lstinline{callable} which unweaves that aspect.
    \item Defining PyDySoFu pointcuts requires only a regular expression
    matching a method name. This can apply to a wide range of join points if
    required, but where method names are provided directly, the join point is
    made clear.
    \item Additional clarity over where aspects \emph{can} be woven is
    introduced by PyDySoFu's transparent weaving of aspect hooks, mitigating
    some of aspect orientation's most prominent criticisms.
\end{itemize}

PyDySoFu therefore satisfies the requirements of this work well: it offers
composition of procedures outside of the scope of an original codebase, makes
what is being composed where clear to a programmer, and makes no significant
changes to Python as a language (thereby requiring users to specialise in fewer
tools). 


\labelledsubsec{Proposed Experiment}{optimisation_with_aspects_experiment}

Aspect orientation's use as a composition tool for model components makes sense
in principle, but it is unclear whether the addition of behaviours to a naive
model would make the model more ``realistic''. It is unclear whether the changes
made would properly represent what might be empirically observed, and while
PyDySoFu's design makes understanding \emph{what} is being composed simpler than
other aspect orientation frameworks, a composed model under this paradigm is
still split across multiple areas of a codebase, making a visual assessment of
whether a model accurately reflects the intended behaviour impractical. 

We can confirm whether aspects can realistically represent changes to a naive
understanding of the real world by comparing their output against empirical
data. For example, if a such a model of behaviour in a system outputs data which
correlates poorly against empirically collected data, a change to that system
would make it more realistic if it improved this correlation, and could be said
to be realistic if the generated data appeared sufficiently ``close'' to the
empirical dataset --- which here means that the correlation between the two is
of statistical significance. Such a change can be aspect-oriented. Therefore, we
can see the application of aspects as the application of packages of potential
improvements to a base model, which can be verified by way of comparison to
known-good datasets.

This is the basis of the experiments in this thesis.

With datasets collected empirically on RPGLite's play, we can build a naive
model of play and aspects to apply that should realistically model data from
players. This can be used to answer the question:

\begin{researchquestion}
    Can aspect-oriented models be said to exhibit realism?
\end{researchquestion}

To answer this question, we will produce a model of play, and develop aspects
which encapsulate different play styles so as to compare the aspect-augmented
datasets and naive datasets against the empirically sourced data. The following
subsections detail the naive model developed and aspects applied to this model
respectively.


\labelledsec{Naive Model}{optimisation_with_aspects_naivemodel}

A naive model of play was developed by separating each stage of the actions
taken by players in the client-side app, and separating them into individual
procedures. To facilitate the retrieval of most information about a simulation
in an applied aspect, the model was written so as to contain the entire
simulation state as mutable function arguments. The model was written as a
workflow, and state of workflow execution was separated into three components:
the actor a function invocation (or ``step'') repesents activity from; the
context of that step in the execution of a workflow; and the context of that
workflow's execution in a broader environment. Incidentally, we found this
structure to allow a flexible and natural implementation of a procedural
simulation, which should translate easily to existing simulation frameworks such
as SimPy\cite{simpy_intro}:

\begin{description}
  \item[Actor ---] allows the function to identify the actor performing the activity
    defined by the function. This argument is any object uniquely identifying an
    actor.
  \item[Context ---] allows the function to determine details of the current
    thread of work being undertaken by the actor. This is necessary because in
    some simulations, the same actor might pause and resume multiple occurrences
    of the same activity --- for example, they might concurrently play three
    different matches in RPGLite. As a result, it is necessary to understand the
    context of the action being performed by the actor in question. This
    argument can be any object uniquely identifying the context of a piece of
    work, but should be mutable (such as a class or dictionary-like object) to
    permit the communication of information across invocations of different
    action-representing functions.
  \item[Environment ---] an actor's actions are often determined by the global
    environment they act within. There may be ancillary details to the
    actor's actions and the context of their particular thread of work which
    they are undertaken within which are used to determine behaviour, such
    as a landscape they traverse or other actors they might choose to
    interact with. Because all actors share access to a global environment,
    this also provides a message passing space, or a space where actors can
    set values and flags other actors might look to, should those details be
    more general than their specific thread of work at a given point in time.
    \footnote{This is different to environments in some other simulation frameworks, such
    as SimPy\cite{simpy_documentation}, where the environment controls scheduling and
    execution: this structure imposes no constraints such as models of time, and
    anticipates that any such functionality should be implemented by the
    programmer. However, an environment such as SimPy's might satisfy a programmer's needs
    when using this particular pattern.}
\end{description}

Each simulation step suitable as a join point receives these three arguments at
a minimum. Aspects applied to these therefore have access to the entire state of
the simulation.

The naive model of RPGLite follows a simple workflow mimicking player
interaction with the client-side application used by real-world players.
Graphically, it would be represented as a flowchart like so:



\inline{TODO FINISH WRITING ABOUT THE NAIVE MODEL'S STEPS}


\labelledsec{Aspects Applied}{optimisation_with_aspects_aspectsdeveloped}

Players can be expected to select stronger characters as they understand
gameplay more. Aspects were therefore developed to track player experience with
characters. As players became more familiar with the characters they play, we
can hypothesise that they will better understand how to play with those
characters, and more often select characters they have success with. 




\inline{Write about \emph{both} aspects so as to answer the hypothesis : Can
aspects be used to generate models of alternative behaviours?}


\labelledsec{Experimental Design}{optimisation_with_aspects_experimental_design}

Having discussed the aspects to apply above, this section describes how we'll
apply those aspects to compare simulations under variation to controlled
simulations, and how such an experiment answers our research question.

This likely won't need any subsections: we've already laid out the groundwork
above as to what's involved technically, so this is statistical, explanation of
our control, our model under test, and how they'll be compared in a rigorous and
scientifically appropriate manner.

\labelledsec{Experimental Results}{optimisation_with_aspects_experimental_results}

Presentation of the results owing from the experiment as described in
\cref{sec:optimisation_with_aspects_experimental_design}, and evaluation of the
research question with respect to these findings.

Again, no subsec likely required here.

\labelledsec{Discussion}{optimisation_with_aspects_discussion}

A closing discussion on what we found, and how the research question was
answered.