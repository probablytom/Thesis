\chapter{Rewriting PyDySoFu}\label{chap:pdsf_rewrite}

\inline{Check the most up-to-date pdsf implementation --- is it actually on that
backup drive?}

The work undertaken in this thesis required in improved implementation of old
tooling. When previously used, PyDySoFu was a proof of concept which could
feasibly produce scientific simulations, but was implemented in a manner which
was not optimised for speed (making it a burden for large simulations), lacked
granularity in the application of its aspect hooks (hooks could only be applied
to entire classes), and most importantly, did not work with Python3 (Python2
support officially ended during this PhD). 

This chapter briefly outlines the new implementation of PyDySoFu, discusses
improvements made to design and performance, and explains some contributions
made to the design of aspect orientation frameworks which addresses some core
issues raised with the paradigm. \inline{Consider adding references to the
sections through this PDSF chapter, depending on how beefy it becomes\ldots{}}


\labelledsec{Requirements for Change}{pdsf3requirements}

As time wore on\revnote{Too informal} with PyDySoFu's original implementation, it became increasingly
clear that a rewrite was required\revnote{sloppily put}. PyDySoFu grew out of an undergraduate
project, and accrued technical debt as a result of being written under extreme
time constraints with little experience. On revisiting, and on reflecting on
other aspect orientation frameworks (as discussed in
\cref{sec:dynamic_aop_review} and \cite{dynamicAOchitchyan}) and the use
previously found for PyDySoFu (see \cite{wallis2018caise, wallis2018genetic}),
it was clear that there were a series of improvements which could be made in the
process of rewriting the tool:

\begin{itemize}
    \item Before this body of work, \pdsf made use of techniques for applying
    aspect hooks which did not translate to the changes Python 3 made to its
    object model. In particular, Python 3 changed its underlying object model,
    using a read-only wrapper class that made the replacement of
    \lstinline{__getattribute__} impossible via the previous route.
    \item PyDySoFu's original implementation made no serious accomodations for
    efficiency. It could be seen as the ``total weaving'' described by
    \citeauthor{dynamicAOchitchyan} in~\cite{dynamicAOchitchyan}, and it was not
    possible to provide additional options to ensure that aspects could be as
    efficiently woven as possible at runtime given a particular use-case\revnote{Could be less sloppy}.
    \item The original \pdsf implementation wove onto a \emph{class}, meaning
    that even properties of the class which were not considered join points were
    still affected by the weaving, even if in a minor way\revnote{Could be less sloppy}. Because
    \lstinline{__getattribute__} retrieves all attributes including special
    builtin attributes and non-callable attributes, these are also returned via
    the modified implementation of \lstinline{__getattribute__}, incurring an
    overhead, albeit small, for all attribute resolutions instead of a desired
    subset.
    \item The original \pdsf implementation made no accommodations for scenarios
    where fuzzing of source code was applied in a ``static'' manner. That is to
    say, where a deterministic modification to source is woven as advice,
    instead of dynamically modifying source code, the same modification would
    still be made every time the target attribute was executed, unless caching
    of results was specifically managed by the aspect applying the change. No
    optimisations were made pertaining to this, but compilation and abstract
    syntax tree editing have the potential to be PyDySoFu's most expensive
    operations.
    \item Unlike other aspect orientation frameworks such as
    AspectJ~\cite{aspectj_intro}, join points could not be specified by pattern.
    Instead, each individual join point must be supplied as a Python object.
    This means that, while the target attributes are still oblivious to the
    advice applied to them, the application of that advice could not be written
    obliviously.
\end{itemize}

As a large number of requirements were left unfulfilled by the original
implementation of PyDySoFu, a new implementation satisfying them was deemed
necessary.


\labelledsec{Python3 Specific Implementation}{pdsf3python3}

Replacing \lstinline{__getattribute__} on the class of a targeted method was no
longer viable in Python 3. A replacement method therefore had to be found. For
clarity: replacing \lstinline{__getattribute__} allowed for hooks to be woven
(at runtime) into likely future targets for advice. These hooks would then
discover and manage the execution of advice around each target. Because advice
can be run before and around a target, and dynamic weaving implies that advice
could be supplied or removed at any time, we look to intercept the calling of
any target, and manage advice immediately before execution. So, the task at
hand is to find a method of attaching additional work to the calling of any
potential target, before that target is executed. We refer to code woven
around a target which manages applied advice as \emph{aspect hooks}.


\labelledsubsec{Abandoned techniques}{pdsf3badweaving}

Rather than ``monkey-patching''\footnote{Monkey-patching is the practice of making on-the-fly changes to object
behaviours / definitions by taking advantage of language properties such as
flexible object structures. Common examples of these structures are objects literally being maps from string
attribute / method names to the associated underlying value, as in Python and Javascript. Monkey-patching
makes use of these simple structures by replacing
values such as the function object mapped to by the original function's name in
the dictionary, effectively changing its behaviour. This is the method by which PyDySoFu originally replaced
\lstinline{_getattribute__} on a class object.} a new version of
\lstinline{__getattribute__} with hooks for weaving aspects, the rewritten
method could be patched to the object itself at a deeper level than used in the
original PyDySoFu implementation. This would make use of Python's
\lstinline{ctypes} api to patch the underlying object. Similar work has been
done in the python community in a project called
ForbiddenFruit~\cite{forbiddenfruit_repo}. Efforts were made to add the required
functionality to ForbiddenFruit --- patching \lstinline{__getattribute__}
directly on the object, or ``cursing'' it in ForbiddenFruit jargon --- but this
was abandoned as the underlying mechanism is particularly unsafe, Python API
changes could render the work unusable in future versions easily, and the
implementation would only work with particular implementations of Python (for
\lstinline{ctypes} to exist, the Python implementation must be written in C).
Community patches existed for cursing \lstinline{__getattribute__} which did not
work, and attempts proved challenging, indicating that this would also be
complicated to maintain over time. There are also efficiency concerns with this
technique depending on its use: weaving advice around a function would mean
monkey-patching the built-in class of functions, which would incur an overhead
from running aspect hooks on \emph{every function call}.

Another approach involved making use of existing Python functionality for
interrupting method calls. As PyDySoFu wraps method calls at execution time,
what is required is to add functionality to the beginning and end of the
execution of a method. Python has built-in functionality for implementing
debuggers, profilers, and similar development tools, which provides exactly this
functionality, as debuggers must be able to --- at any point during execution
marked as a breakpoint --- pause a running program and inspect call stacks, the
values of variables, and so on. As a result, the method \lstinline{settrace()}
allows a developer to specify a hook providing additional functionality to a
program. Making use of this also has issues in our case. Most significantly,
\lstinline{settrace()} catches myriad events in the Python interpreter which
PyDySoFu may not need to concern itself with, incurring significant overhead. In
addition, use of the function overrides previous calls to it, meaning that any
debuggers used by a user of PyDySoFu would be replaced with PyDySoFu's
functionality, which was deemed untenable. However, it is worth noting that the
technique could work in theory, and if future versions of Python allow for
multiple trace handlers being managed by \lstinline{settrace()}, this could
provide an interesting approach when implementing future dynamic aspect
orientation frameworks.

\labelledsubsec{A viable technique: import hooks}{pdsf3importhookdiscussion}

A final available technique was to continue to monkey-patch hooks to discover
and weave aspects, via an alternative method which did not make use of
\lstinline{__getattribute__}. This approach would change the use of PyDySoFu
slightly to make a compromise between performance and obliviousness of aspect
application: when importing a module targeted for aspect weaving, methods
which are potential weaving targets are invisibly monkey-patched with a wrapper
method with a reference to the original\footnote{Necessary to run the originally
targeted method.} and hooks to detect and run dynamically supplied advice.

% \todo{Write an explanation of pdsf3's import hook mechanism}
An important note for discussing the implementation of \pdsf is that almost all
Python functionality operates by use of its ``magic methods''. ``Magic
methods'' are methods beginning and ending with two \lstinline{_} characters.
The Python language documentation specifies sets of magic methods and their
required function signatures which are used internally to implement
functionality; for example, any object with the method \lstinline{__eq__()}
defined can be compared against using the \lstinline{==} operator, and the
\lstinline{__eq__()} magic method is run to determine the outcome of the
operator. Magic methods support more than operator overloading. For example,
anything which defines \lstinline{__len__()} and \lstinline{__getitem__()} is
treated as an immutable container, and adding \lstinline{__setitem__()} and
\lstinline{__delitem__()} makes that container mutable. Any class defining
\lstinline{__call__()} is treated as a callable object (not unlike a function).
More can be found in Python's documentation\cite{py3docs}, although more
focused guides exist in the Python community~\cite{magicmethodguide}.
For the purposes of implementing import-based aspect hook weaving,
magic methods have the affect of making the language ideal, particularly as an environment to implement dynamic
aspect orientation. Python's functionality for importing modules is managed by
\lstinline{builtins.__import__}, which receives module names as strings and
handles package resolution. By monkey-patching the import system, modules can be
modified during the process of importing.

Monkey-patching \lstinline{builtins.__import__} is as simple as replacing the
function object with a new one, which has the effect of changing the behaviour
of Python's \lstinline{import} keyword: because all Python functionality relies
on magic methods implicitly, its behaviour can be altered in this way. However,
our intent is not necessarily to manipulate \emph{all} modules, but a subset of
imports specified by a modeller as suitable for manipulation. If all imported
modules were affected, this would include all invocations of \lstinline{import},
including those made recursively by package implementations, for example.
Therefore, it is important to have a mechanism to enable and disable the weaving
of aspect hooks on each import (effectively, to enable and disable PyDySoFu's
modified import logic). 

\inline{Decide whether this needs to be more thoroughly broken up /
structured\ldots{}}

\inline{Include a discussion of \emph{what} gets hooks added using this method\ldots{}}

This can be done through another use of magic methods in a manner which also
makes clear to a modeller exactly where aspect hooks are being applied: making
use of Python's \lstinline{with} keyword. 

\labelledsubsec{Implementing import hooks}{pdsf3implementingimporthooks}

We are interested in manipulating \lstinline{builtins.__import__} only when
imports are made which introduce modules containing prospective join points;
a developer might import many modules, which do not all require aspect hooks to 
be introduced. We enable this new import
behaviour with a syntax of the form shown in \cref{fig:simple_aspect_hook_weaving_example},
which weaves aspect hooks into all functions and (non-builtin)
class methods within the \lstinline{mymodule} module object added to the local
namespace of the importing stack\footnote{Python's use of the stack namespace in
its importing system means that careless re-importing a module can lead to
multiple copies of it in different function stacks, meaning that the same name
resolution (such as resolving a class by its name in a module) might, after
applying aspect hooks in PyDySoFu, change the behaviour of procedures depending
on where they are called. Scenarios where this might arise are deemed unlikely
enough that the risk of this design decision becoming troublesome are considered
negligible. Still, it would be remiss not to make note of the fact.}. Less
formally: importing \lstinline{with AspectHooks()} applies aspect hooks to all
potential targets of advice in the \lstinline{mymodule} package. The behaviour
of Python's \lstinline{with} keyword is defined by more magic methods: any
object with \lstinline{__enter__()} and \lstinline{__exit__()} defined can be
used here, where \lstinline{__enter__()} is run at the beginning of the enclosed
block, and \lstinline{__exit__()} when leaving the block.

\begin{figure}\label{fig:simple_aspect_hook_weaving_example}
\begin{lstlisting}
with AspectHooks():
    import mymodule
\end{lstlisting}
\caption{Example of importing a module, \lstinline{mymodule}, using \pdsf{}'s import hook design.}
\end{figure}


PyDySoFu caches the original \lstinline{builtins.__import__} object in an
instance of the class, and replaces it with \lstinline{AspectHooks.__import__},
in its \lstinline{__enter__()} method. This is reversed by replacing
\lstinline{builtins.__import__()} with the cached object in its
\lstinline{__exit__()} function. \lstinline{__enter__()} and \lstinline{__exit__()} are the
magic methods corresponding to entering and exiting Python's \lstinline{with} blocks, meaning
that this technique modifies Python's importing only when imports are made
within a block such as that in \cref{fig:simple_aspect_hook_weaving_example}.
The resulting implementation for weaving aspect
hooks is uncomplicated, as can be seen in \cref{fig:aspecthooksmagicmethodswith}.

\begin{figure}[t]
\begin{lstlisting}
class AspectHooks:
    def __enter__(self, *args, **kwargs):
        self.old_import = __import__
        import builtins
        builtins.__import__ = self.__import__

    def __import__(self, *args, **kwargs):
        # ...replacement import logic for performing weaving...

    def __exit__(self, *args, **kwargs):
        builtins.__import__ = self.old_import
\end{lstlisting}
    
    \caption{Magic methods used to enable the \lstinline{with} keyword usage for
    PyDySoFu}
    \label{fig:aspecthooksmagicmethodswith}
\end{figure}

\labelledsubsec{Strengths and weaknesses of import hooks}{pdsf3importhooklimitations}

As a technique for weaving aspect hooks, this new method provides multiple
benefits. Application of aspect hooks is straightforward from the perspective of
a modeller using PyDySoFu, whose code clearly applies aspect hooks and does so
in a legible way for future maintainers. Import hooks' explicit application of hooks to modules
means that places where aspect hooks might be applied equally explicit, and thereby implements an
interpretation of aspect-oriented programming's principle of ``obliviousness'' moderated by clarity.
While join points are oblivious to potential aspect application, callers of join points can expect
to be more aware, and signs for codebase maintainers are left directly within source code.
Aspect hooks can be applied to specific modules
or every module depending on the use of the supplied \lstinline{with} statement,
allowing for total weaving or actual hook weaving~\cite{dynamicAOchitchyan}
depending on their preferences. Further, performance is optimised in
comparison to the previous implementation of PyDySoFu, as hooks are weave-able
at a more granular level (on the level of procedures such as functions and
methods, rather than all attributes of a class).

There are also caveats of this approach that are necessary to address.
As aspect hooks are woven in the new implementation of PyDySoFu via Python's
import functionality, any procedure not imported from a module cannot have
aspect hooks attached.\inline{Consider adding local namespace weaving to pdsf3:
should be easy to implement as a cheeky little monkey-patch\ldots{}} However, as
aspect orientation is primarily concerned with a separation-of-concerns approach
to software architecture, targets are expected to exist in other modules, and we
do not consider this to be a significant limitation.

A more significant limitation of the import hook approach is that the object
with aspect hooks woven exists in the namespace of the module \emph{importing}
the join point. In other words, this method makes it impossible for a module to
make use of aspect hooks that are woven in an unrelated piece of code.\revnote{is this true??? verify, I think not, if imported normally, but a 5 min test seemed to confirm. Putting it down to me being rusty with pdsf.}
We therefore have a ``semi-oblivious'' property to our aspect orientation approach:
targets of advice are unaware of any adaptations made, but \emph{any code making
use of those adaptations must be aware enough to at least apply aspect
hooks}.\footnote{Note that once aspect hooks are applied, advice can still be
supplied from anywhere in the codebase.}

In a manner of speaking, this can be considered to alleviate some concerns with
aspect orientation as a paradigm. Aspect Orientation is criticised for making
reasoning about programs more
difficult~\cite{przybylek2010wrong,Constantinides04aopconsidered,steimann06paradoxical}.
One cause of this is that aspects separate logic from where it is run;
\citeauthor{Constantinides04aopconsidered}~'s comparison with the jokingly
proposed \lstinline{come from}
statement~\cite{clark73comefrom,Constantinides04aopconsidered} is a reminder
that it can be effectively impossible to understand how a program will execute
if the path of execution is not at least linear or clearly decipherable from
source code. Aspect orientation as a paradigm violates this
linearity by design. However, import hooks as implemented in
\cref{fig:aspecthooksmagicmethodswith} allow for aspect-oriented code which can be interpreted in
one of only two ways:

\begin{enumerate}
    \item Looking at the original implementation of a procedure, its intended
    execution is clear. A programmer can make use of this directly and it is
    guaranteed to behave as expected.
    \item Any program making use of a procedure imported from a module will see,
    when the procedure is imported, whether it has had aspect hooks applied. In
    this case its behaviour is unknown --- falling prey to the design flaws
    discussed in the aspect orientation
    literature~\cref{Constantinides04aopconsidered,steimann06paradoxical,przybylek2010wrong}
    --- but this unpredictability is highlighted to the
    programmer.
\end{enumerate}

%======== originally a footnote at the end of the second item above
% \footnote{It is worth noting that a third case technically
%     exists, where a procedure is imported from a module which imports that
%     procedure from another module. If the latter module contains the
%     implementation and the former applies aspect hooks when it imports, then any
%     program making use of the former module will be importing a procedure with
%     aspect hooks applied implicitly. However, these situations are still visible
%     through simple inspection of these chained imports, where other aspect
%     orientation frameworks might apply an aspect to any join point at any time,
%     without this being obviously discoverable by a programer.}
    
As a result, while import hooks are somewhat limited in that they are applied
specifically to imported code and break the traditional AOP concept of
obliviousness in at least a weak manner, these two facts combine to arguably fix
a latent issue in the design of the aspect oriented paradigm. The original
PyDySoFu implementation was able to modify any procedure in a more traditional,
oblivious manner. While this new implementation is clearly more limited as a
result, we consider these limitations an overall benefit to the design of the
tool, and a contribution to aspect orientation framework design.


% \subsection{Improvements to supported aspects}
\labelledsubsec{Weaving process}{pdsf3_weaving_process}

\inline{Describe the improved process of weaving in PDSF3}

\inline{Find the newest copy of PDSF3, make a repo for it, cite the new impl}

% \labelledsec{}

\section{Discussion}

The new implementation of PyDySoFu makes a few contributions, particularly in
comparison to the previous version:

\begin{itemize}
    \item Its new technique of weaving aspect hooks on import, making use of
    Python's \lstinline{with} keyword, improves aspect orientation framework
    design by trading a degree of obliviousness for clarity
    \item Aspect hooks can be applied with more precision than the previous
    implementation of PyDySoFu, meaning:
        \begin{itemize}
        \item Users of the framework can better delineate between total and
        actual hook weaving
        \item Unnecessary overheads from checking dynamically applied aspects at
        each join point are reduced.
        \end{itemize}
    \item 
\end{itemize}

Despite this, there is room for improvement in the design of the framework:

\begin{itemize}
    \item Caching of applied aspects to join points could be implemented. If
    between two invocations of a target no changes have been made to the applied
    aspects, a function object containing the composed aspects from earlier
    invocations should be run. This would permit runtime aspect weaving with
    less overhead, as searching for applied aspects need not be performed at
    every target invocation. Targets should have ``changedness'' flags which are
    set every time an aspect is applied or removed from it.
    \item Our intended use case for aspect orientation for simulation \&
    modelling is in scientific codebases specifically; direct integration with
    the scientific package ecosystem (which is vibrant in Python's community)
    should be made. A good initial project would be integration of aspect
    application in sciunit tests~\cite{sciunit_primer}.
\end{itemize}

