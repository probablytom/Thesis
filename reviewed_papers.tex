\section{Papers Reviewed}  %% More of a dumping ground for notes which can be
                           %% incorporated later than a lit review section, for
                           %% now at least!

\subsection{System composition}

What's been written so far on system composition?

\subsubsection{Holons}\label{review:blair2015holons}

Notes on \cite{blair2015holons}\ldots

\subsubsection{Bigraphs}\label{review:milner2009space}

Obviously, discuss \cite{milner2009space}!

\subsubsection{Optimising modularity}

Some work's been done on finding the optimal degree of modularity in a system in \cite{frenken2012optimal}\ldots

\subsubsection{General Systems theory}

Worth mentioning what it means to compose systems in terms of general systems
theory, see \cite{boulding1956general}, \cite{polanyi1968life}\ldots



\subsection{Formalisms}

What work has been done on well-defined formalisms of systems concepts?

\subsubsection{Bigraphs}

This has already been discussed in \ref{review:milner2009space}, but it's worth
linking back to here as bigraphs are ultimately a relatively successful
formalisation attempt for systems.

\subsubsection{Holons}  %% Not sure this really counts as a formalism attempt,
                        %% but there's definition to be made here for sure!
I suppose technically this counts as a formalism! It's discussed above at \ref{review:blair2015holons}.

\subsubsection{UML \& SysML}  %% I suppose technically this counts

A graphical formalism for designing \& architecting systems. TODO: how well can
we model SOS in  UML/SysML?

\subsubsection{OPM}   

Interesting contrast to UML, also something here in
comparison to how Obashi treats processes (the note from                        
the OPM wikipedia page on Dori accidentally creating a                          
bipartite graph with pobjects and processes is fascinating)                     


\subsubsection{Petri net}

Not very useful for actual system modelling, but
an interesting focus on process instead of
structures...
