\documentclass[12pt]{article}

\usepackage{cleverref}

\usepackage[round]{natbib}
\bibliographystyle{plainnat}

\title{A review of systems literature}  %% what type of literature specifically?
\date{}
\author{Tom Wallis}

\begin{document}
\maketitle


\section{Papers Reviewed}  %% More of a dumping ground for notes which can be
                           %% incorporated later than a lit review section, for
                           %% now at least!

\subsection{System composition}

System composition is a fundamental component of systems of systems research,
and an increasingly important research area --- particularly in areas such as
military systems.\par

There are a few different approaches to system composition.

\subsubsection{Holons}\label{review:blair2015holons}

%% Notes on \cite{blair2015holons}\ldots

Blair et al.\cite{blair2015holons} have an interesting take on system
composition, where they define [sematically] a format for reasoning about
systems-of-systems and their composition. Particularly, what they're interested
in is the concept of a \emph{holon}. Their approach is focused toward systems
engineering in software and IT --- defining the requirements of a system in
order to function correctly, as well as defining what resources or properties it
provides other systems for the purpose of successful composition. The authors
assert that sharing of resources and properties allows for successful
composition, and that in defining what systems require and provide, engineers
can treat the holons --- the composed parts --- as black boxes.\par

Holons compose to create new holons, meaning that systems can be composed in a
nested fashion, which is quite nice. I can see this being useful for software
engineering, as we're already in the practice of defining requirements for our
packages through systems like Maven or PIP, but for fuzzier-defined systems this
kind of well-defined structure (such as a sociotechnical system) seems hard to
achieve.\par

I wonder whether we could use Obashi diagrams to define what the requirements of
a sub-system's dataflows are --- for example, a subgraph of an Obashi model will
have open links which need to be connected to things which are well-defined by
the full model. We could construct sub-graphs with well-defined requirements
instead, and then try to compose them.\par

\subsubsection{Families of Systems}

In \cite{meilich2006system}, the case is made that systems of systems should be
modelled as ``families of systems'' instead --- this largely involves moving
from a hierarchically structured tree of nodes composed of other nodes to a
graph of nodes which interact and work together. The difference seems slight,
but the impact is kind of profound in that the change really affects how we
think about system \emph{decomposition}. The point isn't terribly well made in
the paper, but it's a good idea nonetheless.\par

They note that ``While emergence has been a source of fascination for the
complexity community for some time, we still do not know how to deal with
emergent phenomena in a rigorous way.'' I wonder whether that's room for a
potential research topic (though perhaps we'd need something a little less broad\ldots)\par

``Capabilities that can be assembled or composed on-the-fly will be how
effectiveness will be measured.'' --- this strikes me as something which isn't
done terribly well at the moment. On one end of the scale, micro-service
architectures actually come at this problem pretty head-on and make a little
progress (think of automatic deployment via something like Ansible or Juju) ---
at the same time, rearranging / restructuring that deployment when a component
fails is very hard to do. Something like Kubernetes will just try to boot the
broken component again, but what if the issue isn't some blip and is actually a
broken architecture, where requirements have changed and the deployment itself
needs reworked? (Something like the Holon approach in
\cref{review:blair2015holons} might be helpful.)\par

On the other side of the spectrum, doing this in a socio-ecological system is
\emph{very} hard --- which is an issue, because the ability to re-assemble or
compose on the fly is important for system composition and for re-use of legacy
components, in something like a military system, but it's also very important
for system resilience if requirements change (as above) or if a system might
have changed during or after analysis and the change needs to be understood.
Systems which can be composed or re-arranged on the fly will be harder to assess
--- this is at the core of socio-ecological systems, where the systems \emph{do}
re-arrange themselves (as a result of something like darwinian evolution, where
the dependencies between different species change dynamically, and dependencies
might become unavailable as other species become extinct) --- and I'd imagine
it's what causes socio-ecological systems to be so hard to assess.\par

\subsubsection{Bigraphs}\label{review:milner2009space}

%% Obviously, discuss \cite{milner2009space}!

The seminal work on Bigraph modelling is \cite{milner2009space}, which defines
bigraphs, shows how they can be used, and demonstrates how different
communication calculi can be represented in a bigraphical notation.\par

Bigraphs are effectively the combination of two graphs: a \emph{place} graph,
which is a forest showing how a system is structured, and a \emph{link} graph, a
graph showing how nodes in the place graph communicate. Milner defines bigraphs
graphically but also categorically, which is really nice for the purposes of
formal analysis.\par


\subsubsection{Optimising modularity}

Some work is presented in \cite{frenken2012optimal} to find the optimal degree
of modularity/decomposition in a system, using an NK model. Their approach is
interesting and the paper is actually quite a neat read! 

However, I also feel that there are some problems in their approach.
Particularly, I think that their evaluation metric is a little off: they judge
optimal modularity bsed on search time within their model, rather than the
fitness of the point found. Surely ``optimal modularity'' is when the system
performs as well as it possibly can, rather than reducing search time (which I
imagine would be the convergence point of the system as it changes and
evolves)\ldots\par 

I'm going to put some more time into this paper, as I've not read it in as much
detail as I feel it merits. Labelled as ``to read in-depth''.

%%%%%%%%%%
%% Not filled in properly below here
%%%%%%%%%%

\subsubsection{General Systems theory}

Worth mentioning what it means to compose systems in terms of general systems
theory, see \cite{boulding1956general}, \cite{polanyi1968life}\ldots



\subsection{Formalisms}

What work has been done on well-defined formalisms of systems concepts?

\subsubsection{Why formalise?}

An interesting discussion arises in \cite{naylor1967verification} around whether
it's worthwhile formalising \emph{at all}. For the purposes of the work at hand,
we're certainly interested in creating formalisms, but the nature of those
formalisms is important: what are our models trying to show?\par

The authors note that there are three positions on verifying an arbitrary model:
\emph{rationalism}, where an axiomatic basis of the model can be assumed;
\emph{empiricism}, where only things observed in the world can be used as a
basis of a model; and \emph{positive economics}, where the basis of the model is
irrelevant so long as the predictions it produces are accurate.\par

In constructing a formalism of a system, we're in effect creating a model of how
an arbitrary system might be considered. Many modelling systems / formalisms
posit that the system is well represented by a graph, for example, emphasising
the relationships between the elements of the system. This assumption isn't
strictly necessary, however: one can imagine a systems model where relationships
between nodes are never considered, but emerge from interactions between agents
which are studied more in-depth.\par

I think it's important to consider the nature of \emph{what}'s being modelled
and what the purpose of the model is, then, because the assumptions can affect
even the underlying concepts of the formalism.\par

Note: an interesting criticism of the paper (published alongside it) notes that
a metric of a model's validity might be better taken as the \emph{usefulness} of
the model, because so many models are developed for a specific practical
purpose. If the model is inaccurate in some aspects, but more accurate in
others, then we should consider the cases which we care about for the task at
hand. The criticism outlines four cases where one creates models, and claims
that model utility is a more useful metric than the three focused on in this
paper in three out of four cases. They give some simple equations which allow
one to calculate model utility, which is nice.\par

\subsubsection{Bigraphs}

This has already been discussed in \ref{review:milner2009space}, but it's worth
linking back to here as bigraphs are ultimately a relatively successful
formalisation attempt for systems.

\subsubsection{Bigraphs with Sharing}

Work by Muffy and Michele Sevegnani on introducting sharing to bigraphs, kind of
like Obashi's element persistence. \cite{sevegnani2015bigraphs} This particular
paper is pretty intense, but really interesting and well-written!\par

I think there might be a way of encoding sharing in bigraphs using only Milner's
definitions, but I haven't seen how this is done yet. 

\subsubsection{Holons}\label{review:blair2015holons}  %% Not sure this really counts as a formalism attempt,
%% but there's definition to be made here for sure!
I suppose technically this counts as a formalism! It's discussed above at \ref{review:blair2015holons}.

\subsubsection{UML \& SysML}  %% I suppose technically this counts

A graphical formalism for designing \& architecting systems. TODO: how well can
we model SOS in  UML/SysML?

\subsubsection{OPM}   

Interesting contrast to UML, also something here in
comparison to how Obashi treats processes (the note from                        
the OPM wikipedia page on Dori accidentally creating a                          
bipartite graph with pobjects and processes is fascinating)                     


\subsubsection{Petri net}

Not very useful for actual system modelling, but
an interesting focus on process instead of
structures...

%% Worth mentioning condition-event nets here, which are a special transition
%% diagram based on petri nets from Milner's bigraph work in \cite{milner2009space}


\bibliography{references}

\end{document}