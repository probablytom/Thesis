\documentclass[12pt]{article}

\title{A review of systems literature}  %% what type of literature specifically?
\date{}
\author{Tom Wallis}

\begin{document}
\maketitle


\section{Papers Reviewed}  %% More of a dumping ground for notes which can be
                           %% incorporated later than a lit review section, for
                           %% now at least!

\subsection{System composition}

System composition is a fundamental component of systems of systems research,
and an increasingly important research area --- particularly in areas such as
military systems.\par

There are a few different approaches to system composition.

\subsubsection{Holons}\label{review:blair2015holons}

%% Notes on \cite{blair2015holons}\ldots

Blair et al.\cite{blair2015holons} have an interesting take on system
composition, where they define [sematically] a format for reasoning about
systems-of-systems and their composition. Particularly, what they're interested
in is the concept of a \emph{holon}. Their approach is focused toward systems
engineering in software and IT --- defining the requirements of a system in
order to function correctly, as well as defining what resources or properties it
provides other systems for the purpose of successful composition. The authors
assert that sharing of resources and properties allows for successful
composition, and that in defining what systems require and provide, engineers
can treat the holons --- the composed parts --- as black boxes.\par

Holons compose to create new holons, meaning that systems can be composed in a
nested fashion, which is quite nice. I can see this being useful for software
engineering, as we're already in the practice of defining requirements for our
packages through systems like Maven or PIP, but for fuzzier-defined systems this
kind of well-defined structure (such as a sociotechnical system) seems hard to
achieve.\par

I wonder whether we could use Obashi diagrams to define what the requirements of
a sub-system's dataflows are --- for example, a subgraph of an Obashi model will
have open links which need to be connected to things which are well-defined by
the full model. We could construct sub-graphs with well-defined requirements
instead, and then try to compose them.\par

\subsubsection{Bigraphs}\label{review:milner2009space}

%% Obviously, discuss \cite{milner2009space}!

The seminal work on Bigraph modelling is \cite{milner2009space}, which defines
bigraphs, shows how they can be used, and demonstrates how different
communication calculi can be represented in a bigraphical notation.\par

Bigraphs are effectively the combination of two graphs: a \emph{place} graph,
which is a forest showing how a system is structured, and a \emph{link} graph, a
graph showing how nodes in the place graph communicate. Milner defines bigraphs
graphically but also categorically, which is really nice for the purposes of
formal analysis.\par


\subsubsection{Optimising modularity}

Some work is presented in \cite{frenken2012optimal} to find the optimal degree
of modularity/decomposition in a system, using an NK model. Their approach is
interesting and the paper is actually quite a neat read! 

However, I also feel that there are some problems in their approach.
Particularly, I think that their evaluation metric is a little off: they judge
optimal modularity bsed on search time within their model, rather than the
fitness of the point found. Surely ``optimal modularity'' is when the system
performs as well as it possibly can, rather than reducing search time (which I
imagine would be the convergence point of the system as it changes and
evolves)\ldots\par 

I'm going to put some more time into this paper, as I've not read it in as much
detail as I feel it merits. Labelled as ``to read in-depth''.

%%%%%%%%%%
%% Not filled in properly below here
%%%%%%%%%%

\subsubsection{General Systems theory}

Worth mentioning what it means to compose systems in terms of general systems
theory, see \cite{boulding1956general}, \cite{polanyi1968life}\ldots



\subsection{Formalisms}

What work has been done on well-defined formalisms of systems concepts?

\subsubsection{Bigraphs}

This has already been discussed in \ref{review:milner2009space}, but it's worth
linking back to here as bigraphs are ultimately a relatively successful
formalisation attempt for systems.

\subsubsection{Bigraphs with Sharing}

Work by Muffy and Michele Sevegnani on introducting sharing to bigraphs, kind of
like Obashi's element persistence. \cite{sevegnani2015bigraphs} This particular
paper is pretty intense, but really interesting and well-written!\par

I think there might be a way of encoding sharing in bigraphs using only Milner's
definitions, but I haven't seen how this is done yet. 

\subsubsection{Holons}  %% Not sure this really counts as a formalism attempt,
%% but there's definition to be made here for sure!
I suppose technically this counts as a formalism! It's discussed above at \ref{review:blair2015holons}.

\subsubsection{UML \& SysML}  %% I suppose technically this counts

A graphical formalism for designing \& architecting systems. TODO: how well can
we model SOS in  UML/SysML?

\subsubsection{OPM}   

Interesting contrast to UML, also something here in
comparison to how Obashi treats processes (the note from                        
the OPM wikipedia page on Dori accidentally creating a                          
bipartite graph with pobjects and processes is fascinating)                     


\subsubsection{Petri net}

Not very useful for actual system modelling, but
an interesting focus on process instead of
structures...

%% Worth mentioning condition-event nets here, which are a special transition
%% diagram based on petri nets from Milner's bigraph work in \cite{milner2009space}


\end{document}